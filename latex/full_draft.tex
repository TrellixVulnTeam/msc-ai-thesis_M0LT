\documentclass{report}

\usepackage[final]{neurips_2018}
\usepackage{amsmath}
\usepackage{amsfonts}
\usepackage[utf8]{inputenc} % allow utf-8 input
\usepackage[T1]{fontenc}    % use 8-bit T1 fonts
\usepackage{hyperref}       % hyperlinks
\usepackage{url}            % simple URL typesetting
\usepackage{booktabs}       % professional-quality tables
\usepackage{amsfonts}       % blackboard math symbols
\usepackage{nicefrac}       % compact symbols for 1/2, etc.
\usepackage{microtype}      % microtypography
\usepackage{graphicx}
\usepackage{lipsum}
\usepackage{physics}
\usepackage{enumitem} % for custom bulletpoints
\usepackage{float} % for image position forcing
\usepackage{makeidx} % for indices and table of contents
\usepackage{natbib}
\usepackage{siunitx}
\usepackage{caption}
\usepackage{subcaption}
\usepackage{color, soul}
\usepackage[table]{xcolor}
\usepackage{makecell}
\usepackage{tabstackengine}

\usepackage{mwe} % for minipages/having to separate figures side-by-side

\usepackage{tikz}
\usetikzlibrary{bayesnet}
\usetikzlibrary{arrows}

\usepackage[ruled,vlined]{algorithm2e} % for algorithms/pseudocode

\usepackage{hyperref} % for hyperlink referencing and clickable urls

\usepackage{mathrsfs} % for fancy script letters in mathmode using \mathscr

\usepackage{setspace}

\newcommand{\red}[1]{{\color{red}{#1}}}

\newcommand{\len}[1]{\textcolor{purple}{[L: #1]}} %LJ

\usepackage{amssymb}% http://ctan.org/pkg/amssymb
\usepackage{pifont}% http://ctan.org/pkg/pifont
\newcommand{\cmark}{\ding{51}}%
\newcommand{\xmark}{\ding{55}}%

%%%% For footnotes listing
\usepackage{tocloft}

%% Footnotes-Listing %%
\newcommand{\listfootnotesname}{List of Footnotes}% 'List of Footnotes' title 
\newlistof[chapter]{footnotes}{fnt}{\listfootnotesname}% New 'List of...' for footnotes 
\let\oldfootnote\footnote % Save the old \footnote{...} command 
\renewcommand\footnote[1]{% Redefine the new footnote to also add 'List of Footnote' entries. 
    \refstepcounter{footnotes}% Add and step a reference to the footnote/counter. 
    \oldfootnote{#1}% Make a regular footnote. 
    \addcontentsline{fnt}{footnotes}{\protect 
\numberline{\thefootnotes}#1}% Add the 'List of...' entry. 
}





% \title{A Plug-and-Play Approach to Age-Adaptive Dialogue Generation}
% \title{input{sections/title-page-ai}}

% \author{%
%   Lennert Jansen\\
%   lennertjansen95@gmail.com \\
%   Working draft
% }

\begin{document} 

% \documentclass{article}

\usepackage{graphicx, color}

\usepackage[a4paper,margin=2cm]{geometry}



\newcommand{\red}[1]{{\color{red}{#1}}}



\begin{document}





\begin{titlepage}



\newcommand{\HRule}{\rule{\linewidth}{0.5mm}} % Defines a new command for the horizontal lines, change thickness here

\center % Center everything on the page

 

%----------------------------------------------------------------------------------------

%	HEADING SECTIONS

%----------------------------------------------------------------------------------------



\includegraphics[width=\linewidth]{figures/title_page/uvalogo_regular_p_en.jpg}\\[2.5cm]

\textsc{\Large MSc Artificial Intelligence}\\[0.2cm]

\textsc{\Large Master Thesis}\\[0.5cm] 



%----------------------------------------------------------------------------------------

%	TITLE SECTION

%----------------------------------------------------------------------------------------



\HRule \\[0.4cm]

{ \huge \bfseries \red{A Plug-and-Play Approach to\\ Age-Adaptive Dialogue Generation}}\\[0.4cm] % Title of your document

\HRule \\[0.5cm]

 

%----------------------------------------------------------------------------------------

%	AUTHOR SECTION

%----------------------------------------------------------------------------------------



by\\[0.2cm]

\textsc{\Large \red{Lennert Jansen}}\\[0.2cm] %you name

\red{10488952}\\[1cm]





%----------------------------------------------------------------------------------------

%	DATE SECTION

%----------------------------------------------------------------------------------------



{\Large \today}\\[1cm] % Date, change the \today to a set date if you want to be precise



\red{Number of Credits}\\ %

\red{Period in which the research was carried out}\\[1cm]%



%----------------------------------------------------------------------------------------

%	COMMITTEE SECTION

%----------------------------------------------------------------------------------------

\begin{minipage}[t]{0.4\textwidth}

\begin{flushleft} \large

\emph{Supervisor:} \\

\red{Dr \textsc{Sandro Pezzelle} }% Supervisor's Name

\end{flushleft}

\end{minipage}

~

\begin{minipage}[t]{0.4\textwidth}

\begin{flushright} \large

\emph{Assessor:} \\

\red{Dr \textsc{Raquel Fernández}}\\

\end{flushright}

\end{minipage}\\[2cm]



%----------------------------------------------------------------------------------------

%	LOGO SECTION

%----------------------------------------------------------------------------------------



% \framebox{\rule{0pt}{2.5cm}\rule{2.5cm}{0pt}}\\[0.5cm]

\includegraphics[width=2.5cm]{figures/title_page/illc_no_text_logo.jpg}\\ % Include a department/university logo - this will require the graphicx package

\textsc{\large \red{Institute for Logic, Language and Computation}}\\[1.0cm] % 

 

%----------------------------------------------------------------------------------------



\vfill % Fill the rest of the page with whitespace



\end{titlepage}

\end{document}



% \maketitle

%==================== BEGIN TITLE PAGE ==========================%
\afterpage{%
\newgeometry{a4paper,margin=2cm}
\begin{titlepage}



\newcommand{\HRule}{\rule{\linewidth}{0.5mm}} % Defines a new command for the horizontal lines, change thickness here

\center % Center everything on the page

 

%----------------------------------------------------------------------------------------

%	HEADING SECTIONS

%----------------------------------------------------------------------------------------



\includegraphics[width=\linewidth]{figures/title_page/uvalogo_regular_p_en.jpg}\\[2.5cm]

\textsc{\Large MSc Artificial Intelligence}\\[0.2cm]

\textsc{\Large Master Thesis}\\[0.5cm] 



%----------------------------------------------------------------------------------------

%	TITLE SECTION

%----------------------------------------------------------------------------------------



\HRule \\[0.4cm]

{ \huge \bfseries A Plug-and-Play Approach to\\ Age-Adaptive Dialogue Generation}\\[0.4cm] % Title of your document

\HRule \\[0.5cm]

 

%----------------------------------------------------------------------------------------

%	AUTHOR SECTION

%----------------------------------------------------------------------------------------



by\\[0.2cm]

\textsc{\Large Lennert Jansen}\\[0.2cm] %you name

10488952\\[1cm]





%----------------------------------------------------------------------------------------

%	DATE SECTION

%----------------------------------------------------------------------------------------



{\Large \today}\\[1cm] % Date, change the \today to a set date if you want to be precise



48EC\\ %

January 11, 2021 - \today\\[1cm]%



%----------------------------------------------------------------------------------------

%	COMMITTEE SECTION

%----------------------------------------------------------------------------------------

\begin{minipage}[t]{0.4\textwidth}

\begin{flushleft} \large

\emph{Supervisors:} \\

Dr \textsc{Sandro Pezzelle} \\ Dr \textsc{Raquel Fernández} \\ Dr \textsc{Arabella Sinclair}% Supervisor's Name

\end{flushleft}

\end{minipage}

~

\begin{minipage}[t]{0.4\textwidth}

\begin{flushright} \large

\emph{Assessor:} \\

Dr \textsc{Wilker Aziz}

% \red{Dr \textsc{Raquel Fernández}}\\

\end{flushright}

\end{minipage}\\[2cm]



%----------------------------------------------------------------------------------------

%	LOGO SECTION

%----------------------------------------------------------------------------------------



% \framebox{\rule{0pt}{2.5cm}\rule{2.5cm}{0pt}}\\[0.5cm]

\includegraphics[width=2.5cm]{figures/title_page/illc_no_text_logo.jpg}\\ % Include a department/university logo - this will require the graphicx package

\textsc{\large Institute for Logic, Language and Computation}\\[1.0cm] % 

 

%----------------------------------------------------------------------------------------



\vfill % Fill the rest of the page with whitespace



\end{titlepage}
\clearpage
\restoregeometry
}

%===================== END TITLE PAGE ===========================%

%====================== ABSTRACT ================================%


%====================== END OF ABSTRACT =========================%


%====================== ACKNOWLEDGMENTS ================================%


%====================== END OF ACKNOWLEDGMENTS =========================%

%===================== TABLE OF CONTENTS & FOOTNOTES ===========%
\doublespacing
\tableofcontents
\listoffigures
\listoftables
\listoffootnotes

% \newpage
% \section{Research questions}
% \begin{enumerate}
    \item To what extent can a classifier automatically detect age-related linguistic differences in natural language? And which features are most helpful in age-group detection?
    \begin{enumerate}
        \item Do they (i.e., the linguistic or latent features exploited by the classifier) match the age-related informative features reported in previous work?
        \begin{itemize}
            \item \textbf{Hypothesis:} I expect the age-group of a text's producer to be reliably detectable. Also, I suspect efficacy of the models to increase with their complexity, with complexity going from $n$-gram, to RNN-based, to Transformer-based. I suspect linguistic trends in age found in \cite{pennebaker2003words} and \cite{schler2006effects} to persist in the feature importance of the lexically focused $n$-gram models. Moreover, I expect the encoder-decoder architectures to uncover more latent age-related patterns in language that eluded earlier automated age-detection work, e.g., \cite{nguyen-etal-2011-author}. Finally, I suspect the differences in language use between age-groups to lie more at the structural than lexical level.
            \item \textbf{Implications:} This question aims to lay the foundation of the later research objectives by first establishing whether a discriminator can successfully categorize discourse into correct user age-brackets. I plan to build increasingly complex classifiers, starting from a bag-of-words (BoW) $n$-gram model, progressing to embedding-based architectures (LSTM and BERT). The next entailed goal is to investigate which aspects of users' language use seem to play a role in determining the corresponding age groups. NB: these salient characteristics could also express themselves in the latent spaces of neural classifiers. Finding and understanding such features will not only help me better understand the relationship between age-group and word use, but also provide guidelines for the development of controllable conversational response generators later in the project.
        \end{itemize}
    \end{enumerate}
    
    \item Can large-scale language models (LMs), i.e. GPT-2 \citep{radford2019language} or DialoGPT \citep{zhang2019dialogpt}, be leveraged for text generation, controlled for age-groups?
    \begin{enumerate}
        \item What role does the used data play in the differences in output and performance between the base (conversational) language models and their controlled counterparts?
        \begin{itemize}
            \item \textbf{Hypothesis:} Based on \cite{madotto-etal-2020-plug, dathathri2019plug, pennebaker2003words}, I expect they can. Assuming that age-group is distinctive enough a trait of language use for it to notably affect writing style when controlled for (see previous hypothesis), age can be seen as an attribute input that neatly fits in the the plug-and-play setup proposed by \citeauthor{dathathri2019plug}. My research is conducted using \textbf{(1)} a dataset of blogs labeled by author age, and \textbf{(2)} a collection of transcribed conversations labeled by speaker age. I do not expect the controllability of the guided generation model to be impacted by the fact that one is conversational and the other is not. The ways that I do expect the choice of data for the controllable system to impact its performance and implications are differences in the mode of discourse (written versus spoken), recency (blogs are from 2004, conversations from 2014), and input length. Finally, it is to be expected from \cite{dathathri2019plug, madotto-etal-2020-plug} that the fluency of the base language model will tend to deteriorate after controlling for either dataset.
            \item \textbf{Implications:} To confirm the PPLM setup with age as an attribute input. First step in achieving adaptive dialogue generation, as non-conversational text generation is less restrictive.
        \end{itemize}
    \end{enumerate}
    \item To what degree is such a controlled text generation (CTG) model successful in dialogue that is adaptive w.r.t. age, such that it has a detectable effect on the perception of the user?
    \begin{itemize}
        \item \textbf{Hypothesis:} I expect the perceived quality of the dialogues with the adaptive system to be adequate, but decrease for increasing numbers of conversational turns with a user. It can also be expected that evaluating if generated responses are representative of age-specific vernacular will be more decisive with automatic than human evaluation. Nevertheless, I expect the age-differences in generated language to be noticeable by humans. 
        \item \textbf{Implications:} This is in alignment with the objectives of the Human(e) AI research project. The ultimate challenge of this thesis will be to leverage the grammatical fluency of GPT-2 or DialoGPT and the inexpensive tunability of \citeauthor{dathathri2019plug}'s PPLM and \cite{madotto-etal-2020-plug}'s PPCM setups for age-adaptive neural conversational response generation.
    \end{itemize}
\end{enumerate}

% \textbf{What is novel or contributing about your research?}
% \begin{itemize}
%     \item \textit{To the best of my knowledge, there is no work on controllable text generation for dialogue responses. Moreover, I also haven't found instances of neural conversational response systems adapting word use to the user's age.}
% \end{itemize}


\newpage
\chapter{Introduction}
% \textbf{Things to keep in mind when writing this section:}
% \begin{itemize}
%     \item Doesn't have to be longer than 1 page.
%     \item (optional): anecdotal opener to draw attention.
%     \item Brief introduction to the topic of controllable text generation and adaptive dialogue systems.
%     \item Description of the research problem and your questions.
%     \item Placeholder for my contributions and approach.
%     \item Placeholder for overview of remainder of paper.
%     \item \textit{For consistency, maybe start it off with an introduction that you can re-use for your final version. Of course, this will be rewritten for later versions to include findings.}
%     \item The three most important papers to keep in mind for this section are: PPLM, DialoGPT, and Personalized Dialogue Generation with Diversified Traits
% \end{itemize}


 

In recent years, we have witnessed promising advances in natural language processing (NLP) tasks, such as language modeling, reading comprehension, machine translation, controllable text generation, and conversational response generation \citep{radford2019language, DBLP:journals/corr/BahdanauCB14, dathathri2019plug, madotto-etal-2020-plug}. \cite{vaswani2017attention}'s Transformer architecture plays a central role in many of the state of the art (SotA) solutions to these problems. Transformer-based language models (LMs) pre-trained on massive amounts of textual data, most famously OpenAI's GPT-2 (Generative Pre-trained Transformer-2), have demonstrated their usefulness for several of the aforementioned NLP tasks \citep{radford2019language}. For instance, controllable text generation and producing dialogue responses have improved greatly because of GPT-based hybrid models. %Given the non-trivial requirements for sufficiently massive training corpora and computational resources, GPT-based solutions are worth considering for these two problems.

\len{CTG comes out of the blue in the following paragraph. Introduce it a little bit by describing what is is, and why/how it is an important task.}

\begin{itemize}
    \item Controllable text generation entails generating text samples that possess a predefined textual property, like having a positive sentiment, or being about a certain topic.
    \item Controlling more fine-grained linguistic properties, like resemblance of age-specific vernacular, still poses an important, yet unsolved/insufficiently studied (?) challenge.
    \item Personalized interaction between humans and AI systems is crucial to obtain systems that can be trusted by users and are perceived as natural.
    \item (Age-)adaptive language generation can be used to personalize AI-powered personal assistants like Siri and Alexa, improving user experience and trust.
    \item It is important for AI-power conversational agents to be accessible to varying user profiles, rather than targeted at one particular user group. 
    \item In this work, I/we focus on one aspect that may influence successful personalization of conversational agents: user age profile.
\end{itemize}

Controllable text generation (CTG) aims to enforce abstract properties, like writing style, on the passages being produced. Fine-tuning large-scale LMs for writing-style adaptation is extremely expensive, but \cite{dathathri2019plug} and \cite{li-etal-2020-optimus} propose methods that both excel at the task, while bypassing significant retraining costs. Dialogue response generation is the task of producing replies to a conversational agent's prompts, in a manner that is ideally both non-repetitive and relevant to the course of the conversation. With DialoGPT, \cite{zhang2019dialogpt} also manage to leverage GPT-2's powerful fluency for dialogue tasks, by framing them as language modeling tasks where multi-turn dialogue sessions are seen as long texts.

% CTG and dialogue response generation share the overarching objective of producing grammatically correct text that is distinct from any training instance. LJ/RF: this isn't necessarily correct. Every generated response doesn't have to be distinct from any training objective.

\len{Introduce dialogue response generation a bit more. Also emphasize its importance. And then introduce the combined task and its importance.}
A blend of CTG and dialogue response generation, i.e., controllable dialogue response generation, is an interesting and only partially explored route. It ties closely to one of Artificial Intelligence's long-standing goals of achieving human-like conversation with machines, as humans are known to adapt their language use to the characteristics of their interlocutor \citep{gallois2015communication}. Adaptive dialogue generation is difficult due to the challenge of representing traits, like age, gender, or other persona-labeled traits via language expression \citep{zheng2019personalized}.
% and the lack of persona trait labeled dialogue datasets. \cite{zheng2019personalized} explore the problem of personalized dialogue generation, and introduced \texttt{PersonalDialog} a large-scale multi-turn Chinese Mandarin dialogue dataset with personality trait labeling, and persona-aware adaptive dialogue generation models using RNNs and attention mechanisms. 

In this thesis, I investigate the problem of controllable dialogue generation, with a focus on adapting responses to users' age. As a preliminary research objective, I aim to study to what extent a classifier can detect age-related linguistic differences in natural language, and which features are most helpful in age-group detection. Do they (i.e., the linguistic or latent features exploited by the classifier) match the age-related informative features reported in previous work? 
After empirically confirming that speaker age detection is possible, I explore whether large-scale LMs, e.g. GPT-2, can be leveraged for text generation, controlled for age-groups. And what role does the used data play in the differences in output and performance between regular GPT-2 and controllable GPT-2?
Finally, my research focuses on the degree to which such a CTG model is successful in generating dialogue that is adaptive w.r.t. age, such that it has a detectable effect on the perception of the user.

The remainder of this thesis is structured as follows: Chapter \ref{sec:background} positions the subject of controlled text generation in its theoretical background, and and \ref{sec:related_work} compares it to the most relevant related work. The methodology in Chapter \ref{ch:methods} gives detailed explanations of the most important modeling methods and techniques used for this research.

% The main contributions of this paper are ...

% The remainder of this work is structured as follows, ...

\begin{itemize}
    \item When introducing your own work and proposing your hypothesis, use the following argument: \textit{This idea that age prediction from text is more challenging than topic or sentiment prediction could be an indication that controlled language generation for age-differences is also a more nuanced problem than topical steered text generation.}
\end{itemize}

\chapter{Literature Review}\label{ch:literature_review}
% \len{TODO - Add a brief general introduction to the chapter here. See Sandro's comment. E.g:}

This chapter is a two-part literature review. The first section, Background or Section \ref{sec:background}, provides an overview of this thesis' central problem of controllable dialogue generation and the components involved, i.e., 
modeling age in language, language models, controlled text generation, dialogue, dialogue systems, dialogue generation, Transformers, and Plug-and-Play language models. 
In the second section, Related Work or Section \ref{sec:related_work}, I discuss previous approaches relevant to my work that have been proposed to tackle each of these components, either separately or jointly. Approaches to different but strongly related problems, like text style transfer, are also described in Section \ref{sec:related_work}.

\section{Background}
\label{sec:background}

% \textit{What is the relevant theoretical background material to cover?}
% \begin{itemize}
%     \item Language modelling.
%     \item (Controllable) Text Generation.
%     \item GPT-x.
%     \item Transformers.
%     \item Dialogue Modelling.
%     \begin{itemize}
%         \item Adaptive dialogue systems.
%     \end{itemize}
%     \item Age and language.
%     \begin{itemize}
%         \item The theoretical relationship between age and language.
%         \item Age(-group) detection from text.
%     \end{itemize}
% \end{itemize}

% \textit{Writing goals of this subsection:}
% \begin{itemize}
%     \item Position my research (i.e., dialogue response generation that is adaptive to age-groups) in its relevant background.
%     \item Explain concepts that must be understood to grasp my research topic (i.e., give theoretical background information for the rest of your thesis).
% \end{itemize}

% \len{Keep in mind the following distinction between Background and Related Work - The Background section should give an overview of the problem and the components involved: dialogue, language generation, dialogue generation, age modelling, etc., without focusing on one or the other approach — in Related Work, you describe approaches that have been proposed to tackle each of these components, separately or jointly, and which are related or relevant to your own work for some reason}


% \len{Comment by Sandro - \textit{This intro needs to be more explicitly connected to your work. I would recap once more what you are after (“I focus on controllable generation, i.e. … and apply it to dialogue … This naturally involves having a model for language generation that can be controlled to … In what follows, I introduce the crucial concepts behind this: models for language generation, methods to control the output … dialogue … ”}}

% I focus on controlled text generation, i.e., endowing automatically produced text with certain desired linguistic characteristics, and apply it to dialogue. Moreover, I seek to generate dialogue responses that possess linguistic characteristics of certain age groups. This naturally involves having a model for text generation that can be controlled to write texts passages with different age-related linguistic styles. In what follows, I therefore introduce the crucial components involved in developing such models: modeling age in language, language models, controlled text generation, dialogue, dialogue systems, dialogue generation, Transformers, and Plug-and-Play language models.

% \len{isn't this redundant, given the chapter's introduction?}

\subsection{Language and Age}\label{subsec:background_lang_age}
The relationship between a person's age and use of language is a thoroughly studied subject with a decades-long history and inherent challenges \citep{pennebaker2003words, nguyen2014gender, zheng2019personalized}. Several factors like community membership (e.g., gender, socioeconomic status, or political affiliation), experimental condition (e.g., emotional versus non-emotional disclosure), mode of disclosure (writing versus talking), and other confounding variables complicate the study of age's relation to language \citep{nguyen-etal-2011-author}. The relatively recent advent of widely available computational resources and vast amounts of textual data made it possible to leverage machine learning methods to help detect patterns in language that eluded previous conventional sociolinguistic research. Early computational investigations into the connection between a person's age and use of language are typically a combination of qualitative and statistical methods. For instance, using a mix between their proprietary count-based text analysis framework, Linguistic Inquiry and Word Count (LIWC), and sociolinguistic theory, \cite{pennebaker2003words} study the changes in written and spoken language use with increasing age. They discuss four important areas of a person's character that have been found to change with age: emotional experience and expression, identity and social relationships, time orientation, and cognitive abilities. These four axes and their hypothesized relationships with language use and age can be interpreted in the following ways:
\begin{enumerate}
    \item \textit{Emotional experience and expression:} This is the relationship between increasing age and linguistically observable manifestations of a person's experienced emotions. In practical terms, this is framed as detectable instances of positive and negative affect in language. This complex relationship between age and emotional expression is characterized by decreased levels of negative affect and slightly non-decreasing levels of positive affect. This is also confirmed by the findings of \cite{schler2006effects}.
    
    \item \textit{Sense of identity and social relationships:} These terms refer to developmental tends in one's relation to self and others, as expressed in their language, e.g., as references to self (\textit{I, me, my}, and \textit{we, us, our}) or others (\textit{they, them, theirs}). \cite{pennebaker2003words} report that the \textit{quantity} of social connections decreases, and the \textit{quality} of remaining relationships increases with age.
    
    \item \textit{Time orientation:} This relationship describes how people express their perception of and orientation towards time. For instance, this can be indicated by the use of time-related verb tenses. The authors suggest that older individuals tend to be more past-oriented than their younger future-oriented counterparts.
    
    \item \textit{Cognitive abilities:} This refers to markers of cognitive capacity in language. Aging is expected to be associated with less use of cognitively complex words after a certain mid-adulthood peak. Specifically, the relationship between markers of cognitive complexity in natural language (cognitive mechanisms, causal insight, and exclusive words) and age is hypothesized to be curvilinear, with markers of verbal ability only declining very late in life. 
    % And because verbal ability does not decline until very late in life, markers of verbal ability (e.g., use of big words) are not expected to show changes with age.
\end{enumerate}

\cite{pennebaker2003words} consider the following variables: positive and negative emotions, first-person singular and first-person plural pronouns, social references, time-related words (past-tense, present-tense, and future-tense verbs), big words ($>6$ letters), cognitive mechanisms, causal insight, and exclusive words. Their main findings suggest that for increasing age, people use more positive and fewer negative affect words, use fewer self-references, use more future-tense and fewer past-tense verbs, and exhibit a general pattern of increasing cognitive complexity.

Detectable linguistic differences between age groups can often be attributed to the use of language fads or references to age-specific popular culture. For instance, \cite{schler2006effects} find that the use of slang and neologisms (such as \textit{lol} and \textit{ur}) are strong indicators of youth. Similarly, words like `Facebook', `Instagram', and `Netflix' appear in the most frequently used words by younger participants of conversational data collection efforts, like that of the British National Corpus' spoken component \citep{love-spoken-bnc-2014}.

Despite the demonstrated utility of using handcrafted age-related linguistic features for age group detection from text \citep{schler2006effects}, modern approaches leverage statistical language models that represent a probability distribution over sequences of words, often parameterized by neural network architectures \citep{zheng2019personalized}. These approaches based on language models show impressive efficacy for many NLP tasks and can be trained end-to-end, often not requiring domain knowledge. The following section provides further details about language models.

% More recent studies, like that of \cite{nguyen-etal-2011-author}, \cite{zheng2019personalized}, and \cite{abdallah2020age}, frame age prediction from text as traditional machine learning problems, like linear regression, support vector machines, or neural architectures. These modeling approaches tend to reveal that strong indicators of age lie at the syntactic or structural level of language use, as opposed to the more content-based lexical level. Furthermore, this could explain why automatic detection from text of more content-based traits, like topic or sentiment, tend to be easier problems to solve than age prediction from text. To emphasize one such complicating factor, \cite{nguyen2014gender} argue that differences in language use are often relation to the speaker's social identity, which could differ from their biological identity. 
% This idea that age prediction from text is more challenging than topic or sentiment prediction could be an indication that controlled language generation for age-differences is also a more nuanced problem than topical steered text generation.


% \textbf{Keep in mind:}
% \begin{itemize}
%     \item What is known about the relationship between a speaker's (or author's) age and their linguistic characteristics? I.e., how does language use develop with age according to the existing literature?
%     \item How can I automatically detect a speaker's age(-group) from their utterances using machine learning?
%     \item Link findings from the papers you cite to the hypothesis that linguistic differences in age lie more are the phrase/constructional level than at the single word/lexical level. However, is this really the case? \cite{nguyen-etal-2011-author} mention single words like `well', `like`, and `just' being predictive of age.
%     \item Mention that age differences in language could/should also be interpreted as indications of language fads during a person's formative years.
%     \item Hypothesize that age-related linguistic variation is a more subtle and nuanced factor to control for during text generation than, e.g., topic (science, business, religion), or sentiment.
%     \item Mention confounding factors like community membership (like gender, socioeconomic status, or political affiliation), experimental condition (e.g., emotional versus non-emotional disclosure), mode of disclosure (writing versus talking), and other confounding factors that complicate age prediction from language.
% \end{itemize}

\subsection{Language Models}

Generally speaking, language modeling is central to many NLP tasks. A language model (LM) is a probability distribution over words in a sentence or document. Language models are trained to predict the probability of the next word in a sentence, given the preceding sequence of words. The language modeling task is formulated as an unsupervised distribution estimation problem of data points $\{\textbf{x}_1, ..., \textbf{x}_N\}$ (e.g., documents), each representing sequences (of, e.g., symbols or tokens) of varying lengths $(s_{i, 1}, ..., s_{i, n}), i \in \{1, ..., N\}$. Note that $N$ denotes the corpus size, and $n$ the sequence length of data point $i$. To avoid cluttered notation, the subscript $i$ will sometimes be omitted when discussing an arbitrary data point. The probability distribution over an observation $\textbf{x}$  (i.e., the joint probability of an ordered sequence) can then be factorized as the product of its constituent conditionals \citep{radford2019language}:

\begin{equation}
    p(\textbf{x}) = \prod_{j = 1}^n  p(s_j | s_1, ..., s_{j - 1}).
\end{equation}

This formulation allows language models to detect and learn patterns in language. The learned representations of these patterns can then be used for a plethora of applications, such as classification and text generation. Moreover, this results in a framework for tractable sampling from the unconditional language model $p(\textbf{x})$. $p(\textbf{x})$ can therefore be seen as a base generative model that can generate sample sentences \citep{dathathri2019plug}.

In recent years, the attention-based models, Transformers \citep{vaswani2017attention}, have replaced recurrent neural networks (RNNs) as the dominant architecture for LMs, with significant improvements in distribution estimation, long-range dependency handling, sample diversity, and parallel processing. Another recent development in language modeling is that of pre-training LMs on massive corpora. So-called large-scale general-purpose LMs have demonstrated significant improvements in downstream tasks, i.e., other NLP tasks for which the model was not specifically trained or fine-tuned. Most famously the OpenAI's series of Generative Pre-trained Transformer (GPT) models have improved numerous NLP benchmarks \citep{radford2018improving,radford2019language,brown2020language-models-few-shot-gpt3}. In particular, this series of autoregressive language models has demonstrated impressive advancements in the task of text generation, to produce human-like text.



% \begin{itemize}
%     \item What are language models? / What is language modelling?
%     \item Why is it important for many NLP tasks?
%     \item How is it typically formulated or approached (very brief mathematical explanation)?
%     \item Why is it important (to understand this) for my research?
%     \begin{itemize}
%         \item Tractable sampling from the joint $\rightarrow$ (text) generation.
%     \end{itemize}
%     \item Introduce pre-trained language models.
%     \item Briefly introduce the Transformer (but not in too much technical detail, as it could be necessary to explain it in Methods).
%     \item Same for GPT-2.
% \end{itemize}

\subsection{Text Generation and Controlled Text Generation}
\label{subsec:text_gen_and_ctg}

% \begin{itemize}
%     \item Describe the text generation task.
%     \item What is controllability in text generation? Why is it challenging?
%     \item One of the main challenges in CTG using large-scale LMs is the cost of fine-tuning. Expand on this. How can this be solved?
% \end{itemize}

In \textit{text generation}, a language model $p(\textbf{x})$ is asked to produce text $\textbf{x}$ given a prompt by sampling from the distribution of words that are assigned the highest likelihood of following the prompt \citep{radford2019language}. More concisely, text generation in itself is the task of generating a piece of text given an input text. This process can be seen as sampling from a conditional distribution, $p(\textbf{x} | \texttt{prompt})$. \textit{Controlled text generation} refers to the more restrictive problem of enforcing higher-level linguistic features on the generated text during sampling \citep{dathathri2019plug, prabhumoye-etal-2020-exploring}. This can be seen as a sub-problem of vanilla text generation, because the conditioning factor for the output text is further constrained to also include some predefined textual attribute, $a$. That is, controlled text generation is analogous to sampling from the conditional distribution, $p(\textbf{x} | \texttt{prompt}, a)$, where the attribute, $a$, represents a linguistic characteristic of the text, like sentiment, topic, or writing style. 

% \begin{itemize}
%     \item base language model $p(\textbf{x})$
%     \item attribute model $p(a | \textbf{x})$
%     \item CTG model $p(\textbf{x} | a ) \propto p(a | \textbf{x})p(\textbf{x})$
% \end{itemize}

Controlled text generation or CTG is a more challenging problem than vanilla text generation for several reasons. First, defining the desired attribute to be controlled for in a manner that is intelligible for a machine is a challenge in itself \citep{zheng2019personalized}. Second, like many NLP problems, there are not many parallel corpora \citep{dai2019style}. In the context of controlled generation, parallel corpora are datasets of target and source texts that only differ with respect to some attribute. Furthermore, the measure of attribute adherence is a very vague and ambiguous concept \citep{dathathri2019plug, dai2019style}. Namely, a text can be written to have an extremely positive sentiment in multiple formulations, all of which adhere to the positive sentiment. Another important hurdle for controlled text generation, especially when CTG is combined with leveraging the linguistic power of large-scale language models, is that the cost of fine-tuning or pre-training a model to control for a linguistic attribute can be very high \citep{dathathri2019plug, madotto-etal-2020-plug}. 

%=================================================================================
%%%%%%% THE FOLLOWING PARAGRAPH HAS BEEN MOVED TO THE RELATED WORK SECTION %%%%%%%
% PPLM \citep{dathathri2019plug} is an example of a recent work whose primary focus is to leverage powerful large-scale language models and making them controllable for a wide variety of linguistic attributes, while avoiding incurring significant costs of fine-tuning. Nevertheless avoiding these costs is anything but trivial. The plug and play set up of the PPLM model forms one of the main theoretical foundations of this work. It is both logical and promising for every day engineers to be able to leverage the grammatical fluency of pre-trained language models for more specific downstream tasks, e.g., specifying linguistic characteristic to enforce on an automatically written text. Their setup consists of a symbiosis of GPT-2 as their powerful large-scale language model, and a significantly smaller and therefore easier to train and fine-tune attribute model. This attribute model is often a small classifier, and can range in complexity from a simple bag of words model with a logistic classifier to a more complicated transformer encoder head. The main benefit of this setup is the extensibility it brings with it. Namely, such large-scale language models are open-source and available online and can be tailored to their specific needs using a significantly easier to train attribute model of your own liking. \cite{dathathri2019plug} demonstrate the applicability of their model on a wide variety of tasks ranging from text style transfer to language detoxification (all of which can be seen as sub-problems of controllable text generation). Other real-world applications include: being able to automatically re-write or adjust a draft text for an editorial, automatic generation of brand specific vacancy ads, or personalized chatbot assistance or even personalized education. What this work also provides is a starting point for new applications, namely controllable dialogue generation.
%=================================================================================

% \paragraph{Transformers}

% The transformer architecture in recent years has dominated numerous NLP tasks and has for the basis for many of the state of the art architecture is in natural language processing. Its masked self-attention and compatibility with parallel processing have made it both effective and handling long-range dependencies in texts, as well attractive in terms of training time. This ability to handle long-range dependencies is exploited in particular in the domain of pretraining large language models. Namely this allows for transformer models to be pre-trained by applying language modelling objectives to long stretches of texts that contain longer range dependencies compared to e.g. tweets or short reviews. 

% The transformer architecture mainly consists of a encoder and decoder structure. Where the encoder processes the embedded text inputs and produces a hidden state using self attention mechanisms and fully connected layers. This hidden state is then passed to the decoder layer which Produces an output which contains a distribution over the vocabulary and can be used for next word prediction.

% \paragraph{GPT}

% Following the success of open AI's pre-trained transformer model they went on to produce a series of generative pre-train transformers that I have been trained in a unsupervised language modelling session.

\subsection{Dialogue and Dialogue Systems}
\label{subsec:background_dialogue}\label{subsec:dialogue_generation_as_lm}

% \len{
%     \begin{itemize}
%         \item Think of better title.
%         \item Flesh out this section more.
%         \item Add more references. E.g., \cite{welleck-etal-2019-dialogue} have good definitions for Dialogue Generation and Persona-Based Dialogue.
%         \item Properly introduce the concept of a dialogue with references.
%         \item Properly introduce Dialogue Systems and their general scheme. E.g., \cite{Kushneryk2019IntelligentDS} provide good figures for these concepts.
%         \item Emphasize that the focus of this thesis is on Open-Domain dialogue systems. Talk about open-domain dialogue systems
%         \item \textit{Open-domain conversational systems are a special case of language models where the prefix is the dialogue history and the continuation is a humanlike response.} \citep{madotto-etal-2020-plug}
%         \item Very briefly mention that the focus of your research is on the natural language generation part of a dialogue system.
%     \end{itemize}
% }


% \len{TODO - Add a sub-section here about dialogue (also think of a better title for the sub-section). This is necessary because dialogue is the main focus of your thesis. So before you discuss dialogue response generation (which is the NLP-operationalization of dialogue), formally introduce the concept of dialogue. What is it? What does it look like? How is it different from discourse? Provide some examples like I did in the paper-submission.}

% \len{See Sandro's suggestions for this subsection.}


% Contents: 
% \begin{itemize}
%     \item introduce what is dialogue, 
%     \item what it looks like, 
%     \item how it differs from discourse, etc. 
%     \item Some examples would also be useful. (maybe from the BNC?)
% \end{itemize}

% \begin{itemize}
%     \item Describe dialogue in a sentence. There is no need to dive into an explicit definition of something so general.
%     \item Explain what a dialogue systems is, and which components there are (briefly). Chen et al has good brief definitions.
%     \item The focus of this thesis is on languag generation. (where does the discriminator module fit in tho?)
% \end{itemize}

Because the focus of this thesis is on controlled text generation in the context of dialogue, I provide a section explaining key concepts about dialogue in general and in NLP. This section covers the relevant definition of dialogue, what dialogue systems are and why they are developed and studied, and dialogue generation as a language modeling problem.

A dialogue is a written or spoken conversational exchange between two or more interlocutors (e.g., two people, or a person and a virtual assistant like Amazon Alexa) \citep{MerriamWebster2021Dialogue_Entry}, and it is the most natural form of interaction between people \citep{burtsev-etal-2018-deeppavlov}.
% Between people, dialogue is the most natural way of interaction \citep{burtsev-etal-2018-deeppavlov}.
Generally speaking, a dialogue's purpose is to exchange information or build relationships between interlocutors \citep{bohm2013dialogue}, and it typically consists of interlocutors exchanging utterances in turns. In the context of spoken language analysis, an utterance is considered the smallest unit of speech that is followed by a change of speaker or the end of a conversation, thus representing a dialogue turn \citep{traum1996utterance}.

In the age group detection experiment presented in Chapter \ref{ch:experiment1} (i.e., Experiment 1), I also experiment with a so-called discourse dataset (i.e., a dataset of blog posts), in addition to dialogue data. It is therefore helpful to establish a distinction between dialogue and discourse. Discourse has multiple definitions, but for the purpose of this thesis, I adopt the definition of discourse as a linguistic unit (e.g., a long talk or a piece of writing), typically longer than a sentence, whose purpose is to convey thoughts or ideas \citep{MerriamWebster2021Dialogue_Entry}. Dialogue therefore distinguishes itself from discourse in that it necessarily involves two or more participants exchanging information and contributing to the conversation, whereas discourse can be a one-way exchange of information, like a lecture or blog-post. See Figure \ref{fig:examples_dialogue_blog_post} for an example of a dialogue between people, and discourse in the form of a blog post.


\begin{figure}[H]
     \centering
     \begin{subfigure}[b]{0.45\textwidth}
        \centering
        \includegraphics[width=\textwidth]{figures/dialogue_example.png}
        \caption{\textit{Original image source:} Figure 1 of \cite{liu-etal-2020-impress}.}
        \label{}
     \end{subfigure}
     \hfill
     \begin{subfigure}[b]{0.45\textwidth}
        \centering
        \includegraphics[width=\textwidth]{figures/Example-blog-from-wordpresscom.png}
        \caption{}
        \label{}
     \end{subfigure}
        \caption{Examples of a dialogue ((a) / left), and discourse in the form of a blog post ((b) / right).}
        \label{fig:examples_dialogue_blog_post}
\end{figure}

% \len{\textbf{TODO} - Give an example of a dialogue snippet (from the BNC) and a blog post from the BAC side-by-side}

%  \len{Verify if this is true. Is that really the difference? Why does dialogue require at least two participants?}

It remains an open challenge for NLP researchers to endow a machine with the capability to engage in meaningful dialogue \citep{burtsev-etal-2018-deeppavlov}, and the development of so-called dialogue systems is an active field of study.
A dialogue system is a computer program that supports spoken, text-based, or multi-modal conversational interactions with humans \citep{mctear2020conversational}.
Dialogue systems typically fall into one of two categories: task-oriented and non-task-oriented dialogue systems. The former engages in an interaction with the user to complete some task, whereas the latter engages in general conversational interaction with the user \citep{Kushneryk2019IntelligentDS}. The focus of this thesis is on non-task-oriented dialogue.
The development of dialogue systems distinguishes itself from most NLP domains due to the inherent challenges associated with modeling human conversation: informal, noisy, unstructured, and even erroneous real-world responses, possibly competing goals of interlocutors, and extremely diverse sets of acceptable responses to prompts.
% Due to the unsolved and challenging nature of the problem, the development of so-called dialogue systems is an active field of study.

Nevertheless, \cite{mctear2020conversational} suggests that there are three main motivations for researchers to investigate and develop dialogue systems:

\begin{enumerate}
    \item Dialogue systems can provide users with a convenient and intuitive way to interact with technological services. Furthermore, they can help providers of these services by taking over mundane and straightforward tasks. However, it must be noted that the replacement of human workers by dialogue systems has ethical and societal consequences that should be taken into account by researchers \citep{ivanov2020impact}. 
    \item The development of dialogue systems requires researchers to model human conversational dynamics, and doing so can broaden our understanding of human behavior.
    \item Research about dialogue systems might one day result in human conversational behavior being simulated so accurately that users could be convinced they are interacting with another human (e.g., by passing the Turing test \citep{oppy2003turing}). However, it is important to note that convincing human users that they are conversing with another human, while it is, in fact, a dialogue system, is by no means a requirement for an effective dialogue system. And despite this being a core goal of AI \citep{zheng2019personalized}, the ethics of deceiving humans into believing they are talking to another human are highly debatable.
\end{enumerate}

Developing dialogue systems is a very complex task, and many approaches have been proposed for this \citep{mctear2020conversational}.
These approaches can differ greatly with respect to their methodologies, ranging from the use of recurrent neural networks \citep{li-etal-2016-deep}, to reinforcement learning \citep{mo2018personalizing}, variational auto-encoders \citep{ruan2019condition}, Transformers \citep{madotto-etal-2020-plug}, and multi-modal models \citep{shuster-etal-2021-multi}.
Despite the many types of modules that can comprise a dialogue system, generally, the basic components of a dialogue system are automatic speech recognition (ASR), language understanding (LU), dialogue management (DM), natural language generation (NLG), and text-to-speech synthesis (TTS) \citep{chen-etal-2017-deep}. Figure~\ref{fig:dialogue_system_overview} shows a schematic overview of how information is routed through such a pipeline. Note that the first and last components (i.e., ASR and TTS) are omitted if the dialogue system does not support auditory signals and responses. The focus of this thesis is on dialogue generation, a particular operationalization of dialogue in NLP, which falls under the ``\textit{Natural Language Generation (NLG)}'' category of Figure~\ref{fig:dialogue_system_overview}.

\begin{figure}[H]
    \centering
    % \includegraphics[width=\textwidth]{figures/dialogue_system_schema.png}
    \includegraphics[width=\textwidth]{figures/dialogue_system_overview_jansen.png}
    \caption{Schematic overview of a pipeline for a dialogue system.}
    % \len{Maybe re-do this image}}% \textit{Original image source:} Figure 1 of \cite{chen-etal-2017-deep}.}
    \label{fig:dialogue_system_overview}
\end{figure}



% The focus of my thesis is on dialogue generation, a particular operationalization of dialogue in NLP. 

% Dialogue response generation, sometimes referred to simply as dialogue generation, is the task of automatically generating a response given a prompt \citep{madotto-etal-2020-plug}. 
% Dialogue generation can be framed as a next utterance prediction problem \citep{welleck-etal-2019-dialogue}. In this framework, an utterance is a sequence of tokens representing a dialogue turn. The next utterance utterance, $\textbf{x}_{t + 1}$, is predicted conditioned on a dialogue prefix or prompt, $\textbf{x}_{\leq t}$ (e.g., a single previous dialogue turn, or the entire conversation history). And a succession of utterances can be seen as a dialogue between agents.
% Continuing with the framework of \cite{welleck-etal-2019-dialogue}, a sequence of utterances can be interpreted as a dialogue between agents.

% Computational dialogue modeling distinguishes itself from most NLP domains due to the challenges associated with modeling human conversation: informal, noisy, unstructured, and even erroneous real-world responses, possibly competing goals of interlocutors, or an inherently more diverse set of acceptable responses.

% \len{TODO - work out best way to merge previous and next paragraph(s)}

% Text generation is suitable to tackle tasks such as machine translation, abstractive summarization, and paraphrasing. Dialogue response generation is also a special case of language generation. It can be seen as language generation where the prompt is a turn in a dialogue session. Conversational response generation shares open-domain text generation's overarching objective of producing grammatically correct fluent text, while remaining relevant to the prompt. 
% However, computational dialogue modeling distinguishes itself from most NLP domains due to the challenges associated with modeling human conversation: informal, noisy, unstructured, and even erroneous real-world responses, possibly competing goals of interlocutors, or an inherently more diverse set of acceptable responses. \len{Comment by SF about the previous sentence: \textit{all these aspects could be introduced/described in this subsection about (human) dialogue}}

%%%%%% THE FOLLOWING PARAGRAPH HAS BEEN MOVED TO THE METHODOLOGY. FROM HERE ..... %%%%%%%%%
% \len{Consider moving the following paragraph to the methodology. FROM HERE...}

% Despite these differences, conversational response generation can be modeled in similar ways to open-domain text generation. \cite{zeng-etal-2020-meddialog} suggest to either formulate it in terms of source-target pairs, much like neural machine translation, or as a language modeling objective, where the next token or utterance is conditioned on the dialogue history. To remain close to the training objectives of my baseline models (GPT-2 \citep{radford2019language} and DialoGPT \citep{zhang2019dialogpt}) I choose to adopt the language modeling formulation for conversation modeling. I.e., concatenate all dialogue turns in a multi-turn dialogue session into a long text: $x_1, ..., x_N$. Denote the source sentence or dialogue history as $S = x_1, ..., x_m$ and the target sentence (ground truth response) as $T = x_{m + 1}, ..., x_N$. The conditional probability of dialogue continuation given its history $P(T | S)$ can be written as

% \begin{equation}
%     p(T | S) = \prod_{n = m + 1}^N p(x_n | x_1, ..., x_{n - 1}).
% \end{equation}

% A multi-turn dialogue session $T_1, ..., T_K$ can be written as $p(T_K, ..., T_2 | T_1)$ which is essentially the product of all source-target pairs probabilities $p(T_i | T_1, ..., T_{i - 1})$. This formulation also shows that optimising the single objective $p(T_K, ..., T_2 | T_1)$ is equivalent to optimising all source-target pair probabilities.


% % \begin{itemize}
% %     \item What is (computational) dialogue modeling?
% %     \item What do I aim to understand with CDM?
% %     \item How is (controllable) text generation related to CDM?
% %     \item Is dialogue generation ``simply'' a special case of text generation? If so, explain what are the conditions that constitute this special case.
% % \end{itemize}


% \len{...TO HERE.}
%%%%%%%%%%%%%%%% ...TO HERE

% \subsection{Dialogue Response Generation as a Language Modeling Task} 
% \label{subsec:dialogue_generation_as_lm}

% \len{Depending on how the preceding subsection unfolds, you could remove this subsection's title and merge the two subsections (maybe not just concatenation).}

\subsection{Dialogue Generation and Controlled Dialogue Generation}
\label{subsec:controlled_dialogue_generation}

Dialogue response generation, also referred to simply as dialogue generation, is the task of automatically generating a response given a prompt \citep{madotto-etal-2020-plug}. 
% Dialogue generation can be framed as a next utterance prediction problem \citep{welleck-etal-2019-dialogue}. In this framework, an utterance is a sequence of tokens representing a dialogue turn. The next utterance utterance, $\textbf{x}_{t + 1}$, is predicted conditioned on a dialogue prefix or prompt, $\textbf{x}_{\leq t}$ (e.g., a single previous dialogue turn, or the entire conversation history). And a succession of utterances can be seen as a dialogue between agents.
Dialogue generation can be formulated as a language modeling task \citep{welleck-etal-2019-dialogue}, where the next utterance is conditioned on the dialogue history.
In this framework, an utterance is a sequence of tokens representing a dialogue turn. The next utterance, $\textbf{x}_{t + 1}$, is predicted conditioned on a dialogue prefix or prompt, $\textbf{x}_{\leq t}$ (e.g., a single previous dialogue turn or the entire conversation history). 
% And a succession of utterances can be seen as a dialogue between agents.
% \len{TODO - Do I need the following sentence?}
% To remain close to the training objectives of my baseline models (GPT-2 \citep{radford2019language} and DialoGPT \citep{zhang2019dialogpt}) I choose to adopt the language modeling formulation for conversation modeling. 
When developing dialogue generation models, the language modeling training objective can be formulated as a product of source-target pair probabilities \citep{zhang2019dialogpt}. This consists of concatenating all dialogue turns in a multi-turn dialogue session into a long text: $\textbf{x}_1, ..., \textbf{x}_N$. The source sequence (or dialogue history) is then denoted as $\mathcal{D} = \textbf{x}_1, ..., \textbf{x}_m$ and the target sequence (or ground truth dialogue continuation) as $T = \textbf{x}_{m + 1}, ..., \textbf{x}_N$. The conditional probability of dialogue continuation given its history $P(T | \mathcal{D})$ can be then expressed as

\begin{equation}
    p(T | \mathcal{D}) = \prod_{n = m + 1}^N p(\textbf{x}_n | \textbf{x}_1, ..., \textbf{x}_{n - 1}).
\end{equation}

A multi-turn dialogue session $T_1, ..., T_K$ can be written as $p(T_K, ..., T_2 | T_1)$, which is essentially the product of all source-target pairs probabilities $p(T_i | T_1, ..., T_{i - 1})$. This formulation also shows that optimizing the single objective $p(T_K, ..., T_2 | T_1)$ is equivalent to optimizing all source-target pair probabilities. Sampling from such a dialogue generation model, i.e., generating a next utterance $\textbf{x}_{t + 1}$ based on a dialogue history at turn $t$, $\mathcal{D}_t$, entails sampling from the conditional distribution $p(\textbf{x}_{t + 1} | \mathcal{D}_t)$.

% \subsection{Controlled Dialogue Generation}

% \len{Sandro suggests to move the entire following subsection to the Related Work section. See comment. FROM HERE...}

Analogous to the definition of controlled text generation given earlier in Section \ref{subsec:text_gen_and_ctg}, \textit{controlled dialogue generation} refers to the process of enforcing a particular linguistic style on an automatically generated dialogue response, and it is the core task of this thesis. 
Controlled dialogue generation can be seen as sampling from the conditional distribution $p(\textbf{x}_{t + 1} | \mathcal{D}_t, a)$, where $\textbf{x}_{t + 1}$ is a sequence of tokens representing the next dialogue turn or utterance, $\mathcal{D}_t$ is the dialogue history at dialogue turn $t$, and $a$ denotes a specific linguistic style which utterance $\textbf{x}_{t + 1}$ is desired to possess. Similar to the distinction made between text generation and controlled text generation in Section \ref{subsec:text_gen_and_ctg}, controlled dialogue generation can also be seen as a more restrictive version of dialogue generation, as the probability space being conditioned on is further constrained to also contain the style attribute $a$. This style attribute typically represents concrete linguistic styles, like sentiment or topic \citep{madotto-etal-2020-plug}, and rarely a more abstract style, like the speaking style of a particular age group. However, recent efforts to develop personalized dialogue systems require researchers to develop controlled dialogue generation models that control for more abstract linguistic styles.



% \begin{itemize}
%     \item Controlled dialogue generation can be seen as sampling from the conditional distribution $p(\textbf{x}_{t + 1} | \mathcal{D}_t, a)$, where $\textbf{x}_{t + 1}$ is a sequence of tokens representing the next dialogue turn or utterance, $\mathcal{D}_t$ is the dialogue history at dialogue turn $t$, and $a$ denotes a specific linguistic style which utterance $\textbf{x}_{t + 1}$ is desired to possess.
%     \item How is controlled dialogue generation different from regular dialogue generation?
%     \item How is controlled dialogue generation different from controlled text generation.
%     \item Segue to dialogue personalization
% \end{itemize}

% \len{Revise this section}

\subsection{Personalized Dialogue Generation Models}
Personalized dialogue generation models are special cases of controlled dialogue generation models in that they attempt to constrain generated dialogue responses to possess certain linguistic characteristics that align with a user's persona (e.g., their age, gender, or geographical region).
% Personalized interaction is of crucial importance to obtain dialogue systems that can be trusted by users and perceived as natural \citep{van2019exploring}, but most of all to be accessible to varying user profiles, rather than targeted at one particular user group \citep{zheng2019personalized, zeng-etal-2020-meddialog}.
% Endowing a dialogue system with personality traits to generate human-like conversation is a long-standing goal in AI \citep{edlund2008towards, scheutz2011toward}.
Although this thesis solely focuses on a particular aspect of a user's profile, i.e., their age group, discussing general personalization in dialogue generation models is still useful to develop a broader understanding of controlled dialogue generation. Namely, the distinctions between types of personalization approaches for dialogue generation models and the associated challenges often apply to controlled dialogue generation models more broadly.

Assuming an encoder-decoder setup, personalized dialogue generation models can be classified as one of two types: implicit and explicit personalization models \cite{zheng2019personalized}. For implicit personalization models, each speaker has its own vector representation, which implicitly captures the speaking style of the speaker in the decoding process \citep{ijcai2017-521, li-etal-2016-persona}. These models enjoy the benefit of having a more granular and realistic representation of speaking style, as opposed to a simple discrete set of traits (as is the case for explicit personalization models). On the other hand, it is unclear how speaker style is captured and should be interpreted, as all the information about a speaker's style is encoded in a real-valued vector. Furthermore, these methods suffer from a data sparsity issue because each dialogue should be tagged with a speaker identifier, and there should be sufficient dialogues from each trait group to train a reliable trait-adaptive model. 
% \len{This should be in the related work. See comment.} 
% This last drawback is a bigger hurdle for the method proposed by \cite{zheng2019personalized} than it is for mine, as their work deals with personalization for intersections of multiple traits, whereas this thesis focuses on adaptation to a small number of age groups.

When generating responses, explicit personalization models are conditioned either on a given personal profile \citep{ijcai2018-595}, text-described persona \citep{zhang-etal-2018-personalizing}, or simply an attribute label \citep{madotto-etal-2020-plug}. That is, speaker traits are represented as key-value pairs or descriptions about age, gender, etc. This can be seen as conditioning the decoder's output on an attribute $a$ (as is the case in this thesis). Speakers with the same set of personality traits can share attribute representations, so it does not require a speaker-specific representation vector. Such structured character descriptions are more explicit, straightforward, and interpretable. However, explicit personalization models require manually labeled or crowdsourced datasets for development, making it difficult to scale these models to large-scale dialogue datasets \citep{zheng2019personalized, madotto-etal-2020-plug}.

% \cite{zheng2019personalized} argue that this objective is difficult to reach because of the challenge of representing personality traits via language expression and the lack of large-scale persona-labeled dialogue datasets.

% \len{...TO HERE}

% \begin{enumerate}
%     \item \textit{Implicit personalization models}
%         \begin{itemize}
%             \item Each speaker has its own vector representation, and this vector is fed into the decoder to capture the speaking style of the speaker implicitly.
%             \item \textbf{Pros}: more granular and realistic representation of speaking style, as opposed to a simple discrete set of traits.
%             \item \textbf{Cons}: (1) It is unclear how personality is captured and how it can be interpreted as all the information about a speaker's style is encoded in a real-valued vector. (2) These methods suffer from a data sparsity issue: each dialogue should be tagged with a speaker identifier and there should be sufficient dialogues from each speaker to train a reliable user-specific model.
%         \end{itemize}
%     \item \textit{Explicit personalisation models}
%         \begin{itemize}
%             \item The generated responses are conditioned either on a given personal profile or a text-described persona. I.e., personality is represented via key-value pairs or natural language descriptions about age, gender, etc. So this can be seen as conditioning the decoder's output on an attribute $a$, which represents one of the descriptions in the previous two sentences.
%             \item \textbf{Pros:} (1) The persona of a speaker can be viewed as a composite of diversified personality traits, suggesting that this approach is a sensible approximation of reality. (2) Data sparsity problem is solved. Speakers with same set of personality traits can share attribute representations. So it does not require a speaker-specific representation vector. (3) Such structured personality descriptions are more explicit, straight-forward, and interpretable.
%             \item \textbf{Cons:} (1) These methods are limited to either manually-labelled data or crowdsourced dialogues, and thereby not scalable to large-scale dialogue datasets.
%         \end{itemize}
% \end{enumerate}


\subsection{Transformers}

The Transformer architecture plays a central role in most of the recent advances in NLP. The same holds for the methods used in this thesis to investigate controlled dialogue generation and automated detection of age-related linguistic patterns in dialogue utterances. For a more detailed review of the model architecture, the reader is referred to the original paper \citep{vaswani2017attention}, the annotated and replicated version of the original paper \citep{rush-2018-annotated}, or this blog post\footnote{\url{https://jalammar.github.io/illustrated-transformer/}}.

The Transformer, like most neural sequence processing models, has an encoder-decoder structure. On a high level, the encoder receives an input sequence $\textbf{x} = (x_1, ..., x_n)$ (e.g., a sentence), and maps this to a sequence of latent continuous variables $\textbf{z} = (z_1, ..., z_n)$. The decoder then takes $\textbf{z}$ as input and maps this to an output sequence $\textbf{y} = (y_1, ..., y_m)$. Note that the use of positional encodings of the input and output embeddings enables the Transformer to process and generate sequences in arbitrary order, allowing for a high degree of parallelization. The generation of $\textbf{y}$ happens element-by-element in an auto-regressive fashion, where at step $t$, element $y_{t - 1}$ is also taken as input.

Both the encoder and decoder are comprised of $N$ identical layers (denoted by the `N $\times$' in the left part of Figure \ref{fig:transformer_architecture}). Every sub-layer performs a succession of transformations using multi-head self-attention mechanisms and point-wise, fully connected layers, along with residual connections \citep{he2016residual} around every sub-layer followed by layer normalization \citep{DBLP:journals/corr/BaKH16}. The decoder's first self-attention sub-layer is masked to ensure that the output predictions at sequence position $i$ cannot depend on output positions greater than $i$. Finally, the decoder passes its output through a linear and softmax layer to produce a probability distribution over the problem space (e.g., the vocabulary) from which the most likely symbols for the generated output sequence $\textbf{y}$ can be sampled.

A key aspect of the Transformer architecture is its use of attention \citep{DBLP:journals/corr/BahdanauCB14}. This allows the encoder-decoder architecture to selectively focus on parts of the input sequence to produce a more informative hidden representation. \cite{vaswani2017attention} formulate an attention function as a mapping of queries and sets of key-value pairs to an attention output, where matrices represent the queries $Q$, keys $K$, and values $V$. The attention output is a weighted sum of the values, based on the relevance of the corresponding keys to a query. In particular, they employ scaled dot-product attention:

\begin{equation}
    \texttt{Attention}(Q, K, V) = \texttt{softmax} \left( \frac{QK^T}{\sqrt{d_k}}\right) V.
\end{equation}

Furthermore, \cite{vaswani2017attention} propose to use multi-head attention by using learned linear projections to project the queries, keys and values $h$ times, and apply the aforementioned attention function to these projections in parallel. The concatenation of these attention outputs passed through a linear layer ultimately produces the final output of the Transformer's attention sub-layers. This allows the model to attend to the relevant information from all representation sub-spaces at various sequence positions. See Figure \ref{fig:transformer_architecture} for a schematic illustration of the Transformer's structure described above.


\begin{figure}[H]
    \centering
    \includegraphics[width=\textwidth]{figures/transformer_lillog.png}
    \caption{An overview of the full Transformer model architecture. \textit{Collated image source:} Fig. 17 in this blog post \url{https://lilianweng.github.io/lil-log/2018/06/24/attention-attention.html}. \textit{Original image source:} Figures 1 and 2 in \cite{vaswani2017attention}}
    \label{fig:transformer_architecture}
\end{figure}

\subsection{Causal Language Modeling with Transformers for Dialogue Generation}
\label{subsec:causal_lm_with_transformers_dialogue_gen}
% As described in Section \ref{subsec:background_dialogue}, here I focus on dialogue generation. 
% A formulation of dialogue generation as a language modeling task is described in Section \ref{subsec:dialogue_generation_as_lm}. 
As previously described in Section \ref{subsec:background_dialogue}, 
% Following the conventions of \cite{dathathri2019plug} and \cite{madotto-etal-2020-plug}, 
a dialogue is comprised of multiple alternating turns, i.e., utterances, between more than one speaker. Furthermore, as described in Section \ref{subsec:controlled_dialogue_generation}, the concatenation of the sequences representing utterances is defined as the dialogue history. Because the focus of this thesis is on dialogues between two speakers, the dialogue history at turn $t$ is defined as $\mathcal{D}_t = \{S^{(1)}_1, S^{(2)}_1, ..., S^{(1)}_t\}$, where $S^{(j)}_t$ is speaker $j$'s utterance at time $t$. 
% \cite{madotto-etal-2020-plug} denote speaker $1$ as the user $U$, and speaker $2$ as the conversational system $S$, yielding dialogue history $\mathcal{D}_t = \{U_1, S_1, ..., U_t\}$. 
% This notational convention will also be used for the user-system experiments later on in this thesis.

A Transformer-based language model (denoted $\texttt{LM}$) is used in this thesis to model the distribution of dialogues, using dialogue history at time $t$, $\mathcal{D}_t$, as a prompt to auto-regressively generate the dialogue continuation $S_t$. More specifically, let the concatenation of the dialogue history at $t$ and its continuation, $\{\mathcal{D}_t, S_t\}$, be represented as a sequence of tokens $\textbf{x}= \{x_0, ..., x_n\}$. Then, by recursively applying the product rule of probability (\cite{bishop2006pattern}), the unconditional probability of the sequence $p(\textbf{x})$ can be expressed as:

\begin{equation}
    p(\textbf{x}) = \prod_{i = 1}^n p(x_i | x_0, ..., x_{i - 1}).
\end{equation}

\cite{dathathri2019plug} and \cite{madotto-etal-2020-plug} define the Transformer's decoding process in a recursive fashion. Let $H_t$ denote the conversation history's key-value pairs, i.e., $H_t = \left[ (K_t^{(1)}, V_t^{(1)}), ..., (K_t^{(l)}, V_t^{(l)}) \right]$, with $(K_t^{(i)}, V_t^{(i)})$ representing the key-value pairs from the $\texttt{LM}$'s $i$-th layer generated at all time steps $0$ through $t$. This results in the recurrent decoding process being expressed as:

\begin{equation}
    o_{t + 1}, H_{t + 1} = \texttt{LM} \left( x_t, H_t \right),
\label{eq:transformer_lm_io}
\end{equation}

where $o_{t + 1}$ is the hidden state of the last layer. Finally, after applying a softmax transformation, the next token $x_{t + 1}$ is sampled from the resulting probability distribution, i.e.,  $x_{t + 1} \sim p_{t + 1} = \texttt{softmax} \left( W o_{t + 1} \right)$, where $W$ is a linear mapping from the model's last hidden state to a vector of vocabulary size. This recursive formulation allows for efficient text generation by leveraging cached memories without repeated forward passes.

\subsection{Plug-and-Play Language Modeling}
\label{sec:ppm}

% \len{This section needs 
% \begin{itemize}
%     \item a brief intro paragraph (a couple of sentences) introducing the pplm-method and its central importance to this thesis
%     \item paragraph headers
% \end{itemize}}

The Plug-and-Play approach to controlled text generation, proposed by \cite{dathathri2019plug}, is a core method of the research presented in this thesis. It provides a framework to control the writing style of large pre-trained Transformer-based language models (like GPT-2 \citep{radford2019language}), without incurring relatively high computational costs. What follows is a detailed explanation of the PPLM-method. For additional information, the reader is referred to the original paper \citep{dathathri2019plug}.

\paragraph{Relationship to Bayes' Theorem} The Plug-and-Play Language Model (PPLM) works by using a text classifier, referred to as an attribute model, to control the text generated by a language model. Let $p(X)$ denote the output distribution of a Transformer-based language model (e.g., GPT-2 or DialoGPT), where $X$ represents the generated text. And $p(a | X)$ denotes the attribute model (e.g., a single-layer or BoW classifier) that represents the degree of adherence of text $X$ to a certain attribute $a$ (e.g., topic or sentiment (previous work), or age-group characteristics (this work)). Then PPLM can be seen as modeling the conditional distribution of generated text $X$ given attribute $a$, i.e., $p(X | a)$. Note that Bayes' theorem ties these three definitions together as follows:

\begin{equation}
    p(X | a) \overbrace{=}^{\text{Bayes' theorem}} 
    % \frac{p(a, X)}{p(a)} = 
    \frac{p(X) p(a | X)}{p(a)} \propto
    p(X)p(a | X).
\end{equation}

% \len{TODOs -
% \begin{itemize}
%     \item Maybe this isn't the best spot, but somewhere in the methodology motivate the choice of using a unigram wordlist as BoW, and a linear classifier trained with a transformer-based architecture.
%     \item Also emphasize the continuing narrative between Experiment 1 and Experiment 2. Namely, how the best performing systems in Exp1 are used to inform choices during Exp2 (trigram and BERT best classifiers --> use unigram and linear layer used for PPLM).
%     \item Also explain how the PPLM setup is not compatible with lists of $n$-grams (for $n>1$), as it makes perturbations at the unigram level. And adapting the system to fit trigram BoW's implies retraining the entire underlying language model (like GPT-2) for trigrams, defeating the purpose of PPLM (leveraging large scale language models for controllability, without having to finetune the massive architectures).
%     \item Then go on to argue that the unigram classifier is on par with trigram, making it a viable choice for the 100MIU BoW.
%     \item Also mention that your work extends the original BoW-PPLM by using various empirically generated BoWs, instead of curated wordlists. Making it more reproducible.
%     \item Double-check if you should make this last argument in the methodology, and not in related work or experimental details.
% \end{itemize}}

\paragraph{PPLM Gradient Update Rule} To control the generated text, PPLM shifts the previously mentioned history $H_t$ (see Section \ref{subsec:causal_lm_with_transformers_dialogue_gen} and Equation \ref{eq:transformer_lm_io}), i.e., all Transformer key-value pairs generated up to time $t$, in the direction of the sum of two gradients:

\begin{enumerate}
    \item Ascending $\nabla \log p(a | X)$: maximizing the log-likelihood of the desired attribute $a$ under the conditional attribute model. This enforces attribute control.
    \item Ascending $\nabla \log p(X)$: maximizing the log-likelihood of the generated language under the original (possibly conversational) language model. This promotes fluency of the generated text.
\end{enumerate}

These two incentive-representing gradients are combined with various coefficients, yielding a set of tunable parameters to steer the generated text in the direction of the desired fluency, attribute control, and length.

Let's first focus on the first of the two gradients, i.e., the attribute control promoting $\nabla \log p(a | X)$. $\Delta H_t$ represents the update to history $H_t$ that pushes the distribution of the generated text $X$ in the direction that has a higher likelihood of adhering to desired attribute $a$. The gradient update rule can be expressed as:

\begin{equation}
    \Delta H_t \leftarrow \Delta H_t + \alpha
    \frac{\nabla_{\Delta H_t} \log p(a | H_t + \Delta H_t)}
    {\norm{\nabla_{\Delta H_t} \log p(a | H_t + \Delta H_t)}^{\gamma}}
\label{eq:H_update_rule}
\end{equation}

where $\alpha$ is the step size, and $\gamma$ denotes the normalization term's scaling coefficient. Both step size ($\alpha$) and the scaling coefficient ($\gamma$) influence attribute control. Attribute control can be softened by either decreasing $\alpha$ or increasing $\gamma$ and vice versa. Note that $\alpha = 0$ recovers the original uncontrolled underlying language model (e.g., GPT-2 or DialoGPT). In practice, $\Delta H_t$ is initialized at zero, and the update rule in Equation \ref{eq:H_update_rule} is applied $m$ times (usually 3 to 10), resulting in the updated key-value pair history $\tilde{H}_t  = H_t + \Delta H_t$. Then the updated history $\tilde{H}_t$ is passed through the language model, yielding the updated logits (final Transformer-layer): $\tilde{o}_{t + 1}, H_t = \texttt{LM}(x_t, \tilde{H}_t)$. And finally, the shifted $\tilde{o}_{t + 1}$ is linearly mapped through a softmax layer to produce a new, more attribute-adherent, distribution from which to sample, i.e., $x_{t + 1} \sim \tilde{p}_{t + 1} = \texttt{softmax} \left( W \tilde{o}_{t + 1} \right)$.

\paragraph{Maintaining Fluency of Generated Text} The method described until now will generate attribute-adherent text, but will likely yield fooling examples \citep{nguyen2015deep} that are gibberish to humans, but get assigned high $p(a | x)$ by the attribute model \citep{dathathri2019plug}. That is why \cite{dathathri2019plug} apply two methods to ensure fluency of the generated text. The first is to update $\Delta H_t$ to minimize the Kullback-Leibler (KL) divergence \citep{kullback1951information} (denoted $D_{KL})$ between the shifted and original distributions. In practice, $D_{KL}$ is scaled by a coefficient $\lambda_{KL}$, typically found to work well for most tasks when set to 0.01. Repetitive text generation (i.e., high $p(a | x)$ but low $p(x)$) can therefore sometimes be avoided by increasing $\lambda_{KL}$. The second method to ensure fluency is Post-norm Geometric Mean Fusion \citep{stahlberg-etal-2018-simple} which, instead of directly influencing $\Delta H_t$ like minimizing $D_{KL}$, fuses the altered generative distribution $\tilde{p}_{t + 1}$ with the unconditional language distribution $p(x)$. This is done during generation by sampling the next token as follows:

\begin{equation}
    x_{t + 1} \sim \frac{1}{\beta}
    \left( 
    \tilde{p}_{t + 1}^{\gamma_{gm}} p_{t + 1}^{1 - \gamma_{gm}}
    \right)
    \label{eq:gm_fusion}
\end{equation}

where $\beta$ is a normalization constant, $p_{t + 1}$ and $\tilde{p}_{t + 1}$ denote the original and modified distributions, respectively, and $\gamma_{gn}$ is a scaling term that interpolates between the two distributions. Because the new sampling distribution in Equation \ref{eq:gm_fusion} converges towards the unconditional language model as $\gamma_{gm} \rightarrow 0$, repetitive text generation can be avoided by decreasing the scaling term.


\begin{figure}[H]
    \centering
    \includegraphics[width=\textwidth]{figures/pplm_fig1.png}
    \caption{A schematic overview of the plug-and-play interaction between attribute model $p(a | \textbf{x})$ and language model $p(\textbf{x})$. \textit{Original image source:} Figure 1 of \cite{dathathri2019plug}}
    \label{fig:pplm_schematic_overview}
\end{figure}


\paragraph{Attribute Models} As mentioned at the beginning of this section (i.e., Section \ref{sec:ppm}) and shown in Equation \ref{eq:H_update_rule}, the PPLM-method uses gradient updates that maximize the attribute model's log-likelihood $\log p(a | X)$ to enforce an attribute-specific style on a language model's generated output.
The simplest attribute model is a bag-of-words (BoW), i.e., a list of words that represents the stylistic attribute desired to be enforced on the generated text.
The log-likelihood of a BoW attribute model is the log of the sum of likelihoods of every constituent word of the BoW.
Given a bag-of-words $\{w_1, ..., w_k\}$ that represents an attribute $a$, and the output distribution of the language model $p_{t + 1}$, the attribute model's log-likelihood is
\begin{equation}
    \log p(a | x) = \log \left( \sum_{i}^{k} p_{t + 1} [w_i] \right)
    \label{eq:bow_att_model_loglikelihood}
\end{equation}
Ascending $\nabla \log p(a | x)$ therefore increases the likelihood of words in the BoW, $\{w_1, ..., w_k\}$, being generated. Furthermore, this implies that BoW-based control is exerted most noticeably at the lexical level.

BoW attribute models are known to be capable of capturing accurate representations of attributes like sentiment and topic \citep{li-etal-2018-delete}.
However, it can be desirable to use a more complex attribute model, i.e., a neural discriminator, when the attribute is more abstract and not easily represented as a BoW (e.g., age-specific language) \citep{dathathri2019plug}.
In the PPLM-method, a discriminator attribute model $f := p(a | x)$ is developed by training a singly layered linear discriminator on (presumably attribute-adherent) input sentences $x$ and their corresponding labels $y_x$. More specifically, for an input sentence $x$ of length (in tokens) $t$, the set of hidden states $o_{:t}^{x}$ is collected from the Transformer-based $\texttt{LM}$ (see Equation \ref{eq:transformer_lm_io}).
The set of hidden states is then averaged over all time steps, yielding $\bar{o}_t^x$.
The discriminator $f$ is then trained using the cross-entropy between the discriminator's output $f(\bar{o}_t^x)$ and the label distribution $y$. Because the discriminator attribute model is trained on $\bar{o}_t^x$, it is possible to capture and propagate a more comprehensive representation of $a$, shifting $\texttt{LM}$'s output distribution towards one that favors sentences that align with $a$.

\paragraph{Computational Costs of PPLM} Although the discriminator attribute model requires a separate training schedule, the computational cost of training and using such an attribute model in a PPLM-setup is negligible compared to fine-tuning a typical choice of $\texttt{LM}$ (e.g., GPT-2-medium \citep{radford2019language}) for controlled generation. Namely, the number of trainable parameters comprising the discriminator attribute model is equal to $(L_o \cdot N_a) + N_a$, where $L_o$ is the length of the hidden state vector, and $N_a$ is the number of classes defined in attribute $a$ (e.g., $N_a = 2$ if $a$ represents age, defined by two age groups, younger and older). Note that this is a substantially smaller number of trainable parameters than $\texttt{LM}$ typically has.
For instance, training a PPLM-setup consisting of GPT-2-medium as its underlying language model, a discriminator attribute model, and an attribute $a$ with two classes amounts to training $(1024 \cdot 2) + 2 = 2050$ parameters, whereas fine-tuning the same language model to generate attribute-adherent text for \textit{both} attribute classes entails training $2 \cdot 345\text{M}$ parameters (i.e., one fully fine-tuned model per attribute class).
% The discriminator attribute model's training objective can therefore be expressed as:

% \begin{equation}
%     \log p(a | x) = \log f()
% \end{equation}



% \len{Mention attribute models and their formulas so you can base your hypotheses on them.}
% \textbf{Stuff about BoW attribute models}
% \begin{itemize}
%     % \item As mentioned at the beginning of this section (i.e., Section \ref{sec:ppm}) and shown in Equation \ref{eq:H_update_rule}, the PPLM-method uses gradient updated that maximize the attribute model's log-likelihood $\log p(a | X)$ to enforce an attribute-specific style on a language model's generated output.
%     % \item The simplest attribute model is a bag-of-words (BoW), i.e., a list of words that represents the stylistic attribute desired to be enforced on the generated text.
%     % \item The log-likelihood of a BoW attribute model is the log of the sum of likelihoods of every constituent word of the BoW.
%     % \item Given a bag-of-words $\{w_1, ..., w_k\}$ that represents an attribute $a$, and the output distribution of the language model $p_{t + 1}$, the attribute model's log-likelihood is
%     % \begin{equation}
%     %     \log p(a | x) = \log \left( \sum_{i}^{k} p_{t + 1} [w_i] \right)
%     %     \label{eq:bow_att_model_loglikelihood}
%     % \end{equation}
%     \item Ascending $\nabla \log p(a | x)$ therefore increases the likelihood of words in the BoW, $\{w_1, ..., w_k\}$, being generated.
%     \item Emphasize that BoW-based control is exerted most noticeably at the lexical level.
% \end{itemize}

% \textbf{Stuff about discriminator attribute models}

% \begin{itemize}
%     % \item BoW-based attribute models are known to be capable of capturing accurate representations of attributes like sentiment and topic \citep{li-etal-2018-delete}.
%     % \item However, it can be desirable to use a more complex attribute model, i.e., a neural discriminator, when the attribute is more abstract and not easily easily represented as a BoW (e.g., age-specific language).
%     % \item In the PPLM-method, a discriminator attribute model $f$ is developed by training a singly layered neural discriminator on (presumably attribute-adherent) input sentences $x$ and their corresponding labels $y_x$.
%     % \item More specifically, for an input sentence $x$ of length (in tokens) $t$, 
%     % the set of hidden states $o_{:t}^{x}$ is collected from the Transformer-based $\texttt{LM}$ (see Equation \ref{eq:transformer_lm_io}).
%     % the Transformer-based $\texttt{LM}$ outputs the corresponding hidden state of its last layer, $o_{:t}^{x}$ (see Equation \ref{eq:transformer_lm_io}). 
%     % \item The set of hidden states is then averaged over all time steps, yielding $\bar{o}_t^x$.
%     % \item The discriminator $f$ is then trained on the mean of the hidden states over all time steps, $\bar{o}_t^x$.
%     \item (tl;dr version) The discriminator $f$ is then trained using the cross-entropy between the discriminator's output $f(\bar{o}_t^x)$ and the label distribution $y$.
%     \item The number of trainable parameters comprising the discriminator attribute model is equal to $(L_o \cdot N_a) + N_a$, where $L_o$ is the length of the hidden state vector, and $N_a$ is the number of classes defined in attribute $a$ (e.g., $N_a = 2$ if $a$ represents age, defined by two age groups, younger and older). Note that this is a substantially smaller number of trainable parameters than $\texttt{LM}$ typically has. 
%     \item For instance, training a PPLM-setup consisting of GPT-2-medium \citep{radford2019language} as its underlying language model, a discriminator attribute model, and an attribute $a$ with two classes amounts to training $(1024 \cdot 2) + 2 = 2050$ parameters, whereas fine-tuning the same language model to generate attribute-adherent text for \textit{both} attribute classes entails training $2 \cdot 345\text{M}$ parameters (i.e., one fully fine-tuned model per attribute class).
%     \item The discriminator is optimized in a greedy manner \textbf{REFERENCE}, by only considering the next token being generated when optimizing the odds of the sentence possessing features learned to be attribute-adherent by $f$.
%     \item The optimization objective is expressed as:
%     \begin{equation}
%         \log p(a | x ) = \log f(o_{:t + 1}, o_{t + 2})
%         \label{eq:disc_att_model_loglikelihood}
%     \end{equation}
%     \item
% \end{itemize}



It is worth mentioning that the Plug-and-Play method applied by \cite{dathathri2019plug} and \cite{madotto-etal-2020-plug} is different from fine-tuning. Note that in Equation \ref{eq:H_update_rule} the gradient updates are restricted to the history $H_t$, and do not affect the model's parameters. Because the key-value pairs $(K_t^{(i)}, V_t^{(i)})$ that comprise $H_t$ are activations and not model-weights, the updates are only made to the activation-space. This means that PPLM leaves the underlying (conversational) language model untouched.

Contrary to fine-tuning often massive LMs, PPLM does not incur a significant training cost (depending, of course, on the complexity of the discriminator or attribute model). However, \cite{madotto-etal-2020-plug} show that PPLM needs a fixed number of $m$ update-steps for every generated token. This makes the original PPLM setup unsuitable for online interactive applications, like conversational systems. Addressing this problem, they introduce plug-and-play conversational model (PPCM), which extends PPLM by using the original model setup to generate dialogue datasets with the desired attribute $a$, and then use optimized residual adapters \citep{bapna-firat-2019-simple} to control $\texttt{LM}$'s output distribution. More specifically, \cite{madotto-etal-2020-plug} use PPLM to generate $n$ attribute-adherent dialogue datasets $\mathscr{D}^a = \{\mathcal{D}^1, ..., \mathcal{D}^n\}$, for attribute $a$. These generated dialogue datasets are then used to train the residual adapter modules that control the language model's output distribution.

% \subsection{Language and Age}
% The relationship between a person's age and use of language is a thoroughly studied subject with a decades long history and inherent challenges \citep{pennebaker2003words, nguyen2014gender, zheng2019personalized}. A number factors like community membership (e.g., gender, socioeconomic status, or political affiliation), experimental condition (e.g., emotional versus non-emotional disclosure), mode of disclosure (writing versus talking), and other confounding variables complicate the study of age's relation to language \citep{nguyen-etal-2011-author}. The relatively recent advent of widely available computational resources and vast amounts of textual data made it possible to leverage machine learning methods to help detect patterns in language that eluded conventional sociolinguistic research. Early computational investigations into the connection between a person's age and use of language is typically a combination of qualitative and statistical methods. For instance, using a mix between their proprietary count-based text analysis framework, Linguistic Inquiry and Word Count (LIWC) and sociolinguistic theory, \cite{pennebaker2003words} study the changes in written and spoken language use with increasing age. They discuss four important areas of a person's character that have been found to change with age: emotional experience and expression, identity and social relationships, time orientation, and cognitive abilities. These four axes and their hypothesized relationships with language use and age can be interpreted in the following ways:
% \begin{enumerate}
%     \item \textit{Emotional experience and expression:} This is the relationship between increasing age and linguistically observable manifestations of a person's experienced emotions. In practical terms, this is framed as detectable instances of positive and negative affect in language. This complex relationship between age and emotional expression is characterized by decreased levels of negative affect and slightly non-decreasing levels of positive affect. This is also confirmed by the findings of \cite{schler2006effects}.
    
%     \item \textit{Sense of identity and social relationships:} These terms refer to developmental tends in one's relation to self and others, as expressed in their language, e.g., as references to self (\textit{I, me, my}, and \textit{I, us, my}) or others (\textit{they, them, theirs}). \cite{pennebaker2003words} report that the \textit{quantity} of social connections decreases and the \textit{quality} of remaining relationships increases with age.
    
%     \item \textit{Time orientation:} This relationship describes how people express their perception of and orientation towards time. For instance, this can be indicated by the use of time-related verb tenses. The authors suggest that older individuals tend to be more past-oriented than their younger future-oriented counterparts.
    
%     \item \textit{Cognitive abilities:} This refers to markers of cognitive capacity in language. Aging is expected to be associated with less use of cognitively complex words after a certain mid-adulthood peak. Specifically, the relationship between markers of cognitive complexity in natural language (cognitive mechanisms, causal insight, and exclusive words) and age is hypothesized to be curvilinear. And because verbal ability does not decline until very late in life, markers of verbal ability (e.g., use of big words) are not expected to show changes with age.
% \end{enumerate}

% \cite{pennebaker2003words} consider the following variables: positive and negative emotions, first-person singular and first-person plural pronouns, social references, time-related words (past-tense, present-tense, and future-tense verbs), big words ($>6$ letters), cognitive mechanisms, causal insight, and exclusive words. Their main findings suggest that increasing with age, people use more positive and fewer negative affect words, use fewer self-references, use more future-tense and fewer past-tense verbs, and exhibit a general pattern of increasing cognitive complexity.

% Detectable linguistic differences between age-groups can often be attributed to the use of language fads or references to age-specific popular culture. For instance, \cite{schler2006effects} find that the use of slang and neologisms (such as \textit{lol} and \textit{ur}) are strong indicators of youth. Similarly, words like `facebook', `instagram', and `netflix' appear in the most frequently used words by younger participants of conversational data collection efforts, like that of the British National Corpus' spoken component \citep{love-spoken-bnc-2014}.

% % More recent studies, like that of \cite{nguyen-etal-2011-author}, \cite{zheng2019personalized}, and \cite{abdallah2020age}, frame age prediction from text as traditional machine learning problems, like linear regression, support vector machines, or neural architectures. These modeling approaches tend to reveal that strong indicators of age lie at the syntactic or structural level of language use, as opposed to the more content-based lexical level. Furthermore, this could explain why automatic detection from text of more content-based traits, like topic or sentiment, tend to be easier problems to solve than age prediction from text. To emphasize one such complicating factor, \cite{nguyen2014gender} argue that differences in language use are often relation to the speaker's social identity, which could differ from their biological identity. 
% % This idea that age prediction from text is more challenging than topic or sentiment prediction could be an indication that controlled language generation for age-differences is also a more nuanced problem than topical steered text generation.


% % \textbf{Keep in mind:}
% % \begin{itemize}
% %     \item What is known about the relationship between a speaker's (or author's) age and their linguistic characteristics? I.e., how does language use develop with age according to the existing literature?
% %     \item How can I automatically detect a speaker's age(-group) from their utterances using machine learning?
% %     \item Link findings from the papers you cite to the hypothesis that linguistic differences in age lie more are the phrase/constructional level than at the single word/lexical level. However, is this really the case? \cite{nguyen-etal-2011-author} mention single words like `well', `like`, and `just' being predictive of age.
% %     \item Mention that age differences in language could/should also be interpreted as indications of language fads during a person's formative years.
% %     \item Hypothesize that age-related linguistic variation is a more subtle and nuanced factor to control for during text generation than, e.g., topic (science, business, religion), or sentiment.
% %     \item Mention confounding factors like community membership (like gender, socioeconomic status, or political affiliation), experimental condition (e.g., emotional versus non-emotional disclosure), mode of disclosure (writing versus talking), and other confounding factors that complicate age prediction from language.
% % \end{itemize}

\section{Related Work}
\label{sec:related_work}

% \len{Keep in mind the following distinction between Background and Related Work - The Background section should give
% an overview of the problem and the components involved: dialogue,
% language generation, dialogue generation, age modeling, etc.,
% without focusing on one or the other approach — in Related Work, you
% describe approaches that have been proposed to tackle each of these
% components, separately or jointly, and which are related or relevant to
% your own work for some reason}

\subsection{Automated Age Detection from Language}

% \len{Consider adding a small sub-section about automated age detection from text, because you often bring up the problem and other researchers' approaches to solving it in the Background section.}

% \dots

% \len{work this into this subsection. Taken from workshop paper submission}
Automated (speaker or author) age detection from (spoken or written) language is a challenging task, for which many approaches have been suggested \citep{nguyen-etal-2011-author}. The work of \citet{schler2006effects} focuses on age detection in written discourse using a corpus of blog posts. The authors learn a Multi-Class Real Winnow classifier leveraging a set of pre-determined style- and content-based features, including part-of-speech categories, function words, and the 1000 unigrams with the highest information gain in the training set. They find that content features (lexical unigrams) yield higher accuracy ($\sim$ 74\%) than style features ($\sim$72\%), while their best results are obtained with their combination ($\sim$76\%).
Previous work on age detection in dialogue has focused on speech features, which are known to systematically vary across age groups. For example,  \citet{wolters2009age} learn logistic regression age classifiers from a small dialogue dataset %of 448 dialogues 
using different acoustic cues supplemented with a small set of handcrafted lexical features, while \citet{li2013automatic} develop SVM classifiers using acoustic and prosodic features extracted from scripted utterances spoken by participants interacting with an artificial system. 
% In contrast to this line of work, I investigate whether different age groups can be detected from textual linguistic information rather than voice-related cues. I explore whether, and to what extent, various state-of-the-art NLP models are able to capture such differences in dialogue data as a preliminary step to age-group adaptation by conversational agents.
% The work of~\citet{schler2006effects} focuses on age detection in written discourse using a corpus of blog posts. The authors learn a Multi-Class Real Winnow classifier leveraging a set of pre-determined style- and content-based features, including part-of-speech categories, function words, and the 1000 unigrams with the highest information gain in the training set. They find that content features (lexical unigrams) yield higher accuracy ($\sim$ 74\%) than style features ($\sim$72\%), while their best results are obtained with their combination ($\sim$76\%). 
% I extend this investigation in several key ways: (1) I leverage state-of-the-art NLP models that allow us to learn representations end-to-end, without the need to specify concrete features in advance; (2) I apply this approach to dialogue data, using a large-scale dataset of transcribed, spontaneous open-domain dialogues; (3) I show that text-based models can indeed detect age-related differences in both discourse and dialogue, even in the case of very sparse signal at the level of dialogue utterances; and finally (4) I carry out an in-depth analysis of the models' predictions to gain insight on which elements of language use are most informative.

More recent studies, like that of \cite{nguyen-etal-2011-author}, \cite{nguyen2014gender}, \cite{zheng2019personalized}, and \cite{abdallah2020age}, frame age prediction from text as traditional machine learning problems, like linear regression, support vector machines, or neural architectures. These modeling approaches tend to reveal that strong indicators of age lie at the syntactic or structural level of language use, as opposed to the more content-based lexical level. Furthermore, this could explain why automatic detection from text of more content-based traits, like topic or sentiment, tends to be easier problems to solve than age prediction from text. To emphasize one such complicating factor, \cite{nguyen2014gender} argue that differences in language use are often related to the speaker's social identity, which could differ from their biological identity. 


\subsection{Controlled Text Generation}
Previous approaches to controlled text generation require fine-tuning large Transformer-based language models or training conditional generative LMs from scratch. Most notably, CTRL \citep{keskarCTRL2019}, which achieves controlled generation by training a generative Transformer for several control codes. CTG models that require fine-tuning for control, like CTRL, can produce high-quality fluent text because they are specifically trained to maximize the likelihood of generated sequences, given an attribute (denoted $p(\textbf{x} | a)$), but require training massive language models with computational costs.

Other recent examples of controlled text generation models that are not Transformer-based also exist. \cite{li-etal-2020-optimus} introduce OPTIMUS, a large pre-trained Variational Autoencoder (VAE) \citep{Kingma2014} that can be fine-tuned for specific natural language tasks, like guided sentence generation. They demonstrate OPTIMUS' ability to perform controlled text generation from latent style-embeddings, with fluency at par with GPT-2. They also show how OPTIMUS generalizes better for low-resource languages than BERT \citep{devlin-etal-2019-bert}. Nevertheless, much like the previously mentioned CTG models, OPTIMUS still incurs a high computational cost for fine-tuning per NLP task.

% The plug-and-play language model (PPLM) \citep{dathathri2019plug} is a recent approach to leverage powerful large-scale language models and make them controllable for a wide variety of linguistic attributes, while avoiding incurring significant costs of fine-tuning these massive language models. Nevertheless avoiding these costs is anything but trivial. The plug-and-play setup of the PPLM model forms one of the main theoretical foundations of this work. It is both logical and promising for every day engineers to be able to leverage the grammatical fluency of pre-trained language models for more specific downstream tasks, e.g., specifying linguistic characteristic to enforce on an automatically written text. Their setup consists of a symbiosis of GPT-2 as their powerful large-scale language model, and a significantly smaller and therefore easier to train and fine-tune attribute model. This attribute model is often a small classifier, and can range in complexity from a simple bag of words model with a logistic classifier to a more complicated transformer encoder head. The main benefit of this setup is the extensibility it brings with it. Namely, such large-scale language models are open-source and available online and can be tailored to their specific needs using a significantly easier to train attribute model of your own liking. \cite{dathathri2019plug} demonstrate the applicability of their model on a wide variety of tasks ranging from text style transfer to language detoxification (all of which can be seen as sub-problems of controllable text generation). Other real-world applications include: being able to automatically re-write or adjust a draft text for an editorial, automatic generation of brand specific vacancy ads, or personalized chatbot assistance or even personalized education. What this work also provides is a starting point for new applications, namely controllable dialogue generation.

% \len{TODO: Where does the following sentence fit best? ``The plug-and-play setup of PPLM forms one of the main theoretical foundations of this work.''}

The Plug-and-Play language model (PPLM) \citep{dathathri2019plug} is a recent solution to the problem of high re-training costs of controlled text generation. This approach, inspired by a similar technique for style-control of generated images \citep{nguyen2017plug}, leverages the fluency of large-scale language models when controlling them for a specific linguistic attribute, while avoiding incurring high costs of fine-tuning these massive language models. The main benefit of this setup is its low-cost extensibility. Namely, such large-scale language models are often open-source and available online and can now be tailored to users' specific needs using a significantly easier to train attribute model. 
The original architecture proposed by \citeauthor{dathathri2019plug} uses GPT-2 as a base language model, which provides grammatical fluency, combined with a significantly easier to train attribute model (i.e., a simple BoW or single-layer classifier). Using gradient updates to the activation space of the much smaller attribute model, they manage to generate language that combines (some of) the fluency of GPT-2 with the stylistic control of the attribute model, without the cost of re-training a specialized architecture. They demonstrate that PPLM achieves desirable fluency (i.e., perplexity measured with GPT(-1) \citep{radford2018improving}), as well as measurable attribute control. Their architecture's applicability is also demonstrated on tasks such as controlled story writing and language detoxification. They also show a clear trade-off between attribute control and grammatical correctness and diversity. 

% Recent examples of controllable language generation models that are not Transformer-based also exist. \cite{li-etal-2020-optimus} introduce OPTIMUS, a large pre-trained Variational Autoencoder (VAE) \citep{Kingma2014} that can be fine-tuned for specific natural language tasks, like guided sentence generation. They demonstrate OPTIMUS' ability to perform controlled text generation from latent style embeddings, with fluency at par with GPT-2. They also show how OPTIMUS generalizes better for low-resource languages than BERT \citep{devlin-etal-2019-bert}. Nevertheless, much like the previously mentioned non-plug-and-play CTG models, OPTIMUS still incurs a significant computational cost for fine-tuning per NLP task.

% Dialogue generation is not explored as an original application of PPLM, nor is their tested with more complex attribute models to hopefully allow for less deterioration of fluency as attribute control increases. 



\subsection{Text Style Transfer}
Text style transfer is the task of changing a text's stylistic properties, while retaining its style-independent properties, like content and fluency \citep{dai2019style}. Text style transfer is a closely related problem to controlled text generation. Its similarity lies in trying to modify the output distribution of a text generation model, such that stylistic characteristics of the produced text are controllable, keeping content and fluency preserved. It involves rewriting an input text with a specific style. More formally, given a text $\textbf{x}$, its corresponding style-representing vector $\textbf{s}^{(i)}$, the number of different styles $K$ over which there exists a distribution, and a desired style $\hat{\textbf{s}} \in \{\textbf{s}^{(i)}\}_{i = 1}^{K}$, the goal of text style transfer is to produce output text $\hat{\textbf{x}}$ with style $\hat{\textbf{s}}$, and the style-independent properties of $\textbf{x}$. 
% Text style transfer can also be seen as a special case of (abstractive) summarization, for which Transformers have also demonstrated applicability \citep{baan-etal-2019-abstractive}.

Previous approaches to text style transfer involve passing input text through an RNN-based encoder, yielding a style-dependent latent representation $\textbf{z}$ \citep{zhang2018styletranslation}. Typically, these approaches then attempt to ``disentangle'' $\textbf{z}$ into a style-independent content representation and a latent representation of the stylistic properties of the input text. The subsequent decoder then receives the content representation and a new latent style variable as input to ultimately produce a style-altered output text with unchanged content. This style-disentanglement approach has several drawbacks: \textbf{(1)} It is difficult to evaluate the quality of disentanglement of the latent space. \textbf{(2)} It is hard to capture rich semantic information in the latent representation due to the limited capacity of vector representations (especially for long texts). \textbf{(3)} To disentangle style and content in the latent representations, all previous approaches have to assume all input texts can be encoded by a fixed-size latent vector. \textbf{(4)} Since most previous approaches use RNN-based encoder-decoder frameworks, they have problems capturing long-range dependencies in the input sentences. Furthermore, disentanglement might be unnecessary, as \cite{lample2018multipleattribute} have shown that a proper decoder can perform controlled text generation from an entangled latent representation by ``overwriting'' the original style.

% \begin{enumerate}
%     \item It is difficult to evaluate the quality of disentanglement of the latent space.
%     \item Disentanglement is unnecessary, as contemporary work by \cite{lample2018multipleattribute} has shown a good decoder can perform controllable text generation from an entangled latent representation by ``overwriting'' the original style.
%     \item It is hard to capture rich semantic information in the latent representation due to limited capacity of vector representations (especially for long texts).
%     \item To disentangle style and content in the latent representations, all previous approaches have to assume all input texts can be encoded by a fixed-size latent vector.
%     \item Since most previous approaches use RNN-based encoder-decoder frameworks, they have problems capturing long-range dependencies in the input sentences.
% \end{enumerate}

To address these drawbacks, \cite{dai2019style} propose Style Transformer, a Transformer-based alternative encoder-decoder framework for text style transfer. The authors' approach does not require any manipulation (i.e., disentanglement) of the latent space, eliminates the need for a fixed-size vector representation of the input, and handles long-range dependencies better due to Transformers' attention mechanism. Aside from this being the first application of Transformers for text style transfer, \cite{dai2019style} contribute a novel training algorithm for such models, that boasts significant improvements of results on two text style transfer datasets.

\subsection{Dialogue Generation}
As explained earlier in Section \ref{subsec:background_dialogue}, dialogue generation is the task of automatically generating a response given a user's prompt. \cite{zhang2019dialogpt} introduce DialoGPT, a tunable large-scale language model for generation of conversational responses, trained on Reddit discussion chain data. DialoGPT therefore extends GPT-2 \citep{radford2019language} to address a more restrictive sub-category of text generation, i.e., conversational response generation. DialoGPT inherits from GPT-2 a 12-to-48 layer transformer with layer normalization, a custom initialization scheme that accounts for model depth, and byte pair encodings \citep{sennrich-etal-2016-neural} as a tokenizer. The generation task remains framed as language modeling, where a multi-turn dialogue session is modeled as a long text. 

To address the well-known problem of open-domain text generation models producing bland and uninformative samples, \cite{zhang2019dialogpt} implement a maximum mutual information (MMI) scoring function. MMI uses a pre-trained backward model to predict $p(\texttt{source} | \texttt{target})$: i.e., the source sentences (dialogue history) given the target (responses, dialogue continuation). First, top-K sampling is used to generate a set of hypotheses. Then the probability $p(\texttt{source} | \texttt{hypothesis})$ is used to re-rank all hypotheses. As frequent and repetitive hypotheses can be associated with many possible queries/sources (i.e., a hypothesis that frequently occurs is one that is apparently applicable to many queries), and maximizing backward model likelihood penalizes repetitive hypotheses, MMI yields a lower probability for highly frequent hypotheses, thereby reducing blandness and promoting diversity. 

DialoGPT is evaluated on the Dialog System Technology Challenge (DSTC) 7 track, an end-to-end conversational modeling task in which the goal is to generate conversation responses that go beyond chitchat by injecting information that is grounded in external knowledge. The model achieves state-of-the-art results on both the human and automatic evaluation results by achieving near-human-like responses that are diverse, relevant to the prompt, much like GPT-2 for open-domain text generation. They train three models of small (117M), medium (345M), and large (762M) parameter sizes. The medium-sized 345M model achieves the best automatic evaluation results across most metrics and is used as one of the baselines in later experiments in this thesis. Their Hugging Face PyTorch implementation can be tested here\footnote{\url{https://huggingface.co/microsoft/DialoGPT-medium}}.

% Dialogue generation is the essential precursor to this thesis' ultimate task of controlled dialogue generation.

% \len{The previous sentence feels out of the blue. Consider removing it or think of a way to create a natural flow towards it.}



\subsection{Controlled Dialogue Generation}
As mentioned previously in Section \ref{subsec:controlled_dialogue_generation}, controlled dialogue generation is the task of steering automatically generated conversational responses to possess desired attributes, like sentiment, topic, or more abstract writing style characteristics.
% Nowadays, dialogue systems like Siri, Alexa, or Google Assistant, play large roles making technology easier to use, it is of great commercial interest to be able to control (e.g., personalize) the style of their responses. Medical applications too have been found for controllable dialogue generation.
\cite{zeng-etal-2020-meddialog} explore the applications of fine-tuning large language models, like GPT, on (Mandarin and English) medical consultation data. The resulting dialogue systems succeed at generating clinically correct and human-like responses to patients' medical questions. Medical dialogue systems like these can help make healthcare services more accessible and aid medical doctors to improve patient care.

\cite{zheng2019personalized} investigate the problem of incorporating explicit personal characteristics in dialogue generation to deliver personalized conversation. They introduce a dataset \texttt{PersonalDialog}, which is a large-scale multi-turn dialogue dataset with personality trait labeling (i.e., \texttt{Age}, \texttt{Gender}, \texttt{Location}, \texttt{Interest Tags}, etc.) for a large number of speakers. \cite{zheng2019personalized} also propose persona-aware models that include a trait fusion module in the encoder-decoder framework to capture and address personality traits in dialogue generation. Persona-aware attention mechanisms and bias are used to incorporate personality information in the decoding process. All their tested classification and dialogue generation models are either variations of RNNs (such as LSTMs or gated recurrent units (GRUs)), convolutional neural networks (CNNs), or hybrids of these systems (LSTM-outputs fed into a CNN, known as recurrent convolutional neural networks (RCNNs)). The authors study the influence of age, gender, and location on dialogue classification and generation, and use both automatic (perplexity, trait accuracy, and generated response diversity measures) and human evaluation. They find dialogues to be most distinguishable by gender (about 90.61\% test accuracy), then age (78.32\% test accuracy), and finally location (62.04\% test accuracy). Both automatic and human evaluation of the generated responses show that the best performing models benefit significantly from the persona-aware attention mechanism, possibly making a case to consider more attention-based architectures instead of RNNs.

Although the previously mentioned architectures are able to produce human-like conversational responses, sometimes even leveraging the fluency of large pre-trained LMs, they all suffer from the same computational drawback. They all require massive amounts of computational power to adapt their language styles, because in their cases, guided generation implies fine-tuning (or even re-training) large attribute-specific dialogue datasets. For general controlled text generation, this obstacle is overcome by \cite{dathathri2019plug}'s previously mentioned PPLM setup. The conversational analog of this idea, the plug-and-play conversational model (PPCM), is proposed by \cite{madotto-etal-2020-plug}. Similar to PPLM, PPCM achieves guided dialogue generation via activation-space perturbations using easy to train attribute models. 
% In this configuration, they can achieve controllable response generation without needing dialogue data or having to fine-tune a large language model. 
Due to the computational complexity of PPLM's decoding process, PPLM is unusable as an interactive conversational system. PPCM solves this problem by using residual adapters \citep{bapna-firat-2019-simple} to tweak the decoding procedure such that it does not require more computational resources. 
% See Section \ref{sec:ppm} for a detailed explanation of the mechanisms behind PPLM and PPCM. 
\cite{madotto-etal-2020-plug} show, using both human and automatic evaluation, that PPCM can balance grammatical fluency and high degrees of attribute-adherence in its generated responses. PPCM uses DialoGPT as its base language model, and is tested for topical or sentimental attributes (i.e., positive, negative, sports, business, or science \& tech). 
Previous work on controlled text generation focuses on content (e.g., topical attributes or sentiment), rather than more abstract linguistic features, which I hypothesize are more challenging to model and control. The previously mentioned work by \cite{zheng2019personalized} is a notable exception, as their approach deals with controlling dialogue systems for linguistic features, like age, gender, and geographical region. However, \cite{zheng2019personalized} still suffer from high computational costs, because control is achieved by fine-tuning a large system for every specific set of attributes. 
% Furthermore, their proposed architectures are RNN-based, as opposed to my Transformer-based approach. My work therefore aims to extend the applicability of plug-and-play controlled generation to more abstract linguistic characteristics than those explored by \cite{dathathri2019plug} and \cite{madotto-etal-2020-plug}, and without the significant fine-tuning cost of \cite{zheng2019personalized}.

The work presented in this thesis extends the applicability of the Transformer-based Plug-and-Play controlled generation model to more abstract writing styles, namely the linguistic characteristics associated with certain age groups. Furthermore, I apply this adaptation to the task of dialogue generation. As a preliminary research objective, I aim to use text-based NLP models to detect age-related linguistic features from dialogue and discourse data. This preliminary experiment is presented in the next chapter.

% \len{TODO \begin{itemize}
%     \item Ask for Sandro's feedback on this last rephrased paragraph.
%     \item Is it worth mentioning that Zheng et al 2019 deals with Chinese Mandarin dialogue systems, and mine with English?
% \end{itemize} }

\chapter{Experiment 1: Detecting Age-Related Linguistic Patterns in Dialogue}\label{ch:experiment1}
\section{Introduction}

% \len{Briefly introduce the problem you seek to solve (i.e., detection of age related linguistic features from dialogue and discourse), your hypotheses, and give an overview of the chapter (i.e., data, methods and models, results, and analyses).}

This chapter focuses on our experiments about age detection from text, and the components involved. The problem we tackle in this first phase of experiments is automated detection of age-related linguistic patterns from dialogue and discourse, using current text-based NLP models. Being able to detect and investigate these linguistic differences is important for controlled dialogue generation, because it suggests that adapting automated conversational responses to a user's age is possible. Moreover, it can provide us with insights about which linguistic features are most salient for distinguishing between, and adapting to, different age groups. We expect that the classification models are able to reliably detect age-related differences in both transcribed dialogue and discourse, and the most informative differences to lie at the syntactic-level.

The following section describes the two datasets used for these experiments. There we provide descriptive statistics, examples, and comparisons between the corpora. Section \ref{sec:exp1_methods_exp_setup} covers the problem description in more detail, along with the models used, and our experimental setup. The classification results are presented in Section \ref{sec:exp1_results}. Then for the dialogue classification models, Section \ref{sec:exp1_analyses} contains both quantitative and qualitative analyses of the results.

\section{Data}\label{sec:data}

% Two datasets are considered during the experiments of this thesis: (1) the spoken component of the British National Corpus (BNC or BNC2014) \citep{love-spoken-bnc-2014}, and (2) the Blog Authorship Corpus (BAC, or sometimes referred to as `blog corpus') \citep{schler2006effects}. The first corpus is a collection of transcriptions of everyday conversations in British English, gathered between 2012 and 2016. The second is a dataset of blogs posted on \url{https://www.blogger.com}, gathered in or before August 2004.

% The texts (i.e., blogs or dialogue transcriptions) in both corpora are labeled by age, among other labels. This makes them suitable candidates for training and testing age classifiers. In both cases, the texts are written in a somewhat informal manner, making them more representative of everyday speech. Only the BNC is used for the controllable dialogue generation phase of the experiments, as the BAC is not a conversational dataset. What follows is an overview of both datasets' motivation for use, drawbacks, metadata, descriptive statistics, and pre-processing steps before analyses. 

% \subsection{The British National Corpus (BNC)}
% The conversations of the spoken component of the BNC were typically recorded at home and took place between friends and family of participants. Participants recorded their conversations using their smartphones, allowing for spontaneous recording. These two properties of the BNC's sampling procedure make the recorded dialogues representative of contemporary everyday British English speech. 

% A total of 1251 conversations of 672 speakers constitutes the full corpus, accounting for 10.4 million words. During the recruitment process of the BNC, prospective participants were asked to disclose personal information like age, gender, highest completed level of education, employment, and perceived accent\footnote{For a detailed description of the sampling decisions made during data collection, the reader is referred to the BNC2014 user manual: \url{http://corpora.lancs.ac.uk/bnc2014/doc/BNC2014manual.pdf}}. Other metadata accompany the conversations themselves. Namely, number of speakers taking part in the conversation, ages of those speakers, conversation length, and topic. 

% Nine age-brackets are present in the corpus: 11-18, 19-29, 30-39, 40-49, 50-59, 60-69, 70-79, 80-89, and 90-99. To make the objective of learning to classify and generate conversational responses more straightforward, I only consider dialogues between two participants. Namely, it will be more difficult to distinguish which utterances constitute the relevant responses to something said in a conversation with more than two participants. This filter step results in a remaining dataset of 622 dialogues, almost 460K utterances (i.e., a single turn in a dialogue), and nearly 5M tokens. On average, a dialogue has 736 turns, meaning that the conversations are relatively long. Furthermore, due to minors not being allowed to participate in this project's interactive experiments later on, the age-bracket 11-18 is dropped. This leaves us with a dataset of 522 dialogues, 418K utterances, and 4.4M tokens. Finally, the age-brackets get regrouped into brackets 19-29 and 50 plus, and conversations with speakers aged 30-49 are removed. This gap between to two age-groups is made to minimize the chance of overlapping linguistic characteristics being present between separate age-brackets (e.g., the difference in language use between speakers aged 19-29 and 30-39 is probably smaller than that between age-brackets 19-29 and 50-59). Moreover, only conversations between two participants of the same (new regrouped) age-bracket is kept. This is done to avoid the confounding factor that interlocutors of a certain age-group might adjust their choice of words to the age of the person they are talking to. Only considering dialogues between participants of similar ages is expected to keep the utterances as representative of the age-bracket's linguistic characteristics as the corpus allows. These final filtration steps result in a subset consisting of 237 dialogues, roughly 172.000 utterances, and approximately 1.8M tokens.




% \begin{itemize}
%     \item Talk about pre-processing steps.
%     \item The main limitations of the BNC are its size and imbalance. 
%     \item Weighted loss and weighted random sampling
%     \item Data outdated? Not representative of modern speech.
% \end{itemize}

% \begin{figure}[H]
%     \centering
%     \includegraphics[width=\textwidth]{figures/bnc_alt1_dist_age_new.png}
%     \caption{Distribution of age-brackets per dialogue in 19-29 vs. 50-plus subset of BNC.}
%     \label{fig:dist_age_bnc}
% \end{figure}



% \subsection{The Blog Authorship Corpus (BAC)}

% % \begin{table}[H]
% %     \centering
% %     \begin{tabular}{l | c  c}
% %         \hline
% %         \textbf{Corpus} & BNC & BAC \\
% %         \hline
% %         \textbf{Dialogue?} & Yes & No\\
% %         \textbf{No. words} & $10.4 \cdot 10^6$  & $140 \cdot 10^6$ \\
        
% %         \hline
% %     \end{tabular}
% %     \caption{Summary of corpora properties.}
% %     \label{tab:summary_corpora}
% % \end{table}

% \begin{table}[H]
%     \centering
%     \begin{tabular}{l | c  c}
%         \hline
%         \textbf{Corpus} & BNC & BAC \\
%         \hline
%         \textbf{Dialogue?} & Yes & No\\
%         \textbf{No. words} & $10.4 \cdot 10^6$  & $140 \cdot 10^6$ \\
%         \textbf{No. datapoints} & 64,994 & 677,244 \\
%         \textbf{No. age brackets} & 2 & 3 \\
%         \textbf{Year of collection} & 2016 & 2004 \\
%         \textbf{Mean sample length} & 6.389 & 102.168 \\
%         \textbf{Std. sample length} & 9.929 & 212.935 \\
%         \textbf{Min. sample length} & 1 & 1 \\
%         \textbf{Max sample length} & 724 & 71580 \\
%         \textbf{No. unique topics} & 790 & 40 \\
%         \hline
%     \end{tabular}
%     \caption{Summary of corpora properties \textit{after} pre-processing.}
%     \label{tab:summary_corpora}
% \end{table}

One of the used datasets is one of dialogue data where information about the age of the speakers involved in the conversation is available (see the dialogue snippets in Figure~\ref{fig:ex1}), i.e., the spoken partition of the British National Corpus~\citep{love-spoken-bnc-2014}. It is henceforth referred to as the \emph{dialogue} dataset. For comparison with previous work
and to explore commonalities and differences between various types of language data,
I also experiment with a dataset of discourse data, i.e., the Blog Authorship Corpus used by \citet{schler2006effects}, that is henceforth referred to as the \emph{discourse} dataset.
Below, I briefly describe the two datasets along with the pre-processing steps taken to
make the data suitable for 
the experiments.

%------------ FIGURE -----------------
\begin{figure}[]\centering \small
% \includegraphics[height=3.2cm]{images/white_dog_COCO_train2014_000000522288.jpg}\\
%\includegraphics[height=2.7cm]{images/other_dog.png} \\
% \vspace*{2pt}
\fbox{\begin{minipage}{.5\columnwidth}
\begin{tabular}{@{}l@{\ }l}
& \textbf{age 19-29}\\[2pt]
                  & \textcolor{red}{\textbf{A:}} oh that's cool\textcolor{white}{aaaaaaaaaaaaaaaaaaaaaaaaaaaaaaaaaaaa}\\
% $\leadsto$ & 
& \multicolumn{1}{r}{\textcolor{blue}{\textbf{B:}} different sights and stuff}\\
% $\leadsto$ & 
& \textcolor{red}{\textbf{A:}} oh
\end{tabular}
\end{minipage}
}\\[1pt]
\fbox{\begin{minipage}{.5\columnwidth}
\begin{tabular}{@{}l@{\ }l}
& \textbf{age 50+}\\[1pt]
                  & \textcolor{red}{\textbf{A:}} well quite and I'd have to come back as well\textcolor{white}{aaaaaaaa}\\
% $\leadsto$ 
& \multicolumn{1}{r}{\textcolor{blue}{\textbf{B:}} that's of course}\\

% $\leadsto$ 
& \textcolor{red}{\textbf{A:}} and make up for you know
\end{tabular}
\end{minipage}
}
\caption{Example dialogue snippets from speakers of different age groups (19-29 vs.\ 50+) in the British National Corpus.
% \san{To be double-checked that A and B is correct + why they are not adjacent turns.}
I conjecture that stylistic and lexical differences between age groups can be detected.
In my approach, experiments are conducted at the level of a single utterance.}\label{fig:ex1}
\end{figure}
%-------------END OF FIG---------------


\begin{table*}[h]
\resizebox{\linewidth}{!}{
\begin{tabular}{lllllll}
\toprule
\textbf{dataset}   & \textbf{\# age groups} & \textbf{\# samples} & \textbf{\# tokens} & \textbf{mean length ($\pm$std)} & \textbf{min - max length} & \textbf{\# topics} \\
\midrule
% dialogue  & 2             & 64,967     & 420,122     & 6.5 ($\pm$10.1)         & 1 - 724           & 790              \\
dialogue  & 2             & 67,282     & 787,352     & 11.7 ($\pm$19.0)         & 1 - 1246           & 790              \\
% discourse & 3             & 677,244    & 140M      & 102.2 ($\pm$212.9)     & 1 - 71,580         & 40  \\
discourse & 3             & 678,165    & 137M      & 201.7 ($\pm$415.9)     & 1 - 131,169         & 40  \\
\bottomrule
\end{tabular}
}
    \caption{Descriptive statistics of the datasets used in my experiments. {Length} is the number of tokens in a sample.}

    
    \label{tab:summary_corpora}
\end{table*}

\subsection{Dialogue Dataset}\label{sec:datadialogue}
\label{subsec:dialogue_dataset}

% \san{Here I only describe BNC: remember to make comparisons with BAC wrt length of utterances; number of age brackets, etc.}

This partition of the British National Corpus includes spoken informal open-domain conversations between people that were collected between 2012 and 2016 via crowd-sourcing and then recorded and transcribed by the creators. Dialogues can be between two or more interlocutors and are annotated
% at the utterance level 
along several dimensions, including age and gender, together with geographic and social indicators. Speaker ages in the original dataset are categorized in the following ten brackets: 0-10, 11-18, 19-29, 30-39, 40-49, 50-59, 60-69, 70-79, 80-89, and 90-99.


The focus is on conversations in the British National Corpus that took place between two interlocutors,
and only dialogues
between people of the same age group are considered. 
Furthermore, only dialogues are considered between speakers belonging to two age groups: 19-29 and 50+, in which the conversations are grouped from five original brackets: 50-59, 60-69, 70-79, 80-89, and 90-99.
% one where speakers' age ranges from 19 to 29 (hence, 19-29); one where it is equal to or greater than 50 (hence, 50+). 
The intermediate age brackets are omitted to allow for clearer differentiation.
% The dialogues between speakers of intermediate ages are discarded. 
% These two steps are performed to create sufficiently separated classes of comparable size. 

The dialogues are split into their constituent utterances (e.g., from each dialogue snippet in Figure~\ref{fig:ex1} three utterances are extracted), and further pre-process them by removing non-alphabetical characters.
% and stopwords, by using the Natural Language Toolkit (NLTK) English stopword list \cite{bird2009natural}. 
Only samples that were not empty after pre-processing were kept.
The resulting dialogue dataset that is used for the experiments includes around 67K utterances with an average length of 11.7 tokens. Descriptive statistics of it are reported in Table~\ref{tab:summary_corpora}.

Each conversation in the British National Corpus is annotated with a list of \emph{topics} provided by the speakers during data collection. However, it is desirable to obtain a better picture of how single representative dialogue topics are distributed among the age groups.
% Since topic may constitute a confound for models toward detecting speakers' age group, I leverage this annotation to extract one single topic per utterance. 
% \raq{In which sense do I leverage this information? The analysis I currently have of topics doesn't seem to shed light on whether it is a confound or not... Maybe give less importance to topics (no paragraph heading) - it's simply part of the descriptive statistics?} 
Thus, to extract a single representative topic from this list, I first compute the frequency of all topic labels in the whole dataset. Then, for each utterance, I take the label in the conversation with the highest frequency in the ranking.
In total, the final dataset includes 790 unique topic labels. The distribution of the most frequent ones is reported in Figure~\ref{fig:bnc_age_topic_dist_incl_unk}. As can be seen, frequent topics (besides the frequent \emph{none} label) are \emph{food}, \emph{work}, and \emph{holidays}, which reveals the colloquial and everyday nature of the dialogues in this dataset.

\subsection{Discourse Dataset}
The Blog Authorship Corpus~\citep{schler2006effects} is a collection of blog posts
% dataset is a corpus of blogs
written on \url{https://www.blogger.com}, gathered in or before August 2004. % All blog entries are 
Each blog entry is
written by a single user
whose age, gender, and astrological sign are reported.
The corpus contains almost 700,000 posts by 19,000 unique bloggers (i.e., $\sim$35 posts per blogger on average).
% The original corpus consists of blog entries written by more than 19,000 bloggers, for a total of almost 700,000 posts, and over 140M words. 
For my experiments, similar to~\citet{schler2006effects},
three age groups are considered: 13-17, 23-27, and 33+. The data are pre-processed in the same way as described above, namely by removing non-alphabetical characters. The resulting dataset that is used for the experiments includes slightly more than 678K samples with an average length of 201.7 tokens. Descriptive statistics of it are reported in Table~\ref{tab:summary_corpora}. 
% \len{TODO - Adjust paragraph to W.S. setting.}

% only consider three age brackets, leaving gaps in between the age groups to avoid as much overlapping language use as possible between the classes. Similar to \citeauthor{schler2006effects}, I consider the following age brackets: 13-17, 23-27, and 33+.
% for which information about age of the speaker / writer is available, the spoken component of the British National Corpus (BNC or BNC2014) \citep{love-spoken-bnc-2014}, and the Blog Authorship Corpus (BAC, or sometimes referred to as `blog corpus') \san{let's use just one abbreviation per corpus; I'd say BNC and BAC; but I can also use \emph{discourse} and \emph{dialogue}} \citep{schler2006effects}.

%\paragraph{Topics}
Each sample in the Blog Authorship Corpus is annotated with one topic. In my final discourse dataset, the unique topics present are 40. Figure~\ref{fig:blog_age_topic_dist_incl_unk} reports the distribution of the most frequent ones. As can be noted, frequent topics are \emph{student}, \emph{arts}, and \emph{technology}, which 
reveals that this and the dialogue dataset are rather different with respect to the topic themes, input text length, and how well the topics are organized. 
Namely, the discourse dataset has a clear over-representation of topics related to student life, compared to the more colloquial and everyday subject matter of the dialogue dataset. Furthermore, the descriptive statistics in Table~\ref{tab:summary_corpora} show that the average data-entry length in tokens of the discourse dataset is almost 20 times larger than that of the dialogue dataset. Additionally, the differences number of unique topics (790 in dialogue dataset versus 40 in discourse dataset) between the two datasets illustrates the increased noisiness of the dialogue dataset compared to the discourse dataset. The shorter and noisier form of the dialogue dataset could make it more challenging to perform automated age group detection on its texts, as they are likely to carry less discriminative signal with more noise.
% and has only one topic per data-entry, compared to an average of \len{find average topic list length of bnc} topics per dialogue.


\begin{figure}[H]
     \centering
     \begin{subfigure}[b]{0.45\textwidth}
        \centering
        \includegraphics[width=\textwidth]{figures/bnc_age_topic_histogram.png}
        \caption{}
        \label{fig:bnc_age_topic_dist_incl_unk}
     \end{subfigure}
     \hfill
     \begin{subfigure}[b]{0.45\textwidth}
        \centering
        \includegraphics[width=\textwidth]{figures/blog_age_topic_histogram.png}
        \caption{}
        \label{fig:blog_age_topic_dist_incl_unk}
     \end{subfigure}
        \caption{Distribution of most frequent topics shown by age group in the \textbf{dialogue dataset} ((a) / left) and \textbf{discourse dataset} ((b) / right). Best viewed in color.}
        \label{fig:bnc_blog_age_topic_dist_incl_unk}
\end{figure}








\section{Methodology and Experimental Setup}
\label{sec:exp1_methods_exp_setup}

The current section describes the methodology and experimental details of the automated age-detection experiments. I frame the problem as a $N$-class classification problem: given a fragment of text $X$, I seek to predict
the age class of its speaker/writer.
% its age class. 
For the dialogue dataset, $N=2$, while $N=3$ for the discourse dataset.
I experiment with various models that are briefly described here below. Details on the training and evaluation of models are given at the end of the sub-section.
% : $n$-gram logistic regression, Long Short Term Memory (LSTM) networks, and transformer-based language models BERT and GPT-2.

% \len{TODO - Make this intro/foreword flow a bit smoother. Do I need to mention that I frame the problem as a N-class classification problem etc.?}

\paragraph{\textit{n}-gram} 
% $n$-gram logistic regression models are the simplest classifiers I consider for my task. For these models,
% every 
The simplest models are based on $n$-grams, which have the advantage of being highly interpretable.
Each data entry (i.e., a dialogue utterance or blog post) is split into chunks of all possible contiguous sequences of $n$ tokens. The resulting vectorized features are used by a logistic regression model to estimate the odds of a text sample belonging to a certain age group. Specifically, experiments are performed with unigram, bigram, and trigram models. Note that a bigram model uses unigrams and bigrams, and a trigram model uses unigrams, bigrams, and trigrams.

\paragraph{LSTM and BiLSTM}
%also 
A standard Long Short-Term Memory network \cite[LSTM;][]{hochreiter1997long} is used with two layers, embedding size 512, and hidden layer size 1024. Batch-wise padding is applied to variable-length sequences. The original model's bidirectional extension, the bidirectional LSTM \cite[BiLSTM;][]{schuster1997bidirectional}, is also used.
% as an age-classifier. 
% This version of the LSTM 
BiLSTM more thoroughly leverages forward and backward directed information by combining the hidden states from both directions. Padding is similarly applied to this model, and the following optimal architecture is found via guided grid search: embedding size 64, 2 layers, and hidden layer size 512. Both RNN models are found to perform optimally for a learning rate of $10^{-3}$.

\paragraph{BERT} 
I also experiment with a Transformer-based model, i.e., 
% Finally, I use 
Bidirectional Encoder Representations from Transformers \cite[BERT;][]{devlin-etal-2019-bert} for text classification.
%, as my transformer-based class of models. 
BERT is pre-trained to learn deeply bidirectional language representations from massive amounts of unlabeled textual data. 
The base, uncased version of BERT, is used in two settings: (1) by using its 
pre-trained frozen embeddings (BERT$_{frozen}$) and (2) by fine-tuning the embeddings on the age classification task (BERT$_{FT}$).
% and fine-tuned on the task (BERT-FT). \san{should I use these labels to refer to the BERT models?; otherwise, I can use BERT for the fine-tuned one and BERT-frozen for the other}
% Its frozen pre-trained embeddings, as well as a fine-tuned setup are compared for the task of speaker/author-age prediction.
The BERT embeddings are followed by a dropout layer with dropout probability 0.1, and a linear layer with input size 768.

\paragraph{Experimental Details} % \textit{I report common things across models: objective function, number of epochs (?), early stopping criterion, etc. Plus, e.g., I train the models N times with random initializations}

Both datasets are randomly split into a training (75\%), validation (15\%), and test (10\%) set.
%s accounting for 75\%, 15\%, and 10\% of the full corpus size, respectively.
Each model with a given configuration of hyperparameters is run 5 times with different random initializations. All models are trained on an NVIDIA TitanRTX GPU.

The $n$-gram models are trained in a One-vs-Rest (OvR) fashion, and optimized using the Limited-memory Broyden–Fletcher–Goldfarb–Shanno (L-BFGS) algorithm \citep{liu1989limited}, with a maximum of $10^6$ iterations. The $n$-gram models are trained until convergence or for the maximum number of iterations.

% All neural models 
LSTMs and BERT-based models
are optimized using Adam \citep{DBLP:journals/corr/KingmaB14}, and trained for 10 epochs, with an early stopping patience of 3 epochs.
% All results reported in Tables \ref{tab:bnc_classification} and \ref{tab:blog_classification} are averaged over 5 random initializations. 
The RNN-based models' embeddings are jointly trained, and optimal hyperparameters (i.e., learning rate, embedding size, hidden layer size, and number of layers)  are determined using the validation set and a guided grid-search. %BERT-base-uncased is either used in its unaltered, pre-trained form, or fine-tuned on the validation set for 10 epochs, or until the early stopping criterion is met. 
BERT$_{FT}$ is fine-tuned on the validation set for 10 epochs or until the early stopping criterion is met.
BERT models have a maximum input length of 512 tokens. Sequences exceeding this length are truncated.

\section{Detecting Age-Related Linguistic Patterns in Dialogue and Discourse}
\label{sec:exp1_results}

We first report results on \emph{discourse}
% dataset
to check whether we replicate
previous findings. Then, we focus on \emph{dialogue} to answer our research questions. We report accuracy and $F_1$ for each age group.

\subsection{Classification Performance on Discourse}
Table~\ref{tab:blog_classification} reports the results.
% obtained by all tested models.
% on the \emph{discourse} dataset.
%, BAC. 
As can be seen,
all models are well above the majority class baseline in terms of both accuracy (0.472) and 
% age group-specific 
$F_1$s (0.642). This overall confirms previous evidence \citep{schler2006effects} that language features of (written) \emph{discourse} can predict, to some extent, the age group to which the person
% speaker
belongs. At the same time, BERT fine-tuned on the age classification task stands out as our best-performing model by achieving highest accuracy (0.742) and highest $F_1$ in all age groups.
% BiLSTM and LSTM rank second (0.720) and third (0.714) in terms of accuracy, respectively, while a somehow more mixed pattern is observed for $F_1$ scores.
%, where $n$-gram models come into play.
% , where $n$-gram models can perform on par with or better than LSTMs on certain age groups.
LSTM ranks  second (0.663) in terms of accuracy, while the second best $F_1$ scores for all age groups are distributed among the bigram and trigram models.
Overall, these results indicate that powerful neural models that are capable of representing the linguistic context % (BERT \emph{in primis}) 
have a great advantage on this dataset over simpler $n$-gram models, which are more than 10 accuracy points behind. 

Finally, it should be noted that our best results are slightly lower than those obtained by~\citet{schler2006effects}. This could be due to two main reasons: First, they experiment with a
differently pre-processed and smaller
% (smaller)
% significantly smaller 
dataset than ours.\footnote{They are left with roughly 511K datapoints after pre-processing, 
while we experiment with around 677K.}
% which also has a different majority class and corresponding baseline (see Table~\ref{tab:blog_classification}). 
Second, while in our approach all models are trained end-to-end on the task,~\citet{schler2006effects} use
hand-crafted features that are specific to the dataset, 
which could constitute an advantage.

\subsection{Classification Performance on Dialogue}
Table~\ref{tab:bnc_classification_ws} reports the results obtained on the dialogue dataset. As can be seen, BERT fine-tuned on the task is again the best-performing model in terms of accuracy (0.729), which confirms the effectiveness of this model in detecting age-related linguistic differences. At the same time, it can be noted that the model based on trigrams is basically on par with it in terms of accuracy (0.722) and well above both LSTM and BiLSTM (0.693 and 0.691, respectively). A similar pattern is shown for $F_1$ scores, where BERT fine-tuned and the trigram model achieve comparable performance, with LSTMs being overall behind. 

Overall, our results indicate that predicting the age group to which a speaker belongs, using text-based models, is possible also for \emph{dialogue} data, though the task appears to be somehow more challenging compared to when performed on discourse.
Note that the improvement with respect to the majority/random baseline is lower in dialogue (22.9 and 27 accuracy point increase w.r.t baselines for best model for dialogue and discourse datasets, respectively).

% \san{do we need a comment on the comparable accuracy, but with different N of classes and majority baseline?}\raq{maybe say: note that the improvement wrt to the random baseline is lower in dialogue}.
At the same time, the different ranking of models observed between discourse and dialogue suggests possibly different strategies used by models to solve the task. In particular, the very good performance of the trigram model in \emph{dialogue} suggests that leveraging `local' linguistic features captured by $n$-grams is extremely effective in this setup. This could indicate that differences among various age groups are at the level of
% \textcolor{red}{
local lexical constructions.
% }
% \raq{local lexical constructions?}
% \raq{the next bit doesn't stand up, not only because of topics, but also because without stop words I don't see how we can make any claims regarding style at this stage}
% lexical and stylistic features of the language used, rather than involving topical aspects. \san{if we don't include the analysis on topics, this cannot be claimed}
This deserves further analysis, which we carry out in the next section.
% \san{We should add a comment on how performance is with respect to topics: we refer to the plot}


% Here we have the tables with the numbers (accuracy, precision, etc.) for each model on each dataset. Highlight what are best models on each dataset, what are differences between the two datasets, etc.
% Are there differences or not? Yes. Models above baselines.

% What are the features that drive the prediction? We look into two best-performing models.




% \begin{table*}[h]
%     \centering
%     \begin{tabular}{l c c c c}
%     \toprule
%     \textbf{Model} & \textbf{Accuracy} & $\boldsymbol{F}_1^{(13-17)}$ & $\boldsymbol{F}_1^{(23-27)}$ & $\boldsymbol{F}_1^{(33+)}$\\
%     % -plus)}$\\
%      & $\uparrow$ better & $\uparrow$ better & $\uparrow$ better & $\uparrow$ better\\
%     \midrule
%     % Baseline (
%     Majority class
%     % ) 
%     & 0.472
%     % (0.000) 
%     & * & 0.642
%     % (0.000)
%     & *\\
%     % Best model by \citeauthor
%     \citet{schler2006effects} & 0.762 & 0.860 & 0.748 & 0.504 \\
%     \midrule
%     unigram & 0.603 (0.001) & 0.760 (0.003) & 0.706 (0.001) & 0.491 (0.003)\\
%     bigram & 0.627 (0.001) & 0.788 (0.001) & 0.715 (0.001) & \textcolor{blue}{0.504} (0.002)\\
%     trigram & 0.625 (0.002) & \textcolor{blue}{0.789} (0.001) & 0.716 (0.002) & 0.485 (0.003)\\
%     \midrule
%     LSTM & 0.714 (0.005) & 0.772 (0.007) & 0.740 (0.004) & 0.501 (0.006) \\
%     BiLSTM & \textcolor{blue}{0.720} (0.001) & 0.778 (0.005) & \textcolor{blue}{0.746} (0.001) & 0.486 (0.016)\\
%     \midrule
%     BERT$_{frozen}$
%     % -base-uncased (frozen)
%     & 0.604 (0.001) & 0.627 (0.011) & 0.666 (0.005) & 0.198 (0.018)\\
%     BERT$_{FT}$
%     % -base-uncased (fine-tuned)
%     & \textbf{0.731} (0.002) & \textbf{0.791} (0.003) & \textbf{0.752} (0.005) & \textbf{0.521} (0.020)\\
%     % \hline
%     % GPT-2 medium (frozen) & 0.621 (0.003) & 0.654 (0.012) & 0.678 (0.009) & 0.247 (0.032)\\
%     \bottomrule
%     \end{tabular}
%     \caption{Discourse dataset. \len{\textbf{Excluding} stopwords} Test set results averaged over 5 random initializations. Format: \textit{average metric (standard error)}. Values in \textbf{bold} are the highest in the column; in \textcolor{blue}{blue}, the second highest. *: $F_1$ is actually $0/0$.}
%     % \san{If we agree with the abbreviations used in this table, we could use them also in Table 3}}
%     \label{tab:blog_classification}
% \end{table*}

\begin{table*}[h]
    \centering
    \begin{tabular}{l c c c c}
    \toprule
    \textbf{Model} & \textbf{Accuracy} & $\boldsymbol{F}_1^{(13-17)}$ & $\boldsymbol{F}_1^{(23-27)}$ & $\boldsymbol{F}_1^{(33+)}$\\
    % -plus)}$\\
     & $\uparrow$ better & $\uparrow$ better & $\uparrow$ better & $\uparrow$ better\\
    \midrule
    % Baseline (
    Majority class
    % ) 
    & 0.472
    % (0.000) 
    & * & 0.642
    % (0.000)
    & *\\
    % Best model by \citeauthor
    \citet{schler2006effects}** & 0.762 & 0.860 & 0.748 & 0.504 \\
    % Majority class & 0.430 & * & 0.760 & 0.504 \\
    \midrule
    unigram & 0.601 (0.001) & 0.764 (0.001) & 0.704 (0.001) & 0.498 (0.003)\\
    bigram & 0.625 (0.001) & \textcolor{blue}{0.790} (0.001) & \textcolor{blue}{0.712} (0.001) & \textcolor{blue}{0.518} (0.001)\\
    trigram & 0.623 (0.001) & \textcolor{blue}{0.790} (0.001) & 0.712 (0.002) & 0.498 (0.002)\\
    \midrule
    LSTM & \textcolor{blue}{0.663} (0.005) & 0.748 (0.003) & 0.664 (0.010) & 0.502 (0.004) \\
    BiLSTM & 0.618 (0.008) & 0.732 (0.003) & 0.579 (0.016) & 0.509 (0.004)\\
    \midrule
    BERT$_{frozen}$
    & 0.623 (0.002) & 0.658 (0.006) & 0.678 (0.007) & 0.256 (0.041)\\
    BERT$_{FT}$
    % -base-uncased (fine-tuned)
    & \textbf{0.742} (0.010) & \textbf{0.813} (0.007) & \textbf{0.749} (0.013) & \textbf{0.592} (0.009)\\
    % \hline
    % GPT-2 medium (frozen) & 0.621 (0.003) & 0.654 (0.012) & 0.678 (0.009) & 0.247 (0.032)\\
    \bottomrule
    \end{tabular}
    \caption{Discourse dataset. Test set results averaged over 5 random initializations. Format: \textit{average metric (standard error)}. Values in \textbf{bold} are the highest in the column; in \textcolor{blue}{blue}, the second highest. *: $F_1$ is actually $0/0$. **: these results were obtained with a different final dataset.}
    % \san{If we agree with the abbreviations used in this table, we could use them also in Table 3}}
    \label{tab:blog_classification}
\end{table*}

\begin{table*}[h]
    \centering
    % \resizebox{\columnwidth}{!}{
    \begin{tabular}{l c c c}
    \toprule
    \textbf{Model} & \textbf{Accuracy} & $\boldsymbol{F}_1^{(19-29)}$  & $\boldsymbol{F}_1^{(50+)}$ \\ 
    % -plus)}$ \\
     & $\uparrow$ better & $\uparrow$ better & $\uparrow$ better \\
    \midrule
    Random
    % Baseline (random guessing) 
    & 0.500
    % (0.000) 
    & 0.500
    % (0.000) 
    & 0.500 
    % (0.000)
    \\ \midrule
    unigram & 0.701 (0.007) & 0.708 (0.009)  & 0.693 (0.004)\\
    bigram & 0.719 (0.002) & 0.724 (0.003) & 0.714 (0.003)\\
    trigram &  \textcolor{blue}{0.722} (0.001) & \textcolor{blue}{0.727} (0.003) & \textcolor{blue}{0.717} (0.001)\\ \midrule
    LSTM &  0.693 (0.003) & 0.696 (0.005) & 0.691 (0.007)\\
    BiLSTM & 0.691 (0.009) & 0.702 (0.017) & 0.679 (0.007) \\ \midrule
    BERT$_{frozen}$
    % -base uncased (frozen) 
    & 0.675 (0.003) & 0.677 (0.008) & 0.673 (0.010)\\
    BERT$_{FT}$
    % -base uncased (fine-tuned) 
    & \textbf{0.729} (0.002) & \textbf{0.730} (0.011) & \textbf{0.727} (0.010)\\
    % \hline
    % GPT-2 medium (frozen) & 0.671 (0.003) & 0.665 (0.014) & 0.675 (0.017) \\
    \bottomrule
    \end{tabular}
    % }
    \caption{Dialogue dataset. Test set results averaged over 5 random initializations. Format: \textit{average metric (standard error)}. Values in \textbf{bold} are the highest in the column; in \textcolor{blue}{blue}, the second highest.}
    \label{tab:bnc_classification_ws}
\end{table*}

% \begin{table*}[h]
%     \centering
%     % \resizebox{\columnwidth}{!}{
%     \begin{tabular}{l c c c}
%     \toprule
%     \textbf{Model} & \textbf{Accuracy} & $\boldsymbol{F}_1^{(19-29)}$  & $\boldsymbol{F}_1^{(50+)}$ \\ 
%     % -plus)}$ \\
%      & $\uparrow$ better & $\uparrow$ better & $\uparrow$ better \\
%     \midrule
%     Random
%     % Baseline (random guessing) 
%     & 0.500
%     % (0.000) 
%     & 0.500
%     % (0.000) 
%     & 0.500 
%     % (0.000)
%     \\ \midrule
%     unigram & 0.702 (0.006) & 0.713 (0.006)  & 0.690 (0.006)\\
%     bigram & 0.703 (0.006) & 0.713 (0.005) & 0.693 (0.008)\\
%     trigram &  \textcolor{blue}{0.709} (0.007) & \textbf{0.718} (0.007) & 0.700 (0.008)\\ \midrule
%     LSTM & 0.696 (0.005) & 0.689 (0.018) & \textcolor{blue}{0.701} (0.016)\\
%     BiLSTM & 0.684 (0.007) & 0.688 (0.018) & 0.679 (0.016) \\ \midrule
%     BERT$_{frozen}$
%     % -base uncased (frozen) 
%     & 0.673 (0.005) & 0.679 (0.013) & 0.667 (0.018)\\
%     BERT$_{FT}$
%     % -base uncased (fine-tuned) 
%     & \textbf{0.710} (0.006) & \textcolor{blue}{0.717} (0.007) & \textbf{0.703} (0.014)\\
%     % \hline
%     % GPT-2 medium (frozen) & 0.671 (0.003) & 0.665 (0.014) & 0.675 (0.017) \\
%     \bottomrule
%     \end{tabular}
%     % }
%     \caption{Dialogue dataset \len{Excluding stopwords}. Test set results averaged over 5 random initializations. Format: \textit{average metric (standard error)}. Values in \textbf{bold} are the highest in the column; in \textcolor{blue}{blue}, the second highest.}
%     \label{tab:bnc_classification}
% \end{table*}


\begin{table*}[h]
\resizebox{\linewidth}{!}{
% \small
\begin{tabular}{@{}lllll@{}}
\toprule
\bf actual age & \bf both correct                & \bf both wrong                 & \bf BERT$_{FT}$ correct | trigram wrong                              & \bf trigram correct | BERT$_{FT}$ wrong                                                            \\ \midrule
19-29     & oh that's cool              & A retrospective exhibition & what even on the green slope?             & really?                                                                    \\
19-29     & a text and then I'll do it  & chuck them in those pots   & yeah you told me to do you told me to do  & and she like won't eat any carbs and she's like                            \\
19-29     & yeah                        & mm                         & somebody made the f***ing table           & do you not like total greens? \\ 
% 19-29     & yeah they go and like work for Pixar and stuff                        & yeah but that's for er storage reasons I suppose                         & ...           & ... \\
% 19-29     & I don't know? sounded crazy                        & ...                         & ...           & ... \\
% 19-29     & do you have exams again?                        & ...                         & ...           & ... \\
\midrule
% well er it's not acceptable just to ditch \textless{}unclear\textgreater{} \\ \hline
50+       & I said no I don't have them & yeah                       & really?                                   & my under stairs in the kitchen                                             \\
50+       & that's of course            & no no that's alright       & it's still we we frequently walk that way & in the first place                                                         \\
50+       & oh right                    & what a tragic life         & since this this was new this house?       & thank you very much\\
% 50+       & yeah ah that one in your sewing room hasn't got anything in it                    & ...         & ...       & ...\\
% 50+       & was in the suffragette film and that was very interesting                    & ...         & ...       & ...\\
% 50+       & association and that has to be voted on                    & ...         & ...       & ...\\\bottomrule
\bottomrule
\end{tabular}
}
\caption{Examples
% cherry-picked examples per age group 
where both models are correct/wrong or only BERT$_{FT}$/trigram is correct. Illustrative example: the sentence, \textit{A retrospective exhibition} was uttered by a speaker of age 19-29, but was incorrectly classified by both models (BERT$_{FT}$ and trigram) as coming from a person age 50+.}\label{tab:qualexamples}
\end{table*}

% \begin{figure}[h]
%      \centering
%      \begin{subfigure}[b]{0.45\textwidth}
%         \centering
%         \includegraphics[width=1\columnwidth]{figures/bnc_rb_bert_trigram_acc_freq_topics_w_none.png}
%         \caption{BERT$_{FT}$ and trigram test accuracies per topic for most frequent topics (including none/no info).}
%         \label{fig:bnc_tri_bert_acc_topics_w_none}
%      \end{subfigure}
%      \hfill
%      \begin{subfigure}[b]{0.45\textwidth}
%         \centering
%         \includegraphics[width=1\columnwidth]{figures/bnc_rb_trigram_bert_case_dist_hue_age_ws.png}
%         \caption{Distribution of predicted cases by trigram and BERT$_{FT}$ models for dialogue, split by age groups.}
%         \label{fig:bnc_tri_bert_cases_age}
%      \end{subfigure}
%         \caption{}
%         \label{fig:bnc_blog_age_topic_dist_incl_unk}
% \end{figure}

% \begin{figure}[h]%{0.45\textwidth}
%      \centering
%     %  \begin{subfigure}[b]{0.45\textwidth}
%         \centering
%         \includegraphics[width=0.45\textwidth]{figures/bnc_rb_bert_trigram_acc_freq_topics_w_none.png}
%         \caption{BERT$_{FT}$ and trigram test accuracies per topic for most frequent topics (including none/no info).}
%         \label{fig:bnc_tri_bert_acc_topics_w_none}
% \end{figure}
%      \hfill
% \begin{figure}[h]%{0.45\textwidth}
%     %  \begin{subfigure}[b]{0.45\textwidth}
%         \centering
%         \includegraphics[width=0.45\textwidth]{figures/bnc_rb_trigram_bert_case_dist_hue_age_ws.png}
%         \caption{Distribution of predicted cases by trigram and BERT$_{FT}$ models for dialogue, split by age groups.}
%         \label{fig:bnc_tri_bert_cases_age}
%     %  \end{subfigure}
%         \caption{}
%         \label{fig:bnc_blog_age_topic_dist_incl_unk}
% \end{figure}

\begin{figure}
    \centering
    \begin{minipage}{0.45\textwidth}
        \centering
        \includegraphics[width=0.9\textwidth]{figures/bnc_rb_bert_trigram_acc_freq_topics_w_none.png} % first figure itself
        \caption{BERT$_{FT}$ and trigram test accuracies per topic for most frequent topics (including none/no info).}\label{fig:bnc_tri_bert_acc_topics_w_none}
    \end{minipage}\hfill
    \begin{minipage}{0.45\textwidth}
        \centering
        \includegraphics[width=0.9\textwidth]{figures/bnc_rb_trigram_bert_case_dist_hue_age_ws.png} % second figure itself
        \caption{Distribution of predicted cases by trigram and BERT$_{FT}$ models for dialogue, split by age groups.}\label{fig:bnc_blog_age_topic_dist_incl_unk}
    \end{minipage}
\end{figure}

\section{Age Detection Analyses}
\label{sec:exp1_analyses}
Predominantly with the goal of obtaining insight about age-related distinguishing features in dialogue that can inform our subsequent experiment on age-adaptive dialogue generation (Experiment 2 in Chapter \ref{ch:experiment2}), we focus our analysis on the classification results obtained on the dialogue dataset. In particular, we compare the two best-performing models, namely BERT$_{FT}$
and the one using trigrams, and aim to shed light on  what cues they use to solve the task.
We first analyze how these models perform with respect to utterances of various topics.
Secondly, we compare the prediction patterns of the two models, which allows us to highlight easy and hard examples.
Finally, we focus on the trigram model and report the $n$-grams that turn out to be most informative to distinguish between age groups.


\subsection{Performance Against Topic}

As described in Section~\ref{sec:datadialogue}, each utterance in the dialogue dataset is annotated with one label which is representative of its topic.\footnote{In particular, it represents one of the utterance's topic, i.e., the one most frequently used in the whole data.} This information is not explicitly available to the models.
To explore how the two models deal with utterances in different topical contexts, we compare the accuracy they achieve on the 15 most frequent topics. The results are shown in Figure~\ref{fig:bnc_tri_bert_acc_topics_w_none}. Two main observations can be made: Firstly, some topics turn out to be generally easier/harder than others, i.e., both models achieve higher/lower performance. To illustrate, both models achieve an accuracy well above 70\% on topics like \emph{food}, \emph{holidays} or \emph{university}, while topics such as \emph{tv}, \emph{family} or \emph{no info} appear to be generally more challenging for both models. While this could be due to (a combination of) various factors, one intuitive possibility is that certain topics allow for more discriminative language features, which could be at the level of the lexicon or the style used to talk about them.

Secondly, some topics appear to be easier for one model rather than the other, and \emph{vice versa}. To illustrate, the trigram model outperforms BERT on the topics \emph{weather}, \emph{holidays} and \emph{tv}, while an opposite pattern is observed for \emph{work}, \emph{health}, and \emph{future plans}. We conjecture that these patterns could be indicative of different strategies and cues exploited by various models to make a prediction. 
% We explore this issue more in-depth in the following section, where we compare the predictions by the two models and qualitatively inspect some examples.

\subsection{Comparing Model Predictions}
We split the data for analysis by whether or not both models make the same correct or incorrect prediction, or whether they differ. Table~\ref{tab:bnc_tri_bert_cases} shows the breakdown of these results.
As can be seen, a quite large fraction of samples are correctly classified by both models (63.17\%), while in 19.78\% cases neither of the models make a correct prediction. The remaining cases are comparably split between cases where only one of the two is correct, with BERT slightly outperforming the trigram model by 1.23 percentage points. %, but the other is not.
As shown in Figure~\ref{fig:bnc_blog_age_topic_dist_incl_unk}, the 19-29 age group appears to be be slightly easier compared to the 50+ group, where models are observed to make more errors: the trigram misclassifies 50+ utterances 1.12 times as often as 19-29 utterances, and 1.17 times as often by BERT$_{FT}$.

To qualitatively inspect what the utterances falling into these classes look like, in Table~\ref{tab:qualexamples} we show a few cherry-picked cases for each age group. 
% \raq{
We notice that, not surprisingly, both models have trouble with backchanneling utterances consisting of a single word, such as \emph{yeah}, \emph{mm}, or \emph{really?}, which are used by both age groups.
% }
For example, both models seem to consider \emph{yeah} as a `younger' cue, which leads to wrong predictions when \emph{yeah} is used by a speaker in the 50+ group. As for the utterance \emph{really?}, BERT$_{FT}$ assigns it to the 50+ group, while the trigram model makes the opposite prediction.
% \raq{
This indicates that certain utterances simply do not contain sufficient distinguishing information, and model predictions that are based on them should therefore not be considered reliable.
% should not be considered reliable??

This seems to be particularly the case for short utterances. Indeed, through comparing the average length of the utterances incorrectly classified by both models (rightmost column of Table~\ref{tab:bnc_tri_bert_cases}), we notice that they are much shorter than those belonging to the other cases. This is interesting, and indicates a key challenge in the analysis of dialogue data: 
on average, shorter utterances contain less signal. On the other hand, short utterances can provide rich conversational signal in dialogue; for example, backchanneling, exclamations, or other acknowledging acts. As a consequence, using length alone as a filter is not an appropriate approach, as it can remove aspects of language use key to differentiating speaker groups.


\subsection{Most Informative N-grams}
%\js{todo: analysis on trigrams: both quant and qual because we select them based on their effectiveness (top 10) + perform more qual investigation (e.g. nouns, content words, idiosyncratic expressions; also related to point below) (optional, but very nice) analysis of features wrt topics: are there features that are not dependent on the topic under discussion (i.e., more "linguistic" features)?}

\begin{table}[H]
    \centering
    \begin{tabular}{@{}l l @{\hspace*{25pt}} l l@{}}
    \toprule
    \multicolumn{2}{c}{19-29} & \multicolumn{2}{c}{50+}\\
    \textbf{coef.} & \textbf{n-gram} & \textbf{coef.} & \textbf{n-gram}\\
    \midrule
    -3.20 & um & 2.37 & yes\\
    -2.84 & cool & 2.12 & you know\\
    -2.58 & s**t & 2.09 & wonderful\\
    -2.12 & hmm & 1.90 & how weird\\
    -2.09 & like & 1.84 & chinese\\
    -2.02 & was like & 1.73 & right\\
    -1.96 & love & 1.71 & building\\
    -1.96 & as well & 1.66 & right right\\
    -1.88 & as in & 1.55 & so erm\\
    -1.84 & cute & 1.43 & mm mm\\
    -1.82 & uni & 1.41 & cheers\\
    -1.79 & massive & 1.39 & shed\\
    -1.79 & wanna & 1.37 & pain\\
    -1.79 & f**k & 1.36 & we know\\
    -1.72 & tut & 1.08 & yeah exactly\\
    \bottomrule
    \end{tabular}
    \vspace{3mm}
    \caption{
    % \len{Fix scape between table and caption.} 
    For each age group, top 15 most informative $n$-grams used by the trigram model. \textbf{coef.} is the coefficient (and sign) of the corresponding $n$-gram for the logistic regression model: the higher its absolute value, the higher the utterance's odds to belong to one age group.
    % Greater absolute value of \textbf{coef.} indicates occurrence of the $n$-gram results in the model assigning higher odds to the utterance belonging to a certain age group.
    * indicates masking of foul language.}
    % present in the dialogue.}
    \label{tab:top_ngrams_ws}
\end{table}

% \begin{table}[b!]
%     \centering
%     \begin{tabular}{@{}l l @{\hspace*{25pt}} l l@{}}
%     \toprule
%     \multicolumn{2}{c}{19-29} & \multicolumn{2}{c}{50+}\\
%     \textbf{coef.} & \textbf{n-gram} & \textbf{coef.} & \textbf{n-gram}\\
%     \midrule
%     -3.19 & um & 2.29 & yes\\
%     -2.91 & cool & 2.21 & wonderful\\
%     -2.70 & s**t & 1.91 & building\\
%     -2.25 & cute & 1.86 & right right\\
%     -2.15 & uni & 1.80 & something like\\
%     -2.14 & hmm & 1.73 & garden\\
%     -1.97 & wanna & 1.69 & right\\
%     -1.93 & f**k & 1.68 & ordinary\\
%     -1.91 & like & 1.67 & shed\\
%     -1.85 & massive & 1.63 & operation\\
%     -1.83 & yeah course & 1.58 & born\\
%     -1.81 & love & 1.57 & mother\\
%     -1.79 & tut & 1.55 & photographs\\
%     -1.74 & b***h & 1.51 & email\\
%     -1.68 & like oh & 1.08 & anything like\\
%     \bottomrule
%     \end{tabular}
%     \caption{For each age group, top 15 most informative $n$-grams used by the trigram model. \textbf{coef.} is the coefficient (and sign) of the corresponding $n$-gram for the logistic regression model: the higher its absolute value, the higher the utterance's odds to belong to one age group.
%     % Greater absolute value of \textbf{coef.} indicates occurrence of the $n$-gram results in the model assigning higher odds to the utterance belonging to a certain age group.
%     * indicates masking of foul language.}
%     % present in the dialogue.}
%     \label{tab:top_ngrams}
% \end{table}


Analyzing the most informative $n$-grams 
% \raq{
used by the trigram model
%} 
allows us to qualitatively compare the linguistic differences inherent to each age group. In Table~\ref{tab:top_ngrams_ws} we report the top 15 $n$-grams per group.
% within the BNC. 
% \raq{Table XX shows the top 15 n-grams per group.} 
We find, firstly and intuitively, that colloquial language seems somewhat generational, with unigrams particularly indicative of younger speakers consisting of words such as 
\emph{cool} and \emph{massive}, and
% and foul language such as \emph{s**t}, \emph{f**k} and \emph{b***h}, and
% \textit{`awesome' and `col' `massive' `lol' `literally' `mate'}, 
for older speakers, words like
\emph{wonderful}.
% and  \emph{ordinary}.
% \textit{`cheers' `gosh' `right' `wonderful'} 
% \san{some of these examples are not in the table} 
These unigrams are both informative to the model and indicative of differences in both formality and `slang' use across age groups.

These most informative $n$-grams also indicate differences in back-channeling use between age groups; younger speaker's language is more characterized by the use of \emph{hmm}, \emph{um}, \emph{yeah course}, while the top $n$-grams in the older category will more likely use
\emph{yes}, \emph{right}, \emph{right right}.
% `sure sure' and `right right'. 
% `mhm' `yeah course', while the top n-grams in the older category will more likely use `sure sure' and `right right'.
A feature of younger language also apparent from these examples is in their use of more informal language: \emph{yeah course} rather than \emph{yes}.
% `yeah course' rather than `yes'. 
This informal language use also extends to the use of foul language, which make up a percent of the most informative unigrams shown in Table~\ref{tab:top_ngrams_ws}. % (see, e.g., \emph{s**t}, \emph{f**k} and \emph{b***h}).

Interestingly, while topic words make up many of the most informative $n$-grams for older speakers in Table~\ref{tab:top_ngrams_ws}, younger speakers are more defined by their use of slang words such as \textit{wanna}, foul language, or adjectives such as \textit{cute}, \emph{cool}, and \emph{massive}.
% love}. 
A key finding from~\citet{schler2006effects} is in the sentiment of language playing an important role, something which some of the most informative $n$-grams suggest may also be true for the dialogue dataset. As Table~\ref{tab:top_ngrams_ws} demonstrates, younger speakers use more dramatic language % in the younger speakers 
such as negative foul words, and positive \textit{love, cute, cool}; all words with a strong connotative meaning. 
%\textit{hate, really good, oh gosh, excited, scared}, whereas older speakers are comparatively more moderate. \san{Not sure we have examples of dramatic language in the table anymore; in any case, this paragraph needs to be updated} 
This prompts us to hypothesize that further inspection is needed to determine whether the same sentiment pattern will be true of dialogue as it has been reported to be in discourse.

\begin{table}[H]
    \centering
    \begin{tabular}{@{}l  c  c @{}}
    \toprule
    \textbf{} & \textbf{\% cases} & \textbf{avg.\ length ($\pm$std)*}\\
    \midrule
    both correct & 63.17\% & 13.51 ($\pm$18.98) \\
    both wrong & 19.78\% & 5.82 ($\pm$8.33) \\
    only Trigram correct & 7.91 \% & 10.44 ($\pm$11.66) \\
    only BERT correct & 9.14 \% & 11.53 ($\pm$12.12) \\ \bottomrule
    \end{tabular}
    \caption{Percentage (\% cases) of (non-)overlapping (in)correctly predicted cases between trigram and BERT$_{FT}$. *Utterance length measured in tokens.}
    % \caption{Percentages of (non-)overlapping (in)correctly predicted cases between trigram and BERT$_{FT}$ models for BNC. *Pre-processed sequence length is measured in tokens.}
    \label{tab:bnc_tri_bert_cases}
\end{table}

% \begin{table}[H]
%     \centering
%     \begin{tabular}{@{}l  c  c @{}}
%     \toprule
%     \textbf{} & \textbf{\% cases} & \textbf{avg.\ length ($\pm$std)*}\\
%     \midrule
%     both correct & 63.57\% & 7.33 ($\pm$10.10) \\
%     both wrong & 22.89\% & 3.77 ($\pm$4.53) \\
%     only Trigram correct & 6.68 \% & 6.37 ($\pm$6.16) \\
%     only BERT correct & 6.86 \% & 6.86 ($\pm$7.87) \\ \bottomrule
%     \end{tabular}
%     \caption{Percentage (\% cases) of (non-)overlapping (in)correctly predicted cases between trigram and BERT$_{FT}$. *Utterance length measured in tokens.}
%     % \caption{Percentages of (non-)overlapping (in)correctly predicted cases between trigram and BERT$_{FT}$ models for BNC. *Pre-processed sequence length is measured in tokens.}
%     \label{tab:bnc_tri_bert_cases_no_stopwords}
% \end{table}

% Notes on the imbalanced BNC.
% \begin{itemize}
%     \item Attempts were made to account for the original BNC's bias (i.e., the 19-29 age bracket accounts for roughly 80\% of the total considered subset).
%     \item Method 1: weighted loss.
%     \item Method 2: weighted random sampling (i.e., up-sampling of the minority class).
%     \item Weighted random sampling outperformed weighted loss in terms of validation accuracy and $F_1$ scores, but still failed to surpass the baseline.
%     \item  \textit{In terms of test accuracy}, the $n$-gram models succeeded in beating the baseline (predicting the majority class), whereas the best LSTM and fine-tuned BERT-based failed to do so.
%     \item However, the neural discriminators still outperformed all the other models with respect to minority class $F_1$ scores, indicating that (1) the $n$-gram models aren't very useful for correctly classifying the minority class, and that (2) weighted random sampling improved the models' efficacy with respect to the minority class.
%     \item See Appendix \ref{age_disc_bnc} for these results.
% \end{itemize}




\section{Recap, Discussion \& Conclusion}

\textbf{Keep in mind when writing this section}
\begin{itemize}
    \item Goals of this section: a discussion/conclusion which also paves the way to and motivates Experiment 2.
    \item Recap the results and what we have learned from the analysis.
    \item (Based on the results and analyses) Make hypotheses that are relevant for the following experiment
    \item However, don't repeat (too much) what is already stated in the next section (i.e., the introduction to Experiment 2)
    \item Do I need a discussion here, given that there is a general discussion (i.e., for both experiments) at the end of the thesis?
\end{itemize}

Useful phrases from age detection paper (EMNLP submission)

\begin{itemize}
    \item We investigated whether, and to what extent, NLP models can detect age-related linguistic features in dialogue data.
    \item We showed that, in line with what we observed for discourse, state-of-the-art models are capable of doing so with a reasonable accuracy, in particular when the dialogue fragment is long enough to contain discriminative signal.
    \item At the same time, differently from discourse, we found that much simpler models based on n-grams achieve comparable performance, which suggests that, in dialogue, ‘local’ features can be indicative of the language of speakers from different age groups.
    \item We showed this to be the case, with both lexical and stylistic cues being informative to these (and possibly all) models in performing the task.
\end{itemize}

\textbf{How will these results inform the controlled generation experiments?}

\begin{itemize}
    \item Do I need this? It's literally one of the first things I talk about on the next page. (maybe have a call with sandro about it)
\end{itemize}

\chapter{Experiment 2: Age-Adaptive Dialogue Generation Using PPLM}\label{ch:experiment2}

% \section{Introduction}

% \len{TODO - Emphasize the connection between Experiment 1 and Experiment 2: \textit{I’ve shown that it’s possible to classify younger vs older age groups based on language; I now want to check whether it’s possible to generate language than encompasses these features (2). I’ll use state-of-art models for language generation (1)….}}

% \len{Sandro's advice for this introduction and chapter: \textit{I think something important to stress is the connection at the dialogue-level: I’ve used BNC to train the classifier, which is a dialogue dataset. Now, in generating, I think it’s important that I generate something similar to a dialogue turn, i.e., a response to a dialogue “prompt”.}}

% \len{TODO - Motivate the choice of GPT for generation, and BERT for classification}

% \len{
% \begin{itemize}
%     \item Here make the argument for using the best BERT-ft classifier from Exp1
%     \item BERT more suitable for classification.
%     \item Exp1 uses fully finetuned BERT (ca 110M parameters). For GPT2 (ca 345M parameters) this is infeasible with my computational resources.
%     \item Using a separate model class (i.e., BERT) for evaluation of GPT-based generation models makes the results more generalizable.
% \end{itemize}
% }

\section{Introduction}\label{sec:exp2_intro}

\subsection{Recap of Previous Chapter and Connection to Experiment 1}

% In Experiment 1 (Chapter \ref{ch:experiment1}), I showed that it is possible to classify language by speakers and writers from different age groups based on linguistic features.
In Experiment 1, I studied the extent to which text-based NLP models are able to detect speaker and writer age-related linguistic features in dialogue and discourse data. Then, I investigated which features drive the predictions made by those models. It is shown that a fine-tuned version of BERT, BERT$_{FT}$, is capable of classifying language by speakers from different age groups based on linguistic features. It is also found that much simpler models based on $n$-grams achieve classification performance comparable to that of BERT$_{FT}$, suggesting that, in dialogue, local features (i.e., at the lexical level) can also be indicative of a speaker's age. This was shown to be the case, as both lexical and stylistic cues seem to be informative to these models when predicting speaker age from dialogue utterances.

Now, in Experiment 2, I aim to test whether it is possible to generate dialogue responses that possess these age-indicative features identified in Experiment 1. 
% I also use the insights gained from Experiment 1 (e.g., about the applicability of $n$-gram-based representations of age-specific speaking style) to inform me about the development of controlled dialogue generation models. 
More specifically, I seek to use Plug-and-Play language models (PPLM) \citep{dathathri2019plug} to develop controlled dialogue generation models that can produce conversation responses that contain linguistic elements of the speaking style of a certain age group to the extent that it would be classified as such by Experiment 1's best classifier. The use of the PPLM method is motivated by its capability to control the style of the text generated by a large pre-trained Transformer-based language model (e.g., GPT-2 or DialoGPT), without the computationally expensive requirement to re-train or fine-tune it. Namely, the Plug-and-Play approach uses substantially smaller and less costly to train attribute models to make perturbations to the large language model's activation space, thereby shifting its output distribution towards a desired style. It is important to realize that the PPLM method is fundamentally different from fine-tuning in that it leaves the parameters of the underlying language model unchanged.

The analyses of Experiment 1 largely concerned a comparison of performance between simpler $n$-gram-based discriminators and more sophisticated neural ones (i.e., BERT). This comparison is continued throughout Experiment 2, because now I aim to compare the applicability of PPLM for age-adaptation when adaptation is enforced by an aforementioned attribute model, which can be a simple bag-of-words (BoW) model, or a more complex neural discriminator. Throughout this thesis, using PPLM to control the writing style of generated output text using the former is referred to as \textit{BoW-based control}, and using the latter is referred to as \textit{discriminator-based control}. The insights gained from Experiment 1 (e.g., about the applicability of $n$-gram-based representations of age-specific speaking style) are also used to inform decisions about the development of attribute models. Furthermore, in Experiment 1, the classifiers are trained on the spoken component of the BNC, which is a dialogue dataset. Now, in Experiment 2, I aim to generate something similar to a dialogue turn, i.e., a response to a dialogue prompt, in the style of a certain age group. Table \ref{tab:dialogue_gen_example_exp2} shows examples of how age-adapted dialogue responses to a prompt might look.

\begin{table}[h]%{0.5\linewidth}
    \centering
    {\begin{tabular}{r l}
    \hline
        \ \cellcolor{yellow!25}\textbf{\textit{PROMPT}} & \multicolumn{1}{>{\columncolor{yellow!10}}l}{\tabularCenterstack{l}{Can we talk?}}\\
        \hline 
        \cellcolor{red!25}\textbf{\textit{Age 19-29 style response}} & \multicolumn{1}{>{\columncolor{red!10}}l}{\tabularCenterstack{l}{Like about what?}}\\
        \ \cellcolor{blue!25}\textbf{\textit{Age 50+ style response}} & \multicolumn{1}{>{\columncolor{blue!10}}l}{\tabularCenterstack{l}{What would you like to share about?}}\\
        % \ \cellcolor{red!25}\textbf{\textit{DialoGPT$_{Young}$}} & \multicolumn{1}{>{\columncolor{red!10}}l}{\tabularCenterstack{l}{...}}\\
        % \ \cellcolor{blue!25}\textbf{\textit{DialoGPT$_{Old}$}} & \multicolumn{1}{>{\columncolor{blue!10}}l}{\tabularCenterstack{l}{...}}\\
        % \hline
        % \ \cellcolor{yellow!25}\textbf{\textit{PROMPT}} & \multicolumn{1}{>{\columncolor{yellow!10}}l}{\tabularCenterstack{l}{Hi, how's it going?}}\\
        % \hline 
        % \cellcolor{red!25}\textbf{\textit{GPT-2$_{Young}$}} & \multicolumn{1}{>{\columncolor{red!10}}l}{\tabularCenterstack{l}{...}}\\
        % \ \cellcolor{blue!25}\textbf{\textit{GPT-2$_{Old}$}} & \multicolumn{1}{>{\columncolor{blue!10}}l}{\tabularCenterstack{l}{ I've had my first \\ surgery recently.}}\\
        % \ \cellcolor{red!25}\textbf{\textit{DialoGPT$_{Young}$}} & \multicolumn{1}{>{\columncolor{red!10}}l}{\tabularCenterstack{l}{...}}\\
        % \ \cellcolor{blue!25}\textbf{\textit{DialoGPT$_{Old}$}} & \multicolumn{1}{>{\columncolor{blue!10}}l}{\tabularCenterstack{l}{...}}\\
    \hline
    \end{tabular}}
    \vspace{3mm}
    \caption{Examples of controlled dialogue responses to a prompt (in this case, a dialogue turn). The responses have been generated by DialoGPT after being controlled to generate younger sounding (red row) and older sounding (blue row) language.}\label{tab:dialogue_gen_example_exp2} % examples generated by NP | DGPT | Discrim | Young and NP | DGPT | Discrim | Old
\end{table}%

It is also important to realize that the previously seen classifiers from Experiment 1 are not compatible with PPLM, because the method requires a bag-of-words or linear discriminator. Furthermore, one of the goals of Experiment 1 was to develop the best-performing age-classifiers, which are ultimately used for (1) evaluation of the attribute adherence of the generated responses in Experiment 2, and (2) informing decisions about the choice of BoW-based attribute models by using a BoW comprised of the unigram classifier's most informative unigrams (see Section \ref{subsec:att_model_dev} for further details). It also improves the generalizability of the results to use a separate classifier for evaluation than the discriminators used as attribute models.

Lists of unigrams are used as BoWs (as opposed to lists of trigrams from the best-performing $n$-gram-based classifier in Table \ref{tab:bnc_classification_ws}), because the PPLM setup is not compatible with lists of $n$-grams for $n > 1$, as it relies on perturbations at the unigram-level. Making a PPLM-system compatible with, e.g., trigrams, would amount to re-training the entire underlying language model (like GPT-2), thereby defeating the purpose of PPLM, i.e., leveraging large LMs for controllability, without incurring significant re-training costs. However by allowing the best-performing classifiers to inform decisions about generation, the best unigram-based classifier's list of most informative features is used as an attribute model.

% I use state-of-art models, GPT-2 \citep{radford2019language} and DialoGPT \citep{zhang2019dialogpt}, for (controlled) language generation. 
% The deliberate choice to use BERT-based models for classification (and evaluation of generated output), and GPT-based models for generation is motivated by the following reasons. BERT's encoder-based bidirectional architecture makes it more suitable for sequence classification than for generation \citep{devlin-etal-2019-bert}. By similar reasoning, GPT's decoder-based structure makes it a more suitable choice for generation tasks. Furthermore, Experiment 1's best classifier is a fully fine-tuned BERT model (ca 110M parameters). However, it is computationally highly expensive to fine-tune GPT-2-medium (ca 345M parameters) which makes it infeasible given my computational resources. Finally, using a separate model class (i.e., BERT) for evaluation of GPT-based generation models makes the results more generalizable.

% This chapter covers the methodology, experimental setup, results, and analyses relating to Experiment 2 of this thesis. 
\subsection{Research Objectives, Hypotheses, and Contributions}

Experiment 2 concerns a Plug-and-Play approach to age-adaptive dialogue generation.
The aim is to use PPLM for age-adaptive dialogue generation.
Based on the previously demonstrated detectability of age-related linguistic patterns in dialogue by text-based NLP models, I hypothesize age-adaptive dialogue generation to be possible to the extent that similar patterns can be discerned from generated dialogue responses.
% I hypothesize this to be possible to the extent that a text-based NLP model can detect from the generated dialogue responses linguistic features learned to be associated with certain age groups. 
Furthermore, based on previous work on modeling age-related characteristics in language \citep{pennebaker2003words, schler2006effects, nguyen2014gender, zheng2019personalized}, I presume age-related linguistic style to be a more abstract attribute than the sentiment and topic attributes previously studied by \cite{dathathri2019plug} and \cite{madotto-etal-2020-plug}. This implies that it is probably more challenging to accurately express this attribute as a BoW, than it is to represent it in the latent space of a neural discriminator. \cite{dathathri2019plug} suggest that for such attributes that are difficult to accurately express as wordlists, more sophisticated attribute models (i.e., discriminators) are desirable (see Section \ref{sec:ppm}). It is therefore expected of discriminator-based control to be more detectable than BoW-based control, because a discriminator attribute model is probably able to capture a more comprehensive representation of age-related speaking style, and thus more accurately enforce the linguistic style during generation. However, I also expect this heightened degree of control exerted by discriminator attribute models, combined with the more lexical level control associated with the BoW attribute models (see Equation \ref{eq:bow_att_model_loglikelihood}), to result in BoW-based control taking a smaller toll on the fluency of generated responses. 

% Furthermore, based on the 
% I expect discriminator-based control to be more detectable and invasive at the structural level than BoW-based control. 
% I also expect BoW-based control to take a smaller toll on the fluency of generated responses than discriminator-based control.


% What do you want to say:
% \begin{itemize}
%     \item Age group specific speaking style is an abstract attribute.
%     \item It is probably more challenging to accurately express this attribute as a list of words, than it is to represent it in the latent space of a neural discriminator.
%     \item \cite{dathathri2019plug} suggest that for such attributes that are difficult to accurately express as wordlists, more sophisticated attribute models (i.e., discriminators) are desirable.
%     \item I therefore expect discriminator-based control to be more detectable (?) than BoW-based control, as a discriminator attribute model is probably able to capture a more comprehensive representation of age-related speaking style, and thus more accurately enforce the linguistic style during generation.
%     \item Furthermore, I expect BoW-based control to take a smaller toll on the fluency of generated responses because the formulas say so...
% \end{itemize}

% \len{Explain these hypotheses about BoW-based and discrim-based control by basing them on equations (4) and (5) of \cite{dathathri2019plug} and relating these equations to the presumed impact they will have on fluency. Furthermore, it is also worth mentioning that an attribute model that is more sophisticated than a BoW (i.e., discriminator) is especially useful when expressing the attribute is difficult using a wordlist, and that this is most likely the case for an abstract attribute like age.}
% \len{Wait for their feedback about the paragraph above, but they're probably going to ask you "based on what?" for every hypothesis. Make a list of work and observations you based you hypotheses on.}

% In other words, I seek to use large pre-trained language models for controllable dialogue generation, using activation-space perturbations instead of fine-tuning. Either GPT-2-medium or DialoGPT-medium is used as a base language model. And adaptation of the generated language to a certain age group is achieved using either a linear discriminator or a bag-of-words (BoW), trained or empirically constructed from the dialogue dataset (See Section \ref{subsec:dialogue_dataset}). Using GPT-2 and DialoGPT as baselines, a set of automated and human evaluation metrics are used to evaluate the fluency and control of the proposed models. 
% I expect discriminator-based control to be more detectable and fine-grained than BoW-based control. I also expect BoW-based control to take a smaller toll on fluency.

The Plug-and-Play approach has previously been used by \cite{dathathri2019plug} and \cite{madotto-etal-2020-plug} for language generation, controlled for very concrete writing styles and topics (e.g., negative/positive sentiment, politics, religion, business, tech). I apply the PPLM method to a novel problem, that is, age-related language generation conditioned on dialogue data. More concretely, I extend the work of \cite{dathathri2019plug} in several important ways: (1) I control the generated language for more abstract writing styles, i.e., age-related linguistic style; (2) I use PPLMs to generate dialogue responses; (3) I propose two empirical methods for age-specific BoW development, as opposed to the manually constructed BoWs used in previous work; (4) I carry out in-depth analyses about the relationships between evaluation measures of dialogue response quality and style; (5) finally, I address and study the problem of writing style biases induced by the prompt's style.

% \len{EXAPND ON THIS. However, a fully interactive plug-and-play conversational model is out of the scope of this thesis.}

% \begin{itemize}
%     \item However, a fully interactive plug-and-play conversational model is out of the scope of this thesis.
%     \item PPCM is about interactivity. Implementing a PPCM is beyond the scope because of the engineering requirements: generating massive amounts of dialogue datasets, developing and training residual adapters.
%     \item This thesis is about (1) establishing whether age-related linguistic features are automatically detectable, and (2) if I can enforce them as a writing style of PPLM-generated dialogue responses. Interactivity (i.e., speed of response generation) is a separate engineering problem and beyond the scope.
% \end{itemize}

It must be noted that the activation perturbation steps necessary for PPLMs to exert influence on the style of generated text makes the PPLM-approach too slow to be suitable for online interactive conversational applications. This problem is addressed by \cite{madotto-etal-2020-plug}, who propose the Plug-and-Play Conversational Model (PPCM), an extension of PPLM that uses separately trained residual adapter modules to speed up the PPLM-method, making it suitable for real-time dialogue response generation. Despite their work also being about controlled dialogue generation, this work differs in important ways. They control their output for the same styles and topics as \cite{dathathri2019plug}, and do not experiment with more abstract writing styles, like age-related linguistic characteristics. Overall, this thesis is about establishing whether age-related linguistic features are automatically detectable, and if they can be enforced as a writing style of dialogue responses using PPLMs. The work of \cite{madotto-etal-2020-plug} addresses the problem of interactivity (i.e., speed of response generation), which requires solving a separate, very valuable, engineering problem, which is beyond the scope of this thesis. Nevertheless, the previously mentioned contributions still hold.

The rest of this chapter is structured as follows. Section \ref{sec:exp2_methods} describes the methodology and experimental setup of the controlled dialogue generation experiments. It also covers details about attribute model development, the choice of prompts, and an explanation of the chosen evaluation metrics. The results of the language generation experiments are presented and interpreted in Section \ref{sec:exp2_results}, followed by the outcomes of various quantitative and qualitative analyses in Section \ref{sec:exp2_analyses}. A separate section about data is omitted in this chapter, because the previously mentioned dialogue dataset is used for these experiments. The reader is directed to Section \ref{subsec:dialogue_dataset} for a detailed description of that corpus, and the pre-processing steps that have been taken.

% \len{TODO - Does it suffice to redirect the reader to the previous sub-section about the dialogue dataset? Or am I missing an explanation about the dialogue corpus in this chapter?}



% \section{Data}

\len{Same as Classification data section, but without blog data. What else do I need to say here?}

\section{Methods for Controlled Language Generation using Plug-and-Play Language Models}
\label{sec:exp2_methods}

\textit{Plug-and-play language generation entails using a attribute model to make activation-space perturbations on the output of a pre-trained language model. I explain the important architectures involved, and the plug-and-play method in the following sub-sections. Details about how the generation experiments are setup and evaluated are given at the end of the section.}

% \len{TODO - Do I need this short intro?}

\subsection{Transformers}

The Transformer architecture plays a central role in most of the recent advances in NLP. The same holds for the methods used in this thesis to investigate controlled dialogue generation and speaker/author age detection. A brief explanation about the Transformer therefore in order. For a more detailed review of the model architecture, the reader is referred to the original paper (\citep{vaswani2017attention}) or this excellent blog post: \url{https://jalammar.github.io/illustrated-transformer/}.

The Transformer, like most neural sequence processing models, has an encoder-decoder structure. On a high level, the encoder receives an input sequence $\textbf{x} = (x_1, ..., x_n)$ (e.g., a sentence), and maps this to a sequence of latent continuous variables $\textbf{z} = (z_1, ..., z_n)$. The decoder then takes $\textbf{z}$ as input, and maps this to an output sequence $\textbf{y} = (y_1, ..., y_m)$. Note that the use of positional encodings of the input and output embeddings enables the Transformer to process and generate sequences in arbitrary order, allowing for a high degree of parallelization. The generation of $\textbf{y}$ happens element-by-element in an auto-regressive fashion, where at step $t$, element $y_{t - 1}$ is also taken as input.

Both the encoder and decoder are comprised of $N$ identical layers (denoted by the `N $\times$' in the left part of Figure \ref{fig:transformer_architecture}). Every sub-layer performs a succession of transformations using multi-head self-attention mechanisms and point-wise, fully connected layers, along with residual connections \citep{he2016residual} around every sub-layer followed by layer normalization \citep{DBLP:journals/corr/BaKH16}. The decoder's first self-attention sub-layer is masked to ensure that the output predictions at sequence position $i$ cannot depend on output positions greater than $i$. Finally, the decoder passes its output through a linear and softmax layer to produce a probability distribution over the problem space (e.g., the vocabulary) from which the most likely symbols for the generated output sequence $\textbf{y}$ can be sampled.

A key aspect of the Transformer architecture is its use of attention \citep{DBLP:journals/corr/BahdanauCB14}. This allows the encoder-decoder architecture to selectively focus on parts of the input sequence to produce a more informative hidden representation. \citeauthor{vaswani2017attention} formulate an attention function as a mapping of queries and sets of key-value pairs to an attention output, where matrices represent the queries $Q$, keys $K$, and values $V$. The attention output is a weighted sum of the values, based on the relevance of the corresponding keys to a query. In particular, they employ scaled dot-product attention:

\begin{equation}
    \texttt{Attention}(Q, K, V) = \texttt{softmax} \left( \frac{QK^T}{\sqrt{d_k}}\right) V.
\end{equation}

Furthermore, \cite{vaswani2017attention} propose to use multi-head attention by using learned linear projections to project the queries, keys and values $h$ times, and apply the aforementioned attention function to these projections in parallel. The concatenation of these attention outputs, passed through a linear layer, ultimately produces the final output of the Transformer's attention sub-layers. This allows the model to attend to the relevant information from all representation sub-spaces at various sequence positions. See Figure \ref{fig:transformer_architecture} for an schematic illustration of the Transformer's structure described above.


\begin{figure}[H]
    \centering
    \includegraphics[width=\textwidth]{figures/transformer_lillog.png}
    \caption{An overview of the full Transformer model architecture. \textit{Collated image source:} Fig. 17 in this blog post \url{https://lilianweng.github.io/lil-log/2018/06/24/attention-attention.html}. \textit{Original image source:} Figures 1 and 2 in \cite{vaswani2017attention}}
    \label{fig:transformer_architecture}
\end{figure}

\subsection{Causal Language Modeling with Transformers}

Following the conventions of \cite{dathathri2019plug} and \cite{madotto-etal-2020-plug}, a dialogue is comprised of multiple alternating turns (sometimes referred to as utterances) between more than one speaker. For simplicity, this project only focuses on dialogues between two speakers. The conversation history at turn $t$ is defined as $\mathcal{D}_t = \{S^{(1)}_1, S^{(2)}_1, ..., S^{(1)}_t\}$, where $S^{(j)}_t$ is speaker $j$'s utterance at time $t$. \cite{madotto-etal-2020-plug} denote speaker $1$ as the user $U$, and speaker $2$ as the conversational system $S$, yielding dialogue history $\mathcal{D}_t = \{U_1, S_1, ..., U_t\}$. This notational convention will also be used for the user-system experiments later on in this report.

A Transformer-based language model (denoted $\texttt{LM}$) is used in this thesis to model the distribution of dialogues, using dialogue history at time $t$, $\mathcal{D}_t$, as a prompt to auto-regressively generate the dialogue continuation $S_t$. More specifically, let the concatenation of the dialogue history at $t$ and its continuation, $\{\mathcal{D}_t, S_t\}$, be represented as a sequence of tokens $\textbf{x}= \{x_0, ..., x_n\}$. Then, by recursively applying the product rule of probability (\cite{bishop2006pattern}), the unconditional probability of the sequence $p(\textbf{x})$ can be expressed as:

\begin{equation}
    p(\textbf{x}) = \prod_{i = 1}^n p(x_i | x_0, ..., x_{i - 1}).
\end{equation}

\cite{dathathri2019plug} and \cite{madotto-etal-2020-plug} define the Transformer's decoding process in a recursive fashion. Let $H_t$ denote the conversation history's key-value pairs, i.e., $H_t = \left[ (K_t^{(1)}, V_t^{(1)}), ..., (K_t^{(l)}, V_t^{(l)}) \right]$, with $(K_t^{(i)}, V_t^{(i)})$ representing the key-value pairs from the $\texttt{LM}$'s $i$-th layer generated at all time steps $0$ through $t$. This results in the recurrent dedocing process being expressed as:

\begin{equation}
    o_{t + 1}, H_{t + 1} = \texttt{LM} \left( x_t, H_t \right),
\end{equation}

where $o_{t + 1}$ is the hidden state of the last layer. Finally, after applying a softmax transformation, the next token $x_{t + 1}$ is sampled from the resulting probability distribution, i.e.,  $x_{t + 1} \sim p_{t + 1} = \texttt{softmax} \left( W o_{t + 1} \right)$, where $W$ is a linear mapping from the model's last hidden state to a vector of vocabulary size. This recursive formulation allows for efficient text generation by leveraging cached memories, without repeated forward passes.


\subsection{Conversational Response Generation}

Conversational response generation can be modeled in similar ways to open-domain text generation. \cite{zeng-etal-2020-meddialog} suggest to either formulate it in terms of source-target pairs, much like neural machine translation, or as a language modeling objective, where the next token or utterance is conditioned on the dialogue history. 
% \len{TODO - Do I need the following sentence?}
% To remain close to the training objectives of my baseline models (GPT-2 \citep{radford2019language} and DialoGPT \citep{zhang2019dialogpt}) I choose to adopt the language modeling formulation for conversation modeling. 
More formally, concatenate all dialogue turns in a multi-turn dialogue session into a long text: $x_1, ..., x_N$. Denote the source sentence or dialogue history as $S = x_1, ..., x_m$ and the target sentence (ground truth response) as $T = x_{m + 1}, ..., x_N$. The conditional probability of dialogue continuation given its history $P(T | S)$ can be written as

\begin{equation}
    p(T | S) = \prod_{n = m + 1}^N p(x_n | x_1, ..., x_{n - 1}).
\end{equation}

A multi-turn dialogue session $T_1, ..., T_K$ can be written as $p(T_K, ..., T_2 | T_1)$ which is essentially the product of all source-target pairs probabilities $p(T_i | T_1, ..., T_{i - 1})$. This formulation also shows that optimising the single objective $p(T_K, ..., T_2 | T_1)$ is equivalent to optimising all source-target pair probabilities.


\subsection{Plug-and-Play Modeling}
\label{sec:ppm}

Plug-and-play language model (PPLM) \cite{dathathri2019plug} works by using a text classifier, referred to as an attribute model, to control the text generated by a language model. Let $p(X)$ denote the distribution of a Transformer-based language model (e.g., GPT-2 or DialoGPT), where $X$ represents the generated text. And $p(a | X)$ denotes the attribute model (e.g., a single-layer or BoW classifier) that represents the degree of adherence of text $X$ to a certain attribute $a$ (e.g., style, sentiment, or age-group characteristics). Then PPLM can be seen as modeling the conditional distribution of generated text $X$ given attribute $a$, i.e., $p(X | a)$. Note that Bayes' theorem ties these three definitions together as follows:

\begin{equation}
    p(X | a) \overbrace{=}^{\text{Bayes' theorem}} 
    % \frac{p(a, X)}{p(a)} = 
    \frac{p(X) p(a | X)}{p(a)} \propto
    p(X)p(a | X).
\end{equation}

% \len{TODOs -
% \begin{itemize}
%     \item Maybe this isn't the best spot, but somewhere in the methodology motivate the choice of using a unigram wordlist as BoW, and a linear classifier trained with a transformer-based architecture.
%     \item Also emphasize the continuing narrative between Experiment 1 and Experiment 2. Namely, how the best performing systems in Exp1 are used to inform choices during Exp2 (trigram and BERT best classifiers --> use unigram and linear layer used for PPLM).
%     \item Also explain how the PPLM setup is not compatible with lists of $n$-grams (for $n>1$), as it makes perturbations at the unigram level. And adapting the system to fit trigram BoW's implies retraining the entire underlying language model (like GPT-2) for trigrams, defeating the purpose of PPLM (leveraging large scale language models for controllability, without having to finetune the massive architectures).
%     \item Then go on to argue that the unigram classifier is on par with trigram, making it a viable choice for the 100MIU BoW.
%     \item Also mention that your work extends the original BoW-PPLM by using various empirically generated BoWs, instead of curated wordlists. Making it more reproducible.
%     \item Double-check if you should make this last argument in the methodology, and not in related work or experimental details.
% \end{itemize}}

To control the generated text, PPLM shifts the aforementioned history $H_t$ (i.e., all Transformer key-value pairs generated up to time $t$) in the direction of the sum of two gradients:

\begin{enumerate}
    \item Ascending $\nabla \log p(a | X)$: maximizing the log-likelihood of the desired attribute $a$ under the conditional attribute model. This enforces attribute control.
    \item Ascending $\nabla \log p(X)$: maximizing the log-likelihood of the generated language under the original (possibly conversational) language model. This promotes fluency of the generated text.
\end{enumerate}

These two incentive-representing gradients are combined with various coefficients, yielding a set of tunable knobs to steer the generated text in the direction of the desired fluency, attribute control, and length.

Let's first focus on the first of the two gradients, i.e., the attribute control promoting $\nabla \log p(a | X)$. $\Delta H_t$ represents the update to history $H_t$ that pushes the distribution of the generated text $X$ in the direction that has a higher likelihood of adhering to desired attribute $a$. The gradient update rule can be expressed as:

\begin{equation}
    \Delta H_t \leftarrow \Delta H_t + \alpha
    \frac{\nabla_{\Delta H_t} \log p(a | H_t + \Delta H_t)}
    {\norm{\nabla_{\Delta H_t} \log p(a | H_t + \Delta H_t)}^{\gamma}}
\label{eq:H_update_rule}
\end{equation}

where $\alpha$ is the step size, and $\gamma$ denotes the normalization term's scaling coefficient. Both step size ($\alpha$) and the scaling coefficient ($\gamma$) influence attribute control. Attribute control can be softened by either decreasing $\alpha$ or increasing $\gamma$ and vice versa. Note that $\alpha = 0$ recovers the original uncontrolled underlying language model (e.g., GPT-2 or DialoGPT). In practice, $\Delta H_t$ is initialized at zero, and the update rule in Equation \ref{eq:H_update_rule} is applied $m$ times (usually 3 to 10), resulting in the updated key-value pair history $\tilde{H}_t  = H_t + \Delta H_t$. Then the updated history $\tilde{H}_t$ is passed through the language model, yielding the updated logits (final Transformer-layer): $\tilde{o}_{t + 1}, H_t = \texttt{LM}(x_t, \tilde{H}_t)$. And finally the shifted $\tilde{o}_{t + 1}$ is linearly mapped through a softmax layer to produce a new, more attribute-adherent, distribution from which to sample, i.e., $x_{t + 1} \sim \tilde{p}_{t + 1} = \texttt{softmax} \left( W \tilde{o}_{t + 1} \right)$.

The method described until now will generate attribute-adherent text, but will likely yield fooling examples \citep{nguyen2015deep} that are gibberish to humans, but get assigned high $p(a | x)$ by the attribute model \citep{dathathri2019plug}. That is why \cite{dathathri2019plug} apply two methods to ensure fluency of the generate text. The first is to update $\Delta H_t$ such to minimize the Kullback-Leibler (KL) divergence (denoted $D_{KL})$ between the shifted and original distributions. In practice, $D_{KL}$ is scaled by a coefficient $\lambda_{KL}$, typically found to work well for most tasks when set to 0.01. Repetitive text generation (i.e., high $p(a | x)$ but low $p(x)$) can therefore sometimes be avoided by increasing $\lambda_{KL}$. The second method to ensure fluency is Post-norm Geometric Mean Fusion \citep{stahlberg-etal-2018-simple} which, instead of directly influencing $\Delta H_t$ like minimizing $D_{KL}$, fuses the altered generative distribution $\tilde{p}_{t + 1}$ with the unconditional language distribution $p(x)$. This is done during generation by sampling the next token as follows:

\begin{equation}
    x_{t + 1} \sim \frac{1}{\beta}
    \left( 
    \tilde{p}_{t + 1}^{\gamma_{gm}} p_{t + 1}^{1 - \gamma_{gm}}
    \right)
    \label{eq:gm_fusion}
\end{equation}

where $\beta$ is a normalization constant, $p_{t + 1}$ and $\tilde{p}_{t + 1}$ denote the original and modified distributions, respectively, and $\gamma_{gn}$ is a scaling term that interpolates between the two distributions. Because the new sampling distribution in Equation \ref{eq:gm_fusion} converges towards the unconditional language model as $\gamma_{gm} \rightarrow 0$, repetitive text generation can be avoided by decreasing the scaling term.


\begin{figure}[H]
    \centering
    \includegraphics[width=\textwidth]{figures/pplm_fig1.png}
    \caption{A schematic overview of the plug-and-play interaction between attribute model $p(a | \textbf{x})$ and language model $p(\textbf{x})$. \textit{Original image source:} Figure 1 of \cite{dathathri2019plug}}
    \label{fig:pplm_schematic_overview}
\end{figure}


% \subsubsection{PPLM is not fine-tuning}

It is important to realize that the plug-and-play method applied by \cite{dathathri2019plug} and \cite{madotto-etal-2020-plug} is different from fine-tuning. Note that in Equation \ref{eq:H_update_rule} the gradient updates are restricted to the history $H_t$, and do not affect the model's parameters. Because the key-value pairs $(K_t^{(i)}, V_t^{(i)})$ that comprise $H_t$ are activations and not model-weights, the updates only take place in the activation-space. This means that PPLM leaves the underlying (conversational) language model untouched.

Contrary to fine-tuning often massive LMs, PPLM does not incur a significant training cost (depending of course on the complexity of the discriminator or attribute model). However, \cite{madotto-etal-2020-plug} show that PPLM needs a fixed number of $m$ update-steps to for every generated token. This makes the original PPLM setup unsuitable for online interactive applications, like conversational systems. Addressing this problem, they introduce plug-and-play conversational models (PPCM), which extends PPLM by using the original model setup to generate dialogue datasets with the desired attribute $a$, and then use optimized residual adapters \citep{bapna-firat-2019-simple} to control $\texttt{LM}$'s output distribution. However, a fully interactive plug-and-play conversational model is out of the scope of this thesis.

%%%%%%%%%%%%%%%%%%%%%%%%%%%%%%%%%% METHODOLOGY ABOUT PPCM %%%%%%%%%%%%%%%%%%%%%%%%%%%%%%%%%%%%%%%%%%%%%%%%%%%%%%%%%
%%%%%%%% CONSIDER WHERE THIS INFORMATION MIGHT STILL BE USEFUL. MAYBE A VERY SHORT SUMMARY IN THE DISCUSSION AS A PROPERLY THOUGHT OUT FUTURE RESEARCH DIRECTION

% Residual adapters are optimizable modules stacked on every Transformer-layer of a pre-trained (language) model. The adapter module then steers the Transformer's output distribution without changing the pre-trained model's weights. A Layer Normalization module \citep{DBLP:journals/corr/BaKH16} followed by an auto-encoder with residual a connection constitutes a residual adapter module. More specifically, the residual adapter block can be expressed as the following function composition:

% \begin{equation}
% \begin{gathered}
%     f_{\theta_i} (x) = \texttt{ReLU}(\texttt{LayerNorm} (x) \cdot W_i^E) \cdot W_i^D, \\
%     \texttt{Adapter} (o_{:t}^i) = f_{\theta_i}(o_{:t}^i) + o_{:t}^i
% \end{gathered}
% \end{equation}


% where $o_{:t}^i \in \mathbb{R}^{t \times d}$ denotes the Transformer's $i$-th layer's latent output at step $t$, $d$ is the hidden state's size, $W_i^E$ and $W_i^D$ are learnable parameter-matrices of sizes $d \times m$ and $m \times d$, respectively. Finally, $m$ is the auto-encoder's bottle-neck dimension, which is a tunable hyper-parameter for changing the residual adapter's capacity. In practice, \cite{madotto-etal-2020-plug} use PPLM to generate $n$ attribute-adherent dialogue datasets $\mathscr{D}^a = \{\mathcal{D}^1, ..., \mathcal{D}^n\}$, for attribute $a$. These generated dialogue datasets are then used to train the residual adapter, which they aptly name a plug-and-play adapter, so it can be used to control the language model's output distribution. So for every attribute $a$, they train the plug-and-play adapter's parameters $\Theta_a := \{\theta^{a}_0, ..., \theta^{a}_l\}$, where $\theta^{a}_i := \{W_i^{E, a},W_i^{D, a}\}$, such that negative log-likelihood over the corresponding dialogue dataset $\mathscr{D}^a$ is minimized:

% \begin{equation}
%     \Theta_a \text{ s.t. } 
%     \min \mathcal{L} (\mathscr{D}^a) = - \sum_{k}^{|\mathscr{D}^a|} \sum_{i}^n 
%     \log p(s_i^k | s_{<i}^k, \mathcal{D}^k_t),
% \end{equation}

% where $s_i^k$ is the $i$-th generated token of response $S_t^k = \{s_0^k, ..., s_n^k\}$ with maximum sequence length $n$.

%%%%%%%%%%%%%%%%%%%%%%%%%%%%%%%%%%%%%%%%%%%%%%%%%%%%%%%%%%%%%%%%%%%%%%%%%%%%%%%%%%%%%%%%%%%%%%%%%%%%%%%%%%%%%%%%%%%%

% \textbf{TODO: Motivate the use of discriminators (as opposed to BoW) as attribute models. Emphasize that discriminators do not require human selected word-lists. Frequency-based word-lists, though easily produced based on simple heuristics, still often require human second-opinion to confirm their validity. And mention the use of discriminators that are more complex than single-layer linear classifiers.}

% Finally, in the original PPLM paper, the authors experiment with controlled language generation using as attribute models both trained single-layer discriminators, as bag-of-words (BoW) classifiers, where BoW essentially requires providing a human-selected word-list.

% \len{TODOs -
% \begin{itemize}
%     \item Maybe this isn't the best spot, but somewhere in the methodology motivate the choice of using a unigram wordlist as BoW, and a linear classifier trained with a transformer-based architecture.
%     \item Also emphasize the continuing narrative between Experiment 1 and Experiment 2. Namely, how the best performing systems in Exp1 are used to inform choices during Exp2 (trigram and BERT best classifiers --> use unigram and linear layer used for PPLM).
%     \item Also explain how the PPLM setup is not compatible with lists of $n$-grams (for $n>1$), as it makes perturbations at the unigram level. And adapting the system to fit trigram BoW's implies retraining the entire underlying language model (like GPT-2) for trigrams, defeating the purpose of PPLM (leveraging large scale language models for controllability, without having to finetune the massive architectures).
%     \item Then go on to argue that the unigram classifier is on par with trigram, making it a viable choice for the 100MIU BoW.
%     \item Also mention that your work extends the original BoW-PPLM by using various empirically generated BoWs, instead of curated wordlists. Making it more reproducible.
%     \item Double-check if you should make this last argument in the methodology, and not in related work or experimental details.
% \end{itemize}}

\section{Experimental Details and Evaluation}


The workflow of my controlled text generation experiments can be divided into three phases, attribute model development, generation, and evaluation. The following paragraphs describe and motivate the steps and choices per phase. All experiments are conducted on an NVIDIA TitanRTX GPU.

\subsection{Attribute Model Development}
During attribute model development, either a discriminator is trained on the dialogue data, or a bag-of-words is statistically constructed from the same corpus.

When training the discriminators, the dialogue dataset is randomly split into a training (90\%) and test (10\%) set. The frozen embeddings of either GPT-2-medium \citep{radford2019language} or DialoGPT-medium \citep{zhang2019dialogpt} are fed into trainable linear classifiers, seeking to distinguish between transcribed dialogue utterances from young (ages 19 to 29) and old (ages 50 and over) speakers. The discriminators are trained using Adam \citep{DBLP:journals/corr/KingmaB14} with a learning rate of $1\cdot10^{-4}$ and default values for all other parameters, with a maximum sequence length of 512 tokens, for 20 epochs, and a batch size of 64. The discriminator parameters are used of the epoch with the highest test accuracy.

A simple bag-of-words can also serve as an attribute model. Lists of unigrams are used as BoWs, because the PPLM setup is not compatible with lists of $n$-grams for $n > 1$, as it relies on perturbations at the unigram-level. Making a PPLM-system compatible with, e.g., trigrams, would amount of retraining the entire underlying language model (like GPT-2), thereby defeating the purpose of PPLM, i.e., leveraging large LMs for controllability, without incurring significant re-training costs. However, to continue the narrative between Experiment 1 and 2 (allowing the best-performing classifiers to inform decisions about generation), we use the best unigram's list of most informative features. This is a viable choice, because the unigram and trigram classifiers are on par (See Table~\ref{tab:bnc_classification_ws}).



I extend the methodology of \cite{dathathri2019plug} that relies on curated wordlists, by applying two empirical approaches to extract wordlists from the dialogue corpus. An empirical approach has the benefit of being more reproducible, and not requiring a domain expert to manually curate a list. In the first approach, the BoW consists of the 100 most informative unigrams of the unigram classifier used during the text classification experiments (See Table \ref{tab:bnc_classification_ws} for the results). The most informative unigrams per age groups are deemed the most distinguishing features by the unigram classifier. They could therefore be used to make sensible perturbations to a language model's output, yielding more effective control.

The second method of wordlist extraction is fully frequency-based, these setups are labeled $FB$ in Table \ref{tab:ctg_results_ws}. The goal of this extraction method is to yield two distinct sets of words that are representative of each age group's language. The frequency-based wordlists per age group are constructed from the \textit{imbalanced} dialogue dataset as follows: Per age group, all unique words are ordered by frequency of occurrence in the corpus. For both ordered lists of word counts, the most frequent words are kept that account for at least 85\% of the cumulative probability mass of the full age-specific distribution of words. Then, the words are removed that appear in both lists (i.e., the overlapping set of words is discarded). Of the resulting two non-overlapping ordered lists of words and their numbers of occurrences, only the words are kept that account for at least 85\% of the respective wordlist's summed occurrences. The resulting lists consist of 56 (young), and 92 (old) words. The 85-th percentile cutoff points are chosen to yield wordlists of similar lengths as those used by \cite{dathathri2019plug}. Both pairs of wordlists are included Section \ref{sec:wordlists}.

% \len{TODO - rephrase this last paragraph to make it clearer.}

% Convert these steps into a description of how the FB wordlists are constructed:

% Steps taken to create age-specific wordlists (full imbalanced BNC used):
% \begin{itemize}
%     \item Remove all stopwords. List of stopwords from NLTK's English stopword list. \textbf{TODO: does this make sense? What if differences in use of stopwords are strong indicators of an age-group's speech?}
%     \item Order all unique words by frequency per age-group.
%     \item For both lists, keep the words that account for at least 80\% of the respective cumulative probability densities.
%     \item From both sets of words, remove the words that are in the \textit{union} (i.e., the overlapping set) of the young and old sets.
%     \item For both sets, order the words by frequency.
%     \item For both remaining lists, keep the words that account for at least 80\% of the respective cumulative probability densities.
%     \item \textbf{TODO:} remove curse-words?
%     \item Resulting wordlist lengths:
%         \begin{itemize}
%             \item Young (19-29): 90 words
%             \item Old (50 plus): 225 words
%         \end{itemize}
% \end{itemize}

\subsection{Generation \len{Title too generic?}}

% \len{
% \begin{itemize}
%     \item Argue why you choose to use GPT-based models as base LMs for PPLM setup and not best BERT classifier.
%     \item Make my results easier to compare to results in PPLM (and PPCM?) paper.
%     \item BERT is an encoder-based architecture (i.e., more suitable for classification). GPT(-2) is a decoder-based architecture (i.e., more suitable for generation). FIND A REFERENCE FOR THIS STATEMENT OR AT LEAST VERIFY IT.
%     \item exp1 was about comparing model classes (ngram, rnn, transformer). Same type of comparison persists (bow ie ngrams, gpt2 classifier head (transformer))
%     \item EMPHASIZE HOW EXP1 and EXP2 ARE CONNECTED. Here if fits into narrative, but do it somewhere.
% \end{itemize}
% }

Controlled text generation experiments are performed using PPLM-setups that differ with respect to \textbf{(1)} pre-trained language model (GPT-2 or DialoGPT), \textbf{(2)} type of attribute model (BoW or discriminator), \textbf{(3)} attribute (young, old, or uncontrolled), and \textbf{(4)} whether the model was prompted or not. An unprompted model conditions its generated output on the \texttt{<|endoftext|>} token. BoW-based configurations can also differ in their wordlist extraction method (most informative unigrams or frequency-based). 

Every PPLM-setup generates 30 sequences per output length 6, 12, 24, 30, 36, 42, 48, 54, and 60 tokens. Sub-sample sizes of 30 are chosen to satisfy the Central Limit Theorem (CLT), making it possible to assume the sub-samples approximate normal distributions \citep{CLT2008springer}. The results in Table \ref{tab:ctg_results_ws} are averaged over $N = 30 \cdot 9 = 270$ samples.
% \len{Motivate the choice of lengths and sample size} 
Note that perturbations of the base language model's output can affect the controlled sequence length, so the final sequence length may differ by a few tokens from the given one. All other PPLM-parameters are kept at their default values, as recommended by \cite{dathathri2019plug}. 

Each reported PPLM-configuration represents the best initialization, if the term applies. The pre-trained language models are kept equal across configurations, as using different initializations of these large models is infeasible, and would defeat the purpose of PPLM. A BoW-based setup uses a single list of words as an attribute model, thereby not having random parameters to initialize. And finally, discriminator-based setups use comparatively small linear classifiers (a few hundred parameters), the initialization effects of which have been found to be negligible on performance.

\len{TODO - Motivate the choice of prompts}

\subsection{Evaluation \len{Title too generic?}}

The generated sequences are evaluated along two main axes: fluency and control. Fluency refers to the degree to which a text passage appears natural, grammatical, and non-repetitive. Control is the extent to which the produced language resembles that of the attribute being controlled for. Evaluation is done using both automated metrics and human opinions. Fluency is measured automatically by perplexity (denoted ppl) with respect to a different language model, GPT-1 \citep{radford2018improving}, expressed as

\begin{equation}
    \text{ppl}(\textbf{x}) = \exp \left\{ - \frac{1}{t} \sum_{i}^t \ln p_{\theta}(x_i | x_{<i})\right\}.    
\end{equation}

$\textbf{x}$ represents a sequence of tokens, $t$ is sequence length, $x_i$ is the $i$-th token, and $\theta$ denotes GPT's parameters. Perplexity is a measure of a language model's uncertainty when posed with the task of predicting a succession of words. Assuming a language model to be a reliable representation of relationships within a natural language, low perplexity can serve as a rough proxy for fluency of a text. However, a major caveat of perplexity is that it only measures uncertainty w.r.t. one language model, making it less generalizable. To slightly reduce this effect, we choose to evaluate perplexity with respect to a different language model than the one used for generation (GPT-2 or DialoGPT).

Furthermore, the normalized number of distinct unigrams (Dist-1), bigrams (Dist-2), and trigrams (Dist-3), are used as measures of text diversity. Experiment 1's best BERT-classifier's classification accuracy on a set of sequences generated by a single generation model is used as an automated measure of attribute control. It can be seen as a proxy for control, because it indicates how resemblant of an age group's vernacular a generated text is deemed to be.

Two types of baselines are used when evaluating text generation performance: a pre-trained model baseline, and a corpus-specific baseline. The pre-trained model baseline refers to the uncontrolled language model setting, being either GPT-2 or DialoGPT. Therefore all controlled generation models that use GPT-2 as their language model, should be compared to the uncontrolled GPT-2 baseline. The same holds for DialoGPT. The second type of baseline combines the underlying language model with a bag-of-words consisting of the 100 most common words in the balanced dialogue corpus, irrespective of age. This setting is included to give an indication of how biased the balanced BNC's frequently occurring words might be towards a specific age group.

\len{TODO - Describe human evaluation}


% \textit{How do you measure how representative of the stylistic attribute $a$ the generated text is? Specifically, is the generated text similar to that of the age-group you're controlling for?}

% \paragraph{Fluency}

% \textit{How do you measure how grammatically correct and fluent the generated texts are?}

% \begin{itemize}
%     \item Perplexity
%         \begin{itemize}
%             \item \textbf{TODO:} When explaining and motivating the use of perplexity as an evaluation metric for (controlled) language models, re-read this piece of documentation about perplexity by Hugging face: \url{https://huggingface.co/transformers/perplexity.html}
            
%             \item $\text{PPL}(\textbf{x}) = \exp \left\{ - \frac{1}{t} \sum_{i}^t \ln p_{\theta}(x_i | x_{<i})\right\}$
%         \end{itemize}
%     \item Also checkout this blogpost by The Gradient about Evaluation Metrics for Language Modeling (NB: contains BibTeX citation at the bottom): \url{https://thegradient.pub/understanding-evaluation-metrics-for-language-models/}
% \end{itemize}

% \subsubsection{Human evaluation}

% \textbf{TODO:} find humans.

\newpage
\section{Controlled Dialogue Generation Results}
\label{sec:exp2_results}
\len{NB - If you choose to use confidence intervals instead of std's, be sure to revise this section so it still makes sense.}
% \len{Old text for the unprompted results. FROM HERE....}

% Table~\ref{tab:ctg_results_ws} reports the automated evaluation results of our controllable text generation models. It can be seen that uncontrolled GPT-2 baseline has a slight bias towards generating "young-sounding" language (57.5\% accuracy). Furthermore, it appears that perturbing GPT-2's output distribution with the 100 most common words across all ages results in a slight de-biasing of the generated text (54.1\% accuracy). Achieving detectable control seems possible, because all GPT-2-based models surpass both baselines in terms of accuracy, with the exception of both BoW-setups using the 100 most informative unigrams.

% Frequency-based BoW-models outperform those using the most informative unigrams, as illustrated by their higher average accuracy (66.75\% versus. 53.1\%), and lower average perplexity (27.48 versus 27.90). 
% % \len{Offer an explanation.} 
% Discriminator-based models achieve noticeably better accuracies, with an average improvement of 8.45\% over the best performing BoW-based models. However, discriminator-based models do show more signs of disfluency and repetitiveness compared to the BoW-models, as depicted by the worse perplexities and Dist-$n |_{n = 1,2,3}$ scores.

% % The accuracies of our uncontrolled DialoGPT baseline (78.1\%) and the 100MCW baseline (80.7\%), suggest that DialoGPT is heavily biased towards producing young-sounding language. This can be attributable to DialoGPT having been fine-tuned on Reddit threads, as the majority of Reddit users are between the ages 20 and 29~\footnote{\url{https://www.statista.com/statistics/1125159/reddit-us-app-users-age/}}
% % \len{Find a reference for this statement} 
% \citep{zhang2019dialogpt}. DialoGPT's strong propensity for generating younger sounding language makes it a less desirable choice for our human evaluation experiments, because it requires non-standard parameter settings to produce detectably older sounding text.

% Overall, the results show that, for most models, a plug-and-play approach to controlling generated dialogue responses to possess detectable age-specific linguistic features is achievable. The most promising models being either discriminator-based, or frequency-based bag-of-words models. Discriminator-based models achieve more detectable levels of control than their BoW-based counterparts, at the cost of perplexity and repetitiveness. This could be attributable to the more complex activation-space updates that are used by discriminator-models. Furthermore, GPT-2's preference to generate young-sounding language is severely less pronounced than that of DialoGPT, making it easier to control, given equal parameter settings.

% \len{...TO HERE}

The quantitative results of generating younger (19 to 29) and older (50 and over) sounding responses to neutral prompts are reported in Tables \ref{tab:ctg_results_ws_neutral_prompt_young_models} and \ref{tab:ctg_results_ws_neutral_prompt_old_model}, respectively. In these tables, the underlying language model being used in a setup (i.e., a row in a table) is indicated by the prefixes G- for GPT-2 and D- for DialoGPT. Additionally, both tables report the results of the unperturbed GPT-2 and DialoGPT baselines (labeled G-baseline and D-baseline, respectively), and those of the 100 most common age-independent bag-of-words setups for both GPT-2 and DialoGPT (labeled G-100MCW and D-100MCW, respectively). The accuracies for these two setups (i.e., baseline and 100 most common words) are omitted because they do not aim to generate responses that resemble any target age group. Moreover, two bag-of-words (BoW) setups are reported per underlying language model (GPT-2 or DialoGPT): the frequency-based BoW setup (indicated by the suffix, -BoW$_{FB}$), and the 100 most informative unigram setup (indicated by the suffix, -BoW$_{100MIU}$). Detailed descriptions of how and why these aforementioned wordlists are constructed are provided in Section \ref{subsec:att_model_dev}. Finally, the discriminator-based setups are indicated by the suffix, -Discrim. The aforementioned reporting conventions also hold for the tables containing the results of response generation to younger and older sounding prompts (i.e., Tables \ref{tab:ctg_results_ws_young_prompt_young_model}, \ref{tab:ctg_results_ws_young_prompt_old_model}, \ref{tab:ctg_results_ws_old_prompt_young_model}, and \ref{tab:ctg_results_ws_old_prompt_old_model}). The results of those experiments are discussed in Section \ref{subsec:ctg_anal_prompt_class}.

As one would expect, Tables \ref{tab:ctg_results_ws_neutral_prompt_young_models} and \ref{tab:ctg_results_ws_neutral_prompt_old_model} show that the GPT-2 baseline consistently scores among the best on perplexity (best perplexity compared to young generation models, and second best compared to old-generation models) and diversity of generation (the Dist-$n |_{n = 1,2,3}$ scores are almost consistently in the upper registers). Similarly, the unperturbed DialoGPT baseline also scores best in terms of perplexity, when compared to other DialoGPT-based setups. This means that the responses generated by GPT-2 and DialoGPT are found to be among the least perplexing to GPT-1. This is unsurprising, as GPT-2, and thereby also DialoGPT, are pre-trained in similar fashion to GPT-1 \citep{radford2018improving, radford2019language, zhang2019dialogpt}. Additionally, the target probabilities ($\bar{P}_Y = 0.62$ in Table~\ref{tab:ctg_results_ws_neutral_prompt_young_models}, and $\bar{P}_O = 0.38$ in Table~\ref{tab:ctg_results_ws_neutral_prompt_old_model}) indicate that the GPT-2 baseline is biased towards generating young language. That is, given a neutral prompt, GPT-2 is inclined to produce responses that are likely to contain features learned to be young by BERT$_{FT}$. This could be attributable to GPT-2 being pre-trained on WebText, a corpus of high-quality documents scraped from web pages, which could be over-represented by millennials~\footnote{\url{https://www.statista.com/statistics/272365/age-distribution-of-internet-users-worldwide/}}. Moreover, DialoGPT has very strong bias towards generating younger sounding responses, given a neutral prompt (0.76 average probability to contain detectable young features). This is most likely due to DialoGPT having been fine-tuned on Reddit threads \citep{zhang2019dialogpt}, as the majority of Reddit users are between the ages 20 and 29~\footnote{\url{https://www.statista.com/statistics/1125159/reddit-us-app-users-age/}}.

The G-BoW$_{100MCW}$ setup performs on par with baseline w.r.t. perplexity and diversity (second best perplexity, second best Dist-2, best Dist-3), when compared to the young generation models. A similar pattern can be seen for old models, where G-100MCW has the second best Dist-2, and best Dist-3 scores. Additionally, the target probabilities seem virtually unaffected by the 100MCW setups, suggesting that perturbing GPT-2's output with an age-agnostic bag-of-words of the most frequently used words in the dialogue dataset does not noticeably shift the writing style towards that of the younger or older age group. This is to be expected, as such a wordlist should be an unbiased representation of the dataset's language style, given similar sample sizes per class.

BoW-based models seem to generate responses that are slightly more likely to contain features of the target age (young GPT2-BoW$_{FB}$ models result in 0.06 target probability improvement over the baseline, and old GPT2-BoW$_{FB}$ model in 0.04 target probability improvement over baseline). However, these differences are could also be due to randomness. For the 50-plus style GPT-2-based response models, the BoW$_{100MIU}$ setup does not manage to generate older sounding language than the baseline. Furthermore, BoW-based models seem to barely impact the syntactical structure of generated responses, because changes are made at the token-level \cite{dathathri2019plug}. This is confirmed by the barely altered perplexity and Dist-scores. \len{But this will be studied in more detail in the qualitative analyses.} 

The GPT-2 discriminator-based old-style model manages to generate responses that more convincingly resemble the style of the target group (0.38 $\bar{P}_O$ improvement over the GPT-2 baseline). However, this comes at the cost of perplexity ($+19.65$ compared to baseline) and diversity (much lower dist-scores and higher corresponding standard deviations). By contrast, the GPT-2 discriminator-based young setup does not generate convincingly more young-sounding responses (only 0.04 average target probability improvement over baseline), despite having noticeably worse and less precise perplexity ($4.59$ increase over baseline) and diversity.

The unperturbed DialoGPT baseline produces more perplexing and less diverse text than GPT-2 according to GPT-1 perplexity and the Dist-$n |_{n = 1,2,3}$ scores. The higher perplexity is to be expected, as DialoGPT pre-training and fine-tuning method deviates more from GPT-1's than GPT-2, making it likely to produce more unexpected sentences \citep{zhang2019dialogpt}. When using DialoGPT as an underlying language model, the BoW-based models (including 100MCW) seem to reinforce young-bias for DialoGPT, regardless of age (i.e., for all young-targeted BoW-based models $\bar{P}_Y$ goes up, and $\bar{P}_O$ goes down for all old-targeted BoW-based models). And the discriminator-based DialoGPT models, are superior w.r.t. target probability (+0.10 $\bar{P}_Y$ difference compared to DialoGPT baseline for young-models, and +0.34 $\bar{P}_O$ compared to baseline for old-models). However, this again comes at the cost of much higher and more volatile perplexity.

Overall, according to these results it seems to be possible to control dialogue responses for a certain age group. The used underlying language models are biased (in varying degrees) towards generating younger-sounding language. In these tested PPLM-setups, there seems to be a tradeoff between increased control and decreasing perplexity and diversity of generated language. Furthermore, the BoW-based models achieve less detectable levels of control, but preserve the fluency and diversity of generated text. In other words, the discriminator-based models make more invasive changes to the unperturbed sentences, which can result in less fluent and more repetitive text. However, they do produce more detectably age-appropriate passages.

\textbf{Initial observations and interpretations neutrally prompted ctg results (Tables \ref{tab:ctg_results_ws_neutral_prompt_young_models} and \ref{tab:ctg_results_ws_neutral_prompt_old_model})}
\begin{itemize}
    \item Style of prompt heavily influences control of response. Also confirmed by Tables \ref{tab:ctg_results_ws_young_prompt_young_model}, \ref{tab:ctg_results_ws_young_prompt_old_model}, \ref{tab:ctg_results_ws_old_prompt_young_model}, and \ref{tab:ctg_results_ws_old_prompt_old_model}. Remind reader of ctg formula $p(\textbf{x} | a, \texttt{prompt})$.
    % \item GPT-2 baseline scores best w.r.t. fluency (lowest perplexity) and diversity of generation (highest Dist-2, second highest Dist-3).
    % \item GPT-2 + BoW$_{100MCW}$ on par with baseline w.r.t. fluency and diversity (second best perplexity, second best Dist-2, best Dist-3).
    % \item Very similar pattern for old models: baseline second best perplexity and Dist-3, best Dist-1 and Dist-2; 100MCW second best Dist-2, and best Dist-3.
    % \item Baseline is moderately biased towards generating young language. I.e., GPT-2 is inclined to produce responses, given a neutral prompt, that are likely to contain features learned to be young by BERT$_{FT}$.
    % \item BoW-based models seem to result in language that is slightly more likely to contain features of target age (Young bow-models result in 0.06 probability increase wrt baseline, and old-bow-fb model in 0.04 increase wrt baseline). Differences are also possibly due to randomness.
    % \item BoW-based models barely impact the syntactical structure of generated sequences, i.e., changes made at lexical-level. This is shown by the barely altered perplexity (sometimes improved , G-B$_{100MIU,O}$ 0.25 lower average perplexity than baseline), and Dist-scores. \len{But this will be studied in more detail in the qualitative analyses.} 
    % \item BoW-fb slightly superior to BoW-100MIU -> consistently higher accuracy for BERT$_{FT}$.
    % \item For Old-models, BoW still doesn't fully overcome young-bias to result in convincingly old language.
    % \item Discriminator-based old model manages to generate language that is more convincingly similar to target age (0.38 old prob improvement). This comes at the cost of fluency and diversity (much higher perplexity and std wrt baseline), noticeable lower dist-scores and higher std's.
    % \item GPT2-disc young not convincingly more young sounding response. Noticeably worse and less precise fluency and diversity though. Only 0.04 average probability improvement.
    % \item DialoGPT has very strong young-bias, given a neutral prompt (0.76 average probability to contain detectable young features). --> Most likely due to DialoGPT being GPT-2, fine-tuned on Reddit Thread data \citep{zhang2019dialogpt} and include reddit age dist reference.
    % \item DialoGPT baseline produces less fluent and less diverse text than GPT-2 according to GPT-1 perplexity.
    % \item BoW-based models (including 100MCW) seem to reinforce young-bias for DialoGPT, regardless of age (young prob goes up for both BoW, and old prob goes down for both BoW-models).
    % \item Discriminator based models, again, outperform significantly w.r.t. target probability (0.10 prob difference wrt baseline for young, and 0.34 prob difference wrt baseline for old). Again, at the cost of on average worse and more volatile perplexity.
    \item Janie's suggestion: it could be that there are some detectable young-sounding tokens that make BoW-based young control easier, and that old-sounding features are more salient at the structural/syntactical level and harder to control for.
    % \item Overall, it seems to be possible to control dialogue responses for a certain age group. The used base language models are biased towards generating young-sounding language. In the current PPLM-setup, there seems to be a tradeoff between increased control and decreasing fluency and diversity. BoW-based models achieve less detectable levels of control, but preserve the fluency and diversity of generated text. Discriminator based models make more invasive changes to the unperturbed sentences, which can result in less fluent and more repetitive text. However, they do produce more detectably age-appropriate passages.
    \item \len{TODO - Examine the nonsensical generated sequences that are correctly labeled as old or young. What patterns do you see? It could be that BERT$_{FT}$ is picking up non-language patterns that give away age.}
\end{itemize}

\begin{table*}[h]
    \centering
    \begin{tabular}{l | c c c c | c c}
    \toprule
    \textbf{Model} & \textbf{ppl.} & \textbf{Dist-1} & \textbf{Dist-2} & \textbf{Dist-3} & $\boldsymbol{\bar{P}_Y}$ & \textbf{Acc.}\\
    % -plus)}$\\
     & $\downarrow$ better & $\uparrow$ better & $\uparrow$ better & $\uparrow$ better & $\uparrow$ better & $\uparrow$ better\\
    \midrule
    \midrule
    G-baseline & \textbf{27.50} (6.58) & 0.87 (0.09) & \textbf{0.94} (0.04) & \textcolor{blue}{0.90} (0.06) & 0.62 (0.42) & -\\
    G-100MCW & \textcolor{blue}{27.56} (6.60) & 0.86 (0.10) & \textcolor{blue}{0.93} (0.04) & \textbf{0.90} (0.05) & 0.63 (0.42) & -\\
    \midrule
    G-BoW$_{FB}$ & 27.91 (7.18) & 0.87 (0.10) & 0.93 (0.05) & \textcolor{blue}{0.90} (0.06) & 0.69 (0.41) & 70.4\%\\
    G-BoW$_{100MIU}$ & 28.37 (7.31) & 0.87 (0.09) & \textcolor{blue}{0.93} (0.04) & \textcolor{blue}{0.90} (0.06) & 0.67 (0.41) & 67.4\%\\
    \midrule
    G-Discrim & 32.09 (18.98) & 0.77 (0.20) & 0.86 (0.13) & 0.84 (0.15) & 0.66 (0.43) & 67.8\%\\
    \midrule
    \midrule
    D-baseline & 37.52 (12.06) & 0.86 (0.13) & 0.90 (0.08) & 0.85 (0.10) & 0.76 (0.37) & -\\
    D-100MCW & 37.80 (10.89) & 0.85 (0.14) & 0.89 (0.10) & 0.85 (0.10) & 0.82 (0.33) & -\\
    \midrule
    D-BoW$_{FB}$ & 38.53 (12.64) & 0.87 (0.12) & 0.90 (0.08) & 0.86 (0.10) & 0.82 (0.33) & 83.0\%\\
    D-BoW$_{100MIU}$ & 38.67 (11.70) & \textcolor{blue}{0.88} (0.11) & 0.91 (0.07) & 0.86 (0.10) & \textbf{0.87} (0.28) & \textcolor{blue}{88.5\%}\\
    \midrule
    D-Discrim & 42.01 (16.94) & \textbf{0.90} (0.12) & 0.86 (0.14) & 0.77 (0.22) & \textcolor{blue}{0.86} (0.29) & \textbf{85.9\%}\\
    \bottomrule
    \end{tabular}
    \caption{\len{Neutral prompt - Young model} Results of age-controlled language generation. ppl. is perplexity w.r.t. GPT-1. Dist-n is number of distinct n-grams normalized by text length, as a measure of diversity. $\bar{P}(Young)$ is the sample's average probability to contain features learned to be young by BERT$_{FT}$. Acc. is BERT$_{FT}$'s accuracy when classifying the row's samples.}
    \label{tab:ctg_results_ws_neutral_prompt_young_models}
\end{table*}

\begin{table*}[h]
    \centering
    \begin{tabular}{l | c c c c | c c }
    \toprule
    \textbf{Model} & \textbf{ppl.} & \textbf{Dist-1} & \textbf{Dist-2} & \textbf{Dist-3} & $\boldsymbol{\bar{P}_O}$ & \textbf{Acc.}\\
    % -plus)}$\\
     & $\downarrow$ better & $\uparrow$ better & $\uparrow$ better & $\uparrow$ better & $\uparrow$ better & $\uparrow$ better\\
    \midrule
    \midrule
    G-baseline & \textcolor{blue}{27.50} ($\pm$6.58) & \textbf{0.87} ($\pm$0.09) & \textbf{0.94} ($\pm$0.04) & \textcolor{blue}{0.90} ($\pm$0.06) & 0.38 ($\pm$0.42) & -\\
    G-100MCW & 27.56 ($\pm$6.60) & 0.86 ($\pm$0.10) & \textcolor{blue}{0.93} ($\pm$0.04) & \textbf{0.90} ($\pm$0.05) & 0.37 ($\pm$0.42) & -\\
    \midrule
    G-BoW$_{FB}$ & 27.58 ($\pm$7.07) & 0.86 ($\pm$0.10) & \textcolor{blue}{0.93} ($\pm$0.04) & \textcolor{blue}{0.90} ($\pm$0.06) & 0.42 ($\pm$0.42) & 43.0\%\\
    G-BoW$_{100MIU}$ & \textbf{27.25} ($\pm$6.15) & \textbf{0.87} ($\pm$0.09) & \textcolor{blue}{0.93} ($\pm$0.04) & \textcolor{blue}{0.90} ($\pm$0.06) & 0.38 ($\pm$0.42) & 37.4\%\\
    \midrule
    G-Discrim & 47.15 ($\pm$47.56) & 0.73 ($\pm$0.24) & 0.75 ($\pm$0.28) & 0.75 ($\pm$0.27) & \textbf{0.76} ($\pm$0.36) & \textbf{74.3\%}\\
    \midrule
    \midrule
    D-baseline & 37.52 ($\pm$12.06) & 0.86 ($\pm$0.13) & 0.90 ($\pm$0.08) & 0.85 ($\pm$0.10) & 0.24 ($\pm$0.37) & -\\
    D-100MCW & 37.80 ($\pm$10.89) & 0.85 ($\pm$0.14) & 0.89 ($\pm$0.10) & 0.85 ($\pm$0.10) & 0.18 ($\pm$0.33) & -\\
    \midrule
    D-BoW$_{FB}$ & 37.85 ($\pm$11.17) & 0.87 ($\pm$0.12) & 0.90 ($\pm$0.08) & 0.86 ($\pm$0.09) & 0.22 ($\pm$0.35) & 21.5\%\\
    D-BoW$_{100MIU}$ & 37.91 ($\pm$12.27) & \textcolor{blue}{0.87} ($\pm$0.11) & 0.90 ($\pm$0.07) & 0.85 ($\pm$0.10) & 0.22 ($\pm$0.34) & 21.9\%\\
    \midrule
    D-Discrim & 41.17 ($\pm$20.72) & 0.87 ($\pm$0.12) & 0.89 ($\pm$0.13) & 0.83 ($\pm$0.16) & \textcolor{blue}{0.57} ($\pm$0.41) & \textcolor{blue}{56.7\%}\\
    \bottomrule
    \end{tabular}
    \caption{\len{Neutral prompt - Old model} Results of age-controlled language generation. Perplexity is perplexity w.r.t. GPT-1. Dist-n is number of distinct n-grams normalized by text length, as a measure of diversity. $\boldsymbol{\bar{P}_Y}$ and $\boldsymbol{\bar{P}_O}$ are the respective average young and old probabilities assigned by the best BERT$_{FT}$. Acc. is the best BERT model's accuracy when classifying the row's samples.}
    \label{tab:ctg_results_ws_neutral_prompt_old_model}
\end{table*}


% TABLE 5.6 WITH YOUNG AND OLD PROB. JUST IN CASE %
% \begin{table*}[h]
%     \centering
%     \begin{tabular}{l c c c c c c c}
%     \toprule
%     \textbf{Model} & \textbf{ppl.} & \textbf{Dist-1} & \textbf{Dist-2} & \textbf{Dist-3} & \textbf{Young prob.} & \textbf{Old prob.} & \textbf{Acc.}\\
%     % -plus)}$\\
%      & $\downarrow$ better & $\uparrow$ better & $\uparrow$ better & $\uparrow$ better & - & - & $\uparrow$ better\\
%     \midrule
%     \midrule
%     Baseline & 27.45 ($\pm$7.27) & 0.90 ($\pm$0.10) & 0.92 ($\pm$0.05) & 0.86 ($\pm$0.09) & 0.52 ($\pm$0.32) & 0.48 ($\pm$0.32) & 57.5\%\\
%     \midrule
%     B$_{100MCW}$ & 26.68 ($\pm$8.77) & 0.89 ($\pm$0.10) & 0.92 ($\pm$0.05) & 0.86 ($\pm$0.09) & 0.50 ($\pm$0.29) & 0.50 ($\pm$0.29) & 51.7\%\\
%     B$_{Y, FB}$ & 27.11 ($\pm$7.45) & \textbf{0.91} ($\pm$0.09) & \textbf{0.92} ($\pm$0.04) & \textbf{0.87} ($\pm$0.09) & 0.60 ($\pm$0.29) & 0.40 ($\pm$0.29) & 68.3\%\\
%     B$_{O, FB}$ & 25.99 ($\pm$6.41) & 0.88 ($\pm$0.11) & 0.92 ($\pm$0.05) & 0.86 ($\pm$0.09) & 0.35 ($\pm$0.31) & 0.65 ($\pm$0.31) & 62.5\%\\
%     B$_{Y, 100MIU}$ & 28.48 ($\pm$11.96) & 0.88 ($\pm$0.12) & 0.91 ($\pm$0.06) & 0.86 ($\pm$0.10) & 0.62 ($\pm$0.31) & 0.38 ($\pm$0.31) &69.2\%\\
%     B$_{O, 100MIU}$ & \textbf{25.57} ($\pm$7.44) & 0.88 ($\pm$0.11) & 0.92 ($\pm$0.05) & \textbf{0.87} ($\pm$0.09) & 0.38 ($\pm$0.31) & 0.62 ($\pm$0.31) & 58.3\%\\
%     \midrule
%     D$_{Y, GPT2}$ & 33.02 ($\pm$12.24) & 0.85 ($\pm$0.16) & 0.89 ($\pm$0.07) & 0.83 ($\pm$0.12) & 0.70 ($\pm$0.31) & 0.30 ($\pm$0.31) & 73.9\%\\
%     D$_{O, GPT2}$ & 32.86 ($\pm$18.08) & 0.80 ($\pm$0.21) & 0.84 ($\pm$0.13) & 0.79 ($\pm$0.19) & 0.37 ($\pm$0.29) & 0.63 ($\pm$0.29) & 63.3\%\\
%     D$_{Y, GPT2*}$ & 30.98 ($\pm$13.95) & 0.86 ($\pm$0.15) & 0.90 ($\pm$0.06) & 0.84 ($\pm$0.11) & \textcolor{blue}{0.76} ($\pm$0.29) & 0.29 ($\pm$0.29) & 80.0\%\\
%     D$_{O, GPT2*}$ & 34.81 ($\pm$26.76) & 0.84 ($\pm$0.17) & 0.85 ($\pm$0.13) & 0.77 ($\pm$0.23) & 0.31 ($\pm$0.25) & \textcolor{red}{0.69} ($\pm$0.25) & 75.8\%\\
%     \bottomrule
%     \end{tabular}
%     \caption{Results of age-controlled language generation. Perplexity is perplexity w.r.t. GPT-1. Dist-n is number of distinct n-grams normalized by text length, as a measure of diversity. Young and old accuracy are the assigned probabilities of belonging to the young or old age categories.}
%     \label{tab:ctg_results}
% \end{table*}


%%% THIS ISN'T RELEVANT ANYMORE --CONSIDER DELETING IT %%%%%%%%%%%%%%%%%%%%%%%%%
% \paragraph{Notes on Table \ref{tab:ctg_results}}
% \begin{itemize}
%     \item These are initial results.
%     \item All metrics are averaged over 120 samples: 30 samples per group of sequence lengths 8, 16, 32, and 64.
%     \item Young and old accuracy (last two columns) denote the probability of belonging to the young or old age-groups assigned by the best performing BERT-based classifier.
%     \item I use the same parameter settings as Table 6 of \cite{dathathri2019plug} to make the results comparable, i.e.:
%     \begin{itemize}
%         \item Step-size 0.02. Step-size is $\alpha$ in Equation \ref{eq:H_update_rule}.
%         \item Temperature 1.0.
%         \item Number of update iterations: 3.
%         \item $\gamma$ 1.5.
%         \item GM-scale 0.9.
%         \item KL-scale 0.01.
%     \end{itemize}
%     \item The current baseline is uncontrolled/unperturbed GPT-2.
%     \item There are four settings for BoW-based control:
%     \begin{itemize}
%         \item Young frequency-based wordlist.
%         \item Old frequency-based wordlist.
%         \item Young + most informative unigrams as wordlist.
%         \item Old + most informative unigrams as wordlist.
%     \end{itemize}
%     \item Initial observations:
%     \begin{itemize}
%         \item Fractions of distinct uni-,bi, and trigrams do not change.
%         \item Perplexity seems to improve when controlling generation for each age-group, which isn't necessarily what one would expect.
%         \item The baseline starts off with a higher average probability of belonging to the young age group
%         \item Controlling for young-language does result in a slightly greater assigned probability of belonging to young age-bracket.
%         \item Controlling for old-language results in a doubling of the assigned probability of belonging to the old age-bracket.
%     \end{itemize}
% \end{itemize}

%%%%%%%%%%%%%%%%%%%%%%%%%%%%%%%%%%%%%%%%%%%%%%%%%%%%%%%%%%%%%%%%%%%%%%

\begin{itemize}
    \item \len{Include examples of same original sequence being perturbed differently at the unigram-level between corresponding young and old BoW-based CTG. E.g., \textit{I think you're a nice person (old)} vs. \textit{I think you're a nice guy (young)}}
\end{itemize}

% \subsubsection{Frequency-based wordlist generation}

% Steps taken to create age-specific wordlists (full imbalanced BNC used):
% \begin{itemize}
%     \item Remove all stopwords. List of stopwords from NLTK's English stopword list. \textbf{TODO: does this make sense? What if differences in use of stopwords are strong indicators of an age-group's speech?}
%     \item Order all unique words by frequency per age-group.
%     \item For both lists, keep the words that account for at least 80\% of the respective cumulative probability densities.
%     \item From both sets of words, remove the words that are in the \textit{union} (i.e., the overlapping set) of the young and old sets.
%     \item For both sets, order the words by frequency.
%     \item For both remaining lists, keep the words that account for at least 80\% of the respective cumulative probability densities.
%     \item \textbf{TODO:} remove curse-words?
%     \item Resulting wordlist lengths:
%         \begin{itemize}
%             \item Young (19-29): 90 words
%             \item Old (50 plus): 225 words
%         \end{itemize}
% \end{itemize}

% \subsubsection{Most informative unigrams as wordlists}

% \subsection{Discriminator-based control}




% \textbf{Notes on the experimental details of Table \ref{tab:ctg_results_ws}:}

% \begin{itemize}
%     \item All sequences are generated unconditionally. I.e., from $\texttt{|<endoftext>|}$ token.
%     \item All results are averaged over 240 samples.
%     \item 
% \end{itemize}






\section{Controlled Dialogue Generation Analyses}
\label{sec:exp2_analyses}

By means of quantitative and qualitative analyses, we seek to study which relationships affect the grammatical quality and attribute relevance of the generated responses. The discriminator-based setups, and the BoW-models with the highest average target probabilities (i.e., BoW$_{FB}$ for GPT-2, and BoW$_{100MIU}$ for DialoGPT) are considered for the analyses. In the following sections, we presents a series of analyses about the relationship between perplexity and target probability (Section \ref{subsec:ctg_anal_ppl_target_prob}), the effects of response length on generation quality (Section \ref{subsec:ctg_anal_response_length}), the impact of the prompt's style on generation style and quality (Section \ref{subsec:ctg_anal_prompt_class}), and qualitatively observable patterns in generated samples \ref{subsec:ctg_anal_qualitative}.



\len{TODO - Add examples of generated sequences along with their model's configurations, age-group, etc. Similar to dialogue snippets earlier.}

\subsection{The Relationship between Perplexity and Target Probability \len{Think of better title}}
\label{subsec:ctg_anal_ppl_target_prob}


% \len{NB - THE FOLLOWING PARAGRAPH IS ABOUT THE (DEPRECATED) UNPROMPTED SETUP RESULT. FROM HERE...}
% Figure~\ref{fig:ctg_lineplot_fluency_vs_control} attempts to depict the relationship between fluency and control, as measured by perplexity and BERT's classification accuracy, respectively. On the y-axis, ``mean accuracy'' refers to the average fraction of generated sequences, controlled for young or old language, that are correctly labeled as such by Experiment 1's best BERT classifier. The bars around the averages in Figure~\ref{fig:ctg_lineplot_fluency_vs_control} are 90\% confidence intervals.
% % It can be seen as a proxy for control, because it indicates how resemblant of an age group's vernacular a generated text is deemed to be. 
% % Perplexity, measured by a different language model (GPT-1 \citep{radford2018improving}), is a measure of a language model's uncertainty when posed with the task of predicting a succession of words. Assuming a language model to be a reliable representation of relationships within an actual language, low perplexity can serve as a rough proxy for fluency of a text. However, a major caveat of perplexity is that it only measures uncertainty w.r.t. one language model, making it less generalizable. To slightly reduce this effect, we choose to evaluate perplexity with respect to a different language model than the one used for generation.
% % \len{Disclose the confidence interval level}
% It appears that increasing perplexity is slightly negatively correlated with accuracy. It is also clear from Figure~\ref{fig:ctg_lineplot_fluency_vs_control} that uncertainty about prediction strongly increases for greater perplexity. These two observations indicate that sentences deemed less coherent by GPT-1 tend to be harder to classify by BERT$_{FT}$ with certainty. BERT$_{FT}$ is pre-trained and fine-tuned to pick up syntactic features from dialogue that can indicate a speaker's age. It is therefore plausible that structural deviations from proper syntax (i.e., high perplexity) can obfuscate the age-related linguistic signal BERT$_{FT}$ leverages. Finally, it seems that, on average, the discriminator-based models are more capable of producing correctly classifiable high-perplexity sentences. \len{Why would that be?} However, none of the differences between the average accuracies are statistically significant at the 10\% level, so this conclusion should be taken tentatively.
% \len{...TO HERE.}

\len{\textbf{TODO!!!: Re-write this paragraph s.t. it fits with the new plots.}}
\begin{itemize}
    \item Low, medium, and high perplexity are found using the terciles (i.e., the two points that divide the distribution of perplexities into 3 parts, each containing a third of the population.
    \item low perplexity: $\text{ppl}\leq27.52$
    \item medium perplexity: $27.52 < \text{ppl} \leq 35.63$
    \item high perplexity: $>35.63$
\end{itemize}

\len{TODO - Refine the following text. Especially the last part:}
Figures~\ref{fig:barplot_ppl_target_prob_np_gpt2} and \ref{fig:barplot_ppl_target_prob_np_dgpt} show bar charts depicting the relationship between the average target probability ($y$-axes) and perplexity ($x$-axes) assigned to the samples generated by various PPLM-model setups. The error bars around the average target probabilities are 95\% confidence intervals. Based on the distributions of perplexity among generated samples and the necessity to have sufficiently large sub-sample sizes, perplexity is binned into three consecutive intervals, corresponding to low (0-25), medium (25-50), and high (50+) perplexity. 

The GPT2-based old models (Figures~\ref{subfig:barplot_ppl_target_prob_np_gpt2_disc_old} and \ref{subfig:barplot_ppl_target_prob_np_gpt2_bow_old}) show a clear pattern of increasing perplexity coinciding with higher and more precise assigned target probabilities. Responses with relatively high perplexities (50+) are (at a 5\% level) significantly more likely to contain features learned to be old by BERT$_{FT}$. High-perplexity responses of the GPT2-old models are also assigned probabilities with more precision, as indicated by the narrower confidence regions. For the GPT-2 Young models' responses (Figures~\ref{subfig:barplot_ppl_target_prob_np_gpt2_disc_young} and \ref{subfig:barplot_ppl_target_prob_np_gpt2_bow_young}) we observe a similar pattern of slight increase of average assigned target probability between low (0-25) and medium (25-50) perplexity. However, in both cases this is followed by a large drop in both average assigned target probability and precision for high-perplexity responses. It must be noted that there are no significant differences at the 5\% level between the average $\bar{P}_Y$, so conclusions about the relationship between perplexity and target probability should be taken tentatively.


DialoGPT's strong proclivity to generate younger sounding responses is noticeable in Figures~\ref{subfig:barplot_ppl_target_prob_np_dgpt_disc_young} and \ref{subfig:barplot_ppl_target_prob_np_dgpt_bow_young}, as depicted by the high average target probabilities and relatively narrow confidence intervals. That being said, there seem to be no clear patterns between perplexity and target probability in DialoGPT-based models. It could be that DialoGPT's strong bias makes it less reliable to draw conclusions about the effects of PPLM-control for age-related style, given default parameter settings.

BERT$_{FT}$ seems to have least certainty (i.e., low precision aka high variance) about low-perplexity responses in every case, except DialoGPT-BoW Young (Figure~\ref{subfig:barplot_ppl_target_prob_np_dgpt_bow_young}).

Both sets of graphs show that patterns in the relationship between target probability and perplexity seem to persist between different types of control (i.e., discriminator-based or BoW-based), when holding the age and underlying language model constant.

\begin{figure}[H]
     \centering
     \begin{subfigure}[b]{0.48\textwidth}
        \centering
        \includegraphics[width=1\columnwidth]{figures/exp2/fluency_control/barplot_ppl_target_prob_gpt2_disc_np_y.png}
        \caption{}
        \label{subfig:barplot_ppl_target_prob_np_gpt2_disc_young}
     \end{subfigure}
    %  \hfill
     \quad
     \begin{subfigure}[b]{0.48\textwidth}
        \centering
        \includegraphics[width=1\columnwidth]{figures/exp2/fluency_control/barplot_ppl_target_prob_gpt2_disc_np_o.png}
        \caption{}
        \label{subfig:barplot_ppl_target_prob_np_gpt2_disc_old}
     \end{subfigure}
    \medskip
    \begin{subfigure}[b]{0.48\textwidth}
        \centering
        \includegraphics[width=1\columnwidth]{figures/exp2/fluency_control/barplot_ppl_target_prob_gpt2_bow_fb_np_y.png}
        \caption{}
        \label{subfig:barplot_ppl_target_prob_np_gpt2_bow_young}
     \end{subfigure}
    %  \hfill
    \quad
     \begin{subfigure}[b]{0.48\textwidth}
        \centering
        \includegraphics[width=1\columnwidth]{figures/exp2/fluency_control/barplot_ppl_target_prob_gpt2_bow_fb_np_o.png}
        \caption{}
        \label{subfig:barplot_ppl_target_prob_np_gpt2_bow_old}
     \end{subfigure}
    \caption{Mean target probability ($x$-axes) assigned to GPT-2-based models' samples by BERT$_{FT}$ for increasing ranges of GPT-1 perplexity ($y$-axes). Error bars are 95\% confidence intervals.}
    \label{fig:barplot_ppl_target_prob_np_gpt2}
\end{figure}

\begin{figure}[H]
     \centering
     \begin{subfigure}[b]{0.48\textwidth}
        \centering
        \includegraphics[width=1\columnwidth]{figures/exp2/fluency_control/barplot_ppl_target_prob_dgpt_disc_np_y.png}
        \caption{}
        \label{subfig:barplot_ppl_target_prob_np_dgpt_disc_young}
     \end{subfigure}
    %  \hfill
     \quad
     \begin{subfigure}[b]{0.48\textwidth}
        \centering
        \includegraphics[width=1\columnwidth]{figures/exp2/fluency_control/barplot_ppl_target_prob_dgpt_disc_np_o.png}
        \caption{}
        \label{subfig:barplot_ppl_target_prob_np_dgpt_disc_old}
     \end{subfigure}
    \medskip
    \begin{subfigure}[b]{0.48\textwidth}
        \centering
        \includegraphics[width=1\columnwidth]{figures/exp2/fluency_control/barplot_ppl_target_prob_dgpt_bow_miu_np_y.png}
        \caption{}
        \label{subfig:barplot_ppl_target_prob_np_dgpt_bow_young}
     \end{subfigure}
    %  \hfill
    \quad
     \begin{subfigure}[b]{0.48\textwidth}
        \centering
        \includegraphics[width=1\columnwidth]{figures/exp2/fluency_control/barplot_ppl_target_prob_dgpt_bow_miu_np_o.png}
        \caption{}
        \label{subfig:barplot_ppl_target_prob_np_dgpt_bow_old}
     \end{subfigure}
    \caption{Mean target probability ($y$-axes) assigned to DialoGPT-based models' samples by BERT$_{FT}$ for increasing ranges of GPT-1 perplexity ($x$-axes). Error bars are 95\% confidence intervals.}
    \label{fig:barplot_ppl_target_prob_np_dgpt}
\end{figure}

\len{\textbf{Observed patterns in perplexity target-prob plots}}:

\begin{itemize}
    % \item For GPT2-based Old models (both discrim (Figure~\ref{subfig:barplot_ppl_target_prob_np_gpt2_disc_old}) and bow (Figure~\ref{subfig:barplot_ppl_target_prob_np_gpt2_bow_old})), there is a very clear pattern of increasing perplexity coinciding with higher assigned target probabilities. Responses with relatively high perplexities (50+) are significantly more likely to contain features learned to be old by BERT$_{FT}$. This suggests there could be a tradeoff between increased levels of attribute relevance (i.e., control) and fluency (i.e., lower perplexity). High-perplexity responses are of the GPT2-old models are also assigned probabilities with more precision (i.e., smaller confidence region).
    % \item For the GPT-2 Young models' responses we observe a pattern of slight increase of average assigned target probability between low (0-25) and medium (25-50) perplexity, followed by a extreme decrease of assigned target probability and associated precision for high-perplexity responses.
    % \item DialoGPT's young-bias is noticeable in Figures~\ref{subfig:barplot_ppl_target_prob_np_dgpt_disc_young} and \ref{subfig:barplot_ppl_target_prob_np_dgpt_bow_young}: all high average target probabilities with relatively high certainty.
    % \item No clear pattern of tradeoff between perplexity and target probability in DialoGPT-based models.
    % \item BERT$_{FT}$ seems to have least certainty (i.e., low precision aka high variance) about low-perplexity responses in every case, except DialoGPT-BoW Young (Figure~\ref{subfig:barplot_ppl_target_prob_np_dgpt_bow_young}).
    \item \len{Using the term fluency as a generalization of perplexity can be misleading. Think of another word, just use perplexity, or provide a disclaimer.}
\end{itemize}


\subsection{The Effects of Generated Response Length \len{Think of better title}}
\label{subsec:ctg_anal_response_length}

The number of tokens in a generated response coincides with noticeable differences in our automated evaluation metrics. It is therefore important to get a clearer picture of how the various measures for fluency and control change for different sequence lengths. Moreover, properly understanding these relationship can inform developers of adaptive dialogue systems about preserving output quality and adaptation of responses of arbitrary lengths.

Response length (on the $x$-axes) is plotted against various evaluation metrics in Figures~\ref{fig:lineplots_length_acc_np_gpt2_dgpt} (average BERT$_{FT}$ accuracy), \ref{fig:lineplots_length_ppl_np_gpt2_dgpt} (GPT-1 perplexity), \ref{fig:lineplots_length_dist1_np_gpt2_dgpt} (normalized number of distinct unigrams), \ref{fig:lineplots_length_dist2_np_gpt2_dgpt} (normalized number of distinct bigrams), and \ref{fig:lineplots_length_dist3_np_gpt2_dgpt} (normalized number of distinct trigrams).

% \begin{itemize}
%     \item \textit{Main question: how is generated sequence length related to fluency and control?}
    
%     \item \textit{Study the relationship between generated sequence length (measured in number of tokens) and automated evaluation metrics (i.e., perplexity, dist-n, and accuracy).}
    
%     \item \textit{For every metric and for (all?) models, plot sequence length on the x-axis, and the average metric with confidence intervals on the y-axis.}
    
%     \item \textit{Which patterns do you observe?} 
% \end{itemize}

\begin{figure}[H]
     \centering
     \begin{subfigure}[b]{0.49\textwidth}
        \centering
\includegraphics[width=1\columnwidth]{figures/exp2/sequence_length/lineplot_len_acc_best_gpt2_disc_bow_neutral_prompt_ci_95_errstyle_band.png}
        \caption{}
        \label{subfig:lineplot_length_acc_np_gpt2}
     \end{subfigure}
     \hfill
     \begin{subfigure}[b]{0.49\textwidth}
        \centering
        \includegraphics[width=1\columnwidth]{figures/exp2/sequence_length/lineplot_len_acc_best_dgpt_disc_bow_neutral_prompt_ci_95_errstyle_band.png}
        \caption{}
        \label{subfig:lineplot_length_acc_np_dgpt}
     \end{subfigure}
        \caption{Mean BERT$_{FT}$ accuracy. GPT-2-based models (left), DialoGPT-based models (right). Translucent error bands represent 95\% confidence intervals. Plots best viewed in color.}
        \label{fig:lineplots_length_acc_np_gpt2_dgpt}
\end{figure}

% \len{\textbf{Patterns in Figure~\ref{fig:lineplots_length_acc_np_gpt2_dgpt}}}:
% \begin{itemize}
%     % \item No clear trends.
%     % \item BoW-based Old (GPT-2 and DialoGPT) has significantly lower accuracy at almost every length bracket.
%     % \item Shortest responses generated by GPT-2-based models appear to be most challenging to classify for BERT$_{FT}$.
% \end{itemize}

Figure~\ref{subfig:lineplot_length_acc_np_gpt2} shows a slight upward trend in average accuracy with greater uncertainty for increasing response length for all GPT-2-based models, except for the BoW-based old generation model. That is, longer sequences are, on average, slightly easier to classify, though with less precision. This is probably due to the fact that longer sentences contains more information to base predictions on. 
By contrast, the DialoGPT-based models in Figure~\ref{fig:lineplots_length_acc_np_gpt2_dgpt} do not seem to show a clear general trend that mean accuracy follows for increasing response length. However, it does seem that DialoGPT's strong bias towards generating younger sounding responses causes output from DialoGPT-based young generation models to be much easier to classify than that from the old generation models. Overall, it can be seen that the BoW-based old models (GPT-2 and DialoGPT) are significantly more challenging to classify at almost every length bracket.

\begin{figure}[H]
     \centering
     \begin{subfigure}[b]{0.49\textwidth}
        \centering
\includegraphics[width=1\columnwidth]{figures/exp2/sequence_length/lineplot_len_ppl_best_gpt2_disc_bow_neutral_prompt_ci_95_errstyle_band.png}
        \caption{}
        \label{subfig:lineplot_length_ppl_np_gpt2}
     \end{subfigure}
     \hfill
     \begin{subfigure}[b]{0.49\textwidth}
        \centering
        \includegraphics[width=1\columnwidth]{figures/exp2/sequence_length/lineplot_len_ppl_best_dgpt_disc_bow_neutral_prompt_ci_95_errstyle_band.png}
        \caption{}
        \label{subfig:lineplot_length_ppl_np_dgpt}
     \end{subfigure}
        \caption{Perplexity. GPT-2-based models (left), DialoGPT-based models (right). Translucent error bands represent 95\% confidence intervals. Plots best viewed in color.}
        \label{fig:lineplots_length_ppl_np_gpt2_dgpt}
\end{figure}

% \len{\textbf{Patterns in Figure~\ref{fig:lineplots_length_ppl_np_gpt2_dgpt}}:}
% \begin{itemize}
%     % \item Perplexity decreases for increasing response length. Longer responses are deemed more plausible by GPT-1.
%     % \item GPT2-Discrim Old significantly worse perplexity at 5\% level for medium length responses. (is also best model setup w.r.t. target-prob improvement over baseline if i'm not mistaken).
% \end{itemize}

Figure~\ref{fig:lineplots_length_ppl_np_gpt2_dgpt} shows a clear downward trend for both sets of PPLM-setups. Irrespective of the underlying language model being used, longer responses are deemed less perplexing, with more certainty, by GPT-1 than shorter ones. It is worth emphasizing that the model with the highest target probability improvement over its relevant baseline, G-Discrim$_{Old}$, is found to produce significantly more perplexing responses than its GPT-2-based counterparts at most length brackets (See Figure~\ref{subfig:lineplot_length_ppl_np_gpt2}). This finding resonates with Figure~\ref{subfig:barplot_ppl_target_prob_np_gpt2_disc_old} and the idea, that especially for old-generation models, increased levels of attribute relevance coincide with worse perplexity.

However, it must be noted that the downward slope of perplexity for increasing response length could be attributable to the nature of calculating perplexity, rather than generation properties of the models. Namely, perplexity essentially averages the sum of the negative exponentiated probabilities $p(\texttt{word} | \texttt{context})$, for every word in a sentence. Because the context increases with every successive word, and larger contexts typically result in less uncertainty, shorter sequences are often given unfairly high perplexities.

\begin{figure}[H]
     \centering
     \begin{subfigure}[b]{0.49\textwidth}
        \centering
\includegraphics[width=1\columnwidth]{figures/exp2/sequence_length/lineplot_len_dist1_best_gpt2_disc_bow__neutral_prompt_ci_95_errstyle_band.png}
        \caption{}
        \label{subfig:lineplot_length_dist1_np_gpt2}
     \end{subfigure}
     \hfill
     \begin{subfigure}[b]{0.49\textwidth}
        \centering
        \includegraphics[width=1\columnwidth]{figures/exp2/sequence_length/lineplot_len_dist1_best_dgpt_disc_bow__neutral_prompt_ci_95_errstyle_band.png}
        \caption{}
        \label{subfig:lineplot_length_dist1_np_dgpt}
     \end{subfigure}
        \caption{Dist-1. GPT-2-based models (left), DialoGPT-based models (right). Translucent error bands represent 95\% confidence intervals. Plots best viewed in color.}
        \label{fig:lineplots_length_dist1_np_gpt2_dgpt}
\end{figure}

\begin{figure}[H]
     \centering
     \begin{subfigure}[b]{0.49\textwidth}
        \centering
\includegraphics[width=1\columnwidth]{figures/exp2/sequence_length/lineplot_len_dist2_best_gpt2_disc_bow_neutral_prompt_ci_95_errstyle_band.png}
        \caption{}
        \label{subfig:lineplot_length_dist2_np_gpt2}
     \end{subfigure}
     \hfill
     \begin{subfigure}[b]{0.49\textwidth}
        \centering
        \includegraphics[width=1\columnwidth]{figures/exp2/sequence_length/lineplot_len_dist2_best_dgpt_disc_bow_neutral_prompt_ci_95_errstyle_band.png}
        \caption{}
        \label{subfig:lineplot_length_dist2_np_dgpt}
     \end{subfigure}
        \caption{Dist-2. GPT-2-based models (left), DialoGPT-based models (right). Translucent error bands represent 95\% confidence intervals. Plots best viewed in color.}
        \label{fig:lineplots_length_dist2_np_gpt2_dgpt}
\end{figure}

\begin{figure}[H]
     \centering
     \begin{subfigure}[b]{0.49\textwidth}
        \centering
\includegraphics[width=1\columnwidth]{figures/exp2/sequence_length/lineplot_len_dist3_best_gpt2_disc_bow__neutral_prompt_ci_95_errstyle_band.png}
        \caption{}
        \label{subfig:lineplot_length_dist3_np_gpt2}
     \end{subfigure}
     \hfill
     \begin{subfigure}[b]{0.49\textwidth}
        \centering
        \includegraphics[width=1\columnwidth]{figures/exp2/sequence_length/lineplot_len_dist3_best_dgpt_disc_bow__neutral_prompt_ci_95_errstyle_band.png}
        \caption{}
        \label{subfig:lineplot_length_dist3_np_dgpt}
     \end{subfigure}
        \caption{Dist-3. GPT-2-based models (left), DialoGPT-based models (right). Translucent error bands represent 95\% confidence intervals. Plots best viewed in color.}
        \label{fig:lineplots_length_dist3_np_gpt2_dgpt}
\end{figure}

% \len{\textbf{Patterns in Figures~\ref{fig:lineplots_length_dist1_np_gpt2_dgpt}, \ref{fig:lineplots_length_dist2_np_gpt2_dgpt}, and \ref{fig:lineplots_length_dist3_np_gpt2_dgpt}}:}

% \begin{itemize}
    % \item Figure~\ref{fig:lineplots_length_dist1_np_gpt2_dgpt} shows that diversity w.r.t. unigrams increases for longer responses. This is most likely just due to the fact that longer sentences have an \textit{a priori} higher probability of repeating words. E.g., stopwords like "the" and "of" are likely to appear multiple times in longer sentences. \len{Is this a valid suggestion?}
    % \item The same figure shows an interesting difference: GPT-2-BoW models generate significantly more diverse responses w.r.t. unigrams for almost every bracket of response length. This could be attributable to BoW-based control altering base-GPT2's generated sentences at the token-level, thus being more likely to preserve the unigram diversity of the unperturbed baseline (G-Baseline's Dist-1 is always in the upper register when looking at the columns in Tables \ref{tab:ctg_results_ws_neutral_prompt_young_models} and \ref{tab:ctg_results_ws_neutral_prompt_old_model}.
    % \item Figure~\ref{fig:lineplots_length_dist2_np_gpt2_dgpt} and shows that variety w.r.t. bigrams makes an initial upward jump between response length of 1-10 and 10-20. Dist-2 then follows a mild downward trend for both GPT2- and DialoGPT-based models. The GPT2-based models' trend is more pronounced and less volatile than DialoGPT's.
    % \item Figure \ref{subfig:lineplot_length_dist2_np_gpt2} shows the BoW-based GPT-2 models also produce significantly more diverse language w.r.t. bigrams for most response lengths.
    % \item Figure~\ref{fig:lineplots_length_dist3_np_gpt2_dgpt} shows similar patterns: initial upward jump in trigram diversity between shortest and second-to-shortest length brackets. GPT-2 models then show a slight downward trend from the 20-30 lengths onward. However, DialoGPT actually show a decreasing increase in trigram diversity for all models.
% \end{itemize}

Figure~\ref{fig:lineplots_length_dist1_np_gpt2_dgpt} shows that diversity w.r.t. unigrams decreases for longer responses. This is most likely due to the fact that longer sentences have an \textit{a priori} higher probability of containing repeated words. E.g., stopwords like "the" and "of" are likely to appear multiple times in longer sentences. The same figure shows that GPT-2-BoW models generate significantly more diverse responses w.r.t. unigrams for almost every bracket of response length. This could be attributable to BoW-based control altering base-GPT2's generated sentences at the token-level, thus being more likely to preserve the unigram diversity of the unperturbed baseline (the GPT-2 baseline's Dist-1 is always in the upper register in Tables \ref{tab:ctg_results_ws_neutral_prompt_young_models} and \ref{tab:ctg_results_ws_neutral_prompt_old_model}).

Figure~\ref{fig:lineplots_length_dist2_np_gpt2_dgpt} and shows that variety w.r.t. bigrams makes an initial upward jump between response length of 1-10 and 10-20. Dist-2 then follows a mild downward trend for both GPT2- and DialoGPT-based models However, detailed inspection of Figure~\ref{subfig:lineplot_length_dist1_np_gpt2} shows that only the discriminator-based setups have a negative slope, whereas the BoW-based setups follow a very slight upward trend. Thus, the BoW-based GPT-2 models produce significantly more diverse language w.r.t. bigrams for most response lengths, attributable to the same reason mentioned above.

Figure~\ref{fig:lineplots_length_dist3_np_gpt2_dgpt} shows similar patterns: an initial upward jump in trigram diversity between shortest and second-to-shortest length brackets. GPT-2 models then show a slight downward trend for the discriminator-based setups from the 20-30 lengths onward, while the BoW-based models become slightly more diverse w.r.t. trigrams.

Overall, it can again be seen that BoW-based models generate detectably more diverse responses (with greater precision), and remain to do so as response length increases. The decreasing diversity of discriminator-based generated responses further confirms that more invasive control during generated impedes textual variety.

% \textbf{Patterns in (obsolete) unprompted results \len{DELETE OR MOVE TO APPENDIX. FROM HERE....}:}
% \begin{itemize}
%     \item \len{NB: These are all patterns observed in the unprompted results. Once prompted results are in, we can interpret them as if length of dialogue response.}
%     \item Increasing sequence length is correlated with decreasing perplexity. Longer sentences are deemed more coherent by GPT-2
%     \item Increasing length seems very slighty positively correlated with average accuracy (and more uncertainty). I.e., longer sequences are, on average, easier to classify, though less precision.
%     \item BoW-based models significantly more distinct  w.r.t. unigrams than dirscrim-based.
%     \item Overall, repetitiveness seems to increase as generated sequences become longer.
%     \item The differences between young and old perplexity are smaller for BoW-based models than for Discrim-based. Same pattern holds for repetitiveness.
% \end{itemize}

% \len{NB: These are all patterns observed in the unprompted results. Once prompted results are in, we can interpret them as if length of dialogue response.}

% The number of tokens being generated in an utterance seems to coincide with noticeable differences in our automated evaluation metrics. It is therefore important to get a clearer picture of how the various measures for fluency and control change for varying sequence lengths. Properly understanding this relationship can inform developers of adaptive dialogue systems about preserving output quality for responses of arbitrary lengths.

% Figure~\ref{fig:ctg_lineplots_len_vs_metrics} presents plots of the relationships between generated sequence length (on the x-axes) and average perplexity (Figure~\ref{fig:ctg_lineplot_len_vs_ppl}), average BERT$_{FT}$ accuracy (Figure~\ref{fig:ctg_lineplot_len_vs_acc}), and average normalized number of distinct unigrams (Figure~\ref{fig:ctg_lineplot_len_vs_dist1}), bigrams (Figure~\ref{fig:ctg_lineplot_len_vs_dist2}), and trigrams (Figure~\ref{fig:ctg_lineplot_len_vs_dist3}).


% Starting with Figure~\ref{fig:ctg_lineplot_len_vs_ppl}, it appears that, for all models, increases in generated utterance length coincide with decreases in perplexity. This is most likely attributable to the nature of calculating perplexity than generation properties of the models. Namely, perplexity essentially averages the sum of the negative exponentiated probabilities $p(\texttt{word} | \texttt{context})$, for every word in a sentence. Because the context increases with every successive word, and larger contexts typically result in less uncertainty, shorter sequences are often given unfairly high perplexities.

% In Figure~\ref{fig:ctg_lineplot_len_vs_acc}, we can see that increasing length seems very slightly positively correlated with average accuracy (and more uncertainty). That is, longer sequences are, on average, slightly easier to classify, though with less precision. This is probably due to the fact that longer sentences contains more information to base predictions on.

% Focusing on repetitive use of unigrams, it appears that BoW-based models generate significantly more diverse utterances than discriminator-based models. 

% Repetitiveness w.r.t. bigrams and trigrams seems to increase as generated sequences become longer.

% Overall, the differences between young and old perplexity are smaller for BoW-based models than for Discriminator-based. Same pattern holds for repetitiveness.

% \len{...TO HERE}


% \begin{figure}[H]
%      \centering
%      \begin{subfigure}[b]{0.49\textwidth}
%         \centering
%         \includegraphics[width=\textwidth]{figures/exp2/lineplot_len_dist1_best_gpt2_disc_bow_ci_90_errstyle_bars_nolegend.png}
%         \caption{...}
%         \label{fig:ctg_lineplot_len_vs_dist1}
%      \end{subfigure}
%     %  \hfill
%      \begin{subfigure}[b]{0.49\textwidth}
%         \centering
%         \includegraphics[width=\textwidth]{figures/exp2/lineplot_len_dist2_best_gpt2_disc_bow_ci_90_errstyle_bars_nolegend.png}
%         \caption{...}
%         \label{fig:ctg_lineplot_len_vs_dist2}
%      \end{subfigure}
%         \caption{}
%         \label{}
%     % \hfill
%      \begin{subfigure}[b]{0.49\textwidth}
%         \centering
%         \includegraphics[width=\textwidth]{figures/exp2/lineplot_len_dist3_best_gpt2_disc_bow_ci_90_errstyle_bars_nolegend.png}
%         \caption{...}
%         \label{fig:ctg_lineplot_len_vs_dist3}
%      \end{subfigure}
%         \caption{}
%         \label{}
% \end{figure}

% \begin{figure}[H]
% \centering
% \includegraphics[width=.4\textwidth]{figures/exp2/lineplot_len_ppl_best_gpt2_disc_bow_ci_90_errstyle_bars.png}\quad
% \includegraphics[width=.4\textwidth]{figures/exp2/lineplot_len_acc_best_gpt2_disc_bow_ci_90_errstyle_bars_nolegend.png}
% % \includegraphics[width=.3\textwidth]{figures/exp2/lineplot_len_dist1_best_gpt2_disc_bow_ci_90_errstyle_bars_nolegend.png}

% \medskip
% \includegraphics[width=.4\textwidth]{figures/exp2/lineplot_len_dist1_best_gpt2_disc_bow_ci_90_errstyle_bars_nolegend.png}\quad
% \includegraphics[width=.4\textwidth]{figures/exp2/lineplot_len_dist2_best_gpt2_disc_bow_ci_90_errstyle_bars_nolegend.png}\quad
% \includegraphics[width=.4\textwidth]{figures/exp2/lineplot_len_dist3_best_gpt2_disc_bow_ci_90_errstyle_bars_nolegend.png}

% \caption{Generated sequence length against various automated metrics.}
% \label{fig:seq_len_vs_auto_metrics}
% \end{figure}

\subsection{The Effects of Prompt Class \len{Think of better title}}
\label{subsec:ctg_anal_prompt_class}

Recall that given a conditioning prompt $\texttt{prompt}$, a predefined style attribute $a$, and some controlled dialogue generation model parameterized by $\theta$, generating a stye-controlled piece of text $\textbf{x}$ entails modeling $p_{\theta}(\textbf{x} | a, \texttt{prompt})$. It is therefore reasonable to expect the output distribution of controlled generation model $p_{\theta}$ to depend (to some extent) on the content and style of the conditioning text, $\texttt{prompt}$. Indeed, the content and style of prompts are found to strongly influence the output of neural text generation models \citep{fan-etal-2018-hierarchical, lester2021power}. Thus, studying the effects of a prompt's age-style (i.e., whether a prompt is considered young, old, or neutral by BERT$_{FT}$) on the style and grammatical quality of PPLM-setups is of great importance, as it could inform developers of adaptive dialogue systems about mitigation of prompt-induced biases.

It is worth mentioning that the effects of prompt-style on PPLM-generation are not considered by \cite{dathathri2019plug} or \cite{madotto-etal-2020-plug}. Studying these effects is an important extension of their methods, as not quantitatively taking into account the effects prompt-style obfuscates the degree to which one can conclude whether attribute-relevance is the result of controlled generation or prompt-induced bias.
% This is actually a non-negligible fault in their research, because you can bias the style of generation by giving it a biased prompt. 

Figures~\ref{fig:catplot_prompt_class_target_prob} and \ref{fig:catplot_prompt_class_ppl} depict the average target probability and perplexity over responses generated by the baseline, best BoW-based model, and discriminator-based model, when prompted with a prompt of either young, neutral, or old style. More specifically, each bar represents a metric (target probability or perplexity) averaged over $N=270$ samples generated by a single model, when presented with five prompts of the same age-style. E.g., the blue bar in Figure~\ref{subfig:catplot_prompt_class_target_prob_gpt2_young} represents the average probability of samples generated by GPT-2 + frequency-based BoW to contain features learned to be young by BERT$_{FT}$, when the model was presented young-sounding prompts. The explicit numerical values of Figures~\ref{fig:catplot_prompt_class_target_prob} and \ref{fig:catplot_prompt_class_ppl} are found in Tables \ref{tab:ctg_results_ws_young_prompt_young_model}, \ref{tab:ctg_results_ws_young_prompt_old_model}, \ref{tab:ctg_results_ws_old_prompt_young_model}, and \ref{tab:ctg_results_ws_old_prompt_old_model}.

Ideally, the neutrally prompted baseline's assigned probability of generating age-specific responses should be around 0.50. Furthermore, a prompt should shift the target probability in the direction of the prompt class, e.g., a young prompt should shift a young-model's target prob upwards, and an old-model's target prob downwards. We know from previous results in Tables \ref{tab:ctg_results_ws_neutral_prompt_young_models} and \ref{tab:ctg_results_ws_neutral_prompt_old_model} that the language model baselines are (in varying degrees) biased towards generating younger sounding responses to neutral prompts. Nevertheless, we should expect the impact of prompt-style to persist, albeit to differing degrees based on the (dis)similarity in style between the prompt and response, and the type of attribute model being used (BoW or discriminator).


Figures~\ref{fig:catplot_prompt_class_target_prob} and \ref{fig:catplot_prompt_class_ppl}, show that the models' target probabilities indeed move accordingly with the prompts' styles. E.g., young-prompted young-model achieves highest young target prob, then neutral prompted, and then old prompted (Figures \ref{subfig:catplot_prompt_class_target_prob_gpt2_young} and \ref{subfig:catplot_prompt_class_target_prob_dgpt_young}). The same pattern holds the other way around: an old-prompted old-model has the highest (old) target probability, then neutrally prompted, and then the young- prompted ones (Figures \ref{subfig:catplot_prompt_class_target_prob_gpt2_old} and \ref{subfig:catplot_prompt_class_target_prob_dgpt_old}). In these last two sub-figures, we also clearly see that the discriminator-based old generation models achieve substantial target probability improvements over the baseline (and BoW-based models), for every style of prompt. By contrast, the young-generation models (Figures \ref{subfig:catplot_prompt_class_target_prob_gpt2_young} and \ref{subfig:catplot_prompt_class_target_prob_dgpt_young}) do not show the same pattern: discriminator-based models achieve similarly subtle improvements in target probability over their baselines as the BoW-based models do. Figure~\ref{subfig:catplot_prompt_class_target_prob_gpt2_young} even shows the discriminator-based models to perform worse than the baseline and BoW-based models.

So the class of the prompt strongly influences the style of the generated response.

Overall, Figures~\ref{fig:catplot_prompt_class_target_prob} and \ref{fig:catplot_prompt_class_ppl} and Tables \ref{tab:ctg_results_ws_young_prompt_young_model}, \ref{tab:ctg_results_ws_young_prompt_old_model}, \ref{tab:ctg_results_ws_old_prompt_young_model}, and \ref{tab:ctg_results_ws_old_prompt_old_model} show that the style of the prompt clearly nudges the assigned probability of containing age-related features in the direction of the prompt's style. Moreover, using strongly young-prompt results in heavily reinforced young-bias for GPT2 and DialoGPT. Similarly, using an older sounding prompt results in a slightly neutralized young-bias for both language models. Additionally, the discriminator-based old models (GPT-2 and DialoGPT) always yield the highest relative improvements over the baselines. However, they fail to do so when attempting to generate younger sounding responses, which could suggest that the stylistic features learned to be young and old by BERT$_{FT}$ lie at different syntactical levels, and are not equally challenging to control for. Finally, the aforementioned tables show that the effects of prompt-style are largely limited to the target probabilities. For perplexity and distinctiveness, the young- and old-prompted results show similar patterns to the neutral-prompt setting: BoW-based models achieve smaller increases in control, but maintain relatively desirable perplexity and diversity. Whereas, the discriminator-based models achieve higher levels of control, at the cost of worse perplexity and Dist-scores.

\len{\textbf{TODO - Add table with used prompts and their assigned target probabilities.}}

\begin{figure}[H]
     \centering
     \begin{subfigure}[b]{0.48\textwidth}
        \centering
        \includegraphics[width=1\columnwidth]{figures/exp2/prompt_class/catplot_prompt_class_target_prob_gpt2_young_models.png}
        \caption{}
        \label{subfig:catplot_prompt_class_target_prob_gpt2_young}
     \end{subfigure}
    %  \hfill
     \quad
     \begin{subfigure}[b]{0.48\textwidth}
        \centering
        \includegraphics[width=1\columnwidth]{figures/exp2/prompt_class/catplot_prompt_class_target_prob_gpt2_old_models.png}
        \caption{}
        \label{subfig:catplot_prompt_class_target_prob_gpt2_old}
     \end{subfigure}
    \medskip
    \begin{subfigure}[b]{0.48\textwidth}
        \centering
        \includegraphics[width=1\columnwidth]{figures/exp2/prompt_class/catplot_prompt_class_target_prob_dgpt_young_models.png}
        \caption{}
        \label{subfig:catplot_prompt_class_target_prob_dgpt_young}
     \end{subfigure}
    %  \hfill
    \quad
     \begin{subfigure}[b]{0.48\textwidth}
        \centering
        \includegraphics[width=1\columnwidth]{figures/exp2/prompt_class/catplot_prompt_class_target_prob_dgpt_old_models.png}
        \caption{}
        \label{subfig:catplot_prompt_class_target_prob_dgpt_old}
     \end{subfigure}
    \caption{Target probability. The plots are best viewed in color. \len{TODO: (1) Change Target probability labels to Young or Old probability. The term target probability doesn't hold for baseline models.}}
    \label{fig:catplot_prompt_class_target_prob}
\end{figure}

% \len{\textbf{Patterns in Figure \ref{fig:catplot_prompt_class_target_prob}}:}
% \begin{itemize}
%     % \item The style of the prompt clearly nudges the assigned probability in the direction of the prompt's class.
%     % \item The discrim. old models (GPT-2 and DialoGPT) always yield the highest relative improvements over the baselines.
%     % \item Ideally, you would want (1) the neutrally prompted baseline to be around 0.50. (2) A young or old prompt should shift the target probability in the direction of the prompt class; i.e., a young prompt should shift a young-model's target prob upwards, and an old-model's target prob downwards. (3) ...
%     % \item The models target probabilities (baseline, bow, and discrim) move accordingly with the prompt's class: young prompted young-model has highest young target prob, then neutral prompted, and then old prompted. Same pattern goes the other way around: an old prompted old-model has the highest (old) target prob, then neutral prompted, and then young prompted.
%     % \item So the class of the prompt strongly influences the style of the generated response.
%     % \item \len{Make sure to emphasize that this is also something that hasn't been studied by \cite{dathathri2019plug} or \cite{madotto-etal-2020-plug}; i.e., they never study the effect of their prompt style or sentiment on the style/attribute relevance of generation. This is actually a non-negligible fault in their research, because you can bias the style of generation by giving it a biased prompt. Furthermore, not quantitatively taking into account the prompt style or effect obfuscates the degree to which one can conclude whether attribute relevance is the result of controlled generation or prompt-bias.}
% \end{itemize}

\begin{figure}[H]
     \centering
     \begin{subfigure}[b]{0.48\textwidth}
        \centering
        \includegraphics[width=1\columnwidth]{figures/exp2/prompt_class/catplot_prompt_class_ppl_gpt2_young_models.png}
        \caption{}
        \label{subfig:catplot_prompt_class_ppl_gpt2_young}
     \end{subfigure}
    %  \hfill
     \quad
     \begin{subfigure}[b]{0.48\textwidth}
        \centering
        \includegraphics[width=1\columnwidth]{figures/exp2/prompt_class/catplot_prompt_class_ppl_gpt2_old_models.png}
        \caption{}
        \label{subfig:catplot_prompt_class_ppl_gpt2_old}
     \end{subfigure}
    \medskip
    \begin{subfigure}[b]{0.48\textwidth}
        \centering
        \includegraphics[width=1\columnwidth]{figures/exp2/prompt_class/catplot_prompt_class_ppl_dgpt_young_models.png}
        \caption{}
        \label{subfig:catplot_prompt_class_ppl_dgpt_young}
     \end{subfigure}
    %  \hfill
    \quad
     \begin{subfigure}[b]{0.48\textwidth}
        \centering
        \includegraphics[width=1\columnwidth]{figures/exp2/prompt_class/catplot_prompt_class_ppl_dgpt_old_models.png}
        \caption{}
        \label{subfig:catplot_prompt_class_ppl_dgpt_old}
     \end{subfigure}
    \caption{Perplexity. Mind the differences in scale between the $y$-axes. The plots are best viewed in color. \len{TODO - Move to Appendix}}
    \label{fig:catplot_prompt_class_ppl}
\end{figure}



% Initial observations and interpretations of Tables \ref{tab:ctg_results_ws_young_prompt_young_model}, \ref{tab:ctg_results_ws_young_prompt_old_model}, \ref{tab:ctg_results_ws_old_prompt_young_model}, \ref{tab:ctg_results_ws_old_prompt_old_model}.

% \begin{itemize}
%     % \item Using strongly young-prompt results in heavily reinforced young-bias for GPT2 and DialoGPT.
%     % \item Similarly, old-prompt results in slightly neutralized young-bias for both language models.
%     \item Neutrally prompted patterns persist...
%     \begin{itemize}
%         \item Trade-off between control and fluency + diversity.
%         \item BoW achieves smaller increases in control, but leaves fluency and dist intact.
%         \item Discrim higher levels of control increase wrt baseline, worse quality of text wrt fluency and diversity.
%     \end{itemize}
%     \item Overall, stylistic aspects of prompts heavily influence controllability of response.
% \end{itemize}

\begin{table*}[h]
    \centering
    \begin{tabular}{l | c c c c | c c}
    \toprule
    \textbf{Model} & \textbf{ppl.} & \textbf{Dist-1} & \textbf{Dist-2} & \textbf{Dist-3} & $\boldsymbol{\bar{P}_Y}$ & \textbf{Acc.}\\
    % -plus)}$\\
     & $\downarrow$ better & $\uparrow$ better & $\uparrow$ better & $\uparrow$ better & $\uparrow$ better & $\uparrow$ better\\
    \midrule
    \midrule
    G-baseline & \textcolor{blue}{28.05} ($\pm$6.12) & 0.85 ($\pm$0.13) & 0.91 ($\pm$0.08) & 0.88 ($\pm$0.08) & 0.80 ($\pm$0.33) & -\\
    G-100MCW & \textbf{27.71} ($\pm$6.20) & 0.85 ($\pm$0.12) & 0.91 ($\pm$0.09) & 0.88 ($\pm$0.09) & 0.75 ($\pm$0.37) & -\\
    \midrule
    G-B$_{FB, Y}$ & 28.81 ($\pm$7.09) & 0.86 ($\pm$0.12) & \textbf{0.92} ($\pm$0.08) & \textbf{0.89} ($\pm$0.08) & 0.82 ($\pm$0.32) & 83.3\%\\
    G-B$_{100MIU, Y}$ & 28.49 ($\pm$6.49) & 0.86 ($\pm$0.12) & 0.91 ($\pm$0.08) & 0.88 ($\pm$0.08) & 0.83 ($\pm$0.32) & 83.0\%\\
    \midrule
    G-D$_{Y}$ & 39.32 ($\pm$37.49) & 0.84 ($\pm$0.21) & 0.61 ($\pm$0.40) & 0.57 ($\pm$0.40) & 0.70 ($\pm$0.40) & 70.7\%\\
    \midrule
    \midrule
    D-baseline & 36.69 ($\pm$9.11) & 0.87 ($\pm$0.10) & \textcolor{blue}{0.91} ($\pm$0.06) & 0.87 ($\pm$0.08) & \textcolor{blue}{0.90} ($\pm$0.24) & -\\
    D-100MCW & 36.93 ($\pm$9.18) & 0.86 ($\pm$0.11) & \textcolor{blue}{0.91} ($\pm$0.06) & \textcolor{blue}{0.88} ($\pm$0.07) & 0.90 ($\pm$0.25) & -\\
    \midrule
    D-B$_{FB, Y}$ & 37.35 ($\pm$8.60) & \textcolor{blue}{0.88} ($\pm$0.10) & \textcolor{blue}{0.91} ($\pm$0.06) & 0.87 ($\pm$0.08) & 0.90 ($\pm$0.26) & \textcolor{blue}{90.0\%}\\
    D-B$_{100MIU, Y}$ & 37.87 ($\pm$8.32) & \textcolor{blue}{0.88} ($\pm$0.10) & 0.91 ($\pm$0.07) & 0.87 ($\pm$0.09) & \textbf{0.91} ($\pm$0.24) & \textbf{92.6\%}\\
    \midrule
    D-D$_{Y}$ & 39.22 ($\pm$14.96) & \textbf{0.89} ($\pm$0.12) & 0.86 ($\pm$0.19) & 0.79 ($\pm$0.23) & 0.89 ($\pm$0.25) & 91.1\%\\
    \bottomrule
    \end{tabular}
    \caption{\len{Young prompt - Young models} Results of age-controlled language generation. Perplexity is perplexity w.r.t. GPT-1. Dist-n is number of distinct n-grams normalized by text length, as a measure of diversity. Acc. is the best BERT model's accuracy when classifying the row's samples.}
    \label{tab:ctg_results_ws_young_prompt_young_model}
\end{table*}

\begin{table*}[h]
    \centering
    \begin{tabular}{l | c c c c | c c}
    \toprule
    \textbf{Model} & \textbf{ppl.} & \textbf{Dist-1} & \textbf{Dist-2} & \textbf{Dist-3} & $\boldsymbol{\bar{P}_O}$ & \textbf{Acc.}\\
    % -plus)}$\\
     & $\downarrow$ better & $\uparrow$ better & $\uparrow$ better & $\uparrow$ better & $\uparrow$ better & $\uparrow$ better\\
    \midrule
    \midrule
    G-baseline & \textcolor{blue}{28.05} ($\pm$6.12) & 0.85 ($\pm$0.13) & 0.91 ($\pm$0.08) & 0.88 ($\pm$0.08) & 0.20 ($\pm$0.33) & -\\
    G-100MCW & \textbf{27.71} ($\pm$6.20) & 0.85 ($\pm$0.12) & 0.91 ($\pm$0.09) & 0.88 ($\pm$0.09) & 0.25 ($\pm$0.37) & -\\
    \midrule
    G-B$_{FB, O}$ & 28.54 ($\pm$6.45) & 0.86 ($\pm$0.12) & \textbf{0.92} ($\pm$0.08) & \textbf{0.89} ($\pm$0.08) & 0.23 ($\pm$0.36) & 22.6\%\\
    G-B$_{100MIU, O}$ & 28.18 ($\pm$5.70) & 0.87 ($\pm$0.11) & \textbf{0.92} ($\pm$0.08) & \textcolor{blue}{0.89} ($\pm$0.09) & 0.21 ($\pm$0.34) & 21.5\%\\
    \midrule
    G-D$_{O}$ & 85.40 ($\pm$150.28) & 0.67 ($\pm$0.30) & 0.62 ($\pm$0.31) & 0.62 ($\pm$0.32) & \textbf{0.71} ($\pm$0.40) & \textbf{70.5\%}\\
    \midrule
    \midrule
    D-baseline & 36.69 ($\pm$9.11) & \textcolor{blue}{0.87} ($\pm$0.10) & 0.91 ($\pm$0.06) & 0.87 ($\pm$0.08) & 0.10 ($\pm$0.24) & -\\
    D-100MCW & 36.93 ($\pm$9.18) & 0.86 ($\pm$0.11) & 0.91 ($\pm$0.06) & 0.88 ($\pm$0.07) & 0.10 ($\pm$0.25) & -\\
    \midrule
    D-B$_{FB, O}$ & 37.25 ($\pm$9.45) & 0.87 ($\pm$0.11) & 0.91 ($\pm$0.06) & 0.87 ($\pm$0.08) & 0.12 ($\pm$0.29) & 11.1\%\\
    D-B$_{100MIU, O}$ & 37.04 ($\pm$8.78) & \textbf{0.88} ($\pm$0.10) & \textcolor{blue}{0.91} ($\pm$0.05) & 0.88 ($\pm$0.07) & 0.15 ($\pm$0.32) & 15.2\%\\
    \midrule
    D-D$_{O}$ & 38.46 ($\pm$14.91) & 0.82 ($\pm$0.15) & 0.87 ($\pm$0.15) & 0.83 ($\pm$0.17) & \textcolor{blue}{0.48} ($\pm$0.44) & \textcolor{blue}{47.4\%}\\
    \bottomrule
    \end{tabular}
    \caption{\len{Young prompt - Old models} Results of age-controlled language generation. Perplexity is perplexity w.r.t. GPT-1. Dist-n is number of distinct n-grams normalized by text length, as a measure of diversity. Acc. is the best BERT model's accuracy when classifying the row's samples.}
    \label{tab:ctg_results_ws_young_prompt_old_model}
\end{table*}

\begin{table*}[h]
    \centering
    \begin{tabular}{l | c c c c | c c}
    \toprule
    \textbf{Model} & \textbf{ppl.} & \textbf{Dist-1} & \textbf{Dist-2} & \textbf{Dist-3} & $\boldsymbol{\bar{P}_Y}$ & \textbf{Acc.}\\
    % -plus)}$\\
     & $\downarrow$ better & $\uparrow$ better & $\uparrow$ better & $\uparrow$ better & $\uparrow$ better & $\uparrow$ better\\
    \midrule
    \midrule
    G-baseline & \textcolor{blue}{29.34} ($\pm$10.30) & 0.86 ($\pm$0.09) & \textbf{0.94} ($\pm$0.04) & \textcolor{blue}{0.90} ($\pm$0.06) & 0.60 ($\pm$0.43) & -\\
    G-100MCW & \textbf{29.14} ($\pm$10.11) & 0.86 ($\pm$0.10) & \textcolor{blue}{0.93} ($\pm$0.04) & \textcolor{blue}{0.90} ($\pm$0.06) & 0.60 ($\pm$0.44) & -\\
    \midrule
    G-B$_{Y, FB}$ & 29.61 ($\pm$10.28) & 0.86 ($\pm$0.10) & 0.93 ($\pm$0.04) & \textbf{0.91} ($\pm$0.06) & 0.62 ($\pm$0.43) & 61.1\%\\
    G-B$_{Y, 100MIU}$ & 29.51 ($\pm$0.09) & \textcolor{blue}{0.87} ($\pm$0.09) & 0.93 ($\pm$0.05) & \textcolor{blue}{0.90} ($\pm$0.06) & 0.68 ($\pm$0.42) & 68.5\%\\
    \midrule
    G-D$_{Y}$ & 32.34 ($\pm$19.88) & 0.77 ($\pm$0.20) & 0.84 ($\pm$0.19) & 0.80 ($\pm$0.23) & 0.65 ($\pm$0.43) & 65.4\%\\
    \midrule
    \midrule
    D-baseline & 38.18 ($\pm$12.03) & 0.86 ($\pm$0.12) & 0.90 ($\pm$0.08) & 0.86 ($\pm$0.09) & 0.72 ($\pm$0.38) & -\\
    D-100MCW & 37.73 ($\pm$11.88) & 0.85 ($\pm$0.13) & 0.90 ($\pm$0.08) & 0.86 ($\pm$0.09) & 0.73 ($\pm$0.39) & -\\
    \midrule
    D-B$_{Y, FB}$ & 38.24 ($\pm$11.53) & 0.86 ($\pm$0.12) & 0.90 ($\pm$0.08) & 0.86 ($\pm$0.10) & 0.81 ($\pm$0.34) & \textcolor{blue}{82.6\%}\\
    D-B$_{Y, 100MIU}$ & 38.66 ($\pm$11.57) & 0.85 ($\pm$0.12) & 0.90 ($\pm$0.07) & 0.86 ($\pm$0.09) & \textcolor{blue}{0.81} ($\pm$0.33) & 80.7\%\\
    \midrule
    D-D$_{Y}$ & 42.93 ($\pm$20.18) & \textbf{0.90} ($\pm$0.14) & 0.79 ($\pm$0.22) & 0.68 ($\pm$0.28) & \textbf{0.84} ($\pm$0.30) & \textbf{85.2\%}\\
    \bottomrule
    \end{tabular}
    \caption{\len{Old prompt - Young model} Results of age-controlled language generation. Perplexity is perplexity w.r.t. GPT-1. Dist-n is number of distinct n-grams normalized by text length, as a measure of diversity. Acc. is the best BERT model's accuracy when classifying the row's samples.}
    \label{tab:ctg_results_ws_old_prompt_young_model}
\end{table*}

\begin{table*}[h]
    \centering
    \begin{tabular}{l | c c c c | c c}
    \toprule
    \textbf{Model} & \textbf{ppl.} & \textbf{Dist-1} & \textbf{Dist-2} & \textbf{Dist-3} & $\boldsymbol{\bar{P}_O}$ & \textbf{Acc.}\\
    % -plus)}$\\
     & $\downarrow$ better & $\uparrow$ better & $\uparrow$ better & $\uparrow$ better & $\uparrow$ better & $\uparrow$ better\\
    \midrule
    \midrule
    G-baseline & 29.34 ($\pm$10.30) & \textcolor{blue}{0.86} ($\pm$0.09) & \textbf{0.94} ($\pm$0.04) & \textbf{0.90} ($\pm$0.06) & 0.40 ($\pm$0.43) & -\\
    G-100MCW & 29.14 ($\pm$10.11) & 0.86 ($\pm$0.10) & \textcolor{blue}{0.93} ($\pm$0.04) & \textbf{0.90} ($\pm$0.06) & 0.40 ($\pm$0.44) & -\\
    \midrule
    G-B$_{O, FB}$ & \textbf{28.81} ($\pm$10.10) & 0.86 ($\pm$0.10) & 0.93 ($\pm$0.05) & \textbf{0.90} ($\pm$0.06) & 0.41 ($\pm$0.43) & 41.1\%\\
    G-B$_{O, 100MIU}$ & 29.05 ($\pm$9.80) & \textcolor{blue}{0.86} ($\pm$0.09) & \textcolor{blue}{0.93} ($\pm$0.04) & \textbf{0.90} ($\pm$0.06) & 0.40 ($\pm$0.43) & 39.6\%\\
    \midrule
    G-D$_{O}$ & 95.21 ($\pm$174.42) & 0.65 ($\pm$0.27) & 0.78 ($\pm$0.18) & 0.78 ($\pm$0.18) & \textbf{0.90} ($\pm$0.25) & \textbf{90.3\%}\\
    \midrule
    \midrule
    D-baseline & 38.18 ($\pm$12.03) & 0.86 ($\pm$0.12) & 0.90 ($\pm$0.08) & 0.86 ($\pm$0.09) & 0.28 ($\pm$0.38) & -\\
    D-100MCW & 37.73 ($\pm$11.88) & 0.85 ($\pm$0.13) & 0.90 ($\pm$0.08) & 0.86 ($\pm$0.09) & 0.27 ($\pm$0.39) & -\\
    \midrule
    D-B$_{O, FB}$ & 37.80 ($\pm$11.74) & 0.86 ($\pm$0.12) & 0.90 ($\pm$0.07) & \textcolor{blue}{0.87} ($\pm$0.08) & 0.28 ($\pm$0.39) & 29.3\%\\
    D-B$_{O, 100MIU}$ & 36.93 ($\pm$11.68) & \textbf{0.87} ($\pm$0.12) & 0.90 ($\pm$0.09) & 0.86 ($\pm$0.09) & 0.31 ($\pm$0.41) & 29.6\%\\
    \midrule
    D-D$_{O}$ & 40.08 ($\pm$16.77) & 0.85 ($\pm$0.14) & 0.88 ($\pm$0.10) & 0.83 ($\pm$0.14) & \textcolor{blue}{0.61} ($\pm$0.42) & \textcolor{blue}{61.1\%}\\
    \bottomrule
    \end{tabular}
    \caption{\len{Old prompt - Old models} Results of age-controlled language generation. Perplexity is perplexity w.r.t. GPT-1. Dist-n is number of distinct n-grams normalized by text length, as a measure of diversity. Acc. is the best BERT model's accuracy when classifying the row's samples.}
    \label{tab:ctg_results_ws_old_prompt_old_model}
\end{table*}


% \subsubsection{Quantitative 3: The Effects of PPLM-Parameters on Fluency and Control}

% \begin{itemize}
%     \item \textit{Plot and examine the relationship between fluency and control, and various PPLM-parameters (step-size, number of iterations, temperature, top $k$, gamma, KL-scale).}
%     \item \textit{Which patterns do you observe?}
%     \item \len{How much sense does it make to study this, though? Is that the purpose of my thesis? Hasn't this been studied enough in the PPLM-paper? Which parameters do I choose?}
% \end{itemize}

\subsection{BERT$_{FT}$ Attention Patterns}

\begin{itemize}
    \item \textit{\textbf{NB:} This is more relevant to the classification experiments, than to the controlled generation experiments.}
    
    \item \textit{Use BertViz \citep{vig-2019-multiscale} to visualize what parts of sequences BERT's transformer heads and neurons are focusing on.}
    
    \item \url{https://github.com/jessevig/bertviz}
    
    \item \url{https://towardsdatascience.com/openai-gpt-2-understanding-language-generation-through-visualization-8252f683b2f8}
    
    \item Discussion about interpretability of attention. See Attention is not explanation, Attention is not not explanation, and The Elephant in the Interpretability Room.
\end{itemize}

\len{\textbf{Proposed structure:}}

\begin{itemize}
    \item We use BertViz \citep{vig-2019-multiscale} to visualize and study the attention mechanism maps of BERT$_{FT}$.
    \item An attention weight is naturally interpretable as how much a particular token will be weighted when computing the next representation for the current token \citep{clark-etal-2019-bert}.
    \item However, there is debate in NLP literature about the interpretability of attention mechanisms \len{References to Attention is not explanation, Attention is not \textbf{not} explanation, and The Elephant in the Interpretability Room.}
    \item We refer to an attention head as Head \texttt{layer}-\texttt{head}, where \texttt{layer} and \texttt{head} range from 0 to 11: e.g., Head 3-1 refers to Layer 3, Head 1.
    \item There are $12 \times 12 = 144$ attention heads in BERT-base-uncased.
    \item BERT's pre-processing adds a special token \texttt{[CLS]} to the beginning of an input sequence, and another special token \texttt{[SEP]} to the end. If an input sequence consists of multiple sentences (e.g., a question-answer, or prompt-response input), \texttt{[SEP]} tokens are also used to separate the sentences.
    \item Heads in the same layer tend to exhibit similar behaviors.
    \item \len{Clearly explain how to read the plots and what they do and don't imply.}
    \item \len{Clearly explain that the PPLM-method doesn't change the attention mechanisms of the underlying language models, so visualizing their attention maps would not reveal any age-related attention patterns, whereas BERT$_{FT}$ has been fine-tuned to detect age-related patterns, so visualizing its attention maps makes sense.}
    \item Recurring patterns are observed.
    \item Figures shows examples of specific heads showing similar attention patterns.
    \item E.g., broad attention, virtually equally dispersed among tokens.
    \item Focus on next token.
    \item Focus on special BERT-tokens, \texttt{[CLS]} and \texttt{[SEP]}.
    \item Attention heads in layer 9 seems to exhibit patterns that attend to age-related features.
    \item Focuses on tokens that are associated with their respective age groups (based on empirical observations of this thesis and relevant literature).
    \item E.g., focuses on \textit{awesome}, \textit{facebook}, \textit{cool}, \textit{wanna}, \textit{gonna}, and swear word.
    \item Age-related attention pattern much less pronounced for older age group (could explain why classification performance also worse for that target age), but sensible patterns do seem to be subtly present: focus on \textit{grandfather}, \textit{president}, \textit{workers}, \textit{union}, \textit{fellow}, \textit{greetings}
\end{itemize}

We use visualizations of attention mechanisms \citep{vig-2019-multiscale} in BERT$_{FT}$'s attention heads to analyze recurring patterns when assigning target probabilities to generated prompt-response pairs. An attention weight can be interpreted as an indication of how important particular token is when producing the next representation of the current token \citep{DBLP:journals/corr/BahdanauCB14, clark-etal-2019-bert}. Please note that the PPLM-method does not change the attention weights of the underlying language models, so visualizing the attention weights of our generation models would not reveal any age-related attention patterns. By contrast, BERT$_{FT}$ is fine-tuned to detect age-related patterns, so visualizing its attention maps can inform us about which features are important when assigning a target-age probability to a generated sentence. \len{Maybe add the following sentences to the discussion.} Despite this seemingly natural interpretation, there is much debate about the validity of using attention mechanisms (as opposed to, e.g., saliency methods) as explanations for model output \citep{jain-wallace-2019-attention, wiegreffe-pinter-2019-attention, bastings-filippova-2020-elephant}. However, as \cite{vig-2019-multiscale} and \cite{clark-etal-2019-bert} suggest, attention weight visualization can be used tentatively as a complementary analysis tool to add to sets of different analysis methods to inform researchers about, e.g., possible linguistic patterns that may be attended to by attention-based models. \len{...to here.}

To provide a clear reading experience of the analyses presented below, we revise a few important concepts about BERT and how to read attention weight visualizations. During pre-processing, BERT tokenizes the input text and adds a special token \texttt{[CLS]} to the beginning of the text, and another special token \texttt{[SEP]} is appended to the text. If an input sequence consists of multiple sentences (e.g., a question-answer, or prompt-response input), \texttt{[SEP]} tokens are also used to separate the sentences. BERT$_{FT}$ is a fine-tuned version of BERT-base-uncased \citep{devlin-etal-2019-bert}, which consists of 12 layers of 12 attention heads. We refer to a specific attention head as Head \texttt{layer\_number}-\texttt{head\_number}, where \texttt{layer\_number} and \texttt{head\_number} range from 0 to 11. E.g., Head 3-1 refers to Head 1 in Layer 3. Furthermore, attention weights are visualized as colored lines between tokens of the input sequence, where a thicker line corresponds to a greater attention weight, and the color represents the layer in which the head is present. The input text is displayed twice in parallel columns, to make visualizations of self-attention possible (visualized as lines between identical tokens in the same positions). Because BERT is designed to be deeply bidirectional, tokens can also attend to tokens in previous positions in the input text.

We show attention visualizations of BERT$_{FT}$ when processing prompt-response pairs cherry-picked from the age-targeted prompted results, presented in Tables \ref{tab:ctg_results_ws_young_prompt_young_model} and \ref{tab:ctg_results_ws_old_prompt_old_model}. The generated sequences are chosen to display more pronounced examples of recurring attention patterns. We show two young-targeted generated responses to younger sounding prompts in Figures \ref{fig:bertviz_model_view_ypr1} and \ref{fig:bertviz_model_view_ypr2}, and two old-targeted generated responses to older sounding prompts in Figures \ref{fig:bertviz_model_view_opr1} and \ref{fig:bertviz_model_view_opr2}. All prompts and responses received target probabilities from BERT$_{FT}$ of at least 95\%.

Heads in the same layer have the tendency to attend to similar patterns and linguistic phenomena \cite{clark-etal-2019-bert}. Our analyses seem to confirm this behavior, as recurring patterns are observed for heads in the same layers. For instance, we see (in Figures \ref{subfig:bertviz_model_view_ypr1_broad}, \ref{subfig:bertviz_model_view_ypr2_broad}, \ref{subfig:bertviz_model_view_opr1_broad}, and \ref{subfig:bertviz_model_view_opr2_broad}) that heads in the last layer tend to broadly disperse attention among all tokens. Recurring patterns are also observed in earlier layers, such as Head 2-9 attending to the next token in the sequence (Figures \ref{subfig:bertviz_model_view_ypr1_next}, \ref{subfig:bertviz_model_view_ypr2_next}, \ref{subfig:bertviz_model_view_opr1_next}, and \ref{subfig:bertviz_model_view_opr2_next}), and Head 4-4 attending to the special tokens (Figures \ref{subfig:bertviz_model_view_ypr1_special}, \ref{subfig:bertviz_model_view_ypr2_special}, \ref{subfig:bertviz_model_view_opr1_special}, and \ref{subfig:bertviz_model_view_opr2_special}). Certain heads also seem to pay special attention to age-related linguistic features, specifically certain tokens associated with an age group (mentioned in Section \ref{subsec:ctg_anal_qualitative}). This is most noticeable in Head 9-0 (Figures \ref{subfig:bertviz_model_view_ypr1_age}, \ref{subfig:bertviz_model_view_ypr2_age}, \ref{subfig:bertviz_model_view_opr1_age}, and \ref{subfig:bertviz_model_view_opr2_age}), which seems to consistently devote the majority of its attention to tokens that are found to be indicative of age. For instance, Head 9-0 attends strongly to younger sounding tokens like \textit{facebook}, \textit{awesome}, \textit{cool}, and slang and swear words in Figures \ref{subfig:bertviz_model_view_ypr1_age} and \ref{subfig:bertviz_model_view_ypr2_age}. And the same attention head then focuses strongly on tokens associated with older age in Figures \ref{subfig:bertviz_model_view_opr1_age} and \ref{subfig:bertviz_model_view_opr2_age}: e.g., \textit{workers}, \textit{union}, \textit{greetings}, or \textit{fellow}.




\begin{figure}[H]
     \centering
     \begin{subfigure}[b]{0.22\textwidth}
        \centering
        \includegraphics[width=1\columnwidth]{figures/exp2/bertviz/ypr1/bert_model_view_ypr1_layer_11_head_1.png}
        \captionsetup{font=footnotesize,labelfont=footnotesize}
        \caption{Attends broadly.}
        \label{subfig:bertviz_model_view_ypr1_broad}
     \end{subfigure}
    %  \hfill
     \quad
     \begin{subfigure}[b]{0.22\textwidth}
        \centering
        \includegraphics[width=1\columnwidth]{figures/exp2/bertviz/ypr1/bert_model_view_ypr1_layer_2_head_9.png}
        \captionsetup{font=footnotesize,labelfont=footnotesize}
        \caption{Attends to next token.}
        \label{subfig:bertviz_model_view_ypr1_next}
     \end{subfigure}
    \quad
    \begin{subfigure}[b]{0.22\textwidth}
        \centering
        \includegraphics[width=1\columnwidth]{figures/exp2/bertviz/ypr1/bert_model_view_ypr1_layer_4_head_4.png}
        \captionsetup{font=footnotesize,labelfont=footnotesize}
        \caption{Attends to special tokens.}
        \label{subfig:bertviz_model_view_ypr1_special}
     \end{subfigure}
    %  \hfill
    \quad
     \begin{subfigure}[b]{0.22\textwidth}
        \centering
        \includegraphics[width=1\columnwidth]{figures/exp2/bertviz/ypr1/bert_model_view_ypr1_layer_9_head_0.png}
        \captionsetup{font=footnotesize,labelfont=footnotesize}
        \caption{Age-related attention.}
        \label{subfig:bertviz_model_view_ypr1_age}
     \end{subfigure}
    \caption{Attention weights visualizations of four of BERT$_{FT}$'s attentions heads and the patterns to which they presumably attend when processing representations for a cherry-picked prompt-response pair generated by \textbf{young}-targeted GPT2-Discrim.}
    \label{fig:bertviz_model_view_ypr1}
\end{figure}

\begin{figure}[H]
     \centering
     \begin{subfigure}[b]{0.22\textwidth}
        \centering
        \includegraphics[width=1\columnwidth]{figures/exp2/bertviz/ypr2/bert_model_view_ypr2_layer_11_head_1.png}
        \captionsetup{font=footnotesize,labelfont=footnotesize}
        \caption{Attends broadly.}
        \label{subfig:bertviz_model_view_ypr2_broad}
     \end{subfigure}
    %  \hfill
     \quad
     \begin{subfigure}[b]{0.22\textwidth}
        \centering
        \includegraphics[width=1\columnwidth]{figures/exp2/bertviz/ypr2/bert_model_view_ypr2_layer_2_head_9.png}
        \captionsetup{font=footnotesize,labelfont=footnotesize}
        \caption{Attends to next token.}
        \label{subfig:bertviz_model_view_ypr2_next}
     \end{subfigure}
    \quad
    \begin{subfigure}[b]{0.22\textwidth}
        \centering
        \includegraphics[width=1\columnwidth]{figures/exp2/bertviz/ypr2/bert_model_view_ypr2_layer_4_head_4.png}
        \captionsetup{font=footnotesize,labelfont=footnotesize}
        \caption{Attends to special tokens.}
        \label{subfig:bertviz_model_view_ypr2_special}
     \end{subfigure}
    %  \hfill
    \quad
     \begin{subfigure}[b]{0.22\textwidth}
        \centering
        \includegraphics[width=1\columnwidth]{figures/exp2/bertviz/ypr2/bert_model_view_ypr2_layer_9_head_0.png}
        \captionsetup{font=footnotesize,labelfont=footnotesize}
        \caption{Age-related attention.}
        \label{subfig:bertviz_model_view_ypr2_age}
     \end{subfigure}
    \caption{Attention weights visualizations of four of BERT$_{FT}$'s attentions heads and the patterns to which they presumably attend when processing representations for a cherry-picked prompt-response pair generated by \textbf{young}-targeted GPT2-BoW$_{100MIU}$. \len{\textbf{TODO} - Check if it's ok to have a swear word in this figure.}}
    \label{fig:bertviz_model_view_ypr2}
\end{figure}

\begin{figure}[H]
     \centering
     \begin{subfigure}[b]{0.22\textwidth}
        \centering
        \includegraphics[width=1\columnwidth]{figures/exp2/bertviz/opr1/bert_model_view_opr1_layer_11_head_1.png}
        \captionsetup{font=footnotesize,labelfont=footnotesize}
        \caption{Attends broadly.}
        \label{subfig:bertviz_model_view_opr1_broad}
     \end{subfigure}
    %  \hfill
     \quad
     \begin{subfigure}[b]{0.22\textwidth}
        \centering
        \includegraphics[width=1\columnwidth]{figures/exp2/bertviz/opr1/bert_model_view_opr1_layer_2_head_9.png}
        \captionsetup{font=footnotesize,labelfont=footnotesize}
        \caption{Attends to next token.}
        \label{subfig:bertviz_model_view_opr1_next}
     \end{subfigure}
    \quad
    \begin{subfigure}[b]{0.22\textwidth}
        \centering
        \includegraphics[width=1\columnwidth]{figures/exp2/bertviz/opr1/bert_model_view_opr1_layer_4_head_4.png}
        \captionsetup{font=footnotesize,labelfont=footnotesize}
        \caption{Attends to special tokens.}
        \label{subfig:bertviz_model_view_opr1_special}
     \end{subfigure}
    %  \hfill
    \quad
     \begin{subfigure}[b]{0.22\textwidth}
        \centering
        \includegraphics[width=1\columnwidth]{figures/exp2/bertviz/opr1/bert_model_view_opr1_layer_9_head_0.png}
        \captionsetup{font=footnotesize,labelfont=footnotesize}
        \caption{Age-related attention.}
        \label{subfig:bertviz_model_view_opr1_age}
     \end{subfigure}
    \caption{Attention weights visualizations of four of BERT$_{FT}$'s attentions heads and the patterns to which they presumably attend when processing representations for a cherry-picked prompt-response pair generated by \textbf{old}-targeted GPT2-Discrim.}
    \label{fig:bertviz_model_view_opr1}
\end{figure}

\begin{figure}[H]
     \centering
     \begin{subfigure}[b]{0.22\textwidth}
        \centering
        \includegraphics[width=1\columnwidth]{figures/exp2/bertviz/opr2/bert_model_view_opr2_layer_11_head_1.png}
        \captionsetup{font=footnotesize,labelfont=footnotesize}
        \caption{Attends broadly.}
        \label{subfig:bertviz_model_view_opr2_broad}
     \end{subfigure}
    %  \hfill
     \quad
     \begin{subfigure}[b]{0.22\textwidth}
        \centering
        \includegraphics[width=1\columnwidth]{figures/exp2/bertviz/opr2/bert_model_view_opr2_layer_2_head_9.png}
        \captionsetup{font=footnotesize,labelfont=footnotesize}
        \caption{Attends to next token.}
        \label{subfig:bertviz_model_view_opr2_next}
     \end{subfigure}
    \quad
    \begin{subfigure}[b]{0.22\textwidth}
        \centering
        \includegraphics[width=1\columnwidth]{figures/exp2/bertviz/opr2/bert_model_view_opr2_layer_4_head_4.png}
        \captionsetup{font=footnotesize,labelfont=footnotesize}
        \caption{Attends to special tokens.}
        \label{subfig:bertviz_model_view_opr2_special}
     \end{subfigure}
    %  \hfill
    \quad
     \begin{subfigure}[b]{0.22\textwidth}
        \centering
        \includegraphics[width=1\columnwidth]{figures/exp2/bertviz/opr2/bert_model_view_opr2_layer_9_head_0.png}
        \captionsetup{font=footnotesize,labelfont=footnotesize}
        \caption{Age-related attention.}
        \label{subfig:bertviz_model_view_opr2_age}
     \end{subfigure}
    \caption{Attention weights visualizations of four of BERT$_{FT}$'s attentions heads and the patterns to which they presumably attend when processing representations for a cherry-picked prompt-response pair generated by \textbf{old}-targeted DGPT-Discrim.}
    \label{fig:bertviz_model_view_opr2}
\end{figure}

\subsection{Qualitative Analyses}
\label{subsec:ctg_anal_qualitative}

% \subsubsection{Qualitative 1: Summary Statistics and Qualitative Inspection of Various Cases}

% \begin{itemize}
%     \item \textit{Similar to (error)case analyses of Experiment 1.}
    
%     \item \textit{Provide summary statistics and (\textbf{qualitative}) inspection of generated sequences per case.}
    
%     \item \textit{Cases could be: (1) sequences with low, average, high, or very-high perplexity. (2) (in)correctly classified generated sequences.}
    
%     \item \textit{What patterns do you observe among, e.g., misclassified sequences with low perplexity?}
    
%     \item \textit{Provide table of examples per case and age-group. Similar to table \ref{tab:qualexamples}}
% \end{itemize}

% \len{\textbf{Structure as follows:}}

% \begin{itemize}
%     \item Split the generated responses by their (1) perplexity and (2) target probability/classification case.
%     \item Provide summary statistics 
% \end{itemize}

The generated responses to neutral prompts are split by (1) whether or not BERT$_{FT}$ correctly classified a response as its target class, and (2) the level of perplexity: low ($\text{ppl.} \leq 27.52$), medium ($27.52 < \text{ppl.} \leq 35.63$), or high ($\text{ppl.} > 35.63$). See Section \ref{subsec:ctg_anal_ppl_target_prob} for an explanation of the rationale behind these intervals. Table \ref{tab:ctg_model_case_pcts} shows how the generated responses are distributed among these cases for the best performing BoW-based models, and discriminator-based models for both underlying language models and target age groups. As can be seen, the majority of the responses generated by GPT-2 BoW-based models lie in the low perplexity range, whereas the discriminator-based GPT-2 models also have percentage peaks in the higher perplexity registers (e.g., 37.2\% high-perplexity correctly classified for G-Discrim$_{Old}$). By contrast, the DialoGPT-based models all show the majority of their distributions lying in the medium-to-high perplexity range.

To narrow down the comparison, the qualitative inspection of samples is limited to responses generated by the discriminator-based models of both language models and age groups (i.e., samples generated by G-Discrim$_{Young}$, G-Discrim$_{Old}$, D-Discrim$_{Young}$, and D-Discrim$_{Old}$). Moreover, these models are all among those with the highest target probability improvements over their respective baselines in Tables \ref{tab:ctg_results_ws_neutral_prompt_young_models} and \ref{tab:ctg_results_ws_neutral_prompt_old_model}, so the differences in language style should be most pronounced between these setups. Furthermore, to emphasize the differences between perplexity, we only consider low versus high perplexity samples. Finally, the samples used for qualitative inspection are also split by whether or not they were correctly classified by BERT$_{FT}$. To summarize, qualitative inspection are performed on a total of 16 splits: by model (G-Discrim$_{Young}$, G-Discrim$_{Old}$, D-Discrim$_{Young}$, or D-Discrim$_{Old}$), perplexity (low or high), and classification outcome (correct or incorrect). Table \ref{tab:ctg_case_examples} shows examples of generated responses containing the patterns and observations discussed in the remainder of this section.

\begin{table}
    \begin{subtable}{0.5\linewidth}\centering
    {\begin{tabular}{r l}
    \hline
        \ \cellcolor{yellow!25}\textbf{\textit{PROMPT}} & \multicolumn{1}{>{\columncolor{yellow!10}}l}{\tabularCenterstack{l}{Good weather \\ we're having.}}\\
        \hline 
        \cellcolor{red!25}\textbf{\textit{GPT-2$_{Young}$}} & \multicolumn{1}{>{\columncolor{red!10}}l}{\tabularCenterstack{l}{...}}\\
        \ \cellcolor{blue!25}\textbf{\textit{GPT-2$_{Old}$}} & \multicolumn{1}{>{\columncolor{blue!10}}l}{\tabularCenterstack{l}{...}}\\
        \ \cellcolor{red!25}\textbf{\textit{DialoGPT$_{Young}$}} & \multicolumn{1}{>{\columncolor{red!10}}l}{\tabularCenterstack{l}{...}}\\
        \ \cellcolor{blue!25}\textbf{\textit{DialoGPT$_{Old}$}} & \multicolumn{1}{>{\columncolor{blue!10}}l}{\tabularCenterstack{l}{...}}\\
        \hline
        \ \cellcolor{yellow!25}\textbf{\textit{PROMPT}} & \multicolumn{1}{>{\columncolor{yellow!10}}l}{\tabularCenterstack{l}{Hi, how's it going?}}\\
        \hline 
        \cellcolor{red!25}\textbf{\textit{GPT-2$_{Young}$}} & \multicolumn{1}{>{\columncolor{red!10}}l}{\tabularCenterstack{l}{...}}\\
        \ \cellcolor{blue!25}\textbf{\textit{GPT-2$_{Old}$}} & \multicolumn{1}{>{\columncolor{blue!10}}l}{\tabularCenterstack{l}{ I've had my first \\ surgery recently.}}\\
        \ \cellcolor{red!25}\textbf{\textit{DialoGPT$_{Young}$}} & \multicolumn{1}{>{\columncolor{red!10}}l}{\tabularCenterstack{l}{...}}\\
        \ \cellcolor{blue!25}\textbf{\textit{DialoGPT$_{Old}$}} & \multicolumn{1}{>{\columncolor{blue!10}}l}{\tabularCenterstack{l}{...}}\\
    \hline
    \end{tabular}}
    \caption{Low perplexity \& correctly classified.}\label{subtab:ctg_case_examples_low_ppl_correct}
    \end{subtable}%
    \begin{subtable}{0.5\linewidth}\centering
    {\begin{tabular}{r l}
    \hline
        \ \cellcolor{yellow!25}\textbf{\textit{PROMPT}} & \multicolumn{1}{>{\columncolor{yellow!10}}l}{\tabularCenterstack{l}{Good weather \\ we're having.}}\\
        \hline 
        \cellcolor{red!25}\textbf{\textit{GPT-2$_{Young}$}} & \multicolumn{1}{>{\columncolor{red!10}}l}{\tabularCenterstack{l}{...}}\\
        \ \cellcolor{blue!25}\textbf{\textit{GPT-2$_{Old}$}} & \multicolumn{1}{>{\columncolor{blue!10}}l}{\tabularCenterstack{l}{...}}\\
        \ \cellcolor{red!25}\textbf{\textit{DialoGPT$_{Young}$}} & \multicolumn{1}{>{\columncolor{red!10}}l}{\tabularCenterstack{l}{...}}\\
        \ \cellcolor{blue!25}\textbf{\textit{DialoGPT$_{Old}$}} & \multicolumn{1}{>{\columncolor{blue!10}}l}{\tabularCenterstack{l}{...}}\\
        \hline
        \ \cellcolor{yellow!25}\textbf{\textit{PROMPT}} & \multicolumn{1}{>{\columncolor{yellow!10}}l}{\tabularCenterstack{l}{Tell me about your \\ latest holiday.}}\\
        \hline 
        \cellcolor{red!25}\textbf{\textit{GPT-2$_{Young}$}} & \multicolumn{1}{>{\columncolor{red!10}}l}{\tabularCenterstack{l}{...}}\\
        \ \cellcolor{blue!25}\textbf{\textit{GPT-2$_{Old}$}} & \multicolumn{1}{>{\columncolor{blue!10}}l}{\tabularCenterstack{l}{...}}\\
        \ \cellcolor{red!25}\textbf{\textit{DialoGPT$_{Young}$}} & \multicolumn{1}{>{\columncolor{red!10}}l}{\tabularCenterstack{l}{...}}\\
        \ \cellcolor{blue!25}\textbf{\textit{DialoGPT$_{Old}$}} & \multicolumn{1}{>{\columncolor{blue!10}}l}{\tabularCenterstack{l}{...}}\\
    \hline
    \end{tabular}}
    \caption{Low perplexity \& incorrectly classified.}\label{subtab:ctg_case_examples_low_ppl_incorrect}
    \end{subtable}%
    
    \medskip
    \begin{subtable}{0.5\linewidth}\centering
    {\begin{tabular}{r l}
    \hline
        \ \cellcolor{yellow!25}\textbf{\textit{PROMPT}} & \multicolumn{1}{>{\columncolor{yellow!10}}l}{\tabularCenterstack{l}{Good weather \\ we're having.}}\\
        \hline 
        \cellcolor{red!25}\textbf{\textit{GPT-2$_{Young}$}} & \multicolumn{1}{>{\columncolor{red!10}}l}{\tabularCenterstack{l}{...}}\\
        \ \cellcolor{blue!25}\textbf{\textit{GPT-2$_{Old}$}} & \multicolumn{1}{>{\columncolor{blue!10}}l}{\tabularCenterstack{l}{...}}\\
        \ \cellcolor{red!25}\textbf{\textit{DialoGPT$_{Young}$}} & \multicolumn{1}{>{\columncolor{red!10}}l}{\tabularCenterstack{l}{...}}\\
        \ \cellcolor{blue!25}\textbf{\textit{DialoGPT$_{Old}$}} & \multicolumn{1}{>{\columncolor{blue!10}}l}{\tabularCenterstack{l}{...}}\\
        \hline
        \ \cellcolor{yellow!25}\textbf{\textit{PROMPT}} & \multicolumn{1}{>{\columncolor{yellow!10}}l}{\tabularCenterstack{l}{Tell me about your \\ latest holiday.}}\\
        \hline 
        \cellcolor{red!25}\textbf{\textit{GPT-2$_{Young}$}} & \multicolumn{1}{>{\columncolor{red!10}}l}{\tabularCenterstack{l}{...}}\\
        \ \cellcolor{blue!25}\textbf{\textit{GPT-2$_{Old}$}} & \multicolumn{1}{>{\columncolor{blue!10}}l}{\tabularCenterstack{l}{...}}\\
        \ \cellcolor{red!25}\textbf{\textit{DialoGPT$_{Young}$}} & \multicolumn{1}{>{\columncolor{red!10}}l}{\tabularCenterstack{l}{...}}\\
        \ \cellcolor{blue!25}\textbf{\textit{DialoGPT$_{Old}$}} & \multicolumn{1}{>{\columncolor{blue!10}}l}{\tabularCenterstack{l}{...}}\\
    \hline
    \end{tabular}}
    \caption{High perplexity \& correctly classified.}\label{subtab:ctg_case_examples_high_ppl_correct}
    \end{subtable}%
    \begin{subtable}{0.5\linewidth}\centering
    {\begin{tabular}{r l}
    \hline
        \ \cellcolor{yellow!25}\textbf{\textit{PROMPT}} & \multicolumn{1}{>{\columncolor{yellow!10}}l}{\tabularCenterstack{l}{Good weather \\ we're having.}}\\
        \hline 
        \cellcolor{red!25}\textbf{\textit{GPT-2$_{Young}$}} & \multicolumn{1}{>{\columncolor{red!10}}l}{\tabularCenterstack{l}{...}}\\
        \ \cellcolor{blue!25}\textbf{\textit{GPT-2$_{Old}$}} & \multicolumn{1}{>{\columncolor{blue!10}}l}{\tabularCenterstack{l}{...}}\\
        \ \cellcolor{red!25}\textbf{\textit{DialoGPT$_{Young}$}} & \multicolumn{1}{>{\columncolor{red!10}}l}{\tabularCenterstack{l}{...}}\\
        \ \cellcolor{blue!25}\textbf{\textit{DialoGPT$_{Old}$}} & \multicolumn{1}{>{\columncolor{blue!10}}l}{\tabularCenterstack{l}{...}}\\
        \hline
        \ \cellcolor{yellow!25}\textbf{\textit{PROMPT}} & \multicolumn{1}{>{\columncolor{yellow!10}}l}{\tabularCenterstack{l}{Tell me about your \\ latest holiday.}}\\
        \hline 
        \cellcolor{red!25}\textbf{\textit{GPT-2$_{Young}$}} & \multicolumn{1}{>{\columncolor{red!10}}l}{\tabularCenterstack{l}{...}}\\
        \ \cellcolor{blue!25}\textbf{\textit{GPT-2$_{Old}$}} & \multicolumn{1}{>{\columncolor{blue!10}}l}{\tabularCenterstack{l}{...}}\\
        \ \cellcolor{red!25}\textbf{\textit{DialoGPT$_{Young}$}} & \multicolumn{1}{>{\columncolor{red!10}}l}{\tabularCenterstack{l}{...}}\\
        \ \cellcolor{blue!25}\textbf{\textit{DialoGPT$_{Old}$}} & \multicolumn{1}{>{\columncolor{blue!10}}l}{\tabularCenterstack{l}{...}}\\
    \hline
    \end{tabular}}
    \caption{High perplexity \& incorrectly classified.}\label{subtab:ctg_case_examples_high_ppl_incorrect}
    \end{subtable}%
\caption{A table \len{\textbf{TODO - FILL TABLE}}}\label{tab:ctg_case_examples}
\end{table}

Manual inspection of the sub-samples show that correctly classified low-perplexity samples from models targeted towards the older age group use more formal words than their young-targeted counterparts, e.g., \textit{quite}, \textit{significant}, \textit{powerful}, or \textit{institutions}. Samples generated by these models also show recurring topics that are typically associated with older age, e.g., children (\textit{my son} or \textit{your daughter}), history, or politics. Responses about healthcare related subjects are also more common among the samples generated by these models, as indicated by the use of words like \textit{surgery}. When viewing high-perplexity samples from the same set of models (i.e., those target towards the older age group and correctly classified), we see a substantial increase in the amount of gibberish and nonsensical sequences (space-less sequences of words, repetitions of the same words, or sequences of punctuation marks). When we inspect the incorrectly classified samples from models targeted towards the older class, we also see a considerable increase in the amount of nonsensical strings. Remarkably, we also observe more linguistic patterns associated with younger age in this sub-sample, e.g., words of excitement like \textit{favourite} or \textit{best}, informal vocabulary (\textit{pretty much}).

The low-perplexity correctly classified samples generated by models targeted towards the younger age group also contain clear indications of their target age. We observe more usage of slang words, swear words, and informal language, like \textit{yeah}, \textit{dude}, \textit{cool}, \textit{kinda}, and \textit{lol}. The use of the word \textit{like} as a colloquial adverb, quotative, or filler is also more commonplace among the young-targeted models. \len{Add good examples of use of \textit{like} or remove sentence.} Furthermore, these sub-samples are also characterized by increased use of words of excitement and exclamation: e.g., \textit{awesome}, \textit{really}, \textit{love}, \textit{fun}, or \textit{amazing}. Especially in the correctly classified low-perplexity responses generated by G-Discrim$_{Young}$, we see a strong presence of topics such as dating (indicated by words such as \textit{girlfriend} or \textit{boyfriend)}, having parents (\textit{my dad}), parties, and student life (\textit{roommate}. And interesting pattern observed in this sub-sample is one often associated with millennial or social networking language: the tendency to end a (serious-sounding) statement with \textit{lol} or \textit{haha} as a means of softening the perceived severity of the statement or as a signal of interlocutor involvement \citep{newitz2019you, tagliamonte2008linguistic}. For example, \textit{We have to stop talking about this all going so stupid lol lol}. When inspecting the high-perplexity and/or incorrectly classified samples generated by the younger-targeted models, we observe similar patterns. Namely, a substantial increase in non-alphabetical strings, and gibberish.

When comparing the samples from GPT-2 and DialoGPT-based models, we see that the responses generated by DialoGPT are more dialogic, whereas GPT-2 sometimes generates sequences that look more like sentence completions. This is to be expected from the differences in pre-training methods between the two language models \citep{zhang2019dialogpt}. Furthermore, we see a lot more nonsensical low-perplexity responses generated by DialoGPT-based models. Perplexity remains a rough proxy for fluency, and observations like these confirm this problem. That is, low perplexity often does not imply lack of gibberish or nonsense. DialoGPT's strong bias towards generating younger-sounding language is also noticeable in its generated samples. For example, DialoGPT-based models targeted towards the older group still produce lots of word of excitement. However, this older-targeted DialoGPT-based model does succeed in producing significantly fewer usages of slang or swear words, when compared to its younger-target counterpart.


% \subsubsection{Qualitative 2: Human evaluation of fluency, grammaticality, and relevancy}

% \begin{itemize}
%     \item \textit{Generate and sample text passages for a variety of model-configurations and age-groups.}
%     \item \textit{Have a group of human participants rate these sequences on a scale from 1 to 5 for their (1) fluency, (2) grammaticality, (3) relevance to the prompt (if there is one)}
%     \item \textit{Average the ratings, and compare the human evaluation metrics to the automated evaluation metrics reported in Table \ref{tab:ctg_results_ws}}
% \end{itemize}



\begin{table}[H]
    \centering
    \begin{tabular}{@{}l  c  c  c  c  c  c @{}}
    \toprule
    \textbf{model} & \textbf{low ppl. | \cmark} & \textbf{low ppl. | \xmark} & \textbf{med ppl. | \cmark} & \textbf{med ppl. | \xmark} & \textbf{high ppl. | \cmark} & \textbf{high ppl. | \xmark}\\
    \midrule
    G-BoW$_{Young}$ & 48.0\% & 15.6\% & 15.6\% & 10.4\% & 6.7\% & 3.7\%\\
    G-BoW$_{Old}$ & 25.6\% & 37.7\% & 11.9\% & 14.8\% & 5.6\% & 4.4\%\\
    G-Discrim$_{Young}$ & 33.3\% & 15.9\% & 17.4\% & 8.5\% & 17.0\% & 7.9\%\\
    G-Discrim$_{Old}$ & 23.8\% & 18.2\% & 13.4\% & 3.3\% & 37.2\% & 4.1\%\\
    D-BoW$_{Young}$ & 5.2\% & 0.4\% & 40.0\% & 5.2\% & 43.3\% & 5.9\%\\
    D-BoW$_{Old}$ & 3.0\% & 5.9\% & 10.4\% & 35.2\% & 8.5\% & 37.0\%\\
    D-Discrim$_{Young}$ & 5.9\% & 2.7\% & 23.0\% & 5.6\% & 56.9\% & 5.9\%\\
    D-Discrim$_{Old}$ & 7.0\% & 4.1\% & 19.3\% & 11.5\% & 30.3\% & 27.8\%\\\bottomrule
    \end{tabular}
    \caption{\len{TODO - Turn this table into grouped barcharts.}}
    % \caption{Percentages of (non-)overlapping (in)correctly predicted cases between trigram and BERT$_{FT}$ models for BNC. *Pre-processed sequence length is measured in tokens.}
    \label{tab:ctg_model_case_pcts}
\end{table}

% \begin{table*}[h]
% \resizebox{\linewidth}{!}{
% % \small
% \begin{tabular}{@{}lllll@{}}
% \toprule
% \bf age & \bf both correct                & \bf both wrong                 & \bf BERT$_{FT}$ correct | trigram wrong                              & \bf trigram correct | BERT$_{FT}$ wrong                                                            \\ \midrule
% 19-29     & oh that's cool              & A retrospective exhibition & what even on the green slope?             & really?                                                                    \\
% 19-29     & a text and then I'll do it  & chuck them in those pots   & yeah you told me to do you told me to do  & and she like won't eat any carbs and she's like                            \\
% 19-29     & yeah                        & mm                         & somebody made the f***ing table           & do you not like total greens? \\ \midrule
% % well er it's not acceptable just to ditch \textless{}unclear\textgreater{} \\ \hline
% 50+       & I said no I don't have them & yeah                       & really?                                   & my under stairs in the kitchen                                             \\
% 50+       & that's of course            & no no that's alright       & it's still we we frequently walk that way & in the first place                                                         \\
% 50+       & oh right                    & what a tragic life         & since this this was new this house?       & thank you very much                                                        \\ \bottomrule
% \end{tabular}
% }
% \caption{Examples
% % cherry-picked examples per age group 
% where both models are correct/wrong or only BERT$_{FT}$/trigram is correct.}\label{tab:qualexamples}
% \end{table*}

% Qualitative inspection focuses on low versus high perplexity (i.e., medium perplexity is omitted, both for feasibility (in total Table \ref{tab:ctg_model_case_pcts} describes 48 sets of responses), and only responses generated by the discriminator-based model with the highest target prob. gaing over baseline (compared to its age-opposite counterpart): so that's GPT-2 Discrim old vs. young. And the same for BoW

\len{\textbf{Observations in a few of the best performing models:}}

\begin{itemize}
    \item GPT2 | Discrim-O | Low ppl | BERT correct:
    \begin{itemize}
        \item Not very dialogic 
        \item Use of larger words like "significant", "powerful", "institutions"
        \item Talks about "my son" and children
        \item Talks about work.
        \item Talks about hospitals, surgery, health care (which is to be expected if you look at the old-wordlists)
        \item Older slang words "quite"
        \item Talks about history and politics.
    \end{itemize}
    \item GPT2 | Discrim-O | Low ppl | BERT incorrect:
    \begin{itemize}
        \item Despite low GPT-1 perplexity, considerably more gibberish that BERT-correct counterpart, space-less repetitions of words mostly though, not nonsensical sequences of characters. First of all, this makes a case for a different proxy for fluency (find different measures to consider for future research, and/or suggest human evaluation).
        \item The non-gibberish responses contain more words of excitement and positive sentiment "favourite", "best".
        \item Gibberish most likely obfuscates BERT, hence worse prediction.
        \item Talks about football clubs and premier league
        \item Younger sounding slang "pretty much"
        \item PPLM-discrim old (BERT incorrect) nonsense pattern is lots of punctuation marks, parentheses, and words without spaces.
    \end{itemize}
    \item GPT2 | Discrim-O | High ppl | BERT correct:
    \begin{itemize}
        \item Predominantly gibberish. 
        Many sequences of non-alphabetical characters/punctuation marks (parentheses, apostrophes, etc. )
        \item When non-gibberish, does talk about "my wife" and "work"
    \end{itemize}
    \item GPT2 | Discrim-O | High ppl | BERT incorrect:
    \begin{itemize}
        \item Very small subset (see Table \ref{tab:ctg_model_case_pcts}, and earlier plots about this model having a clear tradeoff between perplexity and target prob)
        \item Mostly difluencies
        \item Formal words like "precious"
    \end{itemize}
    \item GPT2 | Discrim-Y | Low ppl | BERT correct:
    \begin{itemize}
        \item Substantially more use of younger sounding slang words: "yeah", "dude", "cool", "awesome", "like", "kinda", "hot", "horny", "gonna", "lol". NB: "yeah" was also picked up by the trigram in Chapter 3 (make proper reference) to be strongly indicative of younger language
        \item More swear words "fucking", "shit", "dick", "hell"
        \item More words of excitement "awesome" "really" "love", "fun", "amazing"
        \item More talk about dating "girlfriend", "boyfriend", "horny", "sex"
        \item More talk about depression and anxiety
        \item More tech-talk
        \item Lots of disfluency and repetition though.
        \item Talks about dancing and parties
        \item Interesting pattern of millennial language: uttering something serious, but ending with a neutralizing, tension-breaking humoring expression, e.g., ``We have to stop talking about this all going so stupid lol lol.'' ``Shit just got weird LOL."
        \item More topics that relate to student-life: roommate, parties, drinking, bars, clubs, tv series on Netflix
    \end{itemize}
    \item GPT2 | Discrim-Y | Low ppl | BERT incorrect:
    \begin{itemize}
        \item Considerably more nonsense than the BERT-correct counterpart. 
        \item More non-language sequences of characters, \texttt{<|endoftext|>} tokens.
        \item Similar word-use and topics: tv series (HBO Game of Thrones), swear words, "mom" and "dad(dy)"/
        \item PPLM-discrim young (BERT incorrect) nonsense pattern is lots of \texttt{<|endoftext|>} tokens.
    \end{itemize}
    \item GPT2 | Discrim-Y | High ppl | BERT correct:
    \begin{itemize}
        \item Substantially more nonsense, disfluency, gibberish, non-alphabetical characters.
        \item Similar patterns to low-perplexity counterpart. Actually same patterns, but with much more nonsense.
    \end{itemize}
    \item GPT2 | Discrim-Y | High ppl | BERT incorrect:
    \begin{itemize}
        \item Very small subset (look at Table \ref{tab:ctg_model_case_pcts} and perplexity-targetprob plots).
        \item Same patterns
        \item Mostly nonsense.
    \end{itemize}
    \item DialoGPT | Discrim-O | Low ppl | BERT correct:
    \begin{itemize}
        \item Very small subset (See table and graph).
        \item Talks about surgery, hospital etc.
        \item Talks about daughters.
        \item Almost no slang words
        \item More formal language, and complete sentences.
        \item Quite some nonsense
    \end{itemize}
    \item DialoGPT | Discrim-O | Low ppl | BERT incorrect:
    \begin{itemize}
        \item Very small subset (See table and graph).
        \item A lot of nonsense, repeated eot tokens.
        \item Pretty much similar patterns to setup above.
    \end{itemize}
    \item DialoGPT | Discrim-O | High ppl | BERT correct:
    \begin{itemize}
        \item Lots of gibberish
        \item Noticeably, a lot less typically "older" sounding language than GPT-2 counterpart. Which makes sense when you look at the difference in target probabilities.
        \item More words of excitement and exclamation than other old discrim model. Likely due to DialoGPT's young-bias.
        \item Barely any slang or swear words.
        \item Pretty neat formal sentences when not gibberish.
    \end{itemize}
    \item DialoGPT | Discrim-O | High ppl | BERT incorrect:
    \begin{itemize}
        \item A lot more nonsensical/non-language sequences of characters.
        \item Talks about days off from work
        \item words like ``quite'', ``glad''
        \item talks about other people's parents and children.
    \end{itemize}
    \item DialoGPT | Discrim-Y | Low ppl | BERT correct:
    \begin{itemize}
        \item Very small sample size
        \item More informal language "gonna"
        \item Talks about gifs and comments
    \end{itemize}
    \item DialoGPT | Discrim-Y | Low ppl | BERT incorrect:
    \begin{itemize}
        \item Predominantly nonsensical sequences.
        \item Repetitions of words
        \item No slang, or words of excitement or other clear giveaways of young language.
        \item Very small sample size.
    \end{itemize}
    \item DialoGPT | Discrim-Y | High ppl | BERT correct:
    \begin{itemize}
        \item Lots of words of excitement and exclamation marks
        \item slang and informal words: "dude", "buddy", "cool", referring to the basketball team, the Cleveland Cavalliers, as the "Cavs", "howdy"
        \item Talks about going on vacation to the beach with friends.
        \item Uses the word "like"
        \item Uses emojis ":P", ":D"
        \item Fair amount of gibberish.
    \end{itemize}
    \item DialoGPT | Discrim-Y | High ppl | BERT incorrect:
    \begin{itemize}
        \item Mostly gibberish.
        \item Similar patterns to BERT-correct counterpart, just way more nonsense.
    \end{itemize}
\end{itemize}

Overall, the GPT-2 ones are also often sentence completions, and not as often rebuttals as DialoGPT. This is understandable from the point of view of how the models have been pre-trained.

% \chapter{Methodology and experimental setup}\label{ch:methods} % TODO: FIX THIS REFERENCE IN INTRO
% % \textit{Which methods do you use to systematically research your topic (e.g., PPLM, GPT-2, Transformers), and by which you are answering your research questions. This is meant to check the validation and verification of your research.}

% Most important methods to highlight:

% \begin{itemize}
%     \item \textbf{TODO:} Do I need to explain $n$-gram logistic regression, and LSTM models?
%     \item Transformers
%     \item Important pre-trained transformer-based architectures:
%         \begin{itemize}
%             \item GPT-2 (specifically GPT-2 for language generation)
%             \item BERT (specifically BERT for sequence classification)
%             \item DialoGPT*
%         \end{itemize}
%     \item Plug and play language models 
%         \begin{itemize}
%             \item Also emphasize that PPLM is different from fine-tuning (see top comment: \url{https://openreview.net/forum?id=H1edEyBKDS}), i.e., that PPLM updates the activations and not the model-parameters. 
            
%         \end{itemize}
% \end{itemize}

\subsection{Transformers}

The Transformer architecture plays a central role in most of the recent advances in NLP. The same holds for the methods used in this thesis to investigate controlled dialogue generation and speaker/author age detection. A brief explanation of its workings is therefore in order. For a more detailed review of the model architecture, the reader is referred to the original paper (\citep{vaswani2017attention}) or this excellent blog post: \url{https://jalammar.github.io/illustrated-transformer/}.

The Transformer, like most neural sequence processing models, has an encoder-decoder structure. On a high level, the encoder receives an input sequence $\textbf{x} = (x_1, ..., x_n)$ (e.g., a sentence), and maps this to a sequence of latent continuous variables $\textbf{z} = (z_1, ..., z_n)$. The decoder then takes $\textbf{z}$ as input, and maps this to an output sequence $\textbf{y} = (y_1, ..., y_m)$. Note that the use of positional encodings of the input and output embeddings enables the Transformer to process and generate sequences in arbitrary order, allowing for a high degree of parallelization. The generation of $\textbf{y}$ happens element-by-element in an auto-regression fashion, where at step $t$, element $y_{t - 1}$ is taken as input.

Both the encoder and decoder are comprised of $N$ identical layers (denoted by the `N $\times$' in the left part of Figure \ref{fig:transformer_architecture}). Every sub-layer performs a succession of transformations using multi-head self-attention mechanisms and point-wise, fully connected layers, along with residual connections (\cite{he2016residual}) around every sub-layer followed by layer normalization (\cite{DBLP:journals/corr/BaKH16}). The decoder's first self-attention sub-layer is masked to ensure that the output predictions at sequence position $i$ cannot depend on output positions greater than $i$. Finally, the decoder passes its output through a linear and softmax layer to produce a probability distribution over the problem space (e.g., the vocabulary) from which the most likely symbols for the generated output sequence $\textbf{y}$ can be inferred.

A key aspect of the Transformer architecture is its use of attention \cite{DBLP:journals/corr/BahdanauCB14}. This allows the encoder-decoder architecture to selectively focus on parts of the input sequence to produce a more informative hidden representation. \citeauthor{vaswani2017attention} formulate an attention function and a mapping of queries and sets of key-value pairs to an attention output, where matrices represent the queries $Q$, keys $K$, and values $V$. The attention output is a weighted sum of the values, based on the relevance of the corresponding keys to a query. In particular, they employ scaled dot-product attention:

\begin{equation}
    \texttt{Attention}(Q, K, V) = \texttt{softmax} \left( \frac{QK^T}{\sqrt{d_k}}\right) V,
\end{equation}

Furthermore, \citeauthor{vaswani2017attention} propose to use multi-head attention by using learned linear projections to project the queries, keys and values $h$ times, and apply the aforementioned attention function to these projections in parallel. The concatenation of these attention outputs, passed through a linear layer, ultimately produces the final output of the Transformer's attention sub-layers. This allows the model to attend to the relevant information from all representation sub-spaces at various sequence positions. See Figure \ref{fig:transformer_architecture} for an schematic illustration of the Transformer's structure described above.


\begin{figure}[H]
    \centering
    \includegraphics[width=\textwidth]{figures/transformer_lillog.png}
    \caption{An overview of the full Transformer model architecture. \textit{Collated image source:} Fig. 17 in this blog post \url{https://lilianweng.github.io/lil-log/2018/06/24/attention-attention.html}. \textit{Original image source:} Figures 1 and 2 in \cite{vaswani2017attention}}
    \label{fig:transformer_architecture}
\end{figure}

\subsection{Causal language modeling with Transformers}

Following the conventions of \cite{dathathri2019plug} and \cite{madotto-etal-2020-plug}, a dialogue is comprised of multiple alternating turns (sometimes referred to as utterances) between more than one speaker. For simplicity, this project only focuses on dialogues between two speakers. The conversation history at turn $t$ is defined as $\mathcal{D}_t = \{S^{(1)}_1, S^{(2)}_1, ..., S^{(1)}_t\}$, where $S^{(j)}_t$ is speaker $j$'s utterance at time $t$. \citeauthor{madotto-etal-2020-plug} denote speaker $1$ and the user $U$, and speaker $2$ as the conversational system $S$, yielding dialogue history $\mathcal{D}_t = \{U_1, S_1, ..., U_t\}$. This notational convention will also be used for the user-system experiments later on.

A Transformer-based language model (denoted $\texttt{LM}$) is used in this thesis to model the distribution of dialogues, using dialogue history at time $t$, $\mathcal{D}_t$, as a prompt to auto-regressively generate the dialogue continuation $S_t$. More specifically, let the concatenation of the dialogue history at $t$ and its continuation, $\{\mathcal{D}_t, S_t\}$, be represented as a sequence of tokens $\textbf{x}= \{x_0, ..., x_n\}$. By recursively applying the product rule of probability (\cite{bishop2006pattern}), the unconditional probability of the sequence $p(\textbf{x})$ can be expressed as:

\begin{equation}
    p(\textbf{x}) = \prod_{i = 1}^n p(x_i | x_0, ..., x_{i - 1}).
\end{equation}

\cite{dathathri2019plug} and \cite{madotto-etal-2020-plug} define the Transformer's decoding process in a recursive fashion. Let $H_t$ denote the conversation history's key-value pairs, i.e., $H_t = \left[ (K_t^{(1)}, V_t^{(1)}), ..., (K_t^{(l)}, V_t^{(l)}) \right]$, with $(K_t^{(i)}, V_t^{(i)})$ representing the key-value pairs from the $\texttt{LM}$'s $i$-th layer generated at all time steps $0$ through $t$. This results in the recurrent dedocing process being expressed as:

\begin{equation}
    o_{t + 1}, H_{t + 1} = \texttt{LM} \left( x_t, H_t \right),
\end{equation}

where $o_{t + 1}$ is the hidden state of the last layer. Finally, after applying a softmax transformation, the next token $x_{t + 1}$ is sampled from the distribution $p_t = \texttt{softmax} \left( W o_{t + 1} \right)$, where $W$ is a linear mapping from the model's last hidden state to a vector of vocabulary size. This recursive formulation allows for efficient text generation by leveraging cached memories, without repeated forward passes.

\subsection{Plug-and-play modeling}

P\&P $\neq$ fine-tuning!


% \chapter{Data}
% Two datasets are considered during the experiments of this thesis: (1) the spoken component of the British National Corpus (BNC or BNC2014) \citep{love-spoken-bnc-2014}, and (2) the Blog Authorship Corpus (BAC, or sometimes referred to as `blog corpus') \citep{schler2006effects}. The first corpus is a collection of transcriptions of everyday conversations in British English, gathered between 2012 and 2016. The second is a dataset of blogs posted on \url{https://www.blogger.com}, gathered in or before August 2004.

The texts (i.e., blogs or dialogue transcriptions) in both corpora are labeled by age, among other labels. This makes them suitable candidates for training and testing age classifiers. In both cases, the texts are written in a somewhat informal manner, making them more representative of everyday speech. Only the BNC is used for the controllable dialogue generation phase of the experiments, as the BAC is not a conversational dataset. What follows is an overview of both datasets' motivation for use, drawbacks, metadata, descriptive statistics, and pre-processing steps before analyses. 

\subsection{The British National Corpus (BNC)}
The conversations of the spoken component of the BNC were typically recorded at home and took place between friends and family of participants. Participants recorded their conversations using their smartphones, allowing for spontaneous recording. These two properties of the BNC's sampling procedure make the recorded dialogues representative of contemporary everyday British English speech. 

A total of 1251 conversations of 672 speakers constitutes the full corpus, accounting for 10.4 million words. During the recruitment process of the BNC, prospective participants were asked to disclose personal information like age, gender, highest completed level of education, employment, and perceived accent\footnote{For a detailed description of the sampling decisions made during data collection, the reader is referred to the BNC2014 user manual: \url{http://corpora.lancs.ac.uk/bnc2014/doc/BNC2014manual.pdf}}. Other metadata accompany the conversations themselves. Namely, number of speakers taking part in the conversation, ages of those speakers, conversation length, and topic. 

Nine age-brackets are present in the corpus: 11-18, 19-29, 30-39, 40-49, 50-59, 60-69, 70-79, 80-89, and 90-99. To make the objective of learning to classify and generate conversational responses more straightforward, I only consider dialogues between two participants. Namely, it will be more difficult to distinguish which utterances constitute the relevant responses to something said in a conversation with more than two participants. This filter step results in a remaining dataset of 622 dialogues, almost 460K utterances (i.e., a single turn in a dialogue), and nearly 5M tokens. On average, a dialogue has 736 turns, meaning that the conversations are relatively long. Furthermore, due to minors not being allowed to participate in this project's interactive experiments later on, the age-bracket 11-18 is dropped. This leaves us with a dataset of 522 dialogues, 418K utterances, and 4.4M tokens. Finally, the age-brackets get regrouped into brackets 19-29 and 50 plus, and conversations with speakers aged 30-49 are removed. This gap between to two age-groups is made to minimize the chance of overlapping linguistic characteristics being present between separate age-brackets (e.g., the difference in language use between speakers aged 19-29 and 30-39 is probably smaller than that between age-brackets 19-29 and 50-59). Moreover, only conversations between two participants of the same (new regrouped) age-bracket is kept. This is done to avoid the confounding factor that interlocutors of a certain age-group might adjust their choice of words to the age of the person they are talking to. Only considering dialogues between participants of similar ages is expected to keep the utterances as representative of the age-bracket's linguistic characteristics as the corpus allows. These final filtration steps result in a subset consisting of 237 dialogues, roughly 172.000 utterances, and approximately 1.8M tokens.




\begin{itemize}
    \item Talk about pre-processing steps.
    \item The main limitations of the BNC are its size and imbalance. 
    \item Weighted loss and weighted random sampling
    \item Data outdated? Not representative of modern speech.
\end{itemize}

\begin{figure}[H]
    \centering
    \includegraphics[width=\textwidth]{figures/bnc_alt1_dist_age_new.png}
    \caption{Distribution of age-brackets per dialogue in 19-29 vs. 50-plus subset of BNC.}
    \label{fig:dist_age_bnc}
\end{figure}



\subsection{The Blog Authorship Corpus (BAC)}

\begin{table}[H]
    \centering
    \begin{tabular}{l | c  c}
        \hline
        \textbf{Corpus} & BNC & BAC \\
        \hline
        \textbf{Dialogue?} & Yes & No\\
        \textbf{No. words} & $10.4 \cdot 10^6$  & $140 \cdot 10^6$ \\
        
        \hline
    \end{tabular}
    \caption{Summary of corpora properties.}
    \label{tab:summary_corpora}
\end{table}



% \section{Planning}
% 

\begin{longtable}{| c | c | m{10cm} |}
\caption{Timetable for my planning on experimenting, evaluating, and writing.}
\label{tab:plan}\\
    % \begin{tabular}{| c | c | m{10cm} |}
        \hline
        \textbf{Month} & \textbf{Week} & \textbf{Task} \\
        \hline
        June-July & 26 & \begin{itemize} 
                            \item 28/06 Thesis proposal deadline. 
                            \item Pre-summer RPA meeting roundup.
                            \item Migrate from AWS EC2 to Lisa GPU.
                            \item Construct properly representative most frequently used words per age-group (unique to age-group).
                            \item First experiments with GPT-2 + BNC Age discriminator.
                        \end{itemize} \\
        \hline
        July & 27 & \begin{itemize}
                        \item  Break.
                    \end{itemize}\\
        \hline
        July & 28 & \begin{itemize}
                        \item Break.
                    \end{itemize}\\
        \hline
        July & 29 & \begin{itemize}
                        \item Coding for BNC-discrim. + GPT-2
                        \item Coding for BNC-discrim. + DialoGPT
                        \item Latent space and feature inspections age-discriminators.
                        \item PPCM code setup.
                    \end{itemize}\\
        \hline
        July-August & 30 &  \begin{itemize}
                                \item Coding for BAC-discrim + GPT-2
                                \item Coding for BAC-discrim + DialoGPT
                                \item Automatic evaluation age-discrim. + LMs.
                                \item Write short summaries and interpretations of results.
                            \end{itemize}\\
        \hline
        August & 31 & \begin{itemize}
                        \item Finalize writing of age-discrim. + LM results up until now.
                        \item Break from Friday 06/08 to Monday 09/08.
                      \end{itemize}\\
        \hline
        August & 32 & \begin{itemize}
                                \item Coding for user-ready adaptive system (PPCM).
                                \item Writing for results and discussion until now.
                      \end{itemize}\\
        \hline
        August & 33 & \begin{itemize}
                            \item Coding for user-ready adaptive system (PPCM).
                            \item Writing for results and discussion until now.
                       \end{itemize}\\
        \hline
        August & 34 & \begin{itemize}
                        \item Test-runs for user-ready adaptive system (PPCM).
                        \item Writing for results and discussion until now.
                      \end{itemize}\\
        \hline
        August-September & 35 & \begin{itemize}
                                    \item Test-runs for user-ready adaptive system (PPCM).
                                    \item Writing for results and discussion until now.
                                \end{itemize}\\
        \hline
        September & 36 &\begin{itemize}
                                \item First RPA meeting after summer. Set up plans for next phase of experiments.
                        \end{itemize}\\
        \hline
        September & 37 & \begin{itemize}
                                \item Phase 2 user experiments.
                                \item Write tentative results.
                            \end{itemize}\\
        \hline
        September & 38 & \begin{itemize}
                                \item Phase 2 user experiments.
                                \item Write tentative results.
                        \end{itemize}\\
        \hline
        September-October & 39 &\begin{itemize}
                                    \item 01/10 Submit draft of final version for feedback.
                                \end{itemize}\\
        \hline
        October & 40 & \begin{itemize}
                            \item Time for additional experiments and case studies.
                        \end{itemize}\\
        \hline
        October & 41 & \begin{itemize}
                                \item Time for additional experiments and case studies.
                        \end{itemize}\\
        \hline
        October & 42 & \begin{itemize}
                            \item Revise final version.
                        \end{itemize}\\
        \hline
        October & 43 & \begin{itemize}
                            \item Revise final version.
                        \end{itemize}\\
        \hline
        November & 44 & \begin{itemize}
                            \item 01/11 Final thesis submission deadline.
                        \end{itemize}\\ 
        \hline
    % \end{tabular}}
\end{longtable}

% \chapter{Results and analyses}
% \textit{The experiments of this study are divided into two phases: (1) automated age detection from written texts or dialogue transcriptions, and (2) age-adaptive dialogue generation.}

\subsection{Automated age-detection from text}

Three classes of architectures are trained for the task of predicting a writer's or speaker's age given a written text or speech transcription; (1) logistic $n$-gram models, (2) recurrent neural networks (RNNs), specifically, long short-term memory (LSTM), and (3) attention-based BERT encoders.

\begin{table}[H]
    \centering
    \begin{tabular}{c c c}
    \hline
    \textbf{Model} & \textbf{Accuracy} & $\boldsymbol{F}_1$\\
    \hline
    Baseline (predict majority class) & 0.4704 & ...\\
    unigram & 0.6058 & ...\\
    bigram & 0.6247 & ...\\
    trigram & 0.6207 & ...\\
    (2 layer bi-)LSTM & 0.7194 & ...\\
    BERT & ... & ...\\
    \hline
    \end{tabular}
    \caption{Blog corpus age classifiers}
    \label{tab:blog_classification}
\end{table}

\begin{table}[H]
    \centering
    \begin{tabular}{c c c}
    \hline
    \textbf{Model} & \textbf{Accuracy} & $\boldsymbol{F}_1$\\
    \hline
    Baseline (predict majority class) & 0.500 & ...\\
    unigram & ... & ...\\
    bigram & ... & ...\\
    trigram & ... & ...\\
    LSTM & 0.701 & ...\\
    BERT & 0.716 & ...\\
    \hline
    \end{tabular}
    \caption{BNC (balanced) age classifiers}
    \label{tab:bnc_classification}
\end{table}

\begin{figure}[H]
    \centering
    \includegraphics[width=0.5\textwidth]{figures/cm_bert_test_dt_27_May_2021.png}
    \caption{Confusion matrix BERT age classifier on balanced BNC test set.}
    \label{fig:cm_bert_bnc_rb}
\end{figure}

\begin{table}[H]
    \centering
    \begin{tabular}{c c c}
    \hline
    \textbf{Model} & \textbf{Accuracy} & $\boldsymbol{F}_1$\\
    \hline
    Baseline (predict majority class) & 0.807 & ...\\
    unigram & ... & ...\\
    bigram & ... & ...\\
    trigram & ... & ...\\
    LSTM & 0.776 & ...\\
    BERT & 0.786 & ...\\
    \hline
    \end{tabular}
    \caption{BNC age classifiers}
    \label{tab:bnc_classification}
\end{table}

\subsection{Age-adaptive conversational response generation}


% \subsection{Controlled dialogue generation}

Model settings to compare. \textbf{TODO:} Also add a representative model from \cite{dathathri2019plug} as baseline.
\begin{itemize}
    \item Uncontrolled language models:
        \begin{itemize}
            \item \textbf{Baseine: }``Vanilla'' uncontrolled GPT-2-medium
            \item Uncontrolled GPT-2-large (?)
            \item Uncontrolled DialoGPT-medium
            \item Uncontrolled DialoGPT-large (?)
        \end{itemize}
    \item Bag-of-words (BoW) controlled language models:
        \begin{itemize}
            \item GPT-2-medium + Full BNC BoW
            \item GPT-2-medium + 19-29 BNC BoW
            \item GPT-2-medium + 50 plus BNC BoW
            \item DialoGPT-medium + Full BNC BoW
            \item DialoGPT-medium + 19-29 BNC BoW
            \item DialoGPT-medium + 50 plus BNC BoW
        \end{itemize}
    \item Discriminator-based controlled language models. \textbf{TODO:} Which age discriminator(s) should I consider? E.g., 1-layer linear, or Transformer-based discriminators.
        \begin{itemize}
            \item GPT-2-medium + BNC age discriminator
            \item DialoGPT-medium + BNC age discriminator
        \end{itemize}
\end{itemize}

\chapter{Discussion}\label{ch:discussion}
\section{Summary of Key Findings and Interpretations}

In this thesis, the problems of automated age detection from dialogue and age-adaptive controlled dialogue generation are investigated. First, I studied the extent to which purely text-based NLP models can detect age-related linguistic features in dialogue data, and which features drive their predictions. Subsequently, I studied the extent to which age-adaptive dialogue generation is possible using an approach based on Plug-and-Play language models (PPLM) \citep{dathathri2019plug}.

The results of the age detection experiments indicated that a fine-tuned version of BERT, BERT$_{FT}$, is capable of detecting age-related linguistic features in dialogue utterances with reasonable accuracy. BERT$_{FT}$ was particularly useful for age detection when the dialogue fragment is long enough to contain discriminative signal. At the same time, it is observed that much simpler models based on $n$-grams achieve comparable performance, which suggests that, in dialogue, ‘local’ features can be indicative of the language of speakers from different age groups. This is showed to be the case, with both lexical and stylistic cues being informative to these models in this task.


% \textbf{connecting paragraph:}
% \begin{itemize}
%     \item The age detection results informed me about the development of controllable dialogue generation systems using PPLM.
%     \item The presumed locality of age-indicative features, as suggested by the comparable performance achieved by the $n$-gram models, motivates the use of unigram-based bag-of-words attribute models in PPLM-setups for, in comparison to more sophisticated discriminator-based PPLM-setups.
%     \item The analyses of experiment 1 predominantly compare the classification performances and feature importance of $n$-gram and Transformer-based neural models. This comparison is continued in the controlled dialogue generation experiments, in the form of comparing BoWs versus neural discriminators as attribute models.
%     \item The best performing classifier from the age detection experiments, BERT$_{FT}$, is used in the second phase of experiments to evaluate the prompts and generated responses for their resemblance to the linguistic style of the younger or older age group. 
% \end{itemize}

Furthermore, the age detection results informed the development of controllable dialogue generation systems using PPLM. 
% Namely, the presumed locality of age-indicative features, as suggested by the comparable performance achieved by the $n$-gram models, motivates the use of unigram-based bag-of-words attribute models in PPLM-setups for, in comparison to more sophisticated discriminator-based PPLM-setups.
On the one hand, the presumed locality of age-indicative features, as suggested by the comparable performance achieved by the $n$-gram models, motivates the use of unigram-based bag-of-words attribute models in PPLM-setups.
On the other hand, the use of more sophisticated neural discriminator attribute models in PPLM-setups is motivated by the superior performance of neural discriminators in Experiment 1, and the notion that abstract linguistic styles, like those of specific age groups, are not easily represented as a bag-of-words.
This difficulty of expressing an age group's linguistic style as a manually curated bag-of-words motivates the use of empirically generated bag-of-words attribute models (see Section \ref{subsec:att_model_dev}).
Specifically, the empirically generated lists of unigrams per age group used as BoW attribute models are (1) the lists of unigrams deemed most informative by Experiment 1's best-performing unigram-based classifier, and (2) the lists of the most frequently used words by a specific age group, after omitting the overlapping frequently used words between both age groups.
% Specifically, the two types of empirically generated bag-of-words attribute models used in Experiment 2 are 
% Additionally, I used the list of unigrams as an empirically generated attribute model for a BoW-based PPLM setup, specifically the list of unigrams deemed most informative by the best-performing unigram classifier.
Overall, the analyses of Experiment 1 predominantly compare the classification performance and feature importance of $n$-gram and Transformer-based neural models. This comparison is continued in the controlled dialogue generation experiments, in the form of comparing BoWs versus neural discriminators as attribute models in PPLM-setups. Also, the best performing classifier from the age detection experiments, BERT$_{FT}$, is used in the second phase of experiments to evaluate the prompts and generated responses for their resemblance to the linguistic style of the younger or older age group.

The controlled dialogue generation results indicate that it is possible to use PPLM to generate dialogue responses that have been adapted to the styles of different age groups, to an extent that is identifiable by Experiment 1's best classifier. It can be seen that the discriminator-based PPLM-setups typically achieve higher levels of detectable control (i.e., statistical resemblance to a specific writing style) than the BoW-based setups, but generate significantly more perplexing and repetitive responses. This could be attributable to the fact that BoW-based control is more local (i.e., token-level) and less invasive than discriminator-based control, which can operate at the structural level. The results also indicate that the underlying language models used in the PPLM-setups, GPT-2 and DialoGPT, are both biased towards generating younger sounding language (according to the best performing classifier from Experiment 1). This is most likely due to both models having been pre-trained on texts scraped from web pages, which are typically over-represented by millennials \citep{radford2019language, zhang2019dialogpt}. 
% Quantitative analyses of the dialogue generation results revealed that the outputs of some PPLM-setups display a correlation between improved control and worsened perplexity and diversity.
Quantitative analyses of the dialogue generation results revealed that the outputs of some PPLM-setups show a correlation between improved target probability and worsened perplexity and diversity. This could be interpreted as a tradeoff between control and quality.
% control and worsened perplexity and diversity.
Also, the style of the prompt (i.e., whether it is classified by BERT$_{FT}$ as younger, neutral, or older) steers the style of the generated response in the direction of the prompt's style. This observation indicates the importance of taking prompt-induced biases into account when developing controlled generation models. It also appears that BoW-based models are significantly worse at controlling for older sounding language than they are for younger sounding language.
This could be attributable to the the possibility that overcoming the younger biases of the underlying language models (GPT-2 and DialoGPT), to produce older sounding responses, needs more invasive perturbations than BoW-based attribute models can bring about, given the current parameter-settings.
% It could also be that there are some detectable, strongly younger sounding, tokens, that make the token-level perturbations made by BoW-based models more likely to contain younger features detectable by my models, whereas older sounding features are more salient at the structural level and thereby harder to control for. 
% BERT$_{FT}$'s attention weight visualizations could also be evidence of this phenomena, because the attention weights of linguistic-feature attending heads are much more focused for strongly younger sounding input, than the more dispersed attention weights of older sounding sentences (see Figures \ref{fig:bertviz_model_view_ypr1}, \ref{fig:bertviz_model_view_ypr2}, \ref{fig:bertviz_model_view_opr1} and \ref{fig:bertviz_model_view_opr2}). 
Finally, qualitative inspection of generated responses reveal that differences in style are noticeable by use of age-indicative words, formality of language, and the prevalence of certain topics.

% \textbf{Paragraph about generation results}
% \begin{itemize}
%     \item The controlled dialogue generation results indicate that it is possible to use PPLM to generate dialogue responses that have been detectably adapted to the styles of different age groups.
%     \item Discriminator-based control typically achieves higher levels of measurable control than BoW-based models, but generates significantly less fluent responses.
%     \item The underlying language models, GPT-2 and DialoGPT, are both biased towards generating younger sounding language. Due to their training data etc.
%     \item For responses generated by some setups, there seems to be correlation between improved adaptation and worsened perplexity and diversity.
%     \item Controlling for the younger age groups seems easier because of (1) bias?, (2) 
%     \item The style of the prompt (i.e., whether it's classified as young, neutral, or old) steers the style of the generated response in the direction of the prompt's style.
%     \item BoW-based models are significantly worse at controlling for old-language than they are for young.
%     \begin{itemize}
%         \item It could be that there are some detectable, very young-sounding, tokens, that make the token-level perturbations made by BoW-based models more likely to contain detectably young-features, whereas old-sounding features are more salient at the structural syntactical level and thereby harder to control for.
%         \item BERT$_{FT}$ attention weight visualization could also be evidence of this phenomena as the attention weights of linguistic-feature attending heads are much more focused for strongly young-input, than the more dispersed attention weights of old sentences.
%         \item However, it could also just be that old features are being detected by different (combinations of) heads. --> Future research.
%     \end{itemize}
%     \item Longer responses are deemed less perplexing.
%     \item Differences in style, brought about by PPLM-control, are noticeable by use of slang words, certain topics, etc. See qualitative analysis.
% \end{itemize}

% \section{Contributions and Implications}\label{sec:contributions}

% My age detection experiments build on the work of \cite{schler2006effects}, who focus on age detection in written discourse using handcrafted features. I extend their work by: \textbf{(1)} eliminating the need for handcrafted features by learning end-to-end representations using state-of-the-art NLP models; \textbf{(2)} applying this approach to dialogue data, using a dataset of transcribed spontaneous open-domain dialogues; \textbf{(3)} showing that text-based models can indeed detect age-related differences, even in the case of very sparse signals at the level of dialogue utterances; \textbf{(4)} carrying out an in-depth analysis of the models' predictions to gain insight about which elements of language use are most informative. Furthermore, the age detection analyses motivate the use of local features (i.e., BoW-based attribute models) for controlled generation as a viable alternative to neural discriminator-based attribute models. My work on age detection from dialogue can be considered a preliminary step to the modeling of age-related linguistic adaptation by AI conversational systems. In particular, these results informed my work on controlled dialogue generation using PPLM.

% My work on controlled dialogue generation using PPLM builds on previous work on controlled language generation by \cite{dathathri2019plug}, who focus on controlled story writing for concrete styles (e.g., sentiment, or topic). I extend their work in several important ways: \textbf{(1)} I control language generation for more abstract writing styles, i.e., age group specific linguistic style; \textbf{(2)} I use PPLM for dialogue response generation; \textbf{(3)} I propose methods for empirical development of BoW attribute models (as opposed to the manually curated BoWs used by \cite{dathathri2019plug}) and demonstrate their applicability for controlled dialogue generation; \textbf{(4)} I thoroughly study the relationships between dialogue response quality, response style, and response length; 
% (5) I also contribute to a clearer understanding of what age-related features a fine-tuned BERT model seems to focus on when classifying generated responses; 
% and finally \textbf{(5)} I carry out an extensive analysis on the effects of prompt-induced biases on the quality and style-attribute adherence of generated language, which has been overlooked by previous work on Plug-and-Play generation. 
% Despite previous work also focusing on 
% Plug-and-Play 
% conversational models, my work demonstrates a Plug-and-Play approach to controlled dialogue generation, without the need to generate attribute-specific dialogue datasets, or separately optimize residual adapter modules \citep{madotto-etal-2020-plug}.

% Overall, this research is a promising step towards the development of adaptive conversational systems. In particular, the development of age-adaptive conversational systems can benefit from these results. Since consistent language style differences
% were found between age groups, systems whose language generation capabilities aim to be consistent with a given age group should therefore reproduce these patterns. This could be achieved, as I have shown, by embedding Plug-and-Play modules that control the generation of a system’s output, which could lead to better, more natural interactions between human speakers and a conversational system.

% \begin{itemize}
%     \item \len{Refine the statements relating to age detection.}
%     \item The age detection results build on the work of \cite{schler2006effects}, who focus on age detection in written discourse using handcrafted features.
%     \item I extend this work by (1) eliminating the need for handcrafted features by learning end-to-end representations using state-of-the-art NLP models; (2) applying this approach to dialogue data, using a dataset of transcribed spontaneous open-domain dialogues; (3) showing that text-based models can indeed detect age-related differences, even in the case of very sparse signal at the level of dialogue utterances; (4) I carry out an in-depth analysis of the models' predictions to gain insight about which elements of language use are most informative.
%     \item The age detection analyses also motivate the use of local features for controlle generation, in combination with neural representations(?).
%     \item My work can be considered a first step toward the modeling of age-related linguistic adaptation by AI conversational systems. In particular, my results can inform future work on controlled text generation for dialogue agents.
%     \item Themore abstract style attribute to control for in a PPLM setup than previous work.
%     \item This work extends original PPLM work by...
%     \begin{itemize}
%         \item using empirically generated BoWs (While previous research has focused on manually curated wordlists as BoWs, these results demonstrate the applicability of various empirical methods for BoW attribute model development..;
%         \item controlling for a more abstract writing style;
%         \item studying the relationships between response quality, age-adaptation, and response length.
%         \item studying the effects of prompt-induced bias, previous work on PPLM doesn't take prompt-bias into account.
%     \end{itemize}
%     \item Also contributes to a clearer understandig of what features finetuned BERT seems to focus on when classifying generated responses.
%     \item Despite previous work by \cite{madotto-etal-2020-plug} also focusing on Plug-and-Play conversational models, my work demonstrates the achievability of a Plug-and-Play approach to controlled dialogue generation, without the necessity to generate attribute-specific dialogue datasets, or separately optimize residual adapter modules.
%     \item Contributes to using low-cost methods for controlled generation
%     \item Overall, my research is a promising step towards the development of personalized virtual assistants.
% \end{itemize}

\section{Limitations}\label{sec:limitations}

% \len{Double check if you're not undermining your research. The discussion of limitations should aim to strengthen your credibility, not emphasize weaknesses or failures.}

There are several limitations to take into account when interpreting the results presented in this thesis. The generalizability of the age detection results are limited by the coarse granularity with which the age groups are defined. One could imagine that if one aims to leverage the predicted age signal in real-world conversational systems, a fine-grained grouping is more reasonable. Furthermore, age group information remains difficult to identify in short utterances. A non-negligible proportion of utterances seem too short for reliable classification, and it remains to be investigated if NLP models or even humans could identify the targets.
Acoustic signals are not considered in this work, but they are important characteristics of speaking style, and could also be taken into account when developing audio-based conversational systems. On the one hand, this richer signal (i.e., the combination of acoustic and textual information) could lead to better classification performance. On the other hand, the use of acoustic signals could lead to ethical privacy-related issues, that aggregated text-based approaches might have to a lesser extent. 
This is discussed in more detail in Section \ref{sec:considerations}.
% Furthermore, for future applications, the combination of speech analysis could improve performance. Acoustic signals are not considered in this work, but they are important characteristics of speaking style, and could also be taken into account when developing audio-based conversational systems. 

% Due to lack of information about the \textit{none} and \textit{no info} topics, the age detection results cannot rule out confounding factors. This is especially important for the 50-plus age group, where the \textit{none} and \textit{no info} topics apply to nearly 50\% of the utterances. Nevertheless, my analysis of classification performance per topic do not seem to indicate severe differences in the identifiability of age-related signals between topic categories. 

With respect to the evaluation of the generated responses (see Section \ref{subsec:exp_setup_eval}), the reliability of the controlled dialogue generation results is impacted by perplexity (measured by GPT-1) being a crude proxy for fluency. Perplexity also lacks generalizability as an evaluation measure, because it only measures the uncertainty assigned by one language model. Automated evaluation of fluency could be improved by considering an aggregated perplexity measure (i.e., perplexity averaged over an ensemble of language models). Alternatively, indications of a generated response's fluency could be made more reliable by having humans participate in an evaluation study to rate the responses, and taking their opinions into account. 

It is also important to realize that the learned representations for younger and older style used in this research are specific to the BNC \citep{love-spoken-bnc-2014}, and should not be interpreted as general representations of the speaking style of people of ages 19-29 and 50-plus. Even within the context of the BNC, the representations used for classification and generation 
correspond to textual features which have been learned by Experiment 1's models to have a high likelihood of coming from a younger or older speaker, 
and are therefore approximations of the age groups' speaking styles. 

Finally, due to the lack of age-labeled transcribed spontaneous dialogue data, I was constrained to using the same dialogue dataset to develop the attribute models (i.e., the BoWs or neural discriminators used in PPLM-setups), and to train evaluation model (i.e., BERT$_{FT}$). It must be noted that this is still sound practice, because the attribute and evaluation models are not trained jointly, and do not share representations. Moreover, the approach of using an external classifier for evaluation is based on the work of \cite{dathathri2019plug} and \cite{madotto-etal-2020-plug}. At the same time, the assigned target probabilities indicating attribute adherence would be more generalizable if the evaluation model were trained on a different dataset of spontaneous dialogue utterances between speakers of the same age.



% \begin{itemize}
%     \item Can any ML architecture pick up signals from 1-6 token sequences? (See workshop paper submission feedback).
%     \item The age detection classification problem's age groups are defined with coarse granularity. Imagine that if I aim to leverage the predicted age signal in real systems, fine-grained grouping is more reasonable.
%     \item unfortunately, there are still problems especially as far as the identification of age groups with short utterances. Nevertheless, for future applications, the combination of speech analysis could bring to better results. (the sentences seem to short (6~7 words), it is doubtful if  ML models or even human could recognize the targets.)
%     \item What are the "none", "no info", "unknown" labels in the two datasets? This could be very critical given that the none + no info topics take nearly 50\% in the 50+ age group.
%     \item Perplexity is a very crude proxy for fluency. --> human evaluation or more reliable automated evaluation metric for fluency.
%     \item The confounding effects of topic on age detection and controlled generation.
%     \item I need to find out how much of "control" is attributable to method and how much to prompt-bias.
%     \item Interactivity is beyond the scope of this thesis
%     \item The methodological choices about age group granularity were constrained by insufficient age-labeled dialogue data for all age groups of interest.
%     \item The representations for young and old style used in this research are specific to the BNC, and should not be interpreted as generally representative of speaking style of 19-29 and 50-plus.
%     \item Even within the context of the BNC, the representations used for classification and generation are indicative of textual features learned to coincide with utterances from certain age groups, and are not to be confused with general representations of age groups' speaking styles.
%     \item Acoustic signals are not considered in this work, but they are important characteristics of speaking style, and should also be taken into account when developing audio-based conversational systems.
%     \item despite formulation of binary classification requiring young-prob and old-prob to be complementary values, "young" and "old" speaking/writing styles are not semantically opposite styles, like positive vs negative sentiment tend to resemble more \len{Maybe think of a better example of semantic polar opposites. Also, maybe also mention that it still needs to be verified whether they are semantically opposite or not, but that you hypothesize that they are not. Idea for future research.}. So I shouldn't expect certain patterns in, e.g., ppl plots to be exactly opposite.
%     \item Attention is not explanation.?
%     \item Due to the lack of age-labeled transcribed spontaneous dialogue data, I were constrained to using the same dialogue dataset to develop the attribute models (i.e., the BoWs or neural discriminators used in PPLM-setups), and to train evaluation model (i.e., BERT$_{FT}$). It must be noted that this is still sound practice, because the attribute and evaluation models are not trained jointly, and do not share representations. Moreover, the approach is based on the work by \cite{dathathri2019plug} and \cite{madotto-etal-2020-plug}. However, the assigned target probabilities indicating attribute adherence would be more generalizable if the evaluation model were trained on a different dataset of spontaneous dialogue utterances between speakers of the same age.
% \end{itemize}

\section{Future Research Directions}

While the classification task in Experiment 1 was performed at the level of single dialogue utterances, future work may take into account larger dialogue fragments, such as the entire dialogue or a fixed number of turns. This would make the setup more comparable to discourse, but but would require making experimental choices and dealing with extra computational challenges. Moreover, it could be tested whether the language used by a speaker is equally discriminative when talking to a same-age (this work) or a different-age interlocutor. Also, future work might involve leveraging acoustic signals for automated age-detection and controlled generation. Acoustic signals can contain important features indicative of speaking style, and could therefore substantially improve classification and controlled generation performance.

Developing interactive age-adaptive conversational systems was beyond the scope of this thesis. Therefore, an interesting future research objective would be to use Plug-and-Play Conversational Models (PPCM) for age-adaptive dialogue generation \citep{madotto-etal-2020-plug}. PPCM solves a latency problem of PPLM, making it usable in interactive settings. However, the use of PPCM in combination with more abstract speaking styles and BoW-based attribute models has not been researched (to the best of my knowledge). Using PPCM for age-adaptive dialogue generation would require developing and training residual adapter modules, and generating style-specific dialogue datasets. Furthermore, the age detection results suggest the importance of $n$-gram features for age-identification. Therefore, an interesting future research direction would be to adapt PPLM and PPCM to be compatible with $n$-gram lists (for $n>1$) as attribute models. This would require finding a way to by-pass the need to re-train large underlying language models like GPT-2 and DialoGPT for arbitrary $n$-grams. Finally, the current set of tools to analyze which features drive controlled generation in PPLM-setups is limited. It would be valuable to develop probing methods for PPLM-setups. Namely, visualizations or saliency methods targeted towards understanding the activation space perturbations made by PPLM could provide important insights about the effects of attribute models on controlled generated output.

% \len{Another future research idea: PPLM for intersections of styles/attributes, e.g., young-positive, or old-family. As it seems now, age-related style seems to strongly correlate with topics. Would be interesting to see if these age-related style aspects can be isolated from the topics they often coincide with.}
% \begin{itemize}
%     \item While I performed the classification task at the level of single dialogue utterances, future work may take into account larger dialogue fragments, such as the entire dialogue or a fixed number of turns. This would make the setup more comparable to discourse, but but would require making experimental choices and dealing with extra computational challenges. Moreover, it could be tested whether the language used by a speaker is equally discriminative when talking to a same-age (this work) or a different-age interlocutor.
%     \item The confounding effects of topic on age detection and controlled generation. Topic-agnostic age control.
%     \item Developing interactive age-adaptive conversational systems was beyond the scope of this thesis. Therefore, an interesting future research objective would be to use PPCM \citep{madotto-etal-2020-plug} in for age adaptive dialogue generation. 
%     \item Leveraging acoustic signals for automated age-detection and controlled generation.
%     \item Future research idea: adapting the PPLM-setup to work with $n$-gram lists for arbitrary $n$.
%         \begin{itemize}
%             \item Finding a way to by-pass the need to retrain GPT-2 for arbitrary n-grams
%         \end{itemize}
%     \item Future research idea: how to probe PPLM models, because BertViz doesn't work, as the attention weights are unchanged by PPLM.
%     \item Emphasize the importance of PPLM-methods w.r.t. carbon footprint and the ecological cost of (re)training massive language models like GPT-x (Maybe this is better for the introduction?)
% \end{itemize}

\section{Ethical and Environmental Considerations}\label{sec:considerations}

There is growing concern around the ethics, environmental costs, and societal dangers associated with powerful large-scale language models \citep{brown2020language-models-few-shot-gpt3, bender2021dangers}. 
There is a non-negligible societal risk accompanying the recent trend of deploying ever larger language models that are more capable of producing texts that are indistinguishable from human-written ones. 
Namely, such language models can be used to spread misinformation (e.g., fake news), or to perpetuate harmful social biases.
Plug-and-Play language models carry the same risks, if not to a greater extent, as they can be used to produce personalized, and ultimately more convincing text.
As seen in the analysis of generated responses in Section \ref{subsec:ctg_anal_qualitative}, some examples of the models' outputs can be seen as exacerbating biases present in the used dialogue dataset (and society) (e.g., older speakers talking about healthcare, younger speakers talking about partying). This bias-amplification could be attributable to the models also to picking up and generating topic-related features, aside from purely stylistic ones. Future considerations for work on controlled generation should therefore also include methods to ensure adaptation happens at the stylistic level, and avoid amplification of (topical) biases in the data being used.

Furthermore, as previously mentioned in Section \ref{sec:limitations}, style-detection performance of adaptive dialogue systems could be improved by the use of acoustic signals, in addition to purely text-based information. However, privacy concerns should be taken into account when augmenting signals for dialogue systems, as this leads to more user-information being processed, making users easier to identify. Researchers should therefore consider adaptation methods that identify users with as sparse a signal as possible.
% Plug-and-Play language models carry the same risks, if not to a greater extent, as they can be used to produce personalized, and ultimately more convincing text. 

However, \cite{dathathri2019plug} also demonstrate the benevolent applicability of PPLM for language detoxification, which can help to reduce the perpetuation of harmful biases.
Furthermore, Plug-and-Play methods, which avoid substantial re-training costs of massive language models, can help to reduce the energy consumption and dampen the environmental impact of developing modern deep learning models \citep{strubell-etal-2019-energy}.
Additionally, the same methods that generate convincing texts (like GPT-2) can be used to build more sophisticated tools for detecting artificially generated texts \cite{gehrmann-etal-2019-gltr}. 
Overall, I believe that the positive applications of PPLM for, e.g, personalized dialogue generation, outweigh the potential risks of misuse. 

% \begin{itemize}
%     \item Carbon footprint of training large language models and how PPLM helps to alleviate this.
%     \item The societal dangers of language generation w.r.t. e.g. fake news
%     \item What are the dangers of these methods? --> Read up on paper by Ebru et al
%     \item The analyses about the effects of using strongly stylized prompts on the style of generation revealed that there is a considerable prompt-induced bias. Despite, efforts having been made in this study to avoid this bias by using neutral prompts, more research is needed to find out how much of "control" is attributable to method and how much to prompt-bias.
%     \item despite formulation of binary classification requiring young-prob and old-prob to be complementary values, "young" and "old" speaking/writing styles are not semantically opposite styles, like positive vs negative sentiment tend to resemble more \len{Maybe think of a better example of semantic polar opposites. Also, maybe also mention that it still needs to be verified whether they are semantically opposite or not, but that you hypothesize that they are not. Idea for future research.}. So I shouldn't expect certain patterns in, e.g., ppl plots to be exactly opposite.
% \end{itemize}

% \section{Bullet points}

% \begin{itemize}
%     \item \len{Re-iterate this list and recap \textit{why} certain decisions were made.}
%     \item The research question was if it was possible to use PPLM for controlled dialogue generation. Before that, I wanted to confirm that age-related features in transcribed dialogue and discourse are prevalent enough to train various classifiers.
%     \item I performed age classification of utterances.
%     \item BERT was best, then trigram. 
%     \item I studied which features were most salient in driving classification, how classification performance related to dialogue topic, and which cases were most challenging to classify for BERT and trigram.
%     \item \len{\textbf{IMPORTANT:} Explicitly Highlight connection/bridge between experiment1 and experiment2}
%     \item After confirming the detectability of age-related linguistic patterns in dialogue, I seek to control generated responses for age-related traits.
%     \item based on the effectiveness of n-gram based classification model compared to transformer-based, I use PPLM-setups with both bow-based and discriminator-based attribute models.
%     \item I generated dialogue responses to neutral, young, and old prompts using models based on either GPT-2 or DialoGPT, and used discriminator- or bow-based attribute models.
%     \item I analyzed the relationship between perplexity and target probability, the relationship between various evaluation metrics and response length, the effects of prompt class, BERT$_{FT}'s$ attention heads, and qualitative properties of generated samples manually.
% \end{itemize}

% \paragraph{Discussion points about classification}
% \begin{itemize}
%     \item Can any ML architecture pick up signals from 1-6 token sequences? (See workshop paper submission feedback).
%     \item age-related linguistic features that inform classification lie more at the syntactic level than at the lexical level.
%     \item A small discussion point on the effects of stopword omission on classification performance
%     \item \len{Re-read the relevant sections of the workshop paper submission}
% \end{itemize}

% \paragraph{Discussion points about generation}
% \begin{itemize}
%     \item What are the effects of prompts on generation? \len{This should probably be an analysis question.}
%     \item What are the limitations of your setup?
%         \begin{itemize}
%             \item Perplexity is a crude proxy for fluency and grammatical correctness.
%         \end{itemize}
%     \item Does this make the world better? How can this help people? --> It can help personalize virtual assistants (especially useful for new speakers of a language. E.g., the difference between young/informal/spoken French and French that is taught in school and courses is large. User-age personalization can adapt use of language of virtual assistants to variant of language spoken by user.)
%     \item What are the dangers of these methods? --> Read up on paper by Ebru et al
%     \item What are interesting future research directions?
%     \item My research is a promising step towards the development of personalized virtual assistants.
%     \item Keep in mind that...
%         \begin{itemize}
%             \item  The representations for young and old style used in this research are specific to the BNC, and should not be interpreted as generally representative of speaking style of 19-29 and 50-plus.
%             \item Even within the context of the BNC, the representations used for classification and generation are indicative of textual features learned to coincide with utterances from certain age groups.
%             \item despite formulation of binary classification requiring young-prob and old-prob to be complementary values, "young" and "old" speaking/writing styles are not semantically opposite styles, like positive vs negative sentiment tend to resemble more \len{Maybe think of a better example of semantic polar opposites. Also, maybe also mention that it still needs to be verified whether they are semantically opposite or not, but that you hypothesize that they are not. Idea for future research.}. So I shouldn't expect certain patterns in, e.g., ppl plots to be exactly opposite.
%         \end{itemize}
%     \item Future research idea: adapting the PPLM-setup to work with $n$-gram lists for arbitrary $n$.
%         \begin{itemize}
%             \item Finding a way to by-pass the need to retrain GPT-2 for arbitrary n-grams
%         \end{itemize}
%     \item Future research idea: real-time interactivity of age-adaptive conversational systems. I.e., a pipeline that (1) "starts off neutral", (2) classifies user's age based on minimal amount of utterances, (2*) uses bayesian modelling or reinforcement learning to constantly update belief, (3) adapts use of language to perceived user age.
%     \item Future research idea: how to probe PPLM models, because BertViz doesn't work, as the attention weights are unchanged by PPLM.
%     \item Emphasize the importance of PPLM-methods w.r.t. carbon footprint and the ecological cost of (re)training massive language models like GPT-x (Maybe this is better for the introduction?)
% \end{itemize}

\chapter{Conclusion}\label{ch:conclusion}
% \len{Revise this.}

In this thesis, I first investigate the extent to which text-based NLP-models can detect age-related linguistic features in dialogue data and which features drive their predictions. Then, I study the extent to which age-adaptive dialogue generation is possible using Plug-and-Play language models (PPLM) \citep{dathathri2019plug}.

The results of the age detection experiments show that a fine-tuned BERT model is capable of detecting age-related linguistic features in dialogue utterances with reasonable accuracy, especially when the dialogue fragment is long enough to contain discriminative signal. However, simpler $n$-gram-based models achieve comparable performance, suggesting that, in dialogue, ‘local’ features can be indicative of the language of speakers from different age groups. This is shown to be the case, with both lexical and stylistic cues being informative to these models in this task. Furthermore, the age detection results informed the subsequent experiments about controlled dialogue generation using PPLM.

The controlled dialogue generation results show that it is possible to use PPLM to generate dialogue responses that possess detectable linguistic features associated with specific age groups. Discriminator-based PPLM-setups typically achieve higher levels of detectable control 
% (i.e., statistical resemblance to a specific writing style) 
than bag-of-words (BoW) based setups, but generate significantly more perplexing and repetitive responses. This could be attributable to the fact that BoW-based control is more local (i.e., at the token level) and less invasive than discriminator-based control, which can operate at the structural level. 

Overall, I believe that the research presented in this thesis is a promising step towards the development of adaptive conversational systems. In particular, the development of age-adaptive conversational systems can benefit from these results. Because consistent language style differences were found between age groups, systems whose language generation capabilities aim to be consistent with a given age group should reproduce these patterns. This could be achieved, as shown in this work, by embedding Plug-and-Play modules that control the generation of a system’s output, which could lead to more natural interactions between human speakers and a dialogue system. 
In an increasingly automated society, where dialogue systems often take over simple tasks related to, e.g., customer service, perceived natural interaction between users and these systems is crucial to their optimal functioning.
And finally, the development of dialogue systems with which users feel they can naturally interact brings us closer to achieving the longstanding goal of AI-powered human-like conversation.
% Aside from natural interaction between users and dialogue systems bringing us closer to achieving AI-powered human-like conversation, it will also increase the trust of users in these systems. This increased trust can benefit 

% Overall, I believe this thesis is a promising step towards the development of age-adaptive conversational systems, which can lead users experience more trust and ease of use when interacting with these systems.

% References ---------------------------------------------------------------
\newpage
\singlespacing
\bibliographystyle{abbrvnat}
\bibliography{references}

\newpage
\doublespacing
\appendix
\chapter{Supplementary material}
\section{Wordlists for BoW-based Approaches}
\label{sec:wordlists}

\len{TODO - Censor foul language?}

\paragraph{100 Most Informative Unigrams - Young (19-29)} um, cool, shit, hmm, uni, cute, tut, massive, awesome, gym, bitch, lol, grand, pizza, like, excited, yawn, Korea, cigarette, fuck, fairness, Jesus, annoying, Facebook, quicker, definitely, guess, Sunderland, oo, wanna, mountain, scared, piss, love, miss, Middlesbrough, mhm, specifically, ooh, website, roundabout, photo, nope, blanket, management, ridiculous, mental, pregnant, beers, hate, log, fucking, cry, cheaper, skinny, plural, burger, hilarious, hint, drunk, fridge, cousin, coke, genuinely, James, mates, smaller, option, balance, saving, basically, leather, nev, shut, frig, mate, yay, invite, maid, nickname, badly, garlic, CD, jokes, Uzbekistan, boyfriend, date, added, Manchester, blah, shitty, lang, tempted, stadium, wee, eh, baking, city, honestly, exam

\paragraph{100 Most Informative Unigrams - Old (50 plus)} ordinary, Chinese, wonderful, yes, tend, father, photographs, vegetables, hospice, operation, shed, pension, areas, mother, hanging, hospices, glasses, chap, anyhow, tank, surgery, container, cheers, born, church, pain, several, workshop, right, horses, building, extraordinary, vegetarian, biscuit, americano, engine, luck, paint, emperor, lipsy, trombone, occasional, supper, lord, architect, council, roast, schools, bath, asbestos, endometrial, concrete, poodle, recall, diabetes, misty, report, heavens, enormous, lawn, potatoes, email, junk, scabies, mousse, Ebola, churches, sewing, plants, rackets, marmalade, engineering, furniture, photograph, sandwiches, unemployment, xylophone, Piccadilly, flu, claim, arab, nineteen, forgotten, sensible, blancmange, spencer, yards, emails, yellow, scruffy, fungi, garden, boiler, lodge, mostly, Robson, tricky, shark, robin, contracture

\paragraph{Frequency-based Young (19-29)} um,
shit, cool, fucking, definitely, guess, friends, everyone, literally, dad, sounds, weekend, loads, watch, fair, fuck, amazing, friend, ha, huh, hate, fun, stay, girl, holiday, blah, hours, uni, month, horrible, massive, Friday, stupid, film, parents, thirty, spend, mate, honest, change, hope, yourself, annoying, wear, wait, ridiculous, anyone, Saturday, tea, dinner, sit, crazy, hell, pound, nine, expensive

\paragraph{Frequency-based Old (50 plus)} building, may, water, mother, perhaps, door, lots, business, cancer, area, although, worked, open, cut, number, under, young, nineteen, everybody, garden, church, case, shop, children, certainly, set, coffee, email, gave, white, along, doctor, hear, often, possibly, group, father, outside, wonderful, taken, seem, places, green, given, hand, early, women, space, front, language, dear, light, huge, supposed, country, hospital, otherwise, asked, putting, bits, gosh, wall, woman, almost, particularly, across, word, age, rest, flat, turned, decided, finished, needed, red, bin, hospice, running, slightly, its, middle, local, percent, Chinese, paper, check, high, milk, piece, near, nobody, usually

\section{Where to put these?}

\begin{figure}[H]
    \centering
    \includegraphics[width=0.5\textwidth]{figures/cm_bert_test_dt_27_May_2021.png}
    \caption{Confusion matrix BERT age classifier on balanced BNC \textbf{test} set.}
    \label{fig:cm_bert_bnc_rb}
\end{figure}

\begin{figure}[H]
    \centering
    \includegraphics[width=0.5\textwidth]{figures/cm_lstm_test_dt_24_May_2021_09_24_30.png}
    \caption{Confusion matrix LSTM age classifier on balanced BNC \textbf{test} set.}
    \label{fig:cm_lstm_bnc_rb}
\end{figure}

\begin{figure}[H]
    \centering
    \includegraphics[width=0.5\textwidth]{figures/cm_lstm_test_dt_24_May_2021_10_03_19.png}
    \caption{Confusion matrix bi-LSTM age classifier on blog corpus \textbf{test} set.}
    \label{fig:cm_lstm_blog}
\end{figure}

\begin{figure}[H]
    \centering
    \includegraphics[width=0.5\textwidth]{figures/cm_3_gram_bnc_rb_dt_08_Jun_2021_12_05_02.png}
    \caption{Confusion matrix for best trigram age classifier on \textbf{balanced} BNC \textbf{test} set.}
    \label{fig:cm_trigram_bnc_rb}
\end{figure}

\section{Age Discrimination on the Imbalanced British National Corpus \len{Do we even need these plots?}}
\label{age_disc_bnc}

\begin{figure}[H]
    \centering
    \includegraphics[width=0.5\textwidth]{figures/cm_2_gram_bnc_dt_08_Jun_2021_11_33_16.png}
    \caption{Confusion matrix for best bigram age classifier on BNC test set.}
    \label{fig:cm_bigram_bnc}
\end{figure}

\begin{figure}[H]
    \centering
    \includegraphics[width=0.5\textwidth]{figures/cm_lstm_test_dt_22_May_2021_12_41_50.png}
    \caption{Confusion matrix bi-LSTM age classifier on BNC test set.}
    \label{fig:cm_lstm_bnc}
\end{figure}

\begin{table}[b!]
    \centering
    \begin{tabular}{@{}l l @{\hspace*{25pt}} l l@{}}
    \toprule
    \multicolumn{2}{c}{19-29} & \multicolumn{2}{c}{50+}\\
    \textbf{coef.} & \textbf{n-gram} & \textbf{coef.} & \textbf{n-gram}\\
    \midrule
    -3.19 & um & 2.29 & yes\\
    -2.91 & cool & 2.21 & wonderful\\
    -2.70 & s**t & 1.91 & building\\
    -2.25 & cute & 1.86 & right right\\
    -2.15 & uni & 1.80 & something like\\
    -2.14 & hmm & 1.73 & garden\\
    -1.97 & wanna & 1.69 & right\\
    -1.93 & f**k & 1.68 & ordinary\\
    -1.91 & like & 1.67 & shed\\
    -1.85 & massive & 1.63 & operation\\
    -1.83 & yeah course & 1.58 & born\\
    -1.81 & love & 1.57 & mother\\
    -1.79 & tut & 1.55 & photographs\\
    -1.74 & b***h & 1.51 & email\\
    -1.68 & like oh & 1.08 & anything like\\
    \bottomrule
    \end{tabular}
    \caption{\len{Excluding stopwords.} For each age group, top 15 most informative $n$-grams used by the trigram model. \textbf{coef.} is the coefficient (and sign) of the corresponding $n$-gram for the logistic regression model: the higher its absolute value, the higher the utterance's odds to belong to one age group.
    % Greater absolute value of \textbf{coef.} indicates occurrence of the $n$-gram results in the model assigning higher odds to the utterance belonging to a certain age group.
    * indicates masking of foul language.}
    % present in the dialogue.}
    \label{tab:top_ngrams}
\end{table}

% \begin{table}[H]
%     \centering
%     \begin{tabular}{@{}l l @{\hspace*{25pt}} l l@{}}
%     \toprule
%     \multicolumn{2}{c}{19-29} & \multicolumn{2}{c}{50+}\\
%     \textbf{coef.} & \textbf{n-gram} & \textbf{coef.} & \textbf{n-gram}\\
%     \midrule
%     -3.20 & um & 2.37 & yes\\
%     -2.84 & cool & 2.12 & you know\\
%     -2.58 & s**t & 2.09 & wonderful\\
%     -2.12 & hmm & 1.90 & how weird\\
%     -2.09 & like & 1.84 & chinese\\
%     -2.02 & was like & 1.73 & right\\
%     -1.96 & love & 1.71 & building\\
%     -1.96 & as well & 1.66 & right right\\
%     -1.88 & as in & 1.55 & so erm\\
%     -1.84 & cute & 1.43 & mm mm\\
%     -1.82 & uni & 1.41 & cheers\\
%     -1.79 & massive & 1.39 & shed\\
%     -1.79 & wanna & 1.37 & pain\\
%     -1.79 & f**k & 1.36 & we know\\
%     -1.72 & tut & 1.08 & yeah exactly\\
%     \bottomrule
%     \end{tabular}
%     \caption{\len{Including stopwords.} For each age group, top 15 most informative $n$-grams used by the trigram model. \textbf{coef.} is the coefficient (and sign) of the corresponding $n$-gram for the logistic regression model: the higher its absolute value, the higher the utterance's odds to belong to one age group.
%     % Greater absolute value of \textbf{coef.} indicates occurrence of the $n$-gram results in the model assigning higher odds to the utterance belonging to a certain age group.
%     * indicates masking of foul language.}
%     % present in the dialogue.}
%     \label{tab:top_ngrams_ws}
% \end{table}

% \begin{table*}[h]
%     \centering
%     % \resizebox{\columnwidth}{!}{
%     \begin{tabular}{l c c c}
%     \toprule
%     \textbf{Model} & \textbf{Accuracy} & $\boldsymbol{F}_1^{(19-29)}$  & $\boldsymbol{F}_1^{(50+)}$ \\ 
%     % -plus)}$ \\
%      & $\uparrow$ better & $\uparrow$ better & $\uparrow$ better \\
%     \midrule
%     Random
%     % Baseline (random guessing) 
%     & 0.500
%     % (0.000) 
%     & 0.500
%     % (0.000) 
%     & 0.500 
%     % (0.000)
%     \\ \midrule
%     unigram & 0.701 (0.007) & 0.708 (0.009)  & 0.693 (0.004)\\
%     bigram & 0.719 (0.002) & 0.724 (0.003) & 0.714 (0.003)\\
%     trigram &  \textcolor{blue}{0.722} (0.001) & \textcolor{blue}{0.727} (0.003) & \textcolor{blue}{0.717} (0.001)\\ \midrule
%     LSTM &  0.693 (0.003) & 0.696 (0.005) & 0.691 (0.007)\\
%     BiLSTM & 0.691 (0.009) & 0.702 (0.017) & 0.679 (0.007) \\ \midrule
%     BERT$_{frozen}$
%     % -base uncased (frozen) 
%     & 0.675 (0.003) & 0.677 (0.008) & 0.673 (0.010)\\
%     BERT$_{FT}$
%     % -base uncased (fine-tuned) 
%     & \textbf{0.729} (0.002) & \textbf{0.730} (0.011) & \textbf{0.727} (0.010)\\
%     % \hline
%     % GPT-2 medium (frozen) & 0.671 (0.003) & 0.665 (0.014) & 0.675 (0.017) \\
%     \bottomrule
%     \end{tabular}
%     % }
%     \caption{Dialogue dataset \len{Including stopwords.}. Test set results averaged over 5 random initializations. Format: \textit{average metric (standard error)}. Values in \textbf{bold} are the highest in the column; in \textcolor{blue}{blue}, the second highest.}
%     \label{tab:bnc_classification_ws}
% \end{table*}

\begin{table*}[h]
    \centering
    \begin{tabular}{l c c c c}
    \toprule
    \textbf{Model} & \textbf{Accuracy} & $\boldsymbol{F}_1^{(13-17)}$ & $\boldsymbol{F}_1^{(23-27)}$ & $\boldsymbol{F}_1^{(33+)}$\\
    % -plus)}$\\
     & $\uparrow$ better & $\uparrow$ better & $\uparrow$ better & $\uparrow$ better\\
    \midrule
    % Baseline (
    Majority class
    % ) 
    & 0.472
    % (0.000) 
    & * & 0.642
    % (0.000)
    & *\\
    % Best model by \citeauthor
    \citet{schler2006effects} & 0.762 & 0.860 & 0.748 & 0.504 \\
    \midrule
    unigram & 0.601 (0.001) & 0.764 (0.001) & 0.704 (0.001) & 0.498 (0.003)\\
    bigram & 0.625 (0.001) & \textcolor{blue}{0.790} (0.001) & \textcolor{blue}{0.712} (0.001) & \textcolor{blue}{0.518} (0.001)\\
    trigram & 0.623 (0.001) & \textcolor{blue}{0.790} (0.001) & 0.712 (0.002) & 0.498 (0.002)\\
    \midrule
    LSTM & \textcolor{blue}{0.663} (0.005) & 0.748 (0.003) & 0.664 (0.010) & 0.502 (0.004) \\
    BiLSTM & 0.618 (0.008) & 0.732 (0.003) & 0.579 (0.016) & 0.509 (0.004)\\
    \midrule
    BERT$_{frozen}$
    & 0.623 (0.002) & 0.658 (0.006) & 0.678 (0.007) & 0.256 (0.041)\\
    BERT$_{FT}$
    % -base-uncased (fine-tuned)
    & \textbf{0.742} (0.010) & \textbf{0.813} (0.007) & \textbf{0.749} (0.013) & \textbf{0.592} (0.009)\\
    % \hline
    % GPT-2 medium (frozen) & 0.621 (0.003) & 0.654 (0.012) & 0.678 (0.009) & 0.247 (0.032)\\
    \bottomrule
    \end{tabular}
    \caption{Discourse dataset. \len{Including stopwords} Test set results averaged over 5 random initializations. Format: \textit{average metric (standard error)}. Values in \textbf{bold} are the highest in the column; in \textcolor{blue}{blue}, the second highest. *: $F_1$ is actually $0/0$.}
    % \san{If we agree with the abbreviations used in this table, we could use them also in Table 3}}
    \label{tab:blog_classification}
\end{table*}

\begin{table*}[h]
    \centering
    % \resizebox{\columnwidth}{!}{
    \begin{tabular}{l c c c}
    \toprule
    \textbf{Model} & \textbf{Accuracy} & $\boldsymbol{F}_1^{(19-29)}$  & $\boldsymbol{F}_1^{(50+)}$ \\ 
    % -plus)}$ \\
     & $\uparrow$ better & $\uparrow$ better & $\uparrow$ better \\
    \midrule
    Random
    % Baseline (random guessing) 
    & 0.500
    % (0.000) 
    & 0.500
    % (0.000) 
    & 0.500 
    % (0.000)
    \\ \midrule
    unigram & 0.702 (0.006) & 0.713 (0.006)  & 0.690 (0.006)\\
    bigram & 0.703 (0.006) & 0.713 (0.005) & 0.693 (0.008)\\
    trigram &  \textcolor{blue}{0.709} (0.007) & \textbf{0.718} (0.007) & 0.700 (0.008)\\ \midrule
    LSTM & 0.696 (0.005) & 0.689 (0.018) & \textcolor{blue}{0.701} (0.016)\\
    BiLSTM & 0.684 (0.007) & 0.688 (0.018) & 0.679 (0.016) \\ \midrule
    BERT$_{frozen}$
    % -base uncased (frozen) 
    & 0.673 (0.005) & 0.679 (0.013) & 0.667 (0.018)\\
    BERT$_{FT}$
    % -base uncased (fine-tuned) 
    & \textbf{0.710} (0.006) & \textcolor{blue}{0.717} (0.007) & \textbf{0.703} (0.014)\\
    % \hline
    % GPT-2 medium (frozen) & 0.671 (0.003) & 0.665 (0.014) & 0.675 (0.017) \\
    \bottomrule
    \end{tabular}
    % }
    \caption{Dialogue dataset \len{Excluding stopwords}. Test set results averaged over 5 random initializations. Format: \textit{average metric (standard error)}. Values in \textbf{bold} are the highest in the column; in \textcolor{blue}{blue}, the second highest.}
    \label{tab:bnc_classification}
\end{table*}

\begin{table*}[h]
    \centering
    \begin{tabular}{l c c c c c}
    \toprule
    \textbf{Model} & \textbf{ppl.} & \textbf{Dist-1} & \textbf{Dist-2} & \textbf{Dist-3} & \textbf{Acc.}\\
    % -plus)}$\\
     & $\downarrow$ better & $\uparrow$ better & $\uparrow$ better & $\uparrow$ better & $\uparrow$ better\\
    \midrule
    \midrule
    Baseline*** & 27.45 ($\pm$7.27) & 0.90 ($\pm$0.10) & 0.92 ($\pm$0.05) & 0.86 ($\pm$0.09) & 57.5\%\\
    \midrule
    B$_{100MCW}$*** & 26.68 ($\pm$8.77) & 0.89 ($\pm$0.10) & 0.92 ($\pm$0.05) & 0.86 ($\pm$0.09) & 51.7\%\\
    B$_{Y, FB}$ & 27.11 ($\pm$7.45) & \textbf{0.91} ($\pm$0.09) & \textbf{0.92} ($\pm$0.04) & \textbf{0.87} ($\pm$0.09) & 68.3\%\\
    B$_{O, FB}$ & 25.99 ($\pm$6.41) & 0.88 ($\pm$0.11) & 0.92 ($\pm$0.05) & 0.86 ($\pm$0.09) & 62.5\%\\
    B$_{Y, 100MIU}$ & 28.48 ($\pm$11.96) & 0.88 ($\pm$0.12) & 0.91 ($\pm$0.06) & 0.86 ($\pm$0.10) & 69.2\%\\
    B$_{O, 100MIU}$ & \textbf{25.57} ($\pm$7.44) & 0.88 ($\pm$0.11) & 0.92 ($\pm$0.05) & \textbf{0.87} ($\pm$0.09) & 58.3\%\\
    \midrule
    D$_{Y, GPT2}$ & 33.02 ($\pm$12.24) & 0.85 ($\pm$0.16) & 0.89 ($\pm$0.07) & 0.83 ($\pm$0.12) & 73.9\%\\
    D$_{O, GPT2}$ & 32.86 ($\pm$18.08) & 0.80 ($\pm$0.21) & 0.84 ($\pm$0.13) & 0.79 ($\pm$0.19) & 63.3\%\\
    % D$_{Y, GPT2*}$ & 30.98 ($\pm$13.95) & 0.86 ($\pm$0.15) & 0.90 ($\pm$0.06) & 0.84 ($\pm$0.11) & \textbf{80.0\%}\\
    % D$_{O, GPT2*}$ & 34.81 ($\pm$26.76) & 0.84 ($\pm$0.17) & 0.85 ($\pm$0.13) & 0.77 ($\pm$0.23) & 75.8\%\\
    \bottomrule
    \end{tabular}
    \caption{\len{Excluding stopwords.} Results of age-controlled language generation. Perplexity is perplexity w.r.t. GPT-1. Dist-n is number of distinct n-grams normalized by text length, as a measure of diversity. Young and old accuracy are the assigned probabilities of belonging to the young or old age categories.}
    \label{tab:ctg_results}
\end{table*}


\section{Placeholders for Final CTG Results Tables}

\begin{table*}[h]
    \centering
    \begin{tabular}{l | c c c c | c c c}
    \toprule
    \textbf{Model} & \textbf{ppl.} & \textbf{Dist-1} & \textbf{Dist-2} & \textbf{Dist-3} & $\boldsymbol{\bar{P}_Y}$ & $\boldsymbol{\bar{P}_O}$ & \textbf{Acc.}\\
    % -plus)}$\\
     & $\downarrow$ better & $\uparrow$ better & $\uparrow$ better & $\uparrow$ better & & & $\uparrow$ better\\
    \midrule
    \midrule
    G-baseline & 27.50 ($\pm$6.58) & 0.87 ($\pm$0.09) & 0.94 ($\pm$0.04) & 0.90 ($\pm$0.06) & 0.62 ($\pm$0.42) & 0.38 & -\\
    G-100MCW & 27.56 ($\pm$6.60) & 0.86 ($\pm$0.10) & 0.93 ($\pm$0.04) & 0.90 ($\pm$0.05) & 0.63 ($\pm$0.42) & 0.37 & -\\
    \midrule
    G-B$_{FB,Y}$ & 27.91 ($\pm$7.18) & 0.87 ($\pm$0.10) & 0.93 ($\pm$0.05) & 0.90 ($\pm$0.06) & 0.69 ($\pm$0.41) & 0.31 & 70.4\%\\
    G-B$_{FB,O}$ & 27.58 ($\pm$7.07) & 0.86 ($\pm$0.10) & 0.93 ($\pm$0.04) & 0.90 ($\pm$0.06) & 0.58 ($\pm$0.42) & 0.42 & 43.0\%\\
    G-B$_{100MIU,Y}$ & 28.37 ($\pm$7.31) & 0.87 ($\pm$0.09) & 0.93 ($\pm$0.04) & 0.90 ($\pm$0.06) & 0.67 ($\pm$0.41) & 0.33 & 67.4\%\\
    G-B$_{100MIU,O}$ & 27.25 ($\pm$6.15) & 0.87 ($\pm$0.09) & 0.93 ($\pm$0.04) & 0.90 ($\pm$0.06) & 0.62 ($\pm$0.42) & 0.38 & 37.4\%\\
    \midrule
    G-D$_{Y}$ & 32.09 ($\pm$18.98) & 0.77 ($\pm$0.20) & 0.86 ($\pm$0.13) & 0.84 ($\pm$0.15) & 0.66 ($\pm$0.43) & 0.34 & 67.8\%\\
    G-D$_{O}$ & 47.15 ($\pm$47.56) & 0.73 ($\pm$0.24) & 0.75 ($\pm$0.28) & 0.75 ($\pm$0.27) & 0.24 ($\pm$0.36) & 0.76 & 74.3\%\\
    \midrule
    \midrule
    D-baseline & 37.52 ($\pm$12.06) & 0.86 ($\pm$0.13) & 0.90 ($\pm$0.08) & 0.85 ($\pm$0.10) & 0.76 ($\pm$0.37) & 0.24 & -\\
    D-100MCW & 37.80 ($\pm$10.89) & 0.85 ($\pm$0.14) & 0.89 ($\pm$0.10) & 0.85 ($\pm$0.10) & 0.82 ($\pm$0.33) & 0.18 & -\\
    \midrule
    D-B$_{FB,Y}$ & 38.53 ($\pm$12.64) & 0.87 ($\pm$0.12) & 0.90 ($\pm$0.08) & 0.86 ($\pm$0.10) & 0.82 ($\pm$0.33) & 0.18 & 83.0\%\\
    D-B$_{FB,O}$ & 37.85 ($\pm$11.17) & 0.87 ($\pm$0.12) & 0.90 ($\pm$0.08) & 0.86 ($\pm$0.09) & 0.78 ($\pm$0.35) & 0.22 & 21.5\%\\
    D-B$_{100MIU,Y}$ & 38.67 ($\pm$11.70) & 0.88 ($\pm$0.11) & 0.91 ($\pm$0.07) & 0.86 ($\pm$0.10) & 0.87 ($\pm$0.28) & 0.13 & 88.5\%\\
    D-B$_{100MIU,O}$ & 37.91 ($\pm$12.27) & 0.87 ($\pm$0.11) & 0.90 ($\pm$0.07) & 0.85 ($\pm$0.10) & 0.79 ($\pm$0.34) & 0.22 & 21.9\%\\
    \midrule
    D-D$_{Y}$ & 42.01 ($\pm$16.94) & 0.90 ($\pm$0.12) & 0.86 ($\pm$0.14) & 0.77 ($\pm$0.22) & 0.86 ($\pm$0.29) & 0.14 & 85.9\%\\
    D-D$_{O}$ & 41.17 ($\pm$20.72) & 0.87 ($\pm$0.12) & 0.89 ($\pm$0.13) & 0.83 ($\pm$0.16) & 0.43 ($\pm$0.41) & 0.57 & 56.7\%\\
    \bottomrule
    \end{tabular}
    \caption{\len{Neutral prompt} Results of age-controlled language generation. Perplexity is perplexity w.r.t. GPT-1. Dist-n is number of distinct n-grams normalized by text length, as a measure of diversity. $\boldsymbol{\bar{P}_Y}$ and $\boldsymbol{\bar{P}_O}$ are the respective average young and old probabilities assigned by the best BERT$_{FT}$. Acc. is the best BERT model's accuracy when classifying the row's samples.}
    \label{tab:ctg_results_ws_neutral_prompt}
\end{table*}

% \begin{table*}[h]
%     \centering
%     \begin{tabular}{l | c c c c | c c}
%     \toprule
%     \textbf{Model} & \textbf{ppl.} & \textbf{Dist-1} & \textbf{Dist-2} & \textbf{Dist-3} & $\boldsymbol{\bar{P}(Young)}$ & \textbf{Acc.}\\
%     % -plus)}$\\
%      & $\downarrow$ better & $\uparrow$ better & $\uparrow$ better & $\uparrow$ better & $\uparrow$ better & $\uparrow$ better\\
%     \midrule
%     \midrule
%     G-baseline & \textbf{27.50} ($\pm$6.58) & 0.87 ($\pm$0.09) & \textbf{0.94} ($\pm$0.04) & \textcolor{blue}{0.90} ($\pm$0.06) & 0.62 ($\pm$0.42) & -\\
%     G-100MCW & \textcolor{blue}{27.56} ($\pm$6.60) & 0.86 ($\pm$0.10) & \textcolor{blue}{0.93} ($\pm$0.04) & \textbf{0.90} ($\pm$0.05) & 0.63 ($\pm$0.42) & -\\
%     \midrule
%     G-B$_{FB,Y}$ & 27.91 ($\pm$7.18) & 0.87 ($\pm$0.10) & 0.93 ($\pm$0.05) & \textcolor{blue}{0.90} ($\pm$0.06) & 0.69 ($\pm$0.41) & 70.4\%\\
%     G-B$_{100MIU,Y}$ & 28.37 ($\pm$7.31) & 0.87 ($\pm$0.09) & \textcolor{blue}{0.93} ($\pm$0.04) & \textcolor{blue}{0.90} ($\pm$0.06) & 0.67 ($\pm$0.41) & 67.4\%\\
%     \midrule
%     G-D$_{Y}$ & 32.09 ($\pm$18.98) & 0.77 ($\pm$0.20) & 0.86 ($\pm$0.13) & 0.84 ($\pm$0.15) & 0.66 ($\pm$0.43) & 67.8\%\\
%     \midrule
%     \midrule
%     D-baseline & 37.52 ($\pm$12.06) & 0.86 ($\pm$0.13) & 0.90 ($\pm$0.08) & 0.85 ($\pm$0.10) & 0.76 ($\pm$0.37) & -\\
%     D-100MCW & 37.80 ($\pm$10.89) & 0.85 ($\pm$0.14) & 0.89 ($\pm$0.10) & 0.85 ($\pm$0.10) & 0.82 ($\pm$0.33) & -\\
%     \midrule
%     D-B$_{FB,Y}$ & 38.53 ($\pm$12.64) & 0.87 ($\pm$0.12) & 0.90 ($\pm$0.08) & 0.86 ($\pm$0.10) & 0.82 ($\pm$0.33) & 83.0\%\\
%     D-B$_{100MIU,Y}$ & 38.67 ($\pm$11.70) & \textcolor{blue}{0.88} ($\pm$0.11) & 0.91 ($\pm$0.07) & 0.86 ($\pm$0.10) & \textbf{0.87} ($\pm$0.28) & \textcolor{blue}{88.5\%}\\
%     \midrule
%     D-D$_{Y}$ & 42.01 ($\pm$16.94) & \textbf{0.90} ($\pm$0.12) & 0.86 ($\pm$0.14) & 0.77 ($\pm$0.22) & \textcolor{blue}{0.86} ($\pm$0.29) & \textbf{85.9\%}\\
%     \bottomrule
%     \end{tabular}
%     \caption{\len{Neutral prompt - Young model} Results of age-controlled language generation. ppl. is perplexity w.r.t. GPT-1. Dist-n is number of distinct n-grams normalized by text length, as a measure of diversity. $\boldsymbol{\bar{P}_Y}$ and $\boldsymbol{\bar{P}_O}$ are the respective average young and old probabilities assigned by the best BERT$_{FT}$. Acc. is the best BERT model's accuracy when classifying the row's samples.}
%     \label{tab:ctg_results_ws_neutral_prompt_young_models}
% \end{table*}

% \begin{table*}[h]
%     \centering
%     \begin{tabular}{l | c c c c | c c }
%     \toprule
%     \textbf{Model} & \textbf{ppl.} & \textbf{Dist-1} & \textbf{Dist-2} & \textbf{Dist-3} & $\boldsymbol{\bar{P}_O}$ & \textbf{Acc.}\\
%     % -plus)}$\\
%      & $\downarrow$ better & $\uparrow$ better & $\uparrow$ better & $\uparrow$ better & $\uparrow$ better & $\uparrow$ better\\
%     \midrule
%     \midrule
%     G-baseline & \textcolor{blue}{27.50} ($\pm$6.58) & \textbf{0.87} ($\pm$0.09) & \textbf{0.94} ($\pm$0.04) & \textcolor{blue}{0.90} ($\pm$0.06) & 0.38 ($\pm$0.42) & -\\
%     G-100MCW & 27.56 ($\pm$6.60) & 0.86 ($\pm$0.10) & \textcolor{blue}{0.93} ($\pm$0.04) & \textbf{0.90} ($\pm$0.05) & 0.37 ($\pm$0.42) & -\\
%     \midrule
%     G-B$_{FB,O}$ & 27.58 ($\pm$7.07) & 0.86 ($\pm$0.10) & \textcolor{blue}{0.93} ($\pm$0.04) & \textcolor{blue}{0.90} ($\pm$0.06) & 0.42 ($\pm$0.42) & 43.0\%\\
%     G-B$_{100MIU,O}$ & \textbf{27.25} ($\pm$6.15) & \textbf{0.87} ($\pm$0.09) & \textcolor{blue}{0.93} ($\pm$0.04) & \textcolor{blue}{0.90} ($\pm$0.06) & 0.38 ($\pm$0.42) & 37.4\%\\
%     \midrule
%     G-D$_{O}$ & 47.15 ($\pm$47.56) & 0.73 ($\pm$0.24) & 0.75 ($\pm$0.28) & 0.75 ($\pm$0.27) & \textbf{0.76} ($\pm$0.36) & \textbf{74.3\%}\\
%     \midrule
%     \midrule
%     D-baseline & 37.52 ($\pm$12.06) & 0.86 ($\pm$0.13) & 0.90 ($\pm$0.08) & 0.85 ($\pm$0.10) & 0.24 ($\pm$0.37) & -\\
%     D-100MCW & 37.80 ($\pm$10.89) & 0.85 ($\pm$0.14) & 0.89 ($\pm$0.10) & 0.85 ($\pm$0.10) & 0.18 ($\pm$0.33) & -\\
%     \midrule
%     D-B$_{FB,O}$ & 37.85 ($\pm$11.17) & 0.87 ($\pm$0.12) & 0.90 ($\pm$0.08) & 0.86 ($\pm$0.09) & 0.22 ($\pm$0.35) & 21.5\%\\
%     D-B$_{100MIU,O}$ & 37.91 ($\pm$12.27) & \textcolor{blue}{0.87} ($\pm$0.11) & 0.90 ($\pm$0.07) & 0.85 ($\pm$0.10) & 0.22 ($\pm$0.34) & 21.9\%\\
%     \midrule
%     D-D$_{O}$ & 41.17 ($\pm$20.72) & 0.87 ($\pm$0.12) & 0.89 ($\pm$0.13) & 0.83 ($\pm$0.16) & \textcolor{blue}{0.57} ($\pm$0.41) & \textcolor{blue}{56.7\%}\\
%     \bottomrule
%     \end{tabular}
%     \caption{\len{Neutral prompt - Old model} Results of age-controlled language generation. Perplexity is perplexity w.r.t. GPT-1. Dist-n is number of distinct n-grams normalized by text length, as a measure of diversity. $\boldsymbol{\bar{P}_Y}$ and $\boldsymbol{\bar{P}_O}$ are the respective average young and old probabilities assigned by the best BERT$_{FT}$. Acc. is the best BERT model's accuracy when classifying the row's samples.}
%     \label{tab:ctg_results_ws_neutral_prompt_old_model}
% \end{table*}

\begin{table*}[h]
    \centering
    \begin{tabular}{l | c c c c | c c c}
    \toprule
    \textbf{Model} & \textbf{ppl.} & \textbf{Dist-1} & \textbf{Dist-2} & \textbf{Dist-3} & $\boldsymbol{\bar{P}_Y}$ & $\boldsymbol{\bar{P}_O}$ & \textbf{Acc.}\\
    % -plus)}$\\
     & $\downarrow$ better & $\uparrow$ better & $\uparrow$ better & $\uparrow$ better & & & $\uparrow$ better\\
    \midrule
    \midrule
    G-baseline & 28.05 ($\pm$6.12) & 0.85 ($\pm$0.13) & 0.91 ($\pm$0.08) & 0.88 ($\pm$0.08) & 0.80 ($\pm$0.33) & 0.20 & -\\
    G-100MCW & 27.71 ($\pm$6.20) & 0.85 ($\pm$0.12) & 0.91 ($\pm$0.09) & 0.88 ($\pm$0.09) & 0.75 ($\pm$0.37) & 0.25 & -\\
    \midrule
    G-B$_{FB, Y}$ & 28.81 ($\pm$7.09) & 0.86 ($\pm$0.12) & 0.92 ($\pm$0.08) & 0.89 ($\pm$0.08) & 0.82 ($\pm$0.32) & 0.18 & 83.3\%\\
    G-B$_{FB, O}$ & 28.54 ($\pm$6.45) & 0.86 ($\pm$0.12) & 0.92 ($\pm$0.08) & 0.89 ($\pm$0.08) & 0.77 ($\pm$0.36) & 0.23 & 22.6\%\\
    G-B$_{100MIU, Y}$ & 28.49 ($\pm$6.49) & 0.86 ($\pm$0.12) & 0.91 ($\pm$0.08) & 0.88 ($\pm$0.08) & 0.83 ($\pm$0.32) & 0.17 & 83.0\%\\
    G-B$_{100MIU, O}$ & 28.18 ($\pm$5.70) & 0.87 ($\pm$0.11) & 0.92 ($\pm$0.08) & 0.89 ($\pm$0.09) & 0.79 ($\pm$0.34) & 0.21 & 21.5\%\\
    \midrule
    G-D$_{Y}$ & 39.32 ($\pm$37.49) & 0.84 ($\pm$0.21) & 0.61 ($\pm$0.40) & 0.57 ($\pm$0.40) & 0.70 ($\pm$0.40) & 0.30 & 70.7\%\\
    G-D$_{O}$ & 85.40 ($\pm$150.28) & 0.67 ($\pm$0.30) & 0.62 ($\pm$0.31) & 0.62 ($\pm$0.32) & 0.29 ($\pm$0.40) & 0.71 & 70.5\%\\
    \midrule
    \midrule
    D-baseline & 36.69 ($\pm$9.11) & 0.87 ($\pm$0.10) & 0.91 ($\pm$0.06) & 0.87 ($\pm$0.08) & 0.90 ($\pm$0.24) & 0.10 & -\\
    D-100MCW & 36.93 ($\pm$9.18) & 0.86 ($\pm$0.11) & 0.91 ($\pm$0.06) & 0.88 ($\pm$0.07) & 0.90 ($\pm$0.25) & 0.10 & -\\
    \midrule
    D-B$_{FB, Y}$ & 37.35 ($\pm$8.60) & 0.88 ($\pm$0.10) & 0.91 ($\pm$0.06) & 0.87 ($\pm$0.08) & 0.90 ($\pm$0.26) & 0.10 & 90.0\%\\
    D-B$_{FB, O}$ & 37.25 ($\pm$9.45) & 0.87 ($\pm$0.11) & 0.91 ($\pm$0.06) & 0.87 ($\pm$0.08) & 0.88 ($\pm$0.29) & 0.12 & 11.1\%\\
    D-B$_{100MIU, Y}$ & 37.87 ($\pm$8.32) & 0.88 ($\pm$0.10) & 0.91 ($\pm$0.07) & 0.87 ($\pm$0.09) & 0.91 ($\pm$0.24) & 0.09 & 92.6\%\\
    D-B$_{100MIU, O}$ & 37.04 ($\pm$8.78) & 0.88 ($\pm$0.10) & 0.91 ($\pm$0.05) & 0.88 ($\pm$0.07) & 0.85 ($\pm$0.32) & 0.15 & 15.2\%\\
    \midrule
    D-D$_{Y}$ & 39.22 ($\pm$14.96) & 0.89 ($\pm$0.12) & 0.86 ($\pm$0.19) & 0.79 ($\pm$0.23) & 0.89 ($\pm$0.25) & 0.11 & 91.1\%\\
    D-D$_{O}$ & 38.46 ($\pm$14.91) & 0.82 ($\pm$0.15) & 0.87 ($\pm$0.15) & 0.83 ($\pm$0.17) & 0.52 ($\pm$0.44) & 0.48 & 47.4\%\\
    \bottomrule
    \end{tabular}
    \caption{\len{Young prompt} Results of age-controlled language generation. Perplexity is perplexity w.r.t. GPT-1. Dist-n is number of distinct n-grams normalized by text length, as a measure of diversity. Acc. is the best BERT model's accuracy when classifying the row's samples.}
    \label{tab:ctg_results_ws_young_prompt}
\end{table*}

% \begin{table*}[h]
%     \centering
%     \begin{tabular}{l | c c c c | c c}
%     \toprule
%     \textbf{Model} & \textbf{ppl.} & \textbf{Dist-1} & \textbf{Dist-2} & \textbf{Dist-3} & $\boldsymbol{\bar{P}_Y}$ & \textbf{Acc.}\\
%     % -plus)}$\\
%      & $\downarrow$ better & $\uparrow$ better & $\uparrow$ better & $\uparrow$ better & $\uparrow$ better & $\uparrow$ better\\
%     \midrule
%     \midrule
%     G-baseline & \textcolor{blue}{28.05} ($\pm$6.12) & 0.85 ($\pm$0.13) & 0.91 ($\pm$0.08) & 0.88 ($\pm$0.08) & 0.80 ($\pm$0.33) & -\\
%     G-100MCW & \textbf{27.71} ($\pm$6.20) & 0.85 ($\pm$0.12) & 0.91 ($\pm$0.09) & 0.88 ($\pm$0.09) & 0.75 ($\pm$0.37) & -\\
%     \midrule
%     G-B$_{FB, Y}$ & 28.81 ($\pm$7.09) & 0.86 ($\pm$0.12) & \textbf{0.92} ($\pm$0.08) & \textbf{0.89} ($\pm$0.08) & 0.82 ($\pm$0.32) & 83.3\%\\
%     G-B$_{100MIU, Y}$ & 28.49 ($\pm$6.49) & 0.86 ($\pm$0.12) & 0.91 ($\pm$0.08) & 0.88 ($\pm$0.08) & 0.83 ($\pm$0.32) & 83.0\%\\
%     \midrule
%     G-D$_{Y}$ & 39.32 ($\pm$37.49) & 0.84 ($\pm$0.21) & 0.61 ($\pm$0.40) & 0.57 ($\pm$0.40) & 0.70 ($\pm$0.40) & 70.7\%\\
%     \midrule
%     \midrule
%     D-baseline & 36.69 ($\pm$9.11) & 0.87 ($\pm$0.10) & \textcolor{blue}{0.91} ($\pm$0.06) & 0.87 ($\pm$0.08) & \textcolor{blue}{0.90} ($\pm$0.24) & -\\
%     D-100MCW & 36.93 ($\pm$9.18) & 0.86 ($\pm$0.11) & \textcolor{blue}{0.91} ($\pm$0.06) & \textcolor{blue}{0.88} ($\pm$0.07) & 0.90 ($\pm$0.25) & -\\
%     \midrule
%     D-B$_{FB, Y}$ & 37.35 ($\pm$8.60) & \textcolor{blue}{0.88} ($\pm$0.10) & \textcolor{blue}{0.91} ($\pm$0.06) & 0.87 ($\pm$0.08) & 0.90 ($\pm$0.26) & \textcolor{blue}{90.0\%}\\
%     D-B$_{100MIU, Y}$ & 37.87 ($\pm$8.32) & \textcolor{blue}{0.88} ($\pm$0.10) & 0.91 ($\pm$0.07) & 0.87 ($\pm$0.09) & \textbf{0.91} ($\pm$0.24) & \textbf{92.6\%}\\
%     \midrule
%     D-D$_{Y}$ & 39.22 ($\pm$14.96) & \textbf{0.89} ($\pm$0.12) & 0.86 ($\pm$0.19) & 0.79 ($\pm$0.23) & 0.89 ($\pm$0.25) & 91.1\%\\
%     \bottomrule
%     \end{tabular}
%     \caption{\len{Young prompt - Young models} Results of age-controlled language generation. Perplexity is perplexity w.r.t. GPT-1. Dist-n is number of distinct n-grams normalized by text length, as a measure of diversity. Acc. is the best BERT model's accuracy when classifying the row's samples.}
%     \label{tab:ctg_results_ws_young_prompt_young_model}
% \end{table*}

% \begin{table*}[h]
%     \centering
%     \begin{tabular}{l | c c c c | c c}
%     \toprule
%     \textbf{Model} & \textbf{ppl.} & \textbf{Dist-1} & \textbf{Dist-2} & \textbf{Dist-3} & $\boldsymbol{\bar{P}_O}$ & \textbf{Acc.}\\
%     % -plus)}$\\
%      & $\downarrow$ better & $\uparrow$ better & $\uparrow$ better & $\uparrow$ better & $\uparrow$ better & $\uparrow$ better\\
%     \midrule
%     \midrule
%     G-baseline & \textcolor{blue}{28.05} ($\pm$6.12) & 0.85 ($\pm$0.13) & 0.91 ($\pm$0.08) & 0.88 ($\pm$0.08) & 0.20 ($\pm$0.33) & -\\
%     G-100MCW & \textbf{27.71} ($\pm$6.20) & 0.85 ($\pm$0.12) & 0.91 ($\pm$0.09) & 0.88 ($\pm$0.09) & 0.25 ($\pm$0.37) & -\\
%     \midrule
%     G-B$_{FB, O}$ & 28.54 ($\pm$6.45) & 0.86 ($\pm$0.12) & \textbf{0.92} ($\pm$0.08) & \textbf{0.89} ($\pm$0.08) & 0.23 ($\pm$0.36) & 22.6\%\\
%     G-B$_{100MIU, O}$ & 28.18 ($\pm$5.70) & 0.87 ($\pm$0.11) & \textbf{0.92} ($\pm$0.08) & \textcolor{blue}{0.89} ($\pm$0.09) & 0.21 ($\pm$0.34) & 21.5\%\\
%     \midrule
%     G-D$_{O}$ & 85.40 ($\pm$150.28) & 0.67 ($\pm$0.30) & 0.62 ($\pm$0.31) & 0.62 ($\pm$0.32) & \textbf{0.71} ($\pm$0.40) & \textbf{70.5\%}\\
%     \midrule
%     \midrule
%     D-baseline & 36.69 ($\pm$9.11) & \textcolor{blue}{0.87} ($\pm$0.10) & 0.91 ($\pm$0.06) & 0.87 ($\pm$0.08) & 0.10 ($\pm$0.24) & -\\
%     D-100MCW & 36.93 ($\pm$9.18) & 0.86 ($\pm$0.11) & 0.91 ($\pm$0.06) & 0.88 ($\pm$0.07) & 0.10 ($\pm$0.25) & -\\
%     \midrule
%     D-B$_{FB, O}$ & 37.25 ($\pm$9.45) & 0.87 ($\pm$0.11) & 0.91 ($\pm$0.06) & 0.87 ($\pm$0.08) & 0.12 ($\pm$0.29) & 11.1\%\\
%     D-B$_{100MIU, O}$ & 37.04 ($\pm$8.78) & \textbf{0.88} ($\pm$0.10) & \textcolor{blue}{0.91} ($\pm$0.05) & 0.88 ($\pm$0.07) & 0.15 ($\pm$0.32) & 15.2\%\\
%     \midrule
%     D-D$_{O}$ & 38.46 ($\pm$14.91) & 0.82 ($\pm$0.15) & 0.87 ($\pm$0.15) & 0.83 ($\pm$0.17) & \textcolor{blue}{0.48} ($\pm$0.44) & \textcolor{blue}{47.4\%}\\
%     \bottomrule
%     \end{tabular}
%     \caption{\len{Young prompt - Old models} Results of age-controlled language generation. Perplexity is perplexity w.r.t. GPT-1. Dist-n is number of distinct n-grams normalized by text length, as a measure of diversity. Acc. is the best BERT model's accuracy when classifying the row's samples.}
%     \label{tab:ctg_results_ws_young_prompt_old_model}
% \end{table*}

\begin{table*}[h]
    \centering
    \begin{tabular}{l | c c c c | c c c}
    \toprule
    \textbf{Model} & \textbf{ppl.} & \textbf{Dist-1} & \textbf{Dist-2} & \textbf{Dist-3} & $\boldsymbol{\bar{P}_Y}$ & $\boldsymbol{\bar{P}_O}$ & \textbf{Acc.}\\
    % -plus)}$\\
     & $\downarrow$ better & $\uparrow$ better & $\uparrow$ better & $\uparrow$ better & & & $\uparrow$ better\\
    \midrule
    \midrule
    G-baseline & 29.34 ($\pm$10.30) & 0.86 ($\pm$0.09) & 0.94 ($\pm$0.04) & 0.90 ($\pm$0.06) & 0.60 ($\pm$0.43) & 0.40 & -\\
    G-100MCW & 29.14 ($\pm$10.11) & 0.86 ($\pm$0.10) & 0.93 ($\pm$0.04) & 0.90 ($\pm$0.06) & 0.60 ($\pm$0.44) & 0.40 & -\\
    \midrule
    G-B$_{Y, FB}$ & 29.61 ($\pm$10.28) & 0.86 ($\pm$0.10) & 0.93 ($\pm$0.04) & 0.91 ($\pm$0.06) & 0.62 ($\pm$0.43) & 0.38 & 61.1\%\\
    G-B$_{O, FB}$ & 28.81 ($\pm$10.10) & 0.86 ($\pm$0.10) & 0.93 ($\pm$0.05) & 0.90 ($\pm$0.06) & 0.59 ($\pm$0.43) & 0.41 & 41.1\%\\
    G-B$_{Y, 100MIU}$ & 29.51 ($\pm$0.09) & 0.87 ($\pm$0.09) & 0.93 ($\pm$0.05) & 0.90 ($\pm$0.06) & 0.68 ($\pm$0.42) & 0.32 & 68.5\%\\
    G-B$_{O, 100MIU}$ & 29.05 ($\pm$9.80) & 0.86 ($\pm$0.09) & 0.93 ($\pm$0.04) & 0.90 ($\pm$0.06) & 0.60 ($\pm$0.43) & 0.40 & 39.6\%\\
    \midrule
    G-D$_{Y}$ & 32.34 ($\pm$19.88) & 0.77 ($\pm$0.20) & 0.84 ($\pm$0.19) & 0.80 ($\pm$0.23) & 0.65 ($\pm$0.43) & 0.35 & 65.4\%\\
    G-D$_{O}$ & 95.21 ($\pm$174.42) & 0.65 ($\pm$0.27) & 0.78 ($\pm$0.18) & 0.78 ($\pm$0.18) & 0.10 ($\pm$0.25) & 0.90 & 90.3\%\\
    \midrule
    \midrule
    D-baseline & 38.18 ($\pm$12.03) & 0.86 ($\pm$0.12) & 0.90 ($\pm$0.08) & 0.86 ($\pm$0.09) & 0.72 ($\pm$0.38) & 0.28 & -\\
    D-100MCW & 37.73 ($\pm$11.88) & 0.85 ($\pm$0.13) & 0.90 ($\pm$0.08) & 0.86 ($\pm$0.09) & 0.73 ($\pm$0.39) & 0.27 & -\\
    \midrule
    D-B$_{Y, FB}$ & 38.24 ($\pm$11.53) & 0.86 ($\pm$0.12) & 0.90 ($\pm$0.08) & 0.86 ($\pm$0.10) & 0.81 ($\pm$0.34) & 0.19 & 82.6\%\\
    D-B$_{O, FB}$ & 37.8 ($\pm$11.74) & 0.86 ($\pm$0.12) & 0.90 ($\pm$0.07) & 0.87 ($\pm$0.08) & 0.72 ($\pm$0.39) & 0.28 & 29.3\%\\
    D-B$_{Y, 100MIU}$ & 38.66 ($\pm$11.57) & 0.85 ($\pm$0.12) & 0.90 ($\pm$0.07) & 0.86 ($\pm$0.09) & 0.81 ($\pm$0.33) & 0.19 & 80.7\%\\
    D-B$_{O, 100MIU}$ & 36.93 ($\pm$11.68) & 0.87 ($\pm$0.12) & 0.90 ($\pm$0.09) & 0.86 ($\pm$0.09) & 0.69 ($\pm$0.41) & 0.31 & 29.6\%\\
    \midrule
    D-D$_{Y}$ & 42.93 ($\pm$20.18) & 0.90 ($\pm$0.14) & 0.79 ($\pm$0.22) & 0.68 ($\pm$0.28) & 0.84 ($\pm$0.30) & 0.16 & 85.2\%\\
    D-D$_{O}$ & 45.58 ($\pm$38.59) & 0.86 ($\pm$0.14) & 0.86 ($\pm$0.13) & 0.79 ($\pm$0.20) & 0.40 ($\pm$0.42) & 0.60 & 59.7\%\\
    \bottomrule
    \end{tabular}
    \caption{\len{Old prompt} Results of age-controlled language generation. Perplexity is perplexity w.r.t. GPT-1. Dist-n is number of distinct n-grams normalized by text length, as a measure of diversity. Acc. is the best BERT model's accuracy when classifying the row's samples.}
    \label{tab:ctg_results_ws_old_prompt}
\end{table*}

% \begin{table*}[h]
%     \centering
%     \begin{tabular}{l | c c c c | c c}
%     \toprule
%     \textbf{Model} & \textbf{ppl.} & \textbf{Dist-1} & \textbf{Dist-2} & \textbf{Dist-3} & $\boldsymbol{\bar{P}_Y}$ & \textbf{Acc.}\\
%     % -plus)}$\\
%      & $\downarrow$ better & $\uparrow$ better & $\uparrow$ better & $\uparrow$ better & $\uparrow$ better & $\uparrow$ better\\
%     \midrule
%     \midrule
%     G-baseline & \textcolor{blue}{29.34} ($\pm$10.30) & 0.86 ($\pm$0.09) & \textbf{0.94} ($\pm$0.04) & \textcolor{blue}{0.90} ($\pm$0.06) & 0.60 ($\pm$0.43) & -\\
%     G-100MCW & \textbf{29.14} ($\pm$10.11) & 0.86 ($\pm$0.10) & \textcolor{blue}{0.93} ($\pm$0.04) & \textcolor{blue}{0.90} ($\pm$0.06) & 0.60 ($\pm$0.44) & -\\
%     \midrule
%     G-B$_{Y, FB}$ & 29.61 ($\pm$10.28) & 0.86 ($\pm$0.10) & 0.93 ($\pm$0.04) & \textbf{0.91} ($\pm$0.06) & 0.62 ($\pm$0.43) & 61.1\%\\
%     G-B$_{Y, 100MIU}$ & 29.51 ($\pm$0.09) & \textcolor{blue}{0.87} ($\pm$0.09) & 0.93 ($\pm$0.05) & \textcolor{blue}{0.90} ($\pm$0.06) & 0.68 ($\pm$0.42) & 68.5\%\\
%     \midrule
%     G-D$_{Y}$ & 32.34 ($\pm$19.88) & 0.77 ($\pm$0.20) & 0.84 ($\pm$0.19) & 0.80 ($\pm$0.23) & 0.65 ($\pm$0.43) & 65.4\%\\
%     \midrule
%     \midrule
%     D-baseline & 38.18 ($\pm$12.03) & 0.86 ($\pm$0.12) & 0.90 ($\pm$0.08) & 0.86 ($\pm$0.09) & 0.72 ($\pm$0.38) & -\\
%     D-100MCW & 37.73 ($\pm$11.88) & 0.85 ($\pm$0.13) & 0.90 ($\pm$0.08) & 0.86 ($\pm$0.09) & 0.73 ($\pm$0.39) & -\\
%     \midrule
%     D-B$_{Y, FB}$ & 38.24 ($\pm$11.53) & 0.86 ($\pm$0.12) & 0.90 ($\pm$0.08) & 0.86 ($\pm$0.10) & 0.81 ($\pm$0.34) & \textcolor{blue}{82.6\%}\\
%     D-B$_{Y, 100MIU}$ & 38.66 ($\pm$11.57) & 0.85 ($\pm$0.12) & 0.90 ($\pm$0.07) & 0.86 ($\pm$0.09) & \textcolor{blue}{0.81} ($\pm$0.33) & 80.7\%\\
%     \midrule
%     D-D$_{Y}$ & 42.93 ($\pm$20.18) & \textbf{0.90} ($\pm$0.14) & 0.79 ($\pm$0.22) & 0.68 ($\pm$0.28) & \textbf{0.84} ($\pm$0.30) & \textbf{85.2\%}\\
%     \bottomrule
%     \end{tabular}
%     \caption{\len{Old prompt - Young model} Results of age-controlled language generation. Perplexity is perplexity w.r.t. GPT-1. Dist-n is number of distinct n-grams normalized by text length, as a measure of diversity. Acc. is the best BERT model's accuracy when classifying the row's samples.}
%     \label{tab:ctg_results_ws_old_prompt_young_model}
% \end{table*}

% \begin{table*}[h]
%     \centering
%     \begin{tabular}{l | c c c c | c c}
%     \toprule
%     \textbf{Model} & \textbf{ppl.} & \textbf{Dist-1} & \textbf{Dist-2} & \textbf{Dist-3} & $\boldsymbol{\bar{P}_O}$ & \textbf{Acc.}\\
%     % -plus)}$\\
%      & $\downarrow$ better & $\uparrow$ better & $\uparrow$ better & $\uparrow$ better & $\uparrow$ better & $\uparrow$ better\\
%     \midrule
%     \midrule
%     G-baseline & 29.34 ($\pm$10.30) & \textcolor{blue}{0.86} ($\pm$0.09) & \textbf{0.94} ($\pm$0.04) & \textbf{0.90} ($\pm$0.06) & 0.40 ($\pm$0.43) & -\\
%     G-100MCW & 29.14 ($\pm$10.11) & 0.86 ($\pm$0.10) & \textcolor{blue}{0.93} ($\pm$0.04) & \textbf{0.90} ($\pm$0.06) & 0.40 ($\pm$0.44) & -\\
%     \midrule
%     G-B$_{O, FB}$ & \textbf{28.81} ($\pm$10.10) & 0.86 ($\pm$0.10) & 0.93 ($\pm$0.05) & \textbf{0.90} ($\pm$0.06) & 0.41 ($\pm$0.43) & 41.1\%\\
%     G-B$_{O, 100MIU}$ & 29.05 ($\pm$9.80) & \textcolor{blue}{0.86} ($\pm$0.09) & \textcolor{blue}{0.93} ($\pm$0.04) & \textbf{0.90} ($\pm$0.06) & 0.40 ($\pm$0.43) & 39.6\%\\
%     \midrule
%     G-D$_{O}$ & 95.21 ($\pm$174.42) & 0.65 ($\pm$0.27) & 0.78 ($\pm$0.18) & 0.78 ($\pm$0.18) & \textbf{0.90} ($\pm$0.25) & \textbf{90.3\%}\\
%     \midrule
%     \midrule
%     D-baseline & 38.18 ($\pm$12.03) & 0.86 ($\pm$0.12) & 0.90 ($\pm$0.08) & 0.86 ($\pm$0.09) & 0.28 ($\pm$0.38) & -\\
%     D-100MCW & 37.73 ($\pm$11.88) & 0.85 ($\pm$0.13) & 0.90 ($\pm$0.08) & 0.86 ($\pm$0.09) & 0.27 ($\pm$0.39) & -\\
%     \midrule
%     D-B$_{O, FB}$ & 37.80 ($\pm$11.74) & 0.86 ($\pm$0.12) & 0.90 ($\pm$0.07) & \textcolor{blue}{0.87} ($\pm$0.08) & 0.28 ($\pm$0.39) & 29.3\%\\
%     D-B$_{O, 100MIU}$ & 36.93 ($\pm$11.68) & \textbf{0.87} ($\pm$0.12) & 0.90 ($\pm$0.09) & 0.86 ($\pm$0.09) & 0.31 ($\pm$0.41) & 29.6\%\\
%     \midrule
%     D-D$_{O}$ & 45.58 ($\pm$38.59) & 0.86 ($\pm$0.14) & 0.86 ($\pm$0.13) & 0.79 ($\pm$0.20) & \textcolor{blue}{0.60} ($\pm$0.42) & \textcolor{blue}{59.7\%}\\
%     \bottomrule
%     \end{tabular}
%     \caption{\len{Old prompt - Old models} Results of age-controlled language generation. Perplexity is perplexity w.r.t. GPT-1. Dist-n is number of distinct n-grams normalized by text length, as a measure of diversity. Acc. is the best BERT model's accuracy when classifying the row's samples.}
%     \label{tab:ctg_results_ws_old_prompt_old_model}
% \end{table*}

\begin{table*}[h]
    \centering
    \begin{tabular}{l c c c c c}
    \toprule
    \textbf{Model} & \textbf{ppl.} & \textbf{Dist-1} & \textbf{Dist-2} & \textbf{Dist-3} & \textbf{Acc.}\\
    % -plus)}$\\
     & $\downarrow$ better & $\uparrow$ better & $\uparrow$ better & $\uparrow$ better & $\uparrow$ better\\
    \midrule
    \midrule
    GPT-2 baseline & \dots ($\pm$\dots) & \dots ($\pm$\dots) & \dots ($\pm$\dots) & \dots ($\pm$\dots) & \dots\%\\
    GPT-2 100MCW baseline & \dots ($\pm$\dots) & \dots ($\pm$\dots) & \dots ($\pm$\dots) & \dots ($\pm$\dots) & \dots\%\\
    \midrule
    GPT-2 B$_{Y, FB}$ & \dots ($\pm$\dots) & \dots ($\pm$\dots) & \dots ($\pm$\dots) & \dots ($\pm$\dots) & \dots\%\\
    GPT-2 B$_{O, FB}$ & \dots ($\pm$\dots) & \dots ($\pm$\dots) & \dots ($\pm$\dots) & \dots ($\pm$\dots) & \dots\%\\
    GPT-2 B$_{Y, 100MIU}$ & \dots ($\pm$\dots) & \dots ($\pm$\dots) & \dots ($\pm$\dots) & \dots ($\pm$\dots) & \dots\%\\
    GPT-2 B$_{O, 100MIU}$ & \dots ($\pm$\dots) & \dots ($\pm$\dots) & \dots ($\pm$\dots) & \dots ($\pm$\dots) & \dots\%\\
    \midrule
    GPT-2 D$_{Y}$ & \dots ($\pm$\dots) & \dots ($\pm$\dots) & \dots ($\pm$\dots) & \dots ($\pm$\dots) & \dots\%\\
    GPT-2 D$_{O}$ & \dots ($\pm$\dots) & \dots ($\pm$\dots) & \dots ($\pm$\dots) & \dots ($\pm$\dots) & \dots\%\\
    \midrule
    \midrule
    DGPT baseline & \dots ($\pm$\dots) & \dots ($\pm$\dots) & \dots ($\pm$\dots) & \dots ($\pm$\dots) & \dots\%\\
    DGPT-100MCW & \dots ($\pm$\dots) & \dots ($\pm$\dots) & \dots ($\pm$\dots) & \dots ($\pm$\dots) & \dots\%\\
    \midrule
    DGPT B$_{Y, FB}$ & \dots ($\pm$\dots) & \dots ($\pm$\dots) & \dots ($\pm$\dots) & \dots ($\pm$\dots) & \dots\%\\
    DGPT B$_{O, FB}$ & \dots ($\pm$\dots) & \dots ($\pm$\dots) & \dots ($\pm$\dots) & \dots ($\pm$\dots) & \dots\%\\
    DGPT B$_{Y, 100MIU}$ & \dots ($\pm$\dots) & \dots ($\pm$\dots) & \dots ($\pm$\dots) & \dots ($\pm$\dots) & \dots\%\\
    DGPT B$_{O, 100MIU}$ & \dots ($\pm$\dots) & \dots ($\pm$\dots) & \dots ($\pm$\dots) & \dots ($\pm$\dots) & \dots\%\\
    \midrule
    DGPT$_{Y}$ & \dots ($\pm$\dots) & \dots ($\pm$\dots) & \dots ($\pm$\dots) & \dots ($\pm$\dots) & \dots\%\\
    DGPT$_{O}$ & \dots ($\pm$\dots) & \dots ($\pm$\dots) & \dots ($\pm$\dots) & \dots ($\pm$\dots) & \dots\%\\
    \bottomrule
    \end{tabular}
    \caption{\len{Unprompted} Results of age-controlled language generation. Perplexity is perplexity w.r.t. GPT-1. Dist-n is number of distinct n-grams normalized by text length, as a measure of diversity. Acc. is the best BERT model's accuracy when classifying the row's samples.}
    \label{tab:ctg_results_ws_unprompted}
\end{table*}

\begin{figure}[H]
    \centering
    \includegraphics[width=\textwidth]{figures/exp2/lineplot_ppl_acc_best_gpt2_disc_bow_ci_90_errstyle_bars.png}
    \caption{\len{Unprompted} Fluency (perplexity, x-axis) versus control (BERT accuracy, y-axis). Bins chosen to be roughly same order of magnitude. The bars represent 90\% confidence intervals. This plot is best viewed in color.}
    \label{fig:ctg_lineplot_fluency_vs_control}
\end{figure}

\begin{figure}[H]
     \centering
     \begin{subfigure}[b]{0.32\textwidth}
        \centering
        \includegraphics[width=1\columnwidth]{figures/exp2/lineplot_len_ppl_best_gpt2_disc_bow_ci_90_errstyle_bars.png}
        \caption{}
        \label{fig:ctg_lineplot_len_vs_ppl}
     \end{subfigure}
    %  \hfill
    \quad
     \begin{subfigure}[b]{0.32\textwidth}
        \centering
        \includegraphics[width=1\columnwidth]{figures/exp2/lineplot_len_acc_best_gpt2_disc_bow_ci_90_errstyle_bars_nolegend.png}
        \caption{}
        \label{fig:ctg_lineplot_len_vs_acc}
     \end{subfigure}
     \begin{subfigure}[b]{0.32\textwidth}
        \centering
        \includegraphics[width=\textwidth]{figures/exp2/lineplot_len_dist1_best_gpt2_disc_bow_ci_90_errstyle_bars_nolegend.png}
        \caption{}
        \label{fig:ctg_lineplot_len_vs_dist1}
     \end{subfigure}
     
     \medskip
     
     \begin{subfigure}[b]{0.32\textwidth}
        \centering
        \includegraphics[width=\textwidth]{figures/exp2/lineplot_len_dist2_best_gpt2_disc_bow_ci_90_errstyle_bars_nolegend.png}
        \caption{}
        \label{fig:ctg_lineplot_len_vs_dist2}
     \end{subfigure}
     \quad
     \begin{subfigure}[b]{0.32\textwidth}
        \centering
        \includegraphics[width=\textwidth]{figures/exp2/lineplot_len_dist3_best_gpt2_disc_bow_ci_90_errstyle_bars_nolegend.png}
        \caption{}
        \label{fig:ctg_lineplot_len_vs_dist3}
     \end{subfigure}
     
     % Main figure caption and label
        \caption{\len{Unprompted} Various averaged automated evaluation metrics of generated sentences, plotted against increasing sequence length (x-axis), with 90\% confidence intervals. The plots are best viewed in color.}
        \label{fig:ctg_lineplots_len_vs_metrics}
\end{figure}

\section{CTG Results for Unprompted Setup \len{Redundant?}}

Table~\ref{tab:ctg_results_ws} reports the automated evaluation results of our controllable text generation models. It can be seen that uncontrolled GPT-2 baseline has a slight bias towards generating "young-sounding" language (57.5\% accuracy). Furthermore, it appears that perturbing GPT-2's output distribution with the 100 most common words across all ages results in a slight de-biasing of the generated text (54.1\% accuracy). Achieving detectable control seems possible, because all GPT-2-based models surpass both baselines in terms of accuracy, with the exception of both BoW-setups using the 100 most informative unigrams.

Frequency-based BoW-models outperform those using the most informative unigrams, as illustrated by their higher average accuracy (66.75\% versus. 53.1\%), and lower average perplexity (27.48 versus 27.90). 
% \len{Offer an explanation.} 
Discriminator-based models achieve noticeably better accuracies, with an average improvement of 8.45\% over the best performing BoW-based models. However, discriminator-based models do show more signs of disfluency and repetitiveness compared to the BoW-models, as depicted by the worse perplexities and Dist-$n |_{n = 1,2,3}$ scores.

The accuracies of our uncontrolled DialoGPT baseline (78.1\%) and the 100MCW baseline (80.7\%), suggest that DialoGPT is heavily biased towards producing young-sounding language. This can be attributable to DialoGPT having been fine-tuned on Reddit threads, as the majority of Reddit users are between the ages 20 and 29~\footnote{\url{https://www.statista.com/statistics/1125159/reddit-us-app-users-age/}}
% \len{Find a reference for this statement} 
\citep{zhang2019dialogpt}. DialoGPT's strong propensity for generating younger sounding language makes it a less desirable choice for our human evaluation experiments, because it requires non-standard parameter settings to produce detectably older sounding text.

Overall, the results show that, for most models, a plug-and-play approach to controlling generated dialogue responses to possess detectable age-specific linguistic features is achievable. The most promising models being either discriminator-based, or frequency-based bag-of-words models. Discriminator-based models achieve more detectable levels of control than their BoW-based counterparts, at the cost of perplexity and repetitiveness. This could be attributable to the more complex activation-space updates that are used by discriminator-models. Furthermore, GPT-2's preference to generate young-sounding language is severely less pronounced than that of DialoGPT, making it easier to control, given equal parameter settings.

% \textbf{Observations:}
% GPT-2
% \begin{itemize}
%     \item All GPT2-based models beat the GPT-2 baseline, except for the BoW-100MIU.
%     \item GPT-2 has a slight bias towards being classified as young. Has a preference to sound young.
%     \item BoW-based control with the 100 most common words appears to de-bias uncontrolled GPT-2
%     \item The frequency-based BoW models are the best ones. They achieve on average better control than 100MIU w.r.t. the baseline, and have on average a lower perplexity than 100MIU. \len{Compute and explicitly mention these differences}.
%     \item Discriminator-based models achieve (1) much higher accuracy, (2) noticeably worse perplexity, and (3) on average more repetitive unigrams, bigrams, and trigrams.
%     \item Controlling for old language seems harder than for young, as its accuracy is consistently lower.
% \end{itemize}

% \textbf{DialoGPT:}
% \begin{itemize}
%     \item DialoGPT is heavily biased towards younger language. This is most likely due to it being finetuned on Reddit threads. Vast majority of U.S. Reddit users are between the ages of 18 and 19 \url{https://www.statista.com/statistics/261766/share-of-us-internet-users-who-use-reddit-by-age-group/}
%     \item 100 most common words across all ages seems to result in more pronounced young-bias. Could also be due to prompt.
% \end{itemize}

\begin{table*}[h!]
    \centering
    \begin{tabular}{l c c c c c}
    \toprule
    \textbf{Model} & \textbf{ppl.} & \textbf{Dist-1} & \textbf{Dist-2} & \textbf{Dist-3} & \textbf{Acc.}\\
    % -plus)}$\\
     & $\downarrow$ better & $\uparrow$ better & $\uparrow$ better & $\uparrow$ better & $\uparrow$ better\\
    \midrule
    \midrule
    GPT-2 baseline & 29.60 ($\pm$17.57) & \textbf{0.89} ($\pm$0.09) & \textbf{0.93} ($\pm$0.05) & 0.88 ($\pm$0.10) & 57.5\%\\
    GPT-2 100MCW baseline & 27.80 ($\pm$16.44) & 0.83 ($\pm$0.12) & 0.91 ($\pm$0.06) & \textcolor{blue}{0.88} ($\pm$0.09) & 54.1\%\\
    \midrule
    % B$_{Y, FB, 80}$ & 29.13 ($\pm$15.17) & 0.86 ($\pm$0.10) & 0.92 ($\pm$0.06) & 0.88 ($\pm$0.11) & 70\%\\
    % B$_{O, FB, 80}$ & \textbf{26.42} ($\pm$8.87) & 0.86 ($\pm$0.10) & \textcolor{blue}{0.92} ($\pm$0.05) & \textbf{0.89} ($\pm$0.10) & 62.2\%\\
    B$_{Y, FB}$ & 28.16 ($\pm$14.52) & 0.87 ($\pm$0.10) & 0.92 ($\pm$0.06) & 0.88 ($\pm$0.10) & 69.3\%\\
    B$_{O, FB}$ & 26.79 ($\pm$8.89) & \textcolor{blue}{0.88} ($\pm$0.09) & \textcolor{blue}{0.92} ($\pm$0.05) & 0.88 ($\pm$0.10) & 64.2\%\\
    B$_{Y, 100MIU}$ & 29.16 ($\pm$14.91) & \textbf{0.89} ($\pm$0.09) & 0.92 ($\pm$0.06) & 0.88 ($\pm$0.11) & 52.5\%\\
    B$_{O, 100MIU}$ & \textcolor{blue}{26.63} ($\pm$8.36) & 0.87 ($\pm$0.10) & 0.92 ($\pm$0.07) & 0.88 ($\pm$0.10) & 53.7\%\\
    \midrule
    D$_{Y, GPT2}$ & 31.95 ($\pm$14.29) & 0.82 ($\pm$0.17) & 0.87 ($\pm$0.14) & 0.83 ($\pm$0.16) & \textbf{77.7\%}\\
    D$_{O, GPT2}$ & 33.63 ($\pm$24.40) & 0.80 ($\pm$0.18) & 0.87 ($\pm$0.11) & 0.81 ($\pm$0.21) & \textcolor{blue}{72.7}\%\\
    \midrule
    DGPT baseline & 35.20 ($\pm$10.01) & 0.87 ($\pm$0.11) & 0.90 ($\pm$0.07) & 0.87 ($\pm$0.08) & 78.1\%\\
    DGPT-100MCW & 35.64 ($\pm$9.72) & 0.86 ($\pm$0.10) & 0.90 ($\pm$0.06) & 0.87 ($\pm$0.08) & 80.7\%\\
    D*$_{Y, DGPT}$ & 41.54 ($\pm$10.87) & 0.91 ($\pm$0.11) & 0.91 ($\pm$0.06) & 0.86 ($\pm$0.09) & \textbf{84.1}\%\\
    D*$_{O, DGPT}$ & 38.16 ($\pm$10.77) & 0.87 ($\pm$0.11) & 0.91 ($\pm$0.06) & 0.87 ($\pm$0.08) & 55.6\%\\
    \bottomrule
    \end{tabular}
    \caption{\len{Unprompted} Results of age-controlled language generation. Perplexity is perplexity w.r.t. GPT-1. Dist-n is number of distinct n-grams normalized by text length, as a measure of diversity. Acc. is the best BERT model's accuracy when classifying the row's samples.}
    \label{tab:ctg_results_ws}
\end{table*}



\end{document}
