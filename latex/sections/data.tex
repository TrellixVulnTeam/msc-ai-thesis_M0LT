Two datasets are considered during the experiments of this thesis: (1) the spoken component of the British National Corpus (BNC or BNC2014) \citep{love-spoken-bnc-2014}, and (2) the Blog Authorship Corpus (BAC, or sometimes referred to as `blog corpus') \citep{schler2006effects}. The first corpus is a collection of transcriptions of recorded everyday conversations in British English, gathered between 2012 and 2016. The second is a dataset of blogs posted on blogger.com, gathered in or before August 2004.

The texts (i.e., blogs or dialogue transcriptions) in both corpora are labeled by age, among other factors. This makes them suitable candidates for training and testing age classifiers. In both cases, the texts are written in a somewhat informal manner, making them more representative of common language. Only the BNC can be used for the controllable dialogue generation phase of the experiments, as the BAC is not a conversational dataset. What follows is an overview of both datasets' motivation for use, drawbacks, metadata, descriptive statistics, and pre-processing steps before analyses. 

\subsection{The British National Corpus (BNC)}
The conversations of the spoken component of the BNC were typically recorded at home and took place between friends and family of participants. Participants recorded their conversations using their smartphones, allowing for spontaneous recording. These two properties of the BNC's sampling procedure make the recorded dialogues representative of contemporary everyday British English speech. 

A total of 1251 conversations among 672 speakers constitutes the full corpus. 


\subsection{The Blog Authorship Corpus (BAC)}

\begin{table}[H]
    \centering
    \begin{tabular}{l | c  c}
        \hline
        \textbf{Corpus} & BNC & BAC \\
        \hline
        \textbf{Dialogue?} & Yes & No\\
        \textbf{No. words} & & \\
        \hline
    \end{tabular}
    \caption{Summary of corpora properties.}
    \label{tab:summary_corpora}
\end{table}