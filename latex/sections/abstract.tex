Personalized interaction between users and dialogue systems is crucial to obtain systems that are trusted by users and perceived as natural.
Because a person's use of language is known to change with age, personalized dialogue systems should be able to identify a user's age profile, and then generate age-adaptive responses, based on the identified signal.
% Developing a module that first automatically identifies a user's age profile, and another module that then adapts the generated text output to the identified signal, are essential steps to obtaining .
% Being able to automatically identify a user's age profile (based on text input only), and adapt the style of the  to this detected profile is an essential step to obtaining personalized dialogue systems.
% , while preserving privacy?.
Recently, the development of Plug-and-Play language models (PPLM) has lead to computationally relatively inexpensive methods to control the writing style of text outputs generated by large pre-trained Transformer-based language models.
However, their application to enforce abstract writing styles (like age-related language) on dialogue responses is virtually unexplored.
This thesis therefore aims to study the detection and subsequent generation of age-related linguistic features in dialogue settings.
First, I investigate to what extent various purely text-based NLP models can detect age-related linguistic patterns in dialogue.
It is found that a fine-tuned BERT-model is able to distinguish between transcribed dialogue utterances of different age groups with reasonable accuracy, while much simpler models based on $n$-grams are able to do so with comparable performance, which suggests that, in dialogue, "local" features can be indicative of the language of speakers from different age groups.
The presumed locality of age-indicative features identified during age-detection motivates the use of both simple bag-of-words (BoW) attribute models, in addition to more sophisticated neural discriminator attribute models for controlled dialogue generation using PPLM.
The results show that age-adaptive dialogue generation with PPLM is possible to the extent that my text-based classifiers can reliably detect age-related linguistic patterns.
% that age-related linguistic features are detectable by my classifiers.
Furthermore, it is observed that discriminator-based PPLM-setups achieve higher levels of detectable control than BoW-based setups, but generate significantly more perplexing and repetitive responses.
I also carry out extensive quantitative and qualitative analyses, revealing the effects of prompt-induced biases, and the presence of previously studied age-related linguistic patterns in dialogue responses generated by PPLM-setups.
I conclude that both structural and local features should be taken into account when adapting the output of dialogue generation models to certain age groups.
Overall, I believe this work is a promising step towards the development of personalized, age-adaptive, dialogue systems.

% \textbf{draft sentences}

% \begin{itemize}
%     \item Personalized interaction between users dialogue systems is crucial to obtain systems that are trusted by users and perceived as natural and human-like.
%     \item Being able to automatically identify a user's age profile (based on text input only), and adapt the style of output language to this detected profile is an essential step to obtaining personalized dialogue systems, while preserving privacy?.
%     \item controlled dialogue generation?
%     \item Despite the development of Plug-and-Play language models (PPLM) leading to computationally inexpensive methods to control the writing style of output generated by large pre-trained Transformer-based language models, their application to enforce abstract writing styles on dialogue responses is virtually unexplored.
%     \item I therefore first investigate to what extent purely text-based NLP models can detect age-related linguistic patterns in dialogue (and discourse?).
%     \item I then test the degree to which I can control the writing style of generated dialogue responses to possess age-indicative features, identified in the age-detection experiment.
%     \item I do so using Plug-and-Play language models (PPLM)
%     \item I find that a fine-tuned BERT-model is able to distinguish between transcribed dialogue utterances of different age groups with reasonable accuracy, while much simpler models based on $n$-grams are able to do so with comparable performance.
%     \item this suggests that in dialogue, "local" features can be indicative of the language of speakers from different age groups.
%     \item The presumed locality of age-indicative features identified during age-detection motivates the use of both simple BoW attribute models, as well as more sophisticated neural discriminator attribute models for controlled dialogue generation using PPLM.
%     \item My results confirm that age-adaptive dialogue generation with PPLM is possible to the extent that is identifiable by my classifiers.
%     \item I observe that discriminator-based PPLM-setups achieve higher levels of detectable control than BoW-based setups, but generate significantly more perplexing and repetitive responses.
%     \item I conclude that both structural and local features should be taken into account when adapting the output of dialogue generation models to certain age groups
%     \item I believe this work is a promising step towards the development of personalized, age-adaptive, dialogue systems.
% \end{itemize}

% \begin{itemize}
%     \item This thesis presents a Plug-and-Play approach to age-adaptive dialogue generation.
%     \item I also detect age-related linguistic patterns in dialogue using purely text-based NLP models
%     \item plug-and-play language models provide ... with capability to control the output of large Transformer-based language models to possess a desired style
%     \item previous work limited to topic or sentiment, not abstract writing styles
%     \item the research objectives of this thesis consist of two subsequent parts: to detect blablabla, and to generate stuff
%     \item I show that ...
    
% \end{itemize}


% \textbf{Abstract must consist of}
% \begin{itemize}
%     \item interesting/captivating introduction to the field/problem. also raise awareness about importance
%     \item research problem and objectives
%     \item methods 
%     \item key results and contributions
%     \item most important conclusions
% \end{itemize}
