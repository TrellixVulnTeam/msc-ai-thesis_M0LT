\section{Data}

% Two datasets are considered during the experiments of this thesis: (1) the spoken component of the British National Corpus (BNC or BNC2014) \citep{love-spoken-bnc-2014}, and (2) the Blog Authorship Corpus (BAC, or sometimes referred to as `blog corpus') \citep{schler2006effects}. The first corpus is a collection of transcriptions of everyday conversations in British English, gathered between 2012 and 2016. The second is a dataset of blogs posted on \url{https://www.blogger.com}, gathered in or before August 2004.

% The texts (i.e., blogs or dialogue transcriptions) in both corpora are labeled by age, among other labels. This makes them suitable candidates for training and testing age classifiers. In both cases, the texts are written in a somewhat informal manner, making them more representative of everyday speech. Only the BNC is used for the controllable dialogue generation phase of the experiments, as the BAC is not a conversational dataset. What follows is an overview of both datasets' motivation for use, drawbacks, metadata, descriptive statistics, and pre-processing steps before analyses. 

% \subsection{The British National Corpus (BNC)}
% The conversations of the spoken component of the BNC were typically recorded at home and took place between friends and family of participants. Participants recorded their conversations using their smartphones, allowing for spontaneous recording. These two properties of the BNC's sampling procedure make the recorded dialogues representative of contemporary everyday British English speech. 

% A total of 1251 conversations of 672 speakers constitutes the full corpus, accounting for 10.4 million words. During the recruitment process of the BNC, prospective participants were asked to disclose personal information like age, gender, highest completed level of education, employment, and perceived accent\footnote{For a detailed description of the sampling decisions made during data collection, the reader is referred to the BNC2014 user manual: \url{http://corpora.lancs.ac.uk/bnc2014/doc/BNC2014manual.pdf}}. Other metadata accompany the conversations themselves. Namely, number of speakers taking part in the conversation, ages of those speakers, conversation length, and topic. 

% Nine age-brackets are present in the corpus: 11-18, 19-29, 30-39, 40-49, 50-59, 60-69, 70-79, 80-89, and 90-99. To make the objective of learning to classify and generate conversational responses more straightforward, I only consider dialogues between two participants. Namely, it will be more difficult to distinguish which utterances constitute the relevant responses to something said in a conversation with more than two participants. This filter step results in a remaining dataset of 622 dialogues, almost 460K utterances (i.e., a single turn in a dialogue), and nearly 5M tokens. On average, a dialogue has 736 turns, meaning that the conversations are relatively long. Furthermore, due to minors not being allowed to participate in this project's interactive experiments later on, the age-bracket 11-18 is dropped. This leaves us with a dataset of 522 dialogues, 418K utterances, and 4.4M tokens. Finally, the age-brackets get regrouped into brackets 19-29 and 50 plus, and conversations with speakers aged 30-49 are removed. This gap between to two age-groups is made to minimize the chance of overlapping linguistic characteristics being present between separate age-brackets (e.g., the difference in language use between speakers aged 19-29 and 30-39 is probably smaller than that between age-brackets 19-29 and 50-59). Moreover, only conversations between two participants of the same (new regrouped) age-bracket is kept. This is done to avoid the confounding factor that interlocutors of a certain age-group might adjust their choice of words to the age of the person they are talking to. Only considering dialogues between participants of similar ages is expected to keep the utterances as representative of the age-bracket's linguistic characteristics as the corpus allows. These final filtration steps result in a subset consisting of 237 dialogues, roughly 172.000 utterances, and approximately 1.8M tokens.




% \begin{itemize}
%     \item Talk about pre-processing steps.
%     \item The main limitations of the BNC are its size and imbalance. 
%     \item Weighted loss and weighted random sampling
%     \item Data outdated? Not representative of modern speech.
% \end{itemize}

% \begin{figure}[H]
%     \centering
%     \includegraphics[width=\textwidth]{figures/bnc_alt1_dist_age_new.png}
%     \caption{Distribution of age-brackets per dialogue in 19-29 vs. 50-plus subset of BNC.}
%     \label{fig:dist_age_bnc}
% \end{figure}



% \subsection{The Blog Authorship Corpus (BAC)}

% % \begin{table}[H]
% %     \centering
% %     \begin{tabular}{l | c  c}
% %         \hline
% %         \textbf{Corpus} & BNC & BAC \\
% %         \hline
% %         \textbf{Dialogue?} & Yes & No\\
% %         \textbf{No. words} & $10.4 \cdot 10^6$  & $140 \cdot 10^6$ \\
        
% %         \hline
% %     \end{tabular}
% %     \caption{Summary of corpora properties.}
% %     \label{tab:summary_corpora}
% % \end{table}

% \begin{table}[H]
%     \centering
%     \begin{tabular}{l | c  c}
%         \hline
%         \textbf{Corpus} & BNC & BAC \\
%         \hline
%         \textbf{Dialogue?} & Yes & No\\
%         \textbf{No. words} & $10.4 \cdot 10^6$  & $140 \cdot 10^6$ \\
%         \textbf{No. datapoints} & 64,994 & 677,244 \\
%         \textbf{No. age brackets} & 2 & 3 \\
%         \textbf{Year of collection} & 2016 & 2004 \\
%         \textbf{Mean sample length} & 6.389 & 102.168 \\
%         \textbf{Std. sample length} & 9.929 & 212.935 \\
%         \textbf{Min. sample length} & 1 & 1 \\
%         \textbf{Max sample length} & 724 & 71580 \\
%         \textbf{No. unique topics} & 790 & 40 \\
%         \hline
%     \end{tabular}
%     \caption{Summary of corpora properties \textit{after} pre-processing.}
%     \label{tab:summary_corpora}
% \end{table}

We use a dataset of dialogue data where information about the age of the speakers involved in the conversation is available (see the dialogue snippets in Figure~\ref{fig:ex1}), i.e., the spoken partition of the British National Corpus~\citep{love-spoken-bnc-2014}. We henceforth refer to it as our \emph{dialogue} dataset. For comparison with previous work,
and to explore commonalities and differences between various types of language data,
we also experiment with a dataset of discourse, i.e., the Blog Authorship Corpus used by \citet{schler2006effects}, that we henceforth refer to as our \emph{discourse} dataset.
Below, we briefly describe the two datasets along with the pre-processing steps we took to
make the data suitable for 
our experiments.

%------------ FIGURE -----------------
\begin{figure}[]\centering \small
% \includegraphics[height=3.2cm]{images/white_dog_COCO_train2014_000000522288.jpg}\\
%\includegraphics[height=2.7cm]{images/other_dog.png} \\
% \vspace*{2pt}
\fbox{\begin{minipage}{.5\columnwidth}
\begin{tabular}{@{}l@{\ }l}
& \textbf{age 19-29}\\[2pt]
                  & \textcolor{red}{\textbf{A:}} oh that's cool\textcolor{white}{aaaaaaaaaaaaaaaaaaaaaaaaaaaaaaaaaaaa}\\
% $\leadsto$ & 
& \multicolumn{1}{r}{\textcolor{blue}{\textbf{B:}} different sights and stuff}\\
% $\leadsto$ & 
& \textcolor{red}{\textbf{A:}} oh
\end{tabular}
\end{minipage}
}\\[1pt]
\fbox{\begin{minipage}{.5\columnwidth}
\begin{tabular}{@{}l@{\ }l}
& \textbf{age 50+}\\[1pt]
                  & \textcolor{red}{\textbf{A:}} well quite and I'd have to come back as well\textcolor{white}{aaaaaaaa}\\
% $\leadsto$ 
& \multicolumn{1}{r}{\textcolor{blue}{\textbf{B:}} that's of course}\\

% $\leadsto$ 
& \textcolor{red}{\textbf{A:}} and make up for you know
\end{tabular}
\end{minipage}
}
\caption{Example dialogue snippets from speakers of different age groups (19-29 vs.\ 50+) in the British National Corpus.
% \san{To be double-checked that A and B is correct + why they are not adjacent turns.}
We conjecture that stylistic and lexical differences between age groups can be detected.
In our approach, we experiment at the level of the single utterance.}\label{fig:ex1}
\end{figure}
%-------------END OF FIG---------------


\begin{table*}[h]
\resizebox{\linewidth}{!}{
\begin{tabular}{lllllll}
\toprule
\textbf{dataset}   & \textbf{\# age groups} & \textbf{\# samples} & \textbf{\# tokens} & \textbf{mean length ($\pm$std)} & \textbf{min - max length} & \textbf{\# topics} \\
\midrule
% dialogue  & 2             & 64,967     & 420,122     & 6.5 ($\pm$10.1)         & 1 - 724           & 790              \\
dialogue  & 2             & 67,282     & 787,352     & 11.7 ($\pm$19.0)         & 1 - 1246           & 790              \\
% discourse & 3             & 677,244    & 140M      & 102.2 ($\pm$212.9)     & 1 - 71,580         & 40  \\
discourse & 3             & 678,165    & 137M      & 201.7 ($\pm$415.9)     & 1 - 131,169         & 40  \\
\bottomrule
\end{tabular}
}
    \caption{Descriptive statistics of the datasets used in our experiments. {Length} is the number of tokens in a sample.}

    
    \label{tab:summary_corpora}
\end{table*}

\subsection{Dialogue Dataset}\label{sec:datadialogue}
\label{subsec:dialogue_dataset}

% \san{Here we only describe BNC: remember to make comparisons with BAC wrt length of utterances; number of age brackets, etc.}

This partition of the British National Corpus includes spoken informal open-domain conversations between people that were collected between 2012 and 2016 via crowd-sourcing, and then recorded and transcribed by the creators. Dialogues can be between two or more interlocutors, and are annotated
% at the utterance level 
along several dimensions including age and gender together with geographic and social indicators. Speaker ages in the original dataset are categorized in the following ten brackets: 0-10, 11-18, 19-29, 30-39, 40-49, 50-59, 60-69, 70-79, 80-89, and 90-99.


We focus on conversations in the British National Corpus that took place between two interlocutors,
and only consider dialogues
between people of the same age group. 
We then focus on dialogues by speakers belonging to two age groups: 19-29 and 50+, in which we group conversations from five original brackets: 50-59, 60-69, 70-79, 80-89, and 90-99.
% one where speakers' age ranges from 19 to 29 (hence, 19-29); one where it is equal to or greater than 50 (hence, 50+). 
We omit the intermediate age brackets to allow for clearer differentiation.
% The dialogues between speakers of intermediate ages are discarded. 
% These two steps are performed to create sufficiently separated classes of comparable size. 

We split the dialogues into their constituent utterances (e.g., from each dialogue snippet in Figure~\ref{fig:ex1} we extract three utterances), and further pre-process them by removing non-alphabetical characters.
% and stopwords, by using the Natural Language Toolkit (NLTK) English stopword list \cite{bird2009natural}. 
Only samples which were not empty after pre-processing were kept.
The resulting dialogue dataset, that we use for our experiments, includes around 67K utterances with an average length of 11.7 tokens. Descriptive statistics of it are reported in Table~\ref{tab:summary_corpora}.

Each conversation in the British National Corpus is annotated with a list of \emph{topics} provided by the speakers during data collection. However, we wish to obtain a better picture of how single representative dialogue topics are distributed among the age groups.
% Since topic may constitute a confound for models toward detecting speakers' age group, we leverage this annotation to extract one single topic per utterance. 
% \raq{In which sense do we leverage this information? The analysis we currently have of topics doesn't seem to shed light on whether it is a confound or not... Maybe give less importance to topics (no paragraph heading) - it's simply part of the descriptive statistics?} 
Thus, to extract a single representative topic from this list, we first compute the frequency of all topic labels in the whole dataset. Then, for each utterance, we take the label in the conversation with the highest frequency in the ranking.
In total, our final dataset includes 790 unique topic labels. The distribution of the most frequent ones is reported in Figure~\ref{fig:bnc_age_topic_dist_incl_unk}. As can be seen, frequent topics (besides the frequent \emph{none} label) are \emph{food}, \emph{work}, and \emph{holidays}, which reveals the colloquial and everyday nature of the dialogues in this dataset.

\subsection{Discourse Dataset}
% \san{Here we only describes BAC}

% The second corpus is a dataset of 
The Blog Authorship Corpus~\citep{schler2006effects} is a collection of blog posts
% dataset is a corpus of blogs
written on \url{https://www.blogger.com}, gathered in or before August 2004. % All blog entries are 
Each blog entry is
written by a single user
whose age, gender, and astrological sign are reported.
The corpus contains almost 700,000 posts by 19,000 unique bloggers (i.e., $\sim$35 posts per blogger on average).
% The original corpus consists of blog entries written by more than 19,000 bloggers, for a total of almost 700,000 posts, and over 140M words. 
For our experiments, similar to~\citet{schler2006effects},
we consider three age groups: 13-17, 23-27, and 33+. We pre-process the data in the same way as described above, namely by removing non-alphabetical characters. The resulting dataset, that we use for our experiments, includes slightly more than 678K samples with an average length of 201.7 tokens. Descriptive statistics of it are reported in Table~\ref{tab:summary_corpora}. 
% \len{TODO - Adjust paragraph to W.S. setting.}

% only consider three age brackets, leaving gaps in between the age groups to avoid as much overlapping language use as possible between the classes. Similar to \citeauthor{schler2006effects}, we consider the following age brackets: 13-17, 23-27, and 33+.
% for which information about age of the speaker / writer is available, the spoken component of the British National Corpus (BNC or BNC2014) \citep{love-spoken-bnc-2014}, and the Blog Authorship Corpus (BAC, or sometimes referred to as `blog corpus') \san{let's use just one abbreviation per corpus; I'd say BNC and BAC; but we can also use \emph{discourse} and \emph{dialogue}} \citep{schler2006effects}.

%\paragraph{Topics}
Each sample in the Blog Authorship Corpus is annotated with one topic. In our final discourse dataset, the unique topics present are 40. Figure~\ref{fig:blog_age_topic_dist_incl_unk} reports the distribution of the most frequent ones. As can be noted, frequent topics are \emph{student}, \emph{arts}, and \emph{technology}, which 
reveals that this and the dialogue dataset are rather different with respect to the topic themes, input text length, and how well the topics are organized. 
Namely, the discourse dataset has a clear over-representation of topics related to student life, compared to the more colloquial and everyday subject matter of the dialogue dataset. Furthermore, the descriptive statistics in Table~\ref{tab:summary_corpora} show that the average data-entry length in tokens of the discourse dataset is almost 20 times larger than that of the dialogue dataset. Additionally, the differences number of unique topics (790 in dialogue dataset versus 40 in discourse dataset) between the two datasets illustrates the increased noisiness of the dialogue dataset compared to the discourse dataset. The shorter and noisier form of the dialogue dataset could make it more challenging to perform automated age group detection on its texts, as they are likely to carry less and more distorted discriminative signal.
% and has only one topic per data-entry, compared to an average of \len{find average topic list length of bnc} topics per dialogue.


\begin{figure}[H]
     \centering
     \begin{subfigure}[b]{0.45\textwidth}
        \centering
        \includegraphics[width=\textwidth]{figures/bnc_age_topic_histogram.png}
        \caption{}
        \label{fig:bnc_age_topic_dist_incl_unk}
     \end{subfigure}
     \hfill
     \begin{subfigure}[b]{0.45\textwidth}
        \centering
        \includegraphics[width=\textwidth]{figures/blog_age_topic_histogram.png}
        \caption{}
        \label{fig:blog_age_topic_dist_incl_unk}
     \end{subfigure}
        \caption{Distribution of most frequent topics shown by age group, in the \textbf{dialogue dataset} ((a) / left) and \textbf{discourse dataset} ((b) / right). Best viewed in color.}
        \label{fig:bnc_blog_age_topic_dist_incl_unk}
\end{figure}






