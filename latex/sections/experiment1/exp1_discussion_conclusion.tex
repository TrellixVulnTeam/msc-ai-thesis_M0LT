\section{Discussion and Conclusion}\label{sec:exp1_discussion_conclusion}

In this chapter, we studied the extent to which purely text-based NLP models can detect age-related linguistic features in dialogue and discourse data.
Our experiment on the discourse dataset confirmed and extended the work of \cite{schler2006effects} by using end-to-end trained NLP models, instead of handcrafted features, and by also applying this approach to transcribed spontaneous open-domain dialogues.
Focusing on the dialogue dataset, we subsequently carried out and in-depth analysis of the best performing models' predictions to gain insights about the elements of language deemed most informative when classifying dialogue utterances into age groups.
In line with what we observed for discourse, we showed that state-of-the-art NLP models (in particular a fine-tuned version of BERT) are capable of distinguishing between dialogue utterances from different age groups with reasonable accuracy, in particular when the utterance is long enough to contain discriminative signal.
However, different from what we observed for discourse data, we found that much simpler models based on $n$-grams achieve comparable performance to fine-tuned BERT. 
This finding suggests that, in dialogue, ‘local’ features can be indicative of the language of speakers from different age groups. Qualitative inspection seems to confirm this postulation, as we found both lexical and stylistic features to be informative to both the BERT-based and trigram-based models.

Now that we have established the feasibility of detecting age-related linguistic features in dialogue, we aim to endow generated dialogue responses with similarly detectable linguistic features that align with target age groups. 
Our results obtained on the dialogue dataset can inform our subsequent experiment about age-adaptive dialogue generation. 
Primarily, they provide insights about the age-related linguistic features that are most salient when distinguishing between, and subsequently adapting to, different age groups.
Furthermore, our findings motivate the use of local, lexical level linguistic features, when controlling dialogue generation towards a target age group, as a comparison to more syntax-level adaptation.

% \begin{itemize}
%     \item 
%     \item Now that we have established the feasibility of detecting age-related linguistic features in dialogue, we aim to endow generated dialogue responses with similarly detectable linguistic features that align with target age groups.
%     \item Our results obtained on the dialogue dataset can inform our subsequent experiment about age-adaptive dialogue generation. Primarily, they provide insights about the age-related linguistic features that are most salient when distinguishing between, and subsequently adapting to, different age groups. Furthermore, our findings motivate the use of local, lexical level linguistic features, when controlling dialogue generation towards a target age group, as a comparison to more syntax-level adaptation.
% \end{itemize}


% \textbf{Keep in mind when writing this section}
% \begin{itemize}
%     \item Goals of this section: a discussion/conclusion which also paves the way to and motivates Experiment 2.
%     \item Recap the results and what we have learned from the analysis.
%     \item (Based on the results and analyses) Make hypotheses that are relevant for the following experiment
%     \item However, don't repeat (too much) what is already stated in the next section (i.e., the introduction to Experiment 2)
%     \item Do I need a discussion here, given that there is a general discussion (i.e., for both experiments) at the end of the thesis?
% \end{itemize}

% Useful phrases from age detection paper (EMNLP submission)

% \begin{itemize}
%     \item We investigated whether, and to what extent, NLP models can detect age-related linguistic features in dialogue data.
%     \item We showed that, in line with what we observed for discourse, state-of-the-art models are capable of doing so with a reasonable accuracy, in particular when the dialogue fragment is long enough to contain discriminative signal.
%     \item At the same time, differently from discourse, we found that much simpler models based on n-grams achieve comparable performance, which suggests that, in dialogue, ‘local’ features can be indicative of the language of speakers from different age groups.
%     \item We showed this to be the case, with both lexical and stylistic cues being informative to these (and possibly all) models in performing the task.
% \end{itemize}

% \textbf{How will these results inform the controlled generation experiments?}

% \begin{itemize}
%     \item Do I need this? It's literally one of the first things I talk about on the next page. (maybe have a call with sandro about it)
% \end{itemize}